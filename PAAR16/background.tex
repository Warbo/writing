\section{Background}
\label{sec:background}

\subsection{Haskell}
\label{sec:haskell}

\newcommand{\CVar}{\texttt{Var}}
\newcommand{\CLit}{\texttt{Lit}}
\newcommand{\CApp}{\texttt{App}}
\newcommand{\CLam}{\texttt{Lam}}
\newcommand{\CLet}{\texttt{Let}}
\newcommand{\CCase}{\texttt{Case}}
\newcommand{\CType}{\texttt{Type}}
\newcommand{\CLocal}{\texttt{Local}}
\newcommand{\CGlobal}{\texttt{Global}}
\newcommand{\CConstructor}{\texttt{Constructor}}
\newcommand{\CLitNum}{\texttt{LitNum}}
\newcommand{\CLitStr}{\texttt{LitStr}}
\newcommand{\CAlt}{\texttt{Alt}}
\newcommand{\CDataAlt}{\texttt{DataAlt}}
\newcommand{\CLitAlt}{\texttt{LitAlt}}
\newcommand{\CDefault}{\texttt{Default}}
\newcommand{\CNonRec}{\texttt{NonRec}}
\newcommand{\CRec}{\texttt{Rec}}
\newcommand{\CBind}{\texttt{Bind}}

We decided to focus on theory exploration in the Haskell programming language as it has mature, state-of-the-art implementations (\qspec{} \cite{QuickSpec} and \hspec{} \cite{claessen2013automating}). This is evident from the fact that the state-of-the-art equivalent for Isabelle/HOL, the \textsc{Hipster} \cite{Hipster} system, is actually implemented by translating to Haskell and invoking \hspec{}.

Like most functional programming languages, Haskell builds upon $\lambda$-calculus, with extra features such as a strong type system and ``syntactic sugar'' to improve readability. To avoid the relatively large and complex syntax of Haskell proper, we will focus on an intermediate representation of the \textsc{GHC} compiler known as \emph{GHC Core}, described in Appendix \ref{sec:core}.

We will use quoted strings to denote names and literals, e.g. \texttt{Local "foo"}, \texttt{Global "bar"}, \texttt{Constructor "Baz"}, \texttt{LitStr "quux"} and \texttt{LitNum "42"}, and require only that they can be compared for equality.

\begin{figure}
  \begin{haskell}\begin{verbatim}
data Bool = True | False

data Nat = Z | S Nat

not :: Bool -> Bool
not True  = False
not False = True

odd :: Nat -> Bool
odd    Z  = False
odd (S n) = even n

even :: Nat -> Bool
even    Z  = True
even (S n) = odd n\end{verbatim}
  \end{haskell}
  \caption{Haskell datatypes for booleans and natural numbers, followed by some simple function definitions. Note that \hs{odd} and \hs{even} are mutually recursive.}
  \label{fig:haskellexample}
\end{figure}

\begin{figure}
  \begin{small}
    \underline{\texttt{not}}
    \begin{verbatim}
Lam "a" (Case (Var (Local "a"))
              "b"
              (Alt (DataAlt "True")  (Var (Constructor "False")))
              (Alt (DataAlt "False") (Var (Constructor "True"))))
    \end{verbatim}

    \underline{\texttt{odd}}
    \begin{verbatim}
Lam "a" (Case (Var (Local "a"))
              "b"
              (Alt (DataAlt "Z") (Var (Constructor "False")))
              (Alt (DataAlt "S") (App (Var (Global "even"))
                                      (Var (Local  "n")))
                                 "n"))
    \end{verbatim}

    \underline{\texttt{even}}
    \begin{verbatim}
Lam "a" (Case (Var (Local "a"))
              "b"
              (Alt (DataAlt "Z") (Var (Constructor "True")))
              (Alt (DataAlt "S") (App (Var (Global "odd"))
                                      (Var (Local  "n")))
                                 "n"))
    \end{verbatim}
  \end{small}
  \caption{Translations of functions in Figure \ref{fig:haskellexample} into our Core syntax, with fresh variables chosen arbitrarily as \texttt{"a"}, \texttt{"b"}, etc. Notice the introduction of explicit $\lambda$ abstractions (\texttt{Lam}) and the use of \texttt{Case} to represent piecewise definitions. The \texttt{"b"} variables are introduced to preserve sharing.}
  \label{fig:coreexample}
\end{figure}

Figure \ref{fig:haskellexample} shows some simple Haskell function definitions, along with common definitions for boolean and natural number datatypes (analogous to those in Haskell's standard library). The translation to our Core syntax is routine, and shown in Figure \ref{fig:coreexample}. Although Core is more verbose, we can see that similar structure in the Haskell definitions gives rise to similar structure in the Core; for example, the definitions of \hs{odd} and \hs{even} are identical in both languages, except for the particular identifiers used. It is this close correspondence which allows us to analyse Core expressions in place of their more complicated Haskell source.

\subsection{Theory Exploration}
\label{sec:theoryexploration}

We consider the problem of \emph{(automated) theory exploration}, which includes the ability to \emph{generate} conjectures about code, to \emph{prove} those conjectures, and hence output \emph{novel} theorems without guidance from the user. The method of conjecture generation is a key characteristic of any theory exploration system, although all existing implementations rely on brute force enumeration to some degree.

We focus on \qspec{} \cite{QuickSpec}, which conjectures equations about Haskell code. These may be fed into another tool, such as \hspec{}, for subsequent proving. These conjectures are arrived at through the following stages:

\iffalse TODO: Make this more formal?
 V \in Var
 F \in Fun
 T \in Term
 T ::= V | F | T1 T2

 Term ::= VAR | Const | Fun (Term)
or
 Term t ::= x | f | t t'
\fi

\begin{enumerate}
  \item Given a typed signature $\Sigma$ and set of variables $V$, \qspec{} generates a list $terms$ containing the constants (including functions) from $\Sigma$, the variables from $V$ and type-correct function applications $f(x)$, where $f$ and $x$ are elements of $terms$ \iffalse TODO: A little awkward; maybe use the above notation? \fi. To ensure the list is finite, function applications are only nested up to a specified depth (by default, 3).
  \item The elements of $terms$ are grouped into equivalence classes, based on their type.
  \item The equivalence of terms in each class is tested using \qcheck{}: variables are instantiated to particular values, generated randomly, and the resulting closed expressions are evaluated and compared for equality.
  \item If a class is found to have non-equal members, it is split up to separate those members.
  \item The previous steps of testing and splitting are repeated until the classes stabilise (i.e. no differences have been observed for some specified number of repetitions).
  \item For each class, one member is selected and equations are conjectured that it is equal to each of the other members.
\end{enumerate}

Such conjectures can be used in several ways: they can be presented to the user, sent to a more rigorous system like \hspec{} for proving, or even serve as a background theory for an automated theorem prover \cite{claessen2013automating}. As an example, giving the definitions from Figure \ref{fig:haskellexample} to \qspec{}. along with some suitable \hs{Nat} and \hs{Bool} variables, produces equations including \hs{not (not a) == a} and \hs{not (odd x) == even x}.

\iffalse
As an example, we can consider a simple signature containing the expressions from Figure \ref{fig:haskellexample}:

\begin{align*}
  \Sigma_{\texttt{Nat}} = \{\texttt{Z}, \texttt{S}, \texttt{odd}, \texttt{even}\}
\end{align*}

Together with a set of variables, say $V_{\texttt{Nat}} = \{a, b, c\}$, \qspec{}'s enumeration will resemble the following:

\begin{align*}
  terms_{\texttt{Nat}} = [& \texttt{Z},\ \texttt{S},\ \texttt{odd},\ \texttt{even},\ a,\ b,\ c,\ \texttt{S Z},\ \texttt{S}\ a,\ \texttt{S}\ b, \\
                     & \texttt{S}\ c,\ \texttt{odd Z},\ \texttt{odd}\ a,\ \dots ]
\end{align*}

Notice that functions such as \hs{odd} and \hs{even} are valid terms, despite not being applied to any arguments. In addition, Haskell curries multi-argument functions, allowing them to be applied to one argument at a time, as used in the construction of $terms$.

These terms will be grouped into four classes, one each for \hs{Nat}, \hs{Nat -> Nat}, \hs{Nat -> Bool} and \hs{Bool}. As the variables $a$, $b$ and $c$ are instantiated to various randomly-generated numbers, these equivalence classes will be divided, until eventually the equations such as $odd (S a) = even a$ and $even (S a) = odd a$ are conjectured.
\fi

Although complete, this enumeration approach is wasteful: many terms are unlikely to appear in theorems, which requires careful choice by the user of what to include in the signature. For example, we know that addition and multiplication are closely related, and hence obey many algebraic laws. Our machine learning technique aims to predict these kinds of relations between functions, so we can create small signatures which can be explored quickly, yet have the potential to give rise to many equations.

\iffalse
TODO
In fact, there are similarities between the way a TE system like \qspec{} can generalise from checking \emph{particular} properties to \emph{inventing} new ones, and the way counterexample finders like \qcheck{} can generalise from testing \emph{particular} expressions to \emph{inventing} expressions to test. One of our aims is to understand the implications of this generalisation, the lessons that each can learn from the other's approach to term generation, and the consequences for testing and QA in general.
\fi

\subsection{Clustering}
\label{sec:clustering}

Our approach to scaling up \qspec{} takes inspiration from two sources. The first is relevance filtering, which makes expensive algorithms used in theorem proving more practical by only considering clauses deemed ``relevant'' to the problem \cite{meng2009lightweight}.

Relevance is determined by comparing clauses to the target theorem, but theory exploration does not have such a distinguished term. Instead, we are interested in relationships between \emph{all} terms in a signature, and hence we need a different algorithm for considering the relevance of \emph{all terms} to \emph{all other terms}.

A natural fit for this task is \emph{clustering}, which attempts to group similar inputs together in an unsupervised way. Based on their success in discovering relationships and patterns between expressions in Coq and ACL2 (in the ML4PG and ACL2(ml) tools respectively), we hypothesise that clustering methods can fulfil the role of relevance filters for theory exploration: intelligently breaking up large signatures into smaller ones more amenable to brute force enumeration, such that related expressions are explored together.

Due to its use by ML4PG and ACL2(ml), we use \emph{k-means} clustering, implemented in the Weka tool \cite{Holmes.Donkin.Witten:1994} by Lloyd's algorithm \cite{lloyd1982least}, with randomly-selected input elements as the initial clusters. Rather than relying on the user to provide the number of clusters $k$, we use the ``rule of thumb'' given in \cite[pp. 365]{mardia1979multivariate} of clustering $n$ data points into $k = \lceil \sqrt{\frac{n}{2}} \rceil$ clusters.
