\section{Formula for Gallina tokens}\label{sec:gallinasyntax}


We split Gallina tokens into the following groups.

\begin{itemize}
 \item Group 1: \lstinline?forall?, \lstinline?->?.
 \item Group 2: \lstinline?fun?,
 \item Group 3: \lstinline?let?, \lstinline?let fix?, \lstinline?let cofix?.
 \item Group 4: \lstinline?fix?, \lstinline?cofix?.
 \item Group 5: \lstinline?@?,
 \item Group 6: \lstinline?match?, \lstinline?if?.
 \item Group 7: \lstinline?:=?, \lstinline?=>?, \lstinline?is?.
 \item Group 8: \lstinline?Inductive?, \lstinline?CoInductive?.
 \item Group 9: \lstinline?exists?, \lstinline?exists2?.
 \item Group 10: \lstinline?:?, \lstinline?:>?, \lstinline?<:?, \lstinline?
\end{itemize}



The formula for the $j$th Gallina token of the $n$th group is given by the formula $$- (n + \sum_{i=0}^j \frac{1}{10\times 2^{i-1}})$$
