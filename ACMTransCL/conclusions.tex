\section{Conclusions}\label{sec:conclusions}

We have presented several enhancements of the ML4PG system. \emph{Term clustering} adds a new 
functionality to ML4PG: the user can receive suggestions about families of similar definitions, types and lemma shapes (in fact, any Coq terms). 
The \emph{proof-patch method} is employed to analyse the properties of the patches that constitute a proof. The whole syntax of Coq libraries is now subject to \emph{recurrent clustering}, which %combination
%of term clustering and proof-patches
 increases the number and accuracy of families of similar proofs suggested by ML4PG. In addition, this information is taken into account in an automatic proof-generation method. Finally, the different visualisation tools implemented in ML4PG facilitates the interpretation of clusters of similar proof-patches and terms. 

Further improvements in \emph{accuracy} (e.g. including proof contexts into the analysis) and \emph{conceptualisation} for clustered terms and proofs are planned. The families of similar proofs and terms can be the basis to
apply symbolic techniques to, for instance, infer models from proof traces~\cite{GNR14}, or generate auxiliary results using
mutation of lemmas~\cite{lpar13}. We expect that the incorporation of these techniques help in the goal pursued by ML4PG: make the 
proof development easier. 




 
%  \textbf{Related Work}
%  on using statistical machine-learning in ITPs was focused on speeding up proof automation in different ITPs~\cite{lpar-urban,mash,UrbanSPV08}. 
%  Comparing to these tools, we use unsupervised, rather than supervised, learning; and we do not use sparse  machine-learning methods.
%  (See also \cite{KHG13} for a detailed comparison of different machine learning tools applied in various theorem provers.)
%  We do not have a quantitative target when it comes to improving \emph{interactive} proof building experience: 
%  no longer speed up in automated proof search or the number of automatically proven theorems are the main criteria of success.
%  Instead, the user experience is the main parameter we target.
%  We generally follow the ``qualitative'' intuition that ML4PG, being an interactive hint generator, must provide interesting and non-trivial hints on user's demand,
%  and should be flexible and fast enough to do so in real  time,  at any stage of the proof, and relative to any chosen proof library. 
