\section{Conclusions}\label{sec:conclusions}

We have presented several enhancements of the ML4PG system. \emph{Term clustering} adds a new
functionality to ML4PG: the user can receive suggestions about families of similar definitions, types and lemma shapes (in fact, any Coq terms).
The \emph{proof-patch method} is employed to analyse the properties of the patches that constitute a proof. The whole syntax of Coq libraries is now subject to \emph{recurrent clustering}, which
 increases the number and accuracy of families of similar proofs suggested by ML4PG. In addition, this information is taken into account in an automatic proof-generation method. Finally, the different visualisation tools implemented in ML4PG facilitates the interpretation of clusters of similar proof-patches and terms.

Further improvements in \emph{accuracy} (e.g. including proof contexts into the analysis) and \emph{conceptualisation} for clustered terms and proofs are planned. The families of similar proofs and terms can be the basis to
apply symbolic techniques to, for instance, infer models from proof traces~\cite{GNR14}, or generate auxiliary results using
mutation of lemmas~\cite{lpar13}. We expect that the incorporation of these techniques help in the goal pursued by ML4PG: make the
proof development easier.
