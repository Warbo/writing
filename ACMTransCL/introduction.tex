\section{Introduction}\label{sec:introduction}


Development of Interactive Theorem Provers (ITPs) has led to the creation of big libraries and varied infrastructures for formal
mathematical proofs. 
These frameworks usually involve thousands of definitions and theorems (for instance, there are approximately
4200 definitions and 15000 theorems in the formalisation of the Feit-Thompson theorem~\cite{FTT}). % Definitions,
%theorems and proof-strategies used in
Parts of  those libraries can often be re-applied in new domains; however, 
it is a challenge for expert and non-expert users alike to trace them, and find re-usable concepts and proof ideas.


% 
% \begin{figure}
%  \centering
%  \begin{tikzpicture}[scale=.5]
%  
% \draw[fill=gray,draw=gray] (-1.95,.45) rectangle (2.05,-.55);  
% \draw[fill=white] (-2,.5) rectangle (2,-.5); 
% \node (0,0) {{\scriptsize Proof General} };
%  
% 
% \draw[fill=gray,draw=gray] (7.95,.45) rectangle (12.15,-.55);  
% \draw[fill=white] (7.9,.5) rectangle (12.1,-.5); 
% \draw (10,0) node{{\scriptsize MATLAB/Weka}};
% 
% 
% \draw (5,0) node{\textbf{ML4PG}};
% 
% 
% \draw[latex-,shorten <=2pt,shorten >=2pt,dashed] (7.9,.4) .. controls (6,1) and (4,1) .. (2,.4); 
% \draw (5,1.3) node[anchor=north,fill=white]{\emph{\tiny{feature extraction}}};
%  
% \draw[latex-,shorten <=2pt,shorten >=2pt,dashed] (2,-.4) .. controls (4,-1) and (6,-1) .. (7.9,-.4); 
% \draw (5,-1.3) node[anchor=south,fill=white]{\emph{\tiny{proof families}}}; 
%  
%  
% \draw[fill=white,rounded corners] (-2,-2) rectangle (2,-3); 
% \draw (0,-2.3) node{{\tiny Interactive Prover:}} ;
% \draw (0,-2.7) node{{\tiny Coq, SSReflect}} ;
% 
% 
% \draw[dotted,shorten <=2pt,shorten >=2pt,-latex] (0,-.5) -- (0,-2);
% \draw[dotted,shorten <=2pt,shorten >=2pt,-latex] (10,-.5) -- (10,-2); 
%  
% \draw[fill=white,rounded corners] (7.7,-2) rectangle (12.3,-3); 
% \draw (10,-2.3) node{{\tiny Clustering:}} ;
% \draw (10,-2.7) node{{\tiny K-means, Gaussian, \ldots}} ;
%  
%  
% \end{tikzpicture}
% \caption{\emph{ML4PG architecture.}}\label{fig:ml4pg-architecture}
% \end{figure}


% 
% Coq/SSReflect already provides comprehensive search mechanisms to search the corpus of results available in different libraries.
% There are several search commands in Coq:  \lstinline?Search?~, \lstinline?SearchAbout?, \lstinline?SearchPattern?
% and \lstinline?SearchRewrite?~\cite{Coq}. In addition, SSReflect implements its own version of the \lstinline?Search?
% command \cite{SSReflect} -- SSReflect's \lstinline?Search? gathers the functionality of the 4 Coq's search commands.
% The Whelp platform~\cite{AspertiGCTZ04} is a web search engine for mathematical knowledge formalised in Coq, which features 3 functionalities:
% \lstinline?Match? (similar to Coq's \lstinline?Search? command), \lstinline?Hint? (that finds all the theorems which can
% be applied to derive the current goal) and \lstinline?Elim? (that retrieves all the eliminators of a given type).
% %In this section, we compare ML4PG with these tools.  
% 
% %Let us explain first when these tools are usually invoked. 
% The existing search mechanisms can be used in two different scenarios. If the user knows how to continue the proof, but
% he does not remember (or know) the concrete name of the desired auxiliary lemma, it suffices to provide a search pattern to Coq searching engines, e.g.
% using commands of the form
% ``\lstinline?Search "distr"? \lstinline?in bigop?'' or   
% ``\lstinline?Search _ (_ * (\big[_/_]_(_ <- _| _) _))?'',
%  where \texttt{bigop} is a library, \texttt{"distr"} is a pattern in a lemma name, and  \lstinline? _ (_ * (\big[_/_]_(_ <- _| _) _))?
% is a pattern for search.
%  
% %\begin{example}\label{ex:distr}
% % In the proof of Lemma~\ref{lem:nilpotent}, we may want to introduce the term $(1-M)$ inside the sum $\sum_{i=0}^{n-1} M^i$, but 
% % if we have no experience with the bigop library of SSReflect, it is likely that we do not know the name of the concrete lemma. 
% % The lemma which carries out this task can be find using, for instance, the commands \lstinline?Search "distr" in bigop.? or  
% %\lstinline?Search _ (_ * (\big[_/_]_(_ <- _| _) _))?. 
% %\end{example}
% In the second scenario, the user needs help in the middle of the proof. Searching mechanisms can be also useful in this 
% case; for instance, using the \lstinline?Search? command the user can search all the lemmas associated with a concrete term
% of the current goal. The \lstinline?Hint? mechanism of Whelp can find all the theorems which can be applied to derive the current goal.
% 
% %\begin{example}\label{ex:fact}
% % In the proof of Lemma~\ref{lem:factorial}, we can find all the lemmas related to the factorial function using the command
% % \lstinline?SearchAbout factorial?. 
% %\end{example}
% 
% %ML4PG is helpful in the second scenario, since it provides families of similar proofs. Inspecting that family of proofs, 
% %the user can find a pattern to finish the proof that she is formalising. 
% 
% %As we have seen in Examples~\ref{ex:distr} and~\ref{ex:fact}, 
% The above search mechanisms are goal directed and deterministic. That is, the user
% searches  the chosen  libraries for lemmas related to a concrete type, term or pattern. 
% If the patterns defined by the user are present in the given library, then the user is guaranteed to see the relevant lemmas on the screen.
% 
% The situation is more complicated if the user does not know the right pattern to search for. Imagine, for example, being in the middle of constructing a proof, 
% and wishing to get some higher-level hint on how to proceed, wishing there was an ITP expert near, who would suggest a further proof strategy based on his previous experience.


The ML4PG (``Machine-Learning for Proof-General'') tool~\cite{KHG13,CICM13,HK14} was created to help the user in such a situation for the particular case of the Coq system~\cite{Coq} and its SSReflect extension~\cite{SSReflect} developments. ML4PG uses statistical machine-learning algorithms to discover re-usable proof-patterns based on ``previous experience'' acquired from previous developments. A proof-pattern was defined as a correlation between the tactics and the types/shapes of subgoals resulting from the tactic applications within a few proof steps. The resulting tool could indeed find some interesting --- unexpected and yet relevant --- proof-patterns,
across different notation, types, and libraries. Our experiments spanned several subjects: basic mathematical infrastructures, Computer Algebra, Game Theory, and the certification of Java-like bytecode. The results are best summarised in~\cite{CICM13,HK14}.

% 
% In comparison to the more traditional \emph{search engines}, ML4PG is a goal-independent (unsupervised) tool, i.e. the user does not have to know the required pattern in advance.
% But, as many 
% statistical machine-learning applications, it is also non-deterministic. That is, the tool failing to suggest a proof pattern does not mean there is no ``interesting'' pattern to be found.
% Actually, the notion of a proof pattern being ``interesting'' is left to the user's judgement, as well. 


That initial approach had two inherent limitations. First, the essence of a Coq/SSReflect proof is not fully expressible by a sequence of applied tactics. The definitions, types, and shapes of auxiliary lemmas used in a proof can be much more sophisticated and conceptual than a proof script calling them. Therefore, although ML4PG could find interesting, and often useful, sequences of tactics; it could not go further to recognise, for instance, similar definitions. Second, the notion of a proof-pattern being ``interesting'' or useful is left to the user's judgement.
 
In this paper, we present the most recent extensions to ML4PG  involving all kinds of Coq terms --- type declarations, definitions and lemma statements --- into the process of proof-pattern search. This required additional algorithms of feature extraction that reflect the mutual dependency of various proof objects in Coq\rq{}s dependently-typed setting; see Sections~\ref{sec:lemmaclustering} and~\ref{sec:reclemmaclustering}. 
This major step in ML4PG development prompted other improvements. The initial ML4PG was considering features arising from the first 5 proof-steps in a proof, whereas now we treat every proof as a collection of
proof patches, each potentially
representing an interesting proof strategy. Moreover, if say 15th-20th step in one proof resembles a 115th-120th step in another, the tool is now able to detect such patterns deep down.  
The feature extraction algorithms for proof features have been further refined to include the data collected from Coq terms, and now the 
whole syntax of the chosen proof libraries is subject
to \emph{recurrent clustering} --- a novel technique for ML4PG. %, but see~\cite{lpar13}. 
All these extensions to proof-feature extraction are explained in Section~\ref{sec:recurrent}. 

ML4PG used to show, in response to the user's call, a set of similar proofs,
with no hints of why these proofs are deemed similar. We now introduce two extensions that improve ML4PG's user-experience. 
First, we have designed a method to automatically proof theorems based on ML4PG clustering methods (cf. Section~\ref{sec:evaluation}). Additionally, several visualisation tools (using automaton-shape or tree-shape representations) have been developed to show the proof-features that correlate, see Section~\ref{sec:visualisation}. This partially addresses the drawback of the subjective approach to 
the pattern's ``interestingness'' --- now 
the tool clearly declares correlation of which proof features defined the suggested proof pattern. 
Finally, in Section~\ref{sec:relatedwork}, we compare ML4PG with other machine-learning and searching approaches available in the literature, and conclude the paper in Section~\ref{sec:conclusions}.


Examples we use throughout the paper come from several Coq and SSReflect libraries: the basic infrastructure of SSReflect~\cite{SSReflect}, a matrix library~\cite{GarillotEtAl09}, a formalisation of persistent homology~\cite{HCMS12}, the HoTT library~\cite{hottbook}, two formalisation related to Nash equilibrium~\cite{Ves06,nash}, and the formalisation about Java-like bytecode presented in~\cite{HK14}. ML4PG is a part of standard Proof General distribution; the novel features we present here are available at~\cite{HK12}. % and can be used in any Coq or SSReflect development. 





 
%On the contrary, the proof patterns found by
%ML4PG are unknown in advance. In this way, ML4PG can find some unexpected results but which can still be useful for the user. 

%\begin{example}\label{ex:cancel}
 % The command \lstinline?Search _ (\sum_(_ <- _) (_ - _))?  can find Lemmas~\ref{lem:lemma3} and~\ref{lem:fundamental}; this can 
 % be useful in the proof of Lemma~\ref{lem:nilpotent} since the three lemmas follow the same pattern. However, Lemma~\ref{lem:nilpotent2}
 % whose proof is closer to the proof of Lemma~\ref{lem:nilpotent} (see Table~\ref{tab:compare}) cannot be found using the previous search command
 % or other commands involving the terms of Lemma~\ref{lem:nilpotent}: \lstinline?Search _ (\sum_(_ <- _) _ *m _)?, 
 % \lstinline?Search _ (_ *m (_ - _))? or \lstinline?Search _ (_ *m _)? (this last command finds Lemma~\ref{lem:nilpotent2}, but there are 
 % approximately 120 additional suggestions).
%\end{example}

%Finally, we can consider the relevance of suggestions provided by search mechanisms and ML4PG. In both cases, some of the results returned by these tools
%can be useless; however, they can be refined modifying the input of the search mechanisms or changing some parameters in ML4PG. 

%\begin{example}\label{ex:relevance}
% In Benchmark 1, a search of all the lemmas involving big operations produce more than $300$ lemmas. If we refine the search to lemmas involving $\sum$,
% we can find $250$ lemmas; and the command \lstinline?Search _ (\sum_(_ <- _) (_ - _))? produces $4$ lemmas. In the same way, we have seen in Table~\ref{tab:compare}
% that the modification of parameters such as the granularity or the algorithm produce better results. %
%\end{example}

















%The main contributions of this paper are: 

%\begin{itemize}
% \item a new feature extraction method based on the shape of Coq terms, see Section~\ref{sec:lemmaclustering}.
% \item an extension of ML4PG's feature extraction method~\cite{KHG13} to capture information from complete 
% proofs of varied sizes using patches, see Section~\ref{sec:patches}.
% \item an evaluation of these techniques in one of the most relevant developments in theorem proving: 
% the Feit-Thompson theorem~\cite{FTT}, see Section~\ref{sec:evaluation}.
 
 
%\end{itemize}
