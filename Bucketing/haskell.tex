\subsection{Haskell}
\label{sec:haskell}

We decided to focus on the Haskell programming language since it has direct
applications in industry, a large selection of library code for testing and
mature, state-of-the-art MTE tools including \qspec{}~\cite{QuickSpec},
\hspec{}~\cite{claessen2013automating} and
\textsc{Speculate}~\cite{braquehais2017speculate}. Our overall approach is
agnostic to the underlying system, so could also be applied to other targets
such as the Isabelle/HOL, which has the
\textsc{IsaCoSy}~\cite{johansson2009isacosy}, \textsc{IsaScheme}~\cite{omar} and
more recent \textsc{Hipster}~\cite{Hipster} MTE tools. However, since
\textsc{Hipster} works by translating to Haskell and invoking \hspec{}, it seems
that improving the Haskell tools will give the most benefit.

Like most functional programming languages, Haskell builds upon
$\lambda$-calculus, with extra features such as a strong type system and
``syntactic sugar'' to improve readability. To avoid the relatively large and
complex syntax of Haskell proper, we will focus on an intermediate
representation of the \textsc{GHC} compiler known as \emph{GHC Core}, described
in Appendix \ref{sec:core}.

We will use quoted strings to denote names and literals, e.g. \texttt{Local
  "foo"}, \texttt{Global "bar"}, \texttt{Constructor "Baz"}, \texttt{LitStr
  "quux"} and \texttt{LitNum "42"}, and require only that they can be compared
for equality.


\begin{figure}
  \begin{small}
    \underline{\texttt{not}}
    \begin{verbatim}
Lam "a" (Case (Var (Local "a"))
              "b"
              (Alt (DataAlt "True")  (Var (Constructor "False")))
              (Alt (DataAlt "False") (Var (Constructor "True"))))
    \end{verbatim}

    \underline{\texttt{odd}}
    \begin{verbatim}
Lam "a" (Case (Var (Local "a"))
              "b"
              (Alt (DataAlt "Z") (Var (Constructor "False")))
              (Alt (DataAlt "S") (App (Var (Global "even"))
                                      (Var (Local  "n")))
                                 "n"))
    \end{verbatim}

    \underline{\texttt{even}}
    \begin{verbatim}
Lam "a" (Case (Var (Local "a"))
              "b"
              (Alt (DataAlt "Z") (Var (Constructor "True")))
              (Alt (DataAlt "S") (App (Var (Global "odd"))
                                      (Var (Local  "n")))
                                 "n"))
    \end{verbatim}
  \end{small}
  \caption{Translations of functions in Figure \ref{fig:haskellexample} into our
    Core syntax, with fresh variables chosen arbitrarily as \texttt{"a"},
    \texttt{"b"}, etc. Notice the introduction of explicit $\lambda$
    abstractions (\texttt{Lam}) and the use of \texttt{Case} to represent
    piecewise definitions. The \texttt{"b"} variables are introduced to preserve
    sharing.}
  \label{fig:coreexample}
\end{figure}

Figure \ref{fig:haskellexample} shows some simple Haskell function definitions,
along with common definitions for boolean and natural number datatypes
(analogous to those in Haskell's standard library). The translation to our Core
syntax is routine, and shown in Figure \ref{fig:coreexample}. Although Core is
more verbose, we can see that similar structure in the Haskell definitions gives
rise to similar structure in the Core; for example, the definitions of \hs{odd}
and \hs{even} are identical in both languages, except for the particular
identifiers used. It is this close correspondence which allows us to analyse
Core expressions in place of their more complicated Haskell source.
