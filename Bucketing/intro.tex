\section{Introduction}

\emph{(Mathematical) Theory Exploration} (TE) describes an open-ended search for
interesting properties (proven or merely conjectured, depending on the approach)
of a given set of formal definitions, such as a mathematical theory, a
scientific model or a software library. It has been applied to generating
``background lemmas'' for the interactive proof assistant
\textsc{Isabelle/HOL}~\cite{johansson2009isacosy,Hipster} and automated
theorem provers~\cite{claessen2013automating}, as well as for generating
specifications for (pure functional) programming
libraries~\cite{QuickSpec,smallbone2017quick,braquehais2017speculate}.

Despite the clever search strategies employed by these tools, they all rely on
enumerating combinations of definitions in some way, which can cause infeasible
running times on larger inputs (in our experience, a few dozen definitions). To
avoid this, users of these systems must carefully select only small subsets of
their definitions to explore at a time.

We dub such selection the \emph{bucketing problem} for theory exploration and
analyse its effect on the running time of theory exploration tools on a
particular problem set, and on the quality of the statements they are able to
generate. An automated solution to this bucketing problem would accept a large
number of definitions and efficiently select sub-sets which are both small
enough to reasonably explore, whilst still enabling interesting properties to be
discovered.

Our contributions are:

\begin{enumerate}
  \item Defining the \emph{bucketing problem} for theory exploration.
  \item Statistical analysis of theory exploration performance on a large
    problem set, derived from existing theorem proving problem sets.
  \item Empirical analysis of the effect of bucketing on the quantity of
    desirable properties attainable by theory exploration.
  \item Experimental application of machine learning to the bucketing problem,
    via unsupervised \emph{clustering}.
  % \item A novel feature extraction method for transforming Haskell expressions
  %   into a form amenable to off-the-shelf learning algorithms.
  % \item An implementation of these feature extraction and theory exploration
  %   approaches.
  % \item A comparison of our methods with existing approaches, both for theory
  %   exploration in Haskell, and for machine learning in other languages.
\end{enumerate}

We describe theory exploration in more detail, as well as the \qspec{} tool we
use in our experiments, in \S~\ref{sec:theoryexploration}. We introduce the
bucketing problem more formally in \S~\ref{sec:bucketing} and measure its
effect on theory exploration systems in \S~\ref{sec:implementation}. In
\S~\ref{sec:clustering} we highlight the difference betwee bucketing and the
well-known \emph{clustering problem} from machine learning, by demonstrating
that clustering algorithms perform poorly at bucketing.  A variety of related
work is surveyed in \S~\ref{sec:related} and we conclude in
\S~\ref{sec:conclusion} with potential directions for future research.
