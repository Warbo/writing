\section{Introduction}

Functional programming claims to be more ``mathematically tractable'' than other ``conventional'' approaches, due to properties like \emph{referential transparency} and \emph{immutability}, which allow equational reasoning without many of the complications caused by side effects and strict evaluation order \cite{hughes1989functional}. Languages such as Haskell combine these advantages with static type systems to restrict the possible values of every expression, making programs amenable to powerful analysis tools such as property checking \cite{claessen2011quickcheck} and theorem proving \cite{rosen2012proving}.

Whilst such tools can support or refute user-provided properties, we are interested in the related problem of \emph{discovering} such properties automatically: the task of \emph{theory exploration}. Whilst clever algorithms are used to guide their search, existing theory exploration tools, including those for Haskell, ultimately rely on enumerating all (well-typed) expressions. This limits their usefulness, as it is only feasible to explore a small collection of functions at a time, relying on the user to select promising combinations. We investigate the use of machine learning methods to perform this selection automatically, removing the need for user intervention whilst maintaining the ability to discover desirable properties.

Our contributions are:

\begin{enumerate}
  \item The application of machine learning algorithms to theory exploration, for intelligently discovering interesting sub-sets of Haskell libraries, which are more tractable to explore.
  \item A novel feature extraction method for transforming Haskell expressions into a form amenable to off-the-shelf learning algorithms.
  \item An implementation of these feature extraction and theory exploration approaches.
  \item A comparison of our methods with existing approaches, both for theory exploration in Haskell, and for machine learning in other languages.
\end{enumerate}

We describe our treatment of Haskell expressions in \S \ref{sec:haskell} and describe the \qspec{} theory exploration system we build upon in \S \ref{sec:theoryexploration}. We discuss our contributions in more depth in \S \ref{sec:contributions}, and provide implementation details in \S \ref{sec:implementation}. A variety of related work is surveyed in \S \ref{sec:related}, we briefly evaluate our implementations in \S \ref{sec:evaluation} and give several potential directions for future research in \S \ref{sec:future}.
