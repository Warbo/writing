\begin{figure}
  \scalebox{0.45}{\input{images/steppedall.pgf}}
  \scalebox{0.45}{\input{images/steppednontoxic.pgf}}
  \caption{Kaplan-Meier survival plot for running QuickSpec on inputs
    containing various numbers of definitions, sampled from TIP. The x axis
    denotes running time, which we cut short after 300 seconds. The height of
    each line shows the proportion of \qspec{} runs which were still going at
    that time (lower is better). First plot is for all TIP definitions, second
    plot removes runs given ``toxic'' definitions.}
  \label{fig:survival}
\end{figure}

Figure~\ref{fig:survival} shows a Kaplan-Meier plot of \qspec{} running
times, when given inputs containing different numbers of definitions. There is a
clear pattern, with many runs finishing quickly and the remainder occupying a
``long tail'' which we cut off after 300 seconds (based on preliminary
experiments, which showed little difference in survival between 300 seconds and
1 hour).

There is an overall trend for larger inputs to have more runs time-out, except
for the very smallest inputs (which are more constrained by the sampling
procedure). This relationship is linear, which can be explained if the presence
of certain definitions is causing the exploration process to fail (since larger
inputs have a higher chance of containing one of these definitions).

\begin{figure}
  \scalebox{0.45}{\input{images/timeoutsall.pgf}}
  \scalebox{0.45}{\input{images/timeoutsnontoxic.pgf}}
  \caption{Proportion of samples which timed out per size, with least-squares
    linear regression. First plot is for all TIP definitions, second removes
    runs given ``toxic'' definitions.}
  \label{fig:tailsize}
\end{figure}

\begin{figure}
  \scalebox{0.45}{\input{images/proportionsall.pgf}}
  \scalebox{0.45}{\input{images/proportionsnontoxic.pgf}}
  \label{fig:proportions}
  \caption{Definitions, ordered by the ratio of successes to failures of the
    runs they appeared in. First graph contains all TIP definitions, showing
    ``toxic'' definitions which always failed. Second graph only contains runs
    without any toxic definitions.}
\end{figure}

\begin{figure}
  \begin{minted}{scheme}
    (define-fun-rec mult2 ((x Nat) (y Nat) (z Nat)) Nat
      (match x
        (case Z      z)
        (case (S x2) (mult2 x2 y (plus y z)))))
  \end{minted}

  \begin{minted}{scheme}
    (define-fun-rec qexp ((x Nat) (y Nat) (z Nat)) Nat
      (match y
        (case Z     z)
        (case (S n) (qexp x n (mult x z)))))
  \end{minted}

  \begin{minted}{scheme}
    (define-fun-rec op ((x Nat) (y Nat) (z Nat) (x2 Nat)) Nat
      (match x
        (case Z
          (match z
            (case Z      x2)
            (case (S x3) (op Z  y x3 (S x2)))))
        (case (S x4)
          (match z
            (case Z      (op x4 y y  x2))
            (case (S c ) (op x  y c  (S x2)))))))
  \end{minted}

  \iffalse
  \begin{minted}{scheme}
    (define-fun-rec mul3acc ((x Nat) (y Nat) (z Nat)) Nat
      (match x
        (case Z Z)                          ;; Base case for 0 * y * z
        (case (S x2)
          (match y
            (case Z Z)                      ;; Base case for x * 0 * z
            (case (S x3)
              (match z
                (case Z Z)                  ;; Base case for x * y * 0
                (case (S x4)
                  (match x2
                    (case Z
                      (match x3
                        (case Z
                          (match x4
                            (case Z (S Z))  ;; Base case for 1 * 1 * 1
                            (case (S x5)
                              (S (add3acc (mul3acc Z Z x4)
                                          (add3acc (mul3acc (S Z) Z x4)
                                                   (mul3acc Z (S Z) x4)
                                                   (mul3acc Z Z (S Z)))
                                          (add3acc Z Z x4))))))
                        (case (S x6)
                          (S (add3acc (mul3acc Z x3 x4)
                                      (add3acc (mul3acc (S Z) x3 x4)
                                               (mul3acc Z (S Z) x4)
                                               (mul3acc Z x3 (S Z)))
                                      (add3acc Z x3 x4))))))
                    (case (S x7)
                      (S (add3acc (mul3acc x2 x3 x4)
                                  (add3acc (mul3acc (S Z) x3 x4)
                                           (mul3acc x2 (S Z) x4)
                                           (mul3acc x2 x3 (S Z)))
                                  (add3acc x2 x3 x4))))))))))))
  \end{minted}
  \if

  \iffalse
  \begin{minted}{scheme}
    (define-fun-rec mul3 ((x Nat) (y Nat) (z Nat)) Nat
      (match x
        (case Z Z)                          ;; Base case for 0 * y * z
        (case (S x2)
          (match y
            (case Z Z)                      ;; Base case for x * 0 * z
            (case (S x3)
              (match z
                (case Z Z)                  ;; Base case for x * y * 0
                (case (S x4)
                  (match x2
                    (case Z
                      (match x3
                        (case Z
                          (match x4
                            (case Z (S Z))  ;; Base case for 1 * 1 * 1
                            (case (S x5)
                              (S (add3 (mul3 Z Z x4)
                                       (add3 (mul3 (S Z) Z x4)
                                             (mul3 Z (S Z) x4)
                                             (mul3 Z Z (S Z)))
                                       (add3 Z Z x4))))))
                        (case (S x6)
                          (S (add3 (mul3 Z x3 x4)
                                   (add3 (mul3 (S Z) x3 x4)
                                         (mul3 Z (S Z) x4)
                                         (mul3 Z x3 (S Z)))
                                   (add3 Z x3 x4))))))
                    (case (S x7)
                      (S (add3 (mul3 x2 x3 x4)
                               (add3 (mul3 (S Z) x3 x4)
                                     (mul3 x2 (S Z) x4)
                                     (mul3 x2 x3 (S Z)))
                               (add3 x2 x3 x4))))))))))))
  \end{minted}
  \fi
  \caption{``Toxic'' definitions, which consistently cause \qspec{} to fail. Two
    other definitions (\texttt{mul3} and \texttt{mul3acc}) are ommitted due to
    their verbosity.}
  \label{fig:faildefs}
\end{figure}

This does appear to be the case, as Figure~\ref{fig:proportions} shows that five
definitions appeared \emph{only} in failing inputs; these are named
\texttt{mul3}, \texttt{mul3acc}, \texttt{mult2}, \texttt{op} and \texttt{qexp};
their definitions appear in Figure~\ref{fig:faildefs}. All of these are
functions of Peano-encoded natural numbers (\texttt{Nat}), and they cause
exploration to time out by generating very large outputs.

\texttt{mul3} and \texttt{mul3acc} are rather pathological implementations of
multiplication with an accumulator parameter, with many (non-tail) recursive
calls. The \texttt{op} function appears in files named \texttt{weird\_nat\_op},
which assert its commutativity and associativity. Finally, the \texttt{mult2}
and \texttt{qexp} functions are standard tail-recursive definitions of
multiplication and exponentiation, respectively.

Exploring each of these functions on its own does not require much memory, since
Haskell generates the output lazily. However, comparing such large numbers for
equality takes a lot of CPU time as the \texttt{S} constructors are successively
unwrapped from each side, and this is why the timeout is reached. We confirmed
this hypothesis by exploring with a custom data generator which only generates
the values \texttt{Z}, \texttt{S Z} and \texttt{S (S Z)} (0, 1 and 2); this
caused the exploration to finish quickly. Other interventions, like making the
accumulator arguments strict (to prevent space leaks), did not prevent timeouts.

To assess the impact of these problematic definitions, we removed any samples
containing them and repeated our analysis.
