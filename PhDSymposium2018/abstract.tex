\documentclass[]{default}
\usepackage{lmodern}
\usepackage{listings}
\usepackage{amssymb,amsmath}
\usepackage{paralist}
\usepackage{ifxetex,ifluatex}
\usepackage{fixltx2e} % provides \textsubscript
\ifnum 0\ifxetex 1\fi\ifluatex 1\fi=0 % if pdftex
  \usepackage[T1]{fontenc}
  \usepackage[utf8]{inputenc}
\else % if luatex or xelatex
  \ifxetex
    \usepackage{mathspec}
    \usepackage{xltxtra,xunicode}
  \else
    \usepackage{fontspec}
  \fi
  \defaultfontfeatures{Mapping=tex-text,Scale=MatchLowercase}
  \newcommand{\euro}{€}
\fi
% use upquote if available, for straight quotes in verbatim environments
\IfFileExists{upquote.sty}{\usepackage{upquote}}{}
% use microtype if available
\IfFileExists{microtype.sty}{%
\usepackage{microtype}
\UseMicrotypeSet[protrusion]{basicmath} % disable protrusion for tt fonts
}{}
\ifxetex
  \usepackage[setpagesize=false, % page size defined by xetex
              unicode=false, % unicode breaks when used with xetex
              xetex]{hyperref}
\else
  \usepackage[unicode=true]{hyperref}
\fi
\hypersetup{breaklinks=true,
            bookmarks=true,
            pdfauthor={},
            pdftitle={Exploring Software with Symbolic and Statistical Algorithms},
            colorlinks=true,
            citecolor=blue,
            urlcolor=blue,
            linkcolor=magenta,
            pdfborder={0 0 0}}
\urlstyle{same}  % don't use monospace font for urls
\setlength{\parindent}{0pt}
\setlength{\parskip}{6pt plus 2pt minus 1pt}
\setlength{\emergencystretch}{3em}  % prevent overfull lines
\setcounter{secnumdepth}{5}


\title{Exploring Software with Symbolic and Statistical Algorithms}
  \author{
          Chris Warburton \\
      University of Dundee \\
      \texttt{cmwarburton@dundee.ac.uk}
    }
\date{}

\begin{document}
\maketitle
\begin{abstract}
  We propose a benchmarking methodology to evaluate the efficiency and quality
  of \emph{conjecture generation} by automated tools for \emph{mathematical
    theory exploration}. Our approach uses widely available theorem proving
  tasks as a \emph{ground-truth} corpus, and we demonstrate its use on the
  QuickSpec and IsaCoSy tools, finding that the former takes significantly less
  time to produce significantly more ``interesting'' output. By providing a
  standard, cross-tool evaluation technique we allow the disparate approaches
  currently being pursued to be more directly compared. Our hope is to encourage
  innovation and improvements similar to those seen in fields like automated
  theorem proving, where the availability of benchmarks encourages healthy
  competition.
\end{abstract}

\section{Introduction}\label{introduction}

\emph{Conjecture generation} (CG) is the open-ended task of producing
conjectures about a given logical theory (such as a software library) which are
somehow ``interesting'', and is studied as a sub-field of \emph{mathematical
  theory exploration}. CG tools can discover formal specifications for software,
and complement theorem provers by spotting crucial lemmas needed in verification
tasks~\cite{Claessen.Johansson.Rosen.ea:2013}. Increasing the automation of
formal methods in this way can reduce their cost and difficulty, which have
historically limited their use to mission- or life-critical domains like
aerospace and microprocessor design. Promising applications which would benefit
from more practical automated reasoning include correctness proofs for software
engineering~\cite{McKinna:2006,chlipala2011certified,Xi2003}, automated
``assistants'' for STEM
research~\cite{McCarthy_Programs59,lenat:77,benzmuller1997omegamega} and more
powerful education software~\cite{conf/ijcai/TrybulecB85,hendriks2010teaching}.

Progress in this field is hard to ascertain, since there is no standard measure
of ``success'' which we can compare across different approaches. This is
partially due to the inherent ambiguity of the task, i.e. what counts as
``interesting''? Different researchers may have different goals, and a variety
of evaluation methods are employed. We attempt to solve this discrepancy, at
least for the foreseeable future, by defining a standard, unambiguous
benchmarking approach with which to compare conjecture generation tools. Our
contributions include:

\begin{itemize}
\item A general methodology for benchmarking conjecture generation.
\item Resolving the issue of ``interestingness'' through the use of
  theorem-proving benchmarks as a ground-truth.
\item A specific instantiation of this methodology, using the Tons of Inductive
  Problems benchmark as a corpus.
\item Automated tooling to perform this benchmarking.
\item Application of our methodology to the QuickSpec and IsaCoSy MTE tools,
  and a comparison and discussion of the results.
\end{itemize}


\bibliographystyle{plain}
\bibliography{../Bibtex}

\end{document}
