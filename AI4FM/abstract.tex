\documentclass[]{article}
\usepackage{lmodern}
\usepackage{amssymb,amsmath}
\usepackage{ifxetex,ifluatex}
\usepackage{fixltx2e} % provides \textsubscript
\ifnum 0\ifxetex 1\fi\ifluatex 1\fi=0 % if pdftex
  \usepackage[T1]{fontenc}
  \usepackage[utf8]{inputenc}
\else % if luatex or xelatex
  \ifxetex
    \usepackage{mathspec}
    \usepackage{xltxtra,xunicode}
  \else
    \usepackage{fontspec}
  \fi
  \defaultfontfeatures{Mapping=tex-text,Scale=MatchLowercase}
  \newcommand{\euro}{€}
\fi
% use upquote if available, for straight quotes in verbatim environments
\IfFileExists{upquote.sty}{\usepackage{upquote}}{}
% use microtype if available
\IfFileExists{microtype.sty}{%
\usepackage{microtype}
\UseMicrotypeSet[protrusion]{basicmath} % disable protrusion for tt fonts
}{}
\ifxetex
  \usepackage[setpagesize=false, % page size defined by xetex
              unicode=false, % unicode breaks when used with xetex
              xetex]{hyperref}
\else
  \usepackage[unicode=true]{hyperref}
\fi
\hypersetup{breaklinks=true,
            bookmarks=true,
            pdfauthor={Chris Warburton},
            pdftitle={Scaling Automated Theory Exploration},
            colorlinks=true,
            citecolor=blue,
            urlcolor=blue,
            linkcolor=magenta,
            pdfborder={0 0 0}}
\urlstyle{same}  % don't use monospace font for urls
\setlength{\parindent}{0pt}
\setlength{\parskip}{6pt plus 2pt minus 1pt}
\setlength{\emergencystretch}{3em}  % prevent overfull lines
\providecommand{\tightlist}{%
  \setlength{\itemsep}{0pt}\setlength{\parskip}{0pt}}
\setcounter{secnumdepth}{0}

\title{Scaling Automated Theory Exploration}
\author{Chris Warburton}
\date{}
\usepackage[T1]{fontenc}
\usepackage{upquote}

% Redefines (sub)paragraphs to behave more like sections
\ifx\paragraph\undefined\else
\let\oldparagraph\paragraph
\renewcommand{\paragraph}[1]{\oldparagraph{#1}\mbox{}}
\fi
\ifx\subparagraph\undefined\else
\let\oldsubparagraph\subparagraph
\renewcommand{\subparagraph}[1]{\oldsubparagraph{#1}\mbox{}}
\fi

\begin{document}
\maketitle

\section{Abstract}\label{abstract}

In this paper we investigate the \textbf{theory exploration} (TE)
paradigm for computer-assisted Mathematics and identify limitations and
improvements of current approaches. Unlike the theorem-proving paradigm,
which requires user-provided conjectures, TE performs an open-ended
search for theorems satisfying given criteria. We see promise in TE for
identifying new abstractions and connections in libraries of software
and proofs, but realising this potential requires more scalable
algorithms than presently used.

\section{Introduction}\label{introduction}

The \emph{theory exploration} (TE) paradigm provides software support
for traditional Mathematical workflows; namely, deriving ``interesting''
consequences from formal definitions (Buchberger 1999). Early
implementations like \textsc{Theorema} (Buchberger 2000) emphasised
interactivity, in a similar way to computer algebra systems like
\textsc{Mathematica} (in which \textsc{Theorema} is implemented) or
interactive theorem provers. Subsequent systems have investigated
\emph{automated} theory exploration, for tasks such as lemma discovery
(Johansson et al. 2014).

By removing user interaction, automated TE systems require an algorithm
for deciding whether a theorem is ``interesting''. In existing systems,
this is intimately connected to the choice of search algorithm, to
ensure tractability. For example, \textsc{IsaCoSy} (Johansson, Dixon,
and Bundy 2009) discovers equations, which are defined as
``interesting'' if they cannot be simplified using previously discovered
equations. The intuition for such criteria is to avoid special cases of
known theorems, such as \(0 + 0 = 0\), \(0 + 1 = 1\), etc. when we
already know \(\forall x. 0 + x = x\). To work effectively, this
requires general forms to be found \emph{before} special cases.

TE is a ``bottom-up'' process: new theorems are derived from given
knowledge and previous theorems. In particular, there is no special
``goal'' theorem(s) as found in interactive theorem proving; all we have
is our (proxy) ``interesting'' criterion, and the definitions of our
theory. This makes theory exploration a \emph{combinatorial
optimisation} problem, which is well studied in Artificial Intelligence
and Machine Learning. Existing TE systems rely on complete, brute-force
search techniques which do not scale to theories of a significant size;
these could be replaced by a variety of faster, approximate algorithms,
eg. those surveyed in (Blum et al. 2011).

Whilst efficient on a small scale, where we expect the chosen inputs to
be related, we cannot expect an effective output criterion to coincide
so readily with search algorithm and output criterion to be effective
for large inputs containing many irrelevant connections. umbers of also
consider data mining approaches, which infer general rules \emph{after}
enumerating some number of special cases. A variety of
``interestingness'' measures are used in domains such as data mining
(Geng and Hamilton 2006), which may be adapted to a TE setting. These
include objective measures, eg. based on information theory, and
subjective measures which ask for the user's opinion. We can avoid
asking users directly, by treating existing proof libraries and test
suites as representative examples of what is interesting, and hence
construct an oracle; such oracles have already been used for
precision-recall experiments to evaluate the performance of
\textsc{HipSpec} (Claessen et al. 2013).

\section{Tackling Complexity}\label{tackling-complexity}

Even with a slow search algorithm, we can use a divide and conquor
approach to limit the number of allowed combinations, either by using
stricter types to prevent composition, or by partitioning the theory
into small independently-searched sub-theories. Of course, such
restrictions should strike a balance between the efficiency gained and
the potential to forbid some interesting theorems.

\section{Existing Work}\label{existing-work}

Automated theory exploration has been applied to libraries in Isabelle
and Haskell, although we focus on the latter as its implementations are
the most mature (\textsc{Hipster} actually explores Isabelle by
translating code to Haskell first!). Haskell is interesting to target,
since its functional purity and algebraic structure make equational
properties common; recursion and higher-order functions make automation
non-trivial; and since these properties can't be expressed in Haskell
itself (without difficulty (Lindley and McBride 2014), at least), less
effort is spent discovering and proving these properties compared to
proof-oriented systems like Isabelle.

Due to Haskell's relative popularity, there are large code repositories
such as \textsc{Hackage} available to explore, with the potential to
benefit existing library authors and users in comprehending and
maintaining their code (Claessen, Smallbone, and Hughes 2010).

Currently, the most powerful TE system for Haskell is \textsc{HipSpec}.
This uses \textsc{QuickSpec} to search through \emph{expressions}
(combinations of the Haskell terms given by the theory), rather than
searching through the space of equations or proofs directly. Expressions
are grouped into equivalence classes, such that the \textsc{QuickCheck}
counterexample finder cannot distinguish between the elements; equations
relating the members of these classes are then conjectured, and sent to
existing automated theorem provers to try and prove (Rosén 2012). This
approach works well as a lemma generation system, making
\textsc{HipSpec} a capable inductive theorem prover as well as a theory
exploration system (Claessen et al. 2013).

\section{Going Forward}\label{going-forward}

Given this state of the art, we identify the following as potential
areas for improvement:

\begin{itemize}
\tightlist
\item
  Expression enumeration is brute-force; this could scale to larger
  terms and theories using a heuristic algorithm.
\item
  ``Interestingness'' is a fixed part of the algorithm: an equation is
  interesting if it cannot be derived from previous equations. As we
  increase the size of our theory, this becomes unsatisfying in two
  ways:

  \begin{itemize}
  \tightlist
  \item
    The number of irreducible equations grows, making it desirable to
    impose extra conditions of a more subjective nature.
  \item
    Surprising, insightful equations may be discarded, if they are
    actually reducible in some complex, non-obvious way. A more
    subjective interestingness measure could be used to veto such
    rejections.
  \end{itemize}
\item
  The system does not propose candidate equations by data mining
  previous results; generalisation methods like anti-unification could
  do this, and at the same time remove the requirement that general
  forms must be enumerated early.
\item
  All type-safe combinations of the given expressions are tried, whilst
  it may be discernable a priori that some combinations are not worth
  considering (either because they are never related, or because their
  relations are never interesting). A pre-processor could make large
  theories more tractable by selecting combinations which are likely to
  be related, similar to premise selection in automated theorem proving
  (Kühlwein et al. 2012).
\end{itemize}

We are implementing a system called \textsc{ML4HS} to investigate these
ideas. Our initial hypothesis that expressions with similar definitions
are more likely to be related by equational properties than those
without, and hence a similarity-based clustering method such as that of
\textsc{ML4PG} (Heras and Komendantskaya 2013) can be used to implement
a divide and conquor pre-processor.

Since our aim is to scale up theory exploration, we treat entire
\textsc{Hackage} packages as our theories. \textsc{ML4HS} will manage
downloading, compiling, managing dependencies, etc. automatically. This
both eliminates the need to define theories manually, and may be useful
in its own right as a mechanism to execute arbitrary Haskell code from
arbitrary modules in arbitrary packages.

\section{Acknowlegements}\label{acknowlegements}

I am grateful for those who have helped formulate these ideas through
conversation, especially the HipSpec team at Chalmers University (Moa
Johansson, Koen Claessen, Nick Smallbone and Dan Rosén) and my
supervisor Katya Komendantskaya. I also wish to thank the implementors
of the systems we are building on, including HipSpec, QuickSpec and
QuickCheck on the theory exploration side, as well as GHC, Cabal and Nix
on the infrastructure side.

\section*{References}\label{references}
\addcontentsline{toc}{section}{References}

\hyperdef{}{ref-blum2011hybrid}{\label{ref-blum2011hybrid}}
Blum, Christian, Jakob Puchinger, Günther R Raidl, and Andrea Roli.
2011. ``Hybrid Metaheuristics in Combinatorial Optimization: A Survey.''
\emph{Applied Soft Computing} 11 (6). Elsevier: 4135--51.

\hyperdef{}{ref-RISC1482}{\label{ref-RISC1482}}
Buchberger, Bruno. 1999. ``Theory Exploration Versus Theorem Proving.''
RISC Report Series 99-46. Schloss Hagenberg, 4232 Hagenberg, Austria:
Research Institute for Symbolic Computation (RISC), Johannes Kepler
University Linz.

\hyperdef{}{ref-buchberger2000theory}{\label{ref-buchberger2000theory}}
---------. 2000. ``Theory Exploration with Theorema.'' \emph{Analele
Universitatii Din Timisoara, Ser. Matematica-Informatica} 38 (2).
Citeseer: 9--32.

\hyperdef{}{ref-claessen2013automating}{\label{ref-claessen2013automating}}
Claessen, Koen, Moa Johansson, Dan Rosén, and Nicholas Smallbone. 2013.
``Automating Inductive Proofs Using Theory Exploration.'' In
\emph{Automated Deduction--CADE-24}, 392--406. Springer.

\hyperdef{}{ref-QuickSpec}{\label{ref-QuickSpec}}
Claessen, Koen, Nicholas Smallbone, and John Hughes. 2010. ``QuickSpec:
Guessing Formal Specifications Using Testing.'' In \emph{Tests and
Proofs}, edited by Gordon Fraser and Angelo Gargantini, 6143:6--21.
Lecture Notes in Computer Science. Springer Berlin Heidelberg.
\href{http://doi.org/10.1007/978-3-642-13977-2_3}{doi:10.1007/978-3-642-13977-2\_3}.

\hyperdef{}{ref-geng2006interestingness}{\label{ref-geng2006interestingness}}
Geng, Liqiang, and Howard J Hamilton. 2006. ``Interestingness Measures
for Data Mining: A Survey.'' \emph{ACM Computing Surveys (CSUR)} 38 (3).
ACM: 9.

\hyperdef{}{ref-journalsux2fcorrux2fabs-1302-6421}{\label{ref-journalsux2fcorrux2fabs-1302-6421}}
Heras, Jónathan, and Ekaterina Komendantskaya. 2013. ``ML4PG:
proof-mining in Coq.'' \emph{CoRR} abs/1302.6421.
\url{http://dblp.uni-trier.de/db/journals/corr/corr1302.html\#abs-1302-6421}.

\hyperdef{}{ref-johansson2009isacosy}{\label{ref-johansson2009isacosy}}
Johansson, Moa, Lucas Dixon, and Alan Bundy. 2009. ``Isacosy: Synthesis
of Inductive Theorems.'' In \emph{Workshop on Automated Mathematical
Theory Exploration (Automatheo)}.

\hyperdef{}{ref-Hipster}{\label{ref-Hipster}}
Johansson, Moa, Dan Rosén, Nicholas Smallbone, and Koen Claessen. 2014.
``Hipster: Integrating Theory Exploration in a Proof Assistant.'' In
\emph{Intelligent Computer Mathematics}, edited by StephenM. Watt,
JamesH. Davenport, AlanP. Sexton, Petr Sojka, and Josef Urban,
8543:108--22. Lecture Notes in Computer Science. Springer International
Publishing.
\href{http://doi.org/10.1007/978-3-319-08434-3_9}{doi:10.1007/978-3-319-08434-3\_9}.

\hyperdef{}{ref-kuhlwein2012overview}{\label{ref-kuhlwein2012overview}}
Kühlwein, Daniel, Twan van Laarhoven, Evgeni Tsivtsivadze, Josef Urban,
and Tom Heskes. 2012. ``Overview and Evaluation of Premise Selection
Techniques for Large Theory Mathematics.'' In \emph{Automated
Reasoning}, 378--92. Springer.

\hyperdef{}{ref-lindley2014hasochism}{\label{ref-lindley2014hasochism}}
Lindley, Sam, and Conor McBride. 2014. ``Hasochism: The Pleasure and
Pain of Dependently Typed Haskell Programming.'' \emph{ACM SIGPLAN
Notices} 48 (12). ACM: 81--92.

\hyperdef{}{ref-rosen2012proving}{\label{ref-rosen2012proving}}
Rosén, Dan. 2012. ``Proving Equational Haskell Properties Using
Automated Theorem Provers.'' PhD thesis, Master's thesis, University of
Gothenburg, Sweden.

\end{document}
