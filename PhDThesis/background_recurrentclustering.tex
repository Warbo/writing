\section{Recurrent Clustering}

Our recurrent clustering approach takes inspiration from the ML4PG
\cite{journals/corr/abs-1212-3618} and ACL2(ml) \cite{heras2013proof} tools,
used for analysing proofs in Coq and ACL2, respectively. Whilst both transform
syntax trees into matrices, the algorithm of ML4PG most closely resembles ours
as it assigns tokens directly to matrix elements. In contrast, the matrices
produced by ACL2(ml) \emph{summarise} information about the tree; for example,
one column counts the number of variables appearing at each tree level, others
count the number of function symbols which are nullary, unary, binary,
etc. Whilst it may be interesting to contrast our current algorithm with an
alternative based on that of ACL2(ml), it is unclear how such summaries could be
extended to include types, which seems the next logical step for our
approach. The ML4PG algorithm extends trivially, by using (term, type) pairs
instead of just terms.

The way we \emph{use} our clusters to inform theory exploration is actually more
similar to that of ACL2(ml) than ML4PG. ML4PG can either present clusters to the
user for inspection, or produce automata for recreating proofs. In ACL2(ml), the
clusters are used to restrict the search space of a proof search, much like we
restrict the scope of theory exploration.

ACL2(ml) reasons by analogy: finding theorem statements which are similar to the
current goal, and attempting to prove the goal in a similar way. In particular,
the lemmas used to prove a theorem are mutated by substituting symbols for those
which appear in the same cluster. For example, if \texttt{plus} and
\texttt{multiply} are clustered together, and we are trying to prove a goal
involving \texttt{multiply}, then ACL2(ml) might consider an existing theorem
involving \texttt{plus}. The lemmas used to prove that theorem will be mutated,
for example replacing occurrences of \texttt{plus} with \texttt{mult}, in an
attempt to prove the goal.

Whilst we do not currently reason by analogy, this is an interesting area for
future work in theory exploration: given a set of theorems relating particular
terms, we might form conjectures regarding similar terms found through
clustering.

\iffalse
We could expand this a bit, e.g. talking about how we both use Weka, etc.
\fi
