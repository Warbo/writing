\section{Machine Learning}

Machine learning (ML) is a sub-field of artificial intelligence (AI) which
emphasises the use of statistical methods to find patterns and optimise
functions, in contrast with symbolic reasoning approaches like automated theorem
proving (commonly referred to as ``Good Old-Fashioned AI'').

The machine learning umbrella spans many techniques and applications, but one
common distinction is between \emph{supervised} and \emph{unsupervised} tasks.
In supervised learning we allow the system to perform some behaviour, then an
external process assesses whether it behaved well or poorly (akin to Justice
Potter Stewart's test for obscenity: ``I know it when I see it''). The system's
parameters are optimised to try and score highly on the test. Common supervised
learning tasks include function approximation using input/output examples (e.g.
classifying scanned images of numerals) where the test is the proportion of
example inputs which got the correct output, and reinforcement learning where
the system chooses from a set of ``actions'', receives a pair of ``observation''
and ``reward'' (test score) in response, and must accumulate the most reward.

Unsupervised learning does not have the same testable ``success'' criteria as
supervised learning. Instead, the goal is to find patterns and form models of
the target domain, which is more ambitious and open-ended than supervised
learning but also much harder to define, measure and optimise. Examples of this
are prediction and compression tasks such as auto-encoders
(learned compressor/decompressor pairs, rewarded by how well they preserve data
through a round-trip) and adversarial or co-evolution tasks (coupled systems
where each is rewarded by exploiting mis-prediction in the others).

\subsection{Distributed Representations}

A \emph{distributed representation} does not isolate each semantic feature (for
example, presence of absence of some lexeme) to separate, independent
``computing elements'' (e.g. memory locations, such as the elements of a feature
vector). Instead, features are represented as \emph{patterns} across multiple
elements: each feature uses many elements and each element contributes to many
features~\cite{hinton1984distributed}. Such encodings provide simple mechanisms
to represent \emph{combinations} of features (combining their patterns
element-wise), \emph{partial} features (presence of the pattern across a subset
of its elements) and \emph{generalisation} (patterns which are present as
sub-patterns of multiple features).

% Distributed representations may be learned from data, for example by
% \emph{auto-encoding}; or they may be calculated ``from whole cloth'' using
% techniques like hashing. We are particularly interested in distributed
% representations for storing structured data (like syntax trees) in fixed-length
% feature vectors.

Our approach to scaling up \quickspec{} takes inspiration from two sources. The
first is relevance filtering, which makes expensive algorithms used in theorem
proving more practical by limiting the size of their inputs. We describe this
approach in more details in \S~\ref{sec:relevance}. Relevance filtering is a
practical tool which has existing applications in software, such as the
\emph{Sledgehammer} component of the Isabelle/HOL theorem prover.

Despite the idea's promise, we cannot simply invoke existing relevance filter
algorithms in our theory exploration setting. The reason is that relevance
filtering is a supervised learning method, i.e. it would require a distinguished
expression to compare everything against. Theory exploration does not have such
a distinguished expression; instead, we are interested in relationships between
\emph{any} terms generated from a signature, and hence we must consider the
relevance of \emph{all terms} to \emph{all other terms}.

A natural fit for this task is \emph{clustering}, which attempts to group
similar inputs together in an unsupervised way. Based on their success in
discovering relationships and patterns between expressions in Coq and ACL2 (in
the ML4PG and ACL2(ml) tools respectively), we hypothesise that clustering
methods can fulfil the role of relevance filters for theory exploration:
intelligently breaking up large signatures into smaller ones more amenable to
brute force enumeration, such that related expressions are explored together.

