\section{Machine Learning}

Machine learning (ML) is a sub-field of artificial intelligence (AI) which
emphasises the use of statistical methods to find patterns and optimise
functions, in contrast with symbolic reasoning approaches like automated theorem
proving (commonly referred to as ``Good Old-Fashioned AI'').

The machine learning umbrella spans many techniques and applications, but one
common distinction is between \emph{supervised} and \emph{unsupervised} tasks.
In supervised learning we allow the system to perform some behaviour, then an
external process assesses whether it behaved well or poorly (akin to Justice
Potter Stewart's test for obscenity: ``I know it when I see it''). The system's
parameters are optimised to try and score highly on the test. Common supervised
learning tasks include function approximation using input/output examples (e.g.
classifying scanned images of numerals) where the test is the proportion of
example inputs which got the correct output, and reinforcement learning where
the system chooses from a set of ``actions'', receives a pair of ``observation''
and ``reward'' (test score) in response, and must accumulate the most reward.

Unsupervised learning does not have the same testable ``success'' criteria as
supervised learning. Instead, the goal is to find patterns and form models of
the target domain, which is more ambitious and open-ended than supervised
learning but also much harder to define, measure and optimise. Examples of this
are prediction and compression tasks such as auto-encoders
(learned compressor/decompressor pairs, rewarded by how well they preserve data
through a round-trip) and adversarial or co-evolution tasks (coupled systems
where each is rewarded by exploiting mis-prediction in the others).

Our approach to scaling up \quickspec{} takes inspiration from two sources. The
first is relevance filtering, which makes expensive algorithms used in theorem
proving more practical by limiting the size of their inputs. We describe this
approach in more details in \S~\ref{sec:relevance}. Relevance filtering is a
practical tool which has existing applications in software, such as the
\emph{Sledgehammer} component of the Isabelle/HOL theorem prover.

Despite the idea's promise, we cannot simply invoke existing relevance filter
algorithms in our theory exploration setting. The reason is that relevance
filtering is a supervised learning method, i.e. it would require a distinguished
expression to compare everything against. Theory exploration does not have such
a distinguished expression; instead, we are interested in relationships between
\emph{any} terms generated from a signature, and hence we must consider the
relevance of \emph{all terms} to \emph{all other terms}.

A natural fit for this task is \emph{clustering}, which attempts to group
similar inputs together in an unsupervised way. Based on their success in
discovering relationships and patterns between expressions in Coq and ACL2 (in
the ML4PG and ACL2(ml) tools respectively), we hypothesise that clustering
methods can fulfil the role of relevance filters for theory exploration:
intelligently breaking up large signatures into smaller ones more amenable to
brute force enumeration, such that related expressions are explored together.

One major difficulty when applying statistical machine learning algorithms to
\emph{languages}, such as Haskell, is the appearance of recursive
structures. This can lead to nested expressions of arbitrary depth, which are
difficult to compare in numerical ways. One solution, as described in \S
\ref{sec:featureextraction}, is to use \emph{feature extraction}; however, our
method is not the only possible way to encode recursive structures as fixed-size
features.

The simplest way to encode such inputs is to simply choose a desired size, then
pad anything smaller and truncate anything larger. We use this to make our
matrices a uniform size, borrowing the idea from ML4PG. Care must be taken to
ensure that we are not discarding too much information, that we are not
producing features with too many dimensions to be practical, and that there is a
uniform ``meaning'' to each feature across different feature vectors. In our
case, we avoid many of these problems by transforming the recursive structure of
expressions into matrices first; this gives each feature a stable meaning such
as ``the $i$th token from the left at the $j$th level of nesting''.

Truncation works best when the input data is arranged with the most significant
data first (in a sense, it is ``big-endian''). This is the case for Haskell
expressions, since the higher levels of the syntax tree are the most
semantically significant; for example, the lower levels may never even be
evaluated due to laziness. This allows us to truncate more aggressively than if
the leaves were most significant.

By modelling our inputs as points in high-dimensional spaces, we can consider
feature extraction as projection into a lower-dimensional space (known as
\emph{dimension reduction}). Truncation is a trivial dimension reduction
technique; more sophisticated projection functions consider the
\emph{distribution} of the input points, and project with the hyperplane which
preserves as much of the variance as possible (or, equivalently, reduces the
\emph{mutual information} between the points).

Techniques such as \emph{principle component analysis} (PCA) can be used to find
these hyperplanes, but unfortunately require their inputs to already have a
fixed, integer number of dimensions. In the case of our recursive expressions
(which we may consider to have \emph{fractal} dimension), we would need another
pre-processing stage to satisfy this requirement.

There are machine learning algorithms which can handle variable-size input, but
these are often \emph{supervised} algorithms which require an
externally-provided error function to minimise. Error functions can be given for
clustering, for example k-means implicitly minimises the function given in
equation \ref{eq:kmeansobjective}, but unsupervised algorithms may be preferred
for efficiency as they are more direct.

One example of learning from variable-size input is to use \emph{recurrent
  neural networks} (RNNs). These contain cyclic connections between neurons,
unlike the traditional acyclic ``feed-forward'' NNs, allowing state to persist
between observations. In this way, each data point can be divided into a
sequence of arbitrary length, for example an s-expression, and fed into an RNN
one token at a time for processing.

Unfortunately RNNs are difficult to train. The standard way to train NNs is the
back-propagation algorithm; when this is extended to handle cycles we get the
\emph{backpropagation through time} algorithm
\cite{werbos1990backpropagation}. However, this suffers a problem known as the
\emph{vanishing} (or \emph{exploding}) \emph{gradient}: error values change
exponentially as they propagate back through the cycles, which prevents
effective learning of correlations across a sequence, undermining the main
advantage of RNNs. The vanishing gradient problem is the subject of current
research, with countermeasures including \emph{neuroevolution} (using
evolutionary computation techniques rather than back-propagation) and \emph{long
  short-term memory} (LSTM; introducing special nodes to ``store'' state, rather
than having them loop around a cycle \cite{hochreiter1997long}).

Using sequences to represent recursive structures is also problematic: if we
want our learning algorithm to exploit structure (such as the depth of a token),
it will have to discover how to parse the sequences for itself, which seems
wasteful. The \emph{back-propagation through structure} approach
\cite{goller1996learning} is a more direct solution to this problem, using a
feed-forward NN to learn recursive distributed representations
\cite{pollack1990recursive} which correspond to the recursive structure of the
inputs. Such distributed representations can also be used for sequences, which
we can use to encode sub-trees when the branching factor of nodes is not uniform
\cite{kwasny1995tail}. More recent work has investigated storing recursive
structures inside LSTM cells \cite{zhu2015long}.

A simpler alternative for generating recursive distributed representations is to
use circular convolution \cite{conf/ijcai/Plate91}. Although promising results
are shown for its use in \emph{distributed tree kernels}
\cite{zanzotto2012distributed}, our preliminary experiments in applying
circular convolution to functional programming expressions found most of the
information to be lost in the process; presumably as the expressions are too
small.

\emph{Kernel methods} have also been applied to structured information, for
example in \cite{Gartner2003} the input data (including sequences, trees and
graphs) are represented using \emph{generative models}, such as hidden Markov
models, of a fixed size suitable for learning. Many more applications of kernel
methods to structured domains are given in \cite{bakir2007predicting}, which
could be used to learn more subtle relations between expressions than recurrent
clustering alone.

%% FIXME: MERGE THE FOLLOWING INTO THE ABOVE

\subsubsection{Feature Extraction}
\label{sec:featureextraction}

Before describing clustering in detail, we must introduce the idea of
\emph{feature extraction}. This is the conversion of ``raw'' input data, such as
audio, images or (in our case) Core expressions, into a form more suited to
machine learning algorithms. By pre-processing our data in this way, we can
re-use the same ``off-the-shelf'' machine learning algorithms in a variety of
domains.

We use a standard representation of features as a \emph{feature vector} of
numbers $\vect{x} = (x_1, \dots, x_d)$ where $x_i \in \mathbb{R}$ (we use
\textbf{bold face} to represent vectors, including feature
vectors). \footnote{In fact, practical implementations will use an approximate
  format such as IEEE 754 floating point numbers.} For learning purposes this
has some important advantages over raw expressions:

\begin{itemize}
\item All of our feature vectors will be the same size, i.e. they will all have
  length (or \emph{dimension}) $d$. Many machine learning algorithms only work
  with inputs of a uniform size; feature extraction allows us to use these
  algorithms in domains where the size of each input is not known, may vary or
  may even be unbounded. For example, element-wise comparison of feature vectors
  is trivial (compare the $i$th elements for $1 \leq i \leq d$); for expressions
  this is not so straightforward, as their nesting may give rise to very
  different shapes.
\item Unlike our expressions, which are discrete, we can continuously transform
  one feature vector into another. This enables many powerful machine learning
  algorithms to be used, such as those based on \emph{gradient descent} or, in
  our case, arithmetic means.
\item Feature vectors can be chosen to represent the relevant information in a
  more compressed form than the raw data; for example, we might replace verbose,
  descriptive identifiers with sequential numbers. This reduces the input size
  of the machine learning problem, improving efficiency.
\end{itemize}

