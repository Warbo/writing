\section{Machine Learning}

Machine learning (ML) is a sub-field of artificial intelligence (AI) which
emphasises the use of statistical methods to find patterns and optimise
functions, in contrast with symbolic reasoning approaches like automated theorem
proving (commonly referred to as ``Good Old-Fashioned AI'').

The machine learning umbrella spans many techniques and applications, but one
common distinction is between \emph{supervised} and \emph{unsupervised} tasks.
In supervised learning we allow the system to perform some behaviour, then an
external process assesses whether it behaved well or poorly (akin to Justice
Potter Stewart's test for obscenity: ``I know it when I see it''). The system's
parameters are optimised to try and score highly on the test. Common supervised
learning tasks include function approximation using input/output examples (e.g.
classifying scanned images of numerals) where the test is the proportion of
example inputs which got the correct output, and reinforcement learning where
the system chooses from a set of ``actions'', receives a pair of ``observation''
and ``reward'' (test score) in response, and must accumulate the most reward.

Unsupervised learning does not have the same testable ``success'' criteria as
supervised learning. Instead, the goal is to find patterns and form models of
the target domain, which is more ambitious and open-ended than supervised
learning but also much harder to define, measure and optimise. Examples of this
are prediction and compression tasks such as auto-encoders
(learned compressor/decompressor pairs, rewarded by how well they preserve data
through a ``round trip'' compression-then-decompression) and adversarial or
co-evolution tasks (coupled systems where each is rewarded by exploiting
mis-prediction in the others).

We focus on the application of statistical machine learning algorithms to
\emph{languages}, such as logical statements or programming languages, This is
made particularly difficult due to the ubiquity of recursive structures, from
inductive datatypes, to recursive function definitions; from theorem statements,
to proof objects. Such nested structures may extend to arbitrary depth, which
makes them difficult to represent in the fixed numerical terms expected by most
machine learning algorithms.

\subsection{Feature Extraction}

\emph{Feature extraction} is a common pre-processing step for machine learning
(ML). Rather than feeding ``raw'' data straight into our ML algorithm, we only
learn a sample of \emph{relevant} details, known as \emph{features}. This has
a few benefits:

\begin{itemize}
\item \emph{Feature vectors} (ordered sets of features) are chosen to be more
  compact than the data they're extracted from: feature extraction is
  \emph{lossy compression}. This reduces the size of the ML problem, improving
  efficiency (e.g. running time).
\item We avoid learning irrelevant details such as the encoding system used,
  improving \emph{data} efficiency (the number of samples required to spot a
  pattern).
\item Feature vectors are designed to have a fixed size, i.e. they will all have
  length (or \emph{dimension}) $d$. Many machine learning algorithms only work
  with inputs of a uniform size; feature extraction allows us to use these
  algorithms in domains where the size of each input is not known, may vary or
  may even be unbounded. For example, element-wise comparison of feature vectors
  is trivial (compare the $i$th elements for $1 \leq i \leq d$); for expressions
  this is not so straightforward, as their nesting may give rise to very
  different shapes.
\item Unlike our expressions, which are discrete, we can continuously transform
  one feature vector into another. This allows operations like averaging,
  interpolation and gradient \emph{gradient descent} to be used, which are not
  suitable for discrete terms.
\end{itemize}

As an example, say we want to learn relationships between the following program
fragments:

\begin{haskell}
data Maybe a = Nothing | Just a

data Either a b = Left a | Right b
\end{haskell}

We might hope our algorithm discovers relationships like:

\begin{itemize}
  \item Both are valid Haskell code.
  \item Both describe datatypes.
  \item Both datatypes have two constructors.
  \item \hs{Either} is a generalisation of \hs{Maybe} (we can define \hs{Maybe a
      = Either () a} and \hs{Nothing = Left ()}).
  \item There is a symmetry in \hs{Either}: \hs{Either a b} is equivalent to
    \hs{Either b a} if we swap occurences of \hs{Left} and \hs{Right}.
  \item It is trivial to satisfy \hs{Maybe} (using \hs{Nothing}).
  \item It is not trivial to satisfy \hs{Either}; we require an \hs{a} or a
    \hs{b}.
\end{itemize}

However, this is too optimistic. Without our domain-knowledge of Haskell, an ML
algorithm cannot impose any structure on these fragments, and will treat them as
strings of bits. Our high-level hopes are obscured by low-level details: the
desirable patterns of Haskell types are mixed with undesirable patterns of ASCII
bytes, of letter frequency in English words, and so on.

In theory we could throw more computing resources and data at a problem (this is
precisely what \emph{deep learning} techniques do), but available hardware and
corpora are always limited. Instead, feature extraction lets us narrow the ML
problem to what we, with our domain knowlege, consider important.

There is no \emph{fundamental} difference between raw representations and
features: the identity function is a valid feature extractor. Likewise, there is
no crisp distinction between feature extraction and machine learning: a
sufficiently-powerful learner doesn't require feature extraction, and a
sufficiently-powerful feature extractor doesn't require any learning!
\footnote{Consider a classification problem, to assign a label $l \in L$ to each
  input. If we only extract a single feature $f \in L$, we have solved the
  classification problem without using a separate learning step.}

Rather, the terms are distinguished for purely \emph{practical} reasons: by
separating feature extraction from learning, we can distinguish straightforward,
fast data transformation (feature extraction) from complex, slow statistical
analysis (learning). This allows for modularity, separation of concerns, and in
particular allows ``off-the-shelf'' ML to be re-used across a variety of
different domains. Even with no domain knowledge, feature extraction can still
be used to improve efficiency by compressing the input (for example using
\emph{autoencoding}).

\subsubsection{Truncation and Padding}

The simplest way to limit the size of recursive structures is to truncate
anything larger than a particular size (and pad anything smaller). This is the
approach taken by ML4PG~\cite{journals/corr/abs-1302-6421}, which limits itself
to trees with at most 10 levels and 10 elements per level; each tree is
converted to a $30 \times 10$ matrix (3 values per tree node) and learning
takes place on these normalised representations.

Truncation is unsatisfactory in the way it balances \emph{data} efficiency with
\emph{time} efficiency. Specifically, truncation works best when the input data
contains no redundancy and is arranged with the most significant data first (in
a sense, it is ``big-endian''). The less these assumptions hold, the less we can
truncate. Since many ML algorithms scale poorly with input size, it is important
to truncate aggressively, but there may be useful information deep in the trees
which would be discarded.

\subsubsection{Dimension Reduction}

A more sophisticated approach to the problem of reducing input size is to view
it as a \emph{dimension reduction} technique: our inputs can be modelled as
points in a high-dimensional space, which we want to project into a
lower-dimensional space.

Truncation is a trivial dimension reduction technique: take the first $N$
coordinates. More sophisticated projection functions, such as Principle
Component Analysis (PCA) consider the \emph{distribution} of the points, and
project with the hyperplane which preserves as much of the variance as possible
(or, equivalently, reduces the \emph{mutual information} between the points).

Dimension reduction can also be approached as an unsupervised ML task (even if
the data are labelled for a supervised
task~\cite{Oveisi.Oveisi.Erfanian.ea:2012}). One popular approach is the
\emph{autoencoder}, which transforms training data into smaller feature vectors
and back again, whilst trying to minimise reconstruction error. Such compressed
representations can be used as feature vectors for subsequent learning
tasks. More semantically-meaningful features can be found using
\emph{variational autoencoding}~\cite{kingma2013auto}, which add a stochastic
sampling step before decoding, forcing similar data to produce similar vectors.

Since these techniques are effectively ML algorithms in their own right, they
suffer some of the same difficulties we encounter with recursive expressions:

\begin{itemize}
  \item They operate \emph{offline}, requiring all input points up-front
  \item All input points must have the same dimensionality
\end{itemize}

Sophisticated dimension reduction is still useful for \emph{compressing} large,
redundant features into smaller, information-dense representations, and as such
provides a good complement to more flexible but less informative techniques. For
example, we might truncate (and pad) tree structures into large, sparse vectors
of a fixed size; then use PCA to pick only a few of the most informative
dimensions to constitute our resulting feature vectors.

The requirement for offline ``batch'' processing is more difficult to overcome,
since any learning performed for feature extraction will interfere with the
core learning algorithm that's consuming these features (this is why deep
learning is often done greedily).

\subsubsection{Distributed Representations}

A \emph{distributed representation} does not isolate each semantic feature (for
example, presence of absence of some lexeme) to separate, independent elements
of a feature vector. Instead, features are represented as \emph{patterns} across
multiple elements: each feature uses many elements and each element contributes
to many features~\cite{hinton1984distributed}. Such encodings provide simple
mechanisms to represent \emph{combinations} of features (combining their
patterns element-wise), \emph{partial} features (presence of the pattern across
a subset of its elements) and \emph{generalisation} (patterns which are present
as sub-patterns of multiple features).

The compressed representations learned by autoencoders are distributed
representations, and are usually continuous. A discrete alternative is to use
\emph{hashing}, which can also be learned (e.g. using \emph{random
forests}~\cite{vens2011random}).

Particularly relevant to our work is the ability for distributed representations
to approximate recursive data such as syntax trees. This is achieved by defining
(or learning) functions to \emph{combine} the representations of parts into
those of composite objects, to \emph{test} whether or not a given feature vector
represents a composite object, and to \emph{project} composite representations
back into those of their constituents. For example, if $c$ and $d$ are the
compression and decompression routines of an auto-encoder for non-recursive
data, we can define an auto-encoder ($c'$ and $d'$) for binary trees in terms
of a combining function $\bigoplus$, a predicate $\texttt{node?}$ and projection
functions $\pi_{left}$ and $\pi_{right}$ as follows:

\begin{align*}
  c'(x) &= \begin{cases}
             c'(a) \bigoplus c'(b) & \text{if $x = \texttt{node}(a, b)$} \\
             c(x)                  & \text{otherwise}
           \end{cases} \\
  d'(x) &= \begin{cases}
             \texttt{node}(d'(\pi_{left }(x)),
                           d'(\pi_{right}(x))) & \text{if \texttt{node?}(x)} \\
             d(x)                              & \text{otherwise}
           \end{cases}
\end{align*}

With a suitable distance function between trees (e.g. edit distance), we can
learn $c$, $d$, \texttt{node?} and $\bigoplus$ together in an unsupervised
manner, by minimising the round trip error on a training corpus. The resulting
$c'$ function can then directly encode recursive data to fixed-size feature
vector. Analogous constructions can be applied to different recursive
structures.

\subsection{Sequences}

To handle input of variable size, such as natural or formal language, research
attention has been given to the handling of \emph{sequences}. This is a lossless
approach, which splits the input into fixed-size \emph{chunks} (e.g. splitting
text into a sequence of characters), which are fed into an appropriate ML
algorithm one at a time. The sequence is terminated by a sentinel; an
``end-of-sequence'' marker which, by construction, is distinguishable from the
data chunks. Tradeoffs can be made between chunk size and sequence length; for
example, treating code as a sequence of
characters~\cite{cummins2017synthesizing} or lexer tokens~\cite{cummins2017end}.

Not all ML algorithms can be adapted to accept sequences. One notable approach
is to use \emph{recurrent (artificial) neural networks} (RNNs), which allow
arbitrary connections between nodes, including cycles. Compared to
\emph{feed-forward} (FF) neural networks, which are acyclic, the \emph{future
  output} of a RNN may depend arbitrarily on its \emph{past inputs} (in fact,
RNNs are universal computers).

The main problem with RNNs, compared to the more widely-used FF approach, is the
difficulty of training them. If we extend the standard backpropagation algorithm
to handle cycles, we get the \emph{backpropagation through time}
algorithm~\cite{werbos1990backpropagation}. However, this suffers a problem
known as the \emph{vanishing gradient}: error values decay exponentially as they
propagate back through the cycles, which prevents effective learning of delayed
dependencies, undermining the main advantage of RNNs. The vanishing gradient
problem is the subject of current research, with countermeasures including
\emph{neuroevolution} (using evolutionary computation techniques to complement
or replace backpropagation) and \emph{gated neurons} (which are selectively
updated, and hence resilient to spurious overwriting; the most common example
being \emph{long short-term memory} (LSTM)~\cite{hochreiter1997long}).

An alternative to introducing recurrent connections is the use of
\emph{attention}~\cite{vaswani2017attention}. In a (typically FF) neural network
with attention, the network is given a chunk of the input and, in addition to
its usual output, also produces some ``attention'' information specifying which
chunks of the input should be provided next. The network is run again, but with
the input adjusted using this attention information; more output is produced,
along with new attention information, and so on. Such networks do not need to
``remember'' long-distance relationships, whether by recurrent connections or
gated nodes; they instead use the same input over and over again, but use their
``attention'' to emphasise those parts which are relevant to upcoming
output and hence should be taken into account when computing it.

\subsubsection{Recursive Structure}

Using sequences to represent recursive data is problematic: if we want our
learning algorithm to exploit structure (such as the depth of a token), it will
need to discover how to parse the sequences for itself, which is wasteful since
we already have a parsed representation to begin with. The
\emph{back-propagation through structure} approach~\cite{goller1996learning} is
a more direct solution to this problem, using FF neural networks to learn
recursive distributed representations~\cite{pollack1990recursive} which
correspond to the recursive structure of the inputs. Such distributed
representations can also be used for sequences, which we can use to encode
sub-trees when the branching factor of nodes is not
uniform~\cite{kwasny1995tail}. Recursive structure can also be gated inside
LSTM cells~\cite{zhu2015long}.

A simpler alternative for generating recursive distributed representations is to
use circular convolution \cite{conf/ijcai/Plate91}. Although promising results
are shown for its use in \emph{distributed tree kernels}
\cite{zanzotto2012distributed}, our preliminary experiments in applying
circular convolution to functional programming expressions found most of the
information to be lost in the process; presumably as the expressions are too
small.

\emph{Kernel methods} have also been applied to structured information, for
example in \cite{Gartner2003} the input data (including sequences, trees and
graphs) are represented using \emph{generative models}, such as hidden Markov
models, of a fixed size suitable for learning. Many more applications of kernel
methods to structured domains are given in \cite{bakir2007predicting}, which
could be used to learn more subtle relations between expressions than recurrent
clustering alone.

% FIXME: Include clustering
