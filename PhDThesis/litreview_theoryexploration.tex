The prototypical implementation of MTE is the Theorema system of Buchberger and
colleagues~\cite{buchberger,buchberger2016theorema}, which also places a strong
emphasis on user interface and output presentation. Theory exploration in the
Theorema system involves the user formalising their definitions in a consistent,
layered approach; such that reasoning algorithms can exploit this structure in
subsequent proofs, calculations, etc. The potential of this strategy was
evaluated by illustrating the automated synthesis of Buchberger's own Gr\"obner
bases algorithm~\cite{buchberger:04}.

A similar ``layering'' approach is found in the IsaScheme system of
Monta{\~n}o-Rivas \etal{}~\cite{Montano-Rivas.McCasland.Dixon.ea:2012}, which
has also been quantitatively compared against IsaCoSy and HipSpec using
precision/recall analysis~\cite{claessen2013automating}. The name comes from its
embedding in the Isabelle proof assistant and its use of ``schemes'':
higher-order formulae which can be used to generate new concepts and
conjectures. Variables within a scheme are instantiated automatically and this
drives the invention process. For example, the concept of ``repetition'' can be
encoded as a scheme, and instantiated with existing encodings of zero, successor
and addition to produce a definition of multiplication. The same scheme can be
instantiated with this new multiplication function to produce exponentiation.

IsaCoSy and QuickSpec (the conjecture generation component of HipSpec) are
described in more detail in $\S$\ref{sec:existing-tools}, since these are the
tools we chose to evaluate and compare for $\S$\ref{sec:application}. QuickSpec
has since evolved to version 2~\cite{smallbone2017quick}, which replaces the
distinct enumeration and testing steps with a single, iterative algorithm
similar to that of IsaCoSy. Generated conjectures are fed into a Knuth-Bendix
completion algorithm to form a corresponding set of rewrite rules. As
expressions are enumerated, they are simplified using these rules and discarded
if equal to a known expression. If not, QuickCheck tests whether the new
expression can be distinguished from the known expressions through random
testing: those which can are added to the set of known expressions. Those which
cannot be distinguished are conjectured to be equal, and the rewrite rules are
updated.

QuickSpec has also inspired another MTE tool for Haskell called
Speculate~\cite{braquehais2017speculate}, which operates in a similar way but
also makes use of the laws of total orders and Boolean algebra to conjecture
\emph{in}equalities and conditional relations between expressions.

Another notable MTE implementation, distinct from those based in Isabelle and
Haskell, is the MATHsAiD project (Mechanically Ascertaining Theorems from
Hypotheses, Axioms and Definitions)~\cite{roy}. Unlike the tools above, which
generate \emph{conjectures} that may later be sent to automated provers,
MATHsAiD directly generates \emph{theorems}, by making logically valid
inferences from a given set of axioms and definitions. Evaluation of the
interestingness of these theorems was performed qualitatively by the system's
developer, which highlights how these tools could benefit from the availability
of an objective, repeatable, quantitative method of evaluation and comparison
such as ours.
