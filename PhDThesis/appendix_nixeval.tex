\subsection{\texttt{nix-eval}}

Haskell's support for runtime evaluation of code (a la Lisp's \eval) is
limited. The most popular mechanism is the \hint package, which is built
on top of GHC's API and provides functions like
\hs{eval :: MonadInterpreter m => String -> m String}. The \hs{MonadInterpreter}
constraint ensures that GHC can be invoked and for managing its ``session'',
which includes details like compiler flags, a list of imported modules and paths
to compiled packages.

This function doesn't return arbitrary types of value; only \hs{String}s. This
is due to the given expression being passed to the \hs{show} function (e.g.
\hs{eval "length []"} will produce \hs{"0" :: String}, not
\hs{0 :: Int}).\footnote{\hs{show :: Show a => a -> String} is a method of the
  \hs{Show} type class, and is the standard way to equip Haskell types values
  with a printable representation (akin to Java's \texttt{toString} method).
  \hs{show} has no associated laws, but it is conventional to make it the
  inverse of the \hs{read :: Read a => String -> a} method of the \hs{Read} type
  class (where possible).} Serialising results in this way simplifies the
interface between evaluation contexts (in particular, type checking and
inference doesn't need to cross the interface); at the cost of having to
explicitly deserialise results into the desired type as needed.

One major problem with the \hint package is that the available packages
(and GHC version) are hard-coded into the ``outer'' Haskell context. For
example, a Haskell interpreter based on \hint would fail to evaluate
expressions that use packages or GHC features other than those the interpreter
was compiled against.\footnote{The package database could be overridden in
  principle, but compatibility issues and version conflicts would make this
  unreliable and error-prone in practice.} For our research we want a single
program for exploring \hs{arbitrary} Haskell code, which makes approaches like
\hint undesirable.

Our solution is the \nixeval package. This provides a function
\hs{eval :: Expr -> IO (Maybe String)}, which returns a \hs{String} for the same
reasons as \hint, but differs in its handling of expressions and context. Rather
than operating on a raw \hs{String}, \nixeval defines a type \hs{Expr}
containing several fields to keep track of the context in which the expression
is to be evaluated:

\begin{itemize}
\item \hs{eExpr :: String} is the raw code itself.
\item \hs{ePkgs :: [Pkg]} lists the packages which need to be installed.
\item \hs{eMods :: [Mod]} lists the modules which need to be imported.
\item \hs{eFlags :: [Flag]} lists flags which need passing to the compiler (e.g.
  to enable certain language features).
\item \hs{ePreamble :: [String]} lists code that needs to be defined before
  evaluating (for example, custom function definitions).
\end{itemize}

The types \hs{Pkg}, \hs{Mod} and \hs{Flag} are just wrappers containing a
\hs{String}; they are kept distinct to allow the type checker to catch
mistakes.\footnote{Avoiding such distinctions (i.e. using types to track
  representations but not semantics) is known as ``stringly-typed programming''
  and contributes to widespread errors like injection attacks. In the case of
  binary choices this principle is known as ``boolean blindness''.} These
wrapper types, as well as \hs{Expr} itself, implement the \hs{IsString} type
class, which allows GHC's \texttt{OverloadedStrings} language extension to parse
them from quoted literals, for example:

\begin{haskell}
textToUpper :: Expr
textToUpper = withPkgs       ["text"     ]  -- This is a  Pkg
                (withMods    ["Data.Text"]  -- This is a  Mod
                  (withFlags ["-O2"      ]  -- This is a  Flag
                    "toUpper"))             -- This is an Expr
\end{haskell}

The main advantage over \hint is the reification of packages, which can be
computed in arbitrary ways (including from user input). When an \hs{Expr} is
passed to our \hs{eval} function, a new \texttt{runhaskell} process is launched
with the given \hs{eFlags} (hence the presence of \hs{IO}). The elements of
\hs{eMods} are prefixed with \hs{"import "} and fed into its standard input
stream, followed by the contents of \hs{ePreamble}. The \hs{eExpr} code is
wrapped in parentheses and prefixed with \hs{"main = Prelude.putStr"} which acts
as the entry point of the generated code (\hs{Prelude.putStr} writes a
\hs{String} to standard output).\footnote{The code generation for \hs{eExpr} is
  actually user-configurable, in case the default is too simplistic. Unlike
  \hint we never force calls to the \hs{show} method, since normal function
  calls are easier to specify and override than type class instances.} If the
subprocess exits successfully, its standard output is returned (wrapped in
\hs{Just}); otherwise it is written to the standard error stream and a
\hs{Nothing} value is returned (the subprocess inherits the parent's standard
error stream).

The \texttt{runhaskell} command can be invoked in two ways: if
\texttt{runhaskell} is available via the \texttt{PATH} environment variable, and
the contents of \hs{ePkgs} are all installed in its package database, then we
invoke that command directly. If not, we use the \texttt{nix-shell} command of
the Nix package manager to invoke \texttt{runhaskell} in a sandbox~\cite{TODO}.
We use the Nix language to request a sandbox containing GHC and all of the
elements of \hs{ePkgs}. We also allow the default Nix packages to be overridden
via an environment variable, which allows the user to define their own packages,
select particular versions, or perform any other arbitrary setup (since the Nix
language is Turing-complete).

Nix packages are immutable and (mostly) reproducible, which allows sandboxes and
their components to be re-used, and even shared across machines. However, any
package which isn't cached in this way must be built from source, which can
cause \hs{eval} to take an arbitrarily long time. Even a cached sandbox can take
a few seconds to initialise (largely due to I/O latency as Nix computes and
inspects the sandbox components). To avoid too much overhead, it is advisable to
invoke \hs{eval} as few times as possible (preferably only once). If we need to
dynamically evaluate multiple parts of a program, it might be preferable to move
them all into one big \hs{Expr} value.

\begin{figure}
  \begin{haskell}
-- An Expr representing a list of the given Exprs
asList :: [Expr] -> Expr
asList = foldr (\x -> (("(:)" $$ x) $$)) "[]"

-- Composes the given list of functions using the (.) operator
compose :: [Expr] -> Expr
compose [x]    = x
compose (x:xs) = ("(.)" $$ x) $$ compose xs

-- Apply x to n undefined arguments; result is undefined with x's output type
addCalls :: Int -> Expr -> Expr
addCalls 0 x = x
addCalls n x = addCalls (n-1) (x $$ "undefined")

-- Coerces values to have the output type of a known function "f" (of arity a)
converter :: Expr
converter = ("flip" $$ "asTypeOf") $$ addCalls a "f"

-- Coerces a value to the output type of "f", serialises it then hashes it
hasher :: Expr
hasher = compose [withPkgs ["murmur-hash"]
                    (qualified "Data.Digest.Murmur32" "asWord32"),
                  withPkgs ["murmur-hash"]
                    (qualified "Data.Digest.Murmur32" "hash32"),
                  withPkgs ["cereal"]
                    (qualified "Data.Serialize" "runPut"),
                  withPkgs ["cereal"]
                    (qualified "Data.Serialize" "put"),
                  converter]
  \end{haskell}
  \caption{Example usage of \nixeval functions, taken from \textsc{ML4HS}, which
    invokes \qspec on arbitrary user-provided functions.}
  \label{fig:nix-eval}
\end{figure}

Using \hs{String} functions to manipulate such complicated values soon becomes
unwieldy, so a suite of combinators are provided for composing \hs{Expr} values
in useful ways; examples are shown in Figure~\ref{fig:nix-eval}. Besides the
\hs{withFoo} functions shown before, the \hs{qualified} function can reduce the
ambiguity of names while avoiding repetition of module names: a call like
\hs{qualified "Data.Text" "toUpper"} returns an \hs{Expr} of the name
\hs{"Data.Text.toUpper"} with the \hs{"Data.Text"} \hs{Mod}ule in its
\hs{eMods} list. \hs{foo $$ bar} denotes function application (analogous to
Haskell's standard \hs{$}%$ operator), by juxtaposing the \hs{eExpr} of \hs{foo}
with that of \hs{bar} (wrapping both in parentheses to ensure precedence); the
contexts are appended. Finally, the \hs{asString :: Show a => a -> Expr}
function calls \hs{show} on the given value and wraps the resulting \hs{String}
into an \hs{Expr}; this is especially useful for representing literal
\hs{String}s, since their \hs{show} instance will wrap them in quotes and
perform any required escaping.

Despite being more structured than raw \hs{String}s, the the above API still
suffers from a form of "stringly-typed programming"; notably, that every encoded
expression has the same type: \hs{Expr}. We can avoid this through the use of
\emph{phantom types}: overly-restrictive type annotations, which are unrelated
to the encoding but must still match up in order to pass the type checker. In
particular we can represent typed expressions like this:

\begin{haskell}
newtype TypedExpr t = TE Expr
\end{haskell}

Notice that the parameter \hs{t} doesn't appear on the right hand side: the
values of \hs{TypedExpr} are just those of \hs{Expr} (the phantom type is
\emph{erased} after type-checking). The \hs{TE} constructor is hence polymorphic
in \hs{t}, so \hs{TE x :: TypedExpr T} is valid for \emph{any} choice of \hs{x}
and \hs{T} (similar to how \hs{Nil :: List T} is valid for any choice of
\hs{T}). In particular, this includes the type of the expression
\emph{represented by} \hs{x}, for example \hs{TE "True" :: TypedExpr Bool} is a
type-correct expression.

Haskell will infer the most general type of an expression if none is provided
(i.e. \hs{forall t. TypedExpr t}), but will honour more restricted annotations
if given. This allows the types represented by literal expressions to be
asserted, for example:

\begin{haskell}
nothing' :: TypedExpr (Maybe a)
nothing' = TE (qualified "Data.Maybe" "Nothing")

just' :: TypedExpr (a -> Maybe a)
just' = TE (qualified "Data.Maybe" "Just")
\end{haskell}

Annotations on the combinator library can propagate and check these phantom
types, most notably:

\begin{haskell}
-- Typed function application
($$$) :: TypedExpr (a -> b) -> TypedExpr a -> TypedExpr b
TE f $$$ TE x = TE (f $$ x)

-- Only expressions of type String can be evaluated, since we use Prelude.putStr
tEval :: TypedExpr String -> IO (Maybe String)
tEval (TE x) = eval x
\end{haskell}

Haskell will accept any annotation, so care must be taken to ensure that they
correspond correctly to the encoded expressions. If user-provided literals
aren't required then the \hs{TE} constructor can be kept private to prevent
unsound assertions or coercions. However, there is a more subtle problem with
such \hs{TypedExpr}s: the whole point of \nixeval is to manipulate and evaluate
code whose dependencies are unknown or unavailable at compile time. For example,
we cannot assign a meaningful type at compile time to an expression which has
been generated from user input. Even if we know the value of an expression at
compile time, its type may come from a dependency that our program doesn't have
access to. In these cases the \hs{TypedExpr} approach can still be useful for
auxiliary code whose types are known to us; those unknown or unavailable parts
can either be unwrapped into raw \hs{Expr}s, or dummy types can be used instead.
For example, if we are manipulating code involving the \hs{Data.Text.Text} type,
but we don't have or want the \texttt{text} package in our dependencies, we can
declare an empty type like \hs{data TEXT} (note that this declaration has no
right-hand side) and use this in our annotations in place of
\hs{Data.Text.Text}. The compiler will type-check our code as desired, ensuring
functions are called on appropriate arguments, \hs{tEval} is given on a
\hs{String}, etc.

In the course of writing MLSpec we experimented with both raw \hs{Expr} and
\hs{TypedExpr} values. Whilst we found the latter easier for constructing large
expressions, we ultimately found it even more productive to move as much
functionality as possible into a standalone Cabal package
(\texttt{mlspec-helper}) and have MLSpec call into that code from simple
\hs{Expr} values.
