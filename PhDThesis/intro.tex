\chapter{Introduction}

As computers and software become more capable, and as our reliance on them
increases, the importance of \emph{understanding}, \emph{predicting} and
\emph{verifying} these systems grows; which is undermined by their
ever-increasing complexity. Mathematics provides us with powerful, systematic
methods of reasoning, which we can bring to bear on this challenge; in
particular those of \emph{formal logic} and \emph{statistics}. By (partially)
mechanising these approaches, in the fields of \emph{theorem proving} and
\emph{machine learning}, respectively, we can leverage these increasing machine
capabilities and direct them for the purpose of analysis. However, the question
still remains: on what should we focus that analysis?

In this work, we investigate the notion of \emph{interestingness} in the
exploration of formal systems (an area known as \emph{theory exploration}) as a
way to make productive use of resources in an often intractable domain. To keep
things concrete, we focus our formal analysis on equational formulae describing
programs in the Haskell language, for reasons elaborated in \S
\ref{sec:haskell}. Conceptually, we maintain a broader view, and survey many
related areas which may offer insights on the problem.

Appeals to interestingness arise when more direct measures, such as utility, are
not available. For example, the inclusion of particular statements in a program
or proof development can be easily justified based on their contribution to the
overall solution; however, in a \emph{library} there is no particular problem
being solved, in which case we must judge statements on less direct criteria,
such as how ``interesting'' they may be to our users. Appeals to interestingness
abound in the history of computer-assisted reasoning; for example, in 1971
Plotkin \cite{plotkin1971further} considered the task of \textquote{discovering
  theorems $T$ from a system of axioms $Ax$}, and in particular the questions
\textquote{Under what conditions is $T$ an interesting, possible theorem in the
  system $Ax$?} and \textquote{Is there a way to generate (most) interesting
  possible theorems?}. Despite such widespread use of the term, there is no
standard definition of what makes a formal object, whether it is an axiom, a
conjecture, a proof, etc., ``interesting''; although many ad-hoc heuristics have
been proposed.

\iffalse
We begin our undertaking in \S \ref{sec:background} by introducing the Haskell
language, as well as the relevant fields of verification for context. We define
a formal framework for our investigation, and show how it relates to the
existing theorem proving landscape. A selection of theorem proving scenarios
which \emph{require} exploration are discussed in \S \ref{sec:examples}, whilst
related work, including existing defintions of interestingness, is surveyed in
\S \ref{sec:related}. We also review the use of exploration in other fields of
Artificial Intelligence and Machine Learning, where researchers are
experimenting with replacing \emph{explicit} goals and rewards with
\emph{implicit} alternatives such as interestingness. Recent efforts in this
area have lead to the emergence of principled theories, mostly based around
(algorithmic) information theory, which may be adapted to our theory exploration
context.

% TODO: Do we need this?
We discuss our present contributions in \S \ref{sec:current} and future research
directions in \S \ref{sec:future}, before concluding in \S \ref{sec:conclusion}.
\fi
