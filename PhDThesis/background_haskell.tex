\subsection{Haskell}
\label{sec:haskell}

We mostly focus our attention on the Haskell programming language, both as an
implementation vehicle and as our representation of functions, properties, etc.
to explore. We choose Haskell since it combines formal, logical underpinnings
which aid reasoning (compared to more popular languages like Java or C), yet it
is still popular enough to sustain a rich ecosystem of tooling and a large
corpus of existing code (compared to more formal languages like Coq or
Isabelle).

The following features make Haskell especially useful for our purposes, although
many are also present in other languages such as StandardML and Coq (which we
also use, but only when needed for compatibility with prior work):

\begin{description}

\item{Functional}: All control flow in Haskell is performed by function
  abstraction and application, which we can reason about using standard rules of
  inference such as \emph{modus ponens}.

\item{Pure}: Execution of actions (e.g. reading files) is separate to evaluation
  of expressions; hence our reasoning can safely ignore complicated external and
  non-local interactions.

\item{Statically Typed}: Expressions are constrained by \emph{types}, which can
  be used to eliminate unwanted combinations of values, and hence reduce search
  spaces; \emph{static} types can be deduced syntactically, without having to
  execute the code.

\item{Non-strict}: If an evaluation strategy exists for $\beta$-normalising an
  expression (i.e. performing function calls) without diverging, then a
  non-strict evaluation strategy will not diverge when evaluating that
  expression. This is rather technical, but in simple terms it allows us to
  reason effectively about a Turing-complete language, where evaluation may not
  terminate. For example, when reasoning about \emph{pairs} of values \hs{(x,
    y)} and projection functions \hs{fst} and \hs{snd}, we might want to use an
  ``obvious'' rule such as
  $\forall \text{\hs{x y}}, \text{\hs{x}} = \text{\hs{fst (x, y)}}$. Haskell's
  non-strict semantics makes this equation valid; whilst it would \emph{not} be
  valid in the strict setting common to most other languages, where the
  expression \hs{fst (x, y)} will diverge if \hs{y} diverges (and hence alter
  the semantics, if \hs{x} doesn't diverge).

\item{Algebraic Data Types}: These provide a rich grammar for building up
  user-defined data representations, and an inverse mechanism to inspect these
  data by \emph{pattern-matching}. For our purposes, the useful consequences of
  ADTs and pattern-matching include their amenability for inductive proofs and
  the fact they are \emph{closed}; i.e. an ADT's declaration specifies all of
  the normal forms for that type. This makes exhaustive case analysis trivial,
  which would be impossible for \emph{open} types (for example, consider classes
  in an object oriented language, where new subclasses can be introduced at any
  time).

\item{Parametricity}: This allows Haskell \emph{values} to be parameterised over
  \emph{type-level} objects; provided those objects are never inspected. This
  has the \emph{practical} benefit of enabling \emph{polymorphism}: for example,
  we can write a polymorphic identity function \hs{id :: forall t. t ->
    t}. \footnote{Read ``\hs{a :: b}'' as ``\hs{a} has type \hs{b}'' and ``\hs{a
      -> b}'' as ``the type of functions from \hs{a} to \hs{b}''.} Conceptually,
  this function takes \emph{two} parameters: a type \hs{t} \emph{and} a value of
  type \hs{t}; yet only the latter is available in the function body,
  e.g. \hs{id x = x}. This inability to inspect type-level arguments gives us
  the \emph{theoretical} benefit of being able to characterise the behaviour of
  polymorphic functions from their type alone, a technique known as
  \emph{theorems for free} \cite{wadler1989theorems}.

\item{Type classes}: Along with their various extensions, type classes are
  interfaces which specify a set of operations over a type (or other type-level
  object, such as a \emph{type constructor}). Many type classes also specify a
  set of \emph{laws} which their operations should obey but, lacking a simple
  mechanism to enforce this, laws are usually considered as documentation. As a
  simple example, we can define a type class \hs{Semigroup} with the following
  operation and associativity law:

\begin{lstlisting}[language=Haskell, xleftmargin=.2\textwidth, xrightmargin=.2\textwidth]
op :: forall t. Semigroup t => t -> t -> t
\end{lstlisting}

$$\forall \text{\hs{x y z}}, \text{\hs{op x (op y z)}} = \text{\hs{op (op x y) z}}$$

The notation \hs{Semigroup t =>} is a \emph{type class constraint}, which
restricts the possible types \hs{t} to only those which implement
\hs{Semigroup}. \footnote{Alternatively, we can consider \hs{Semigroup t} as the
  type of ``implementations of \hs{Semigroup} for \hs{t}'', in which case
  \hs{=>} has a similar role to \hs{->} and we can consider \hs{op} to take
  \emph{four} parameters: a type \hs{t}, an implementation of \hs{Semigroup t}
  and two values of type \hs{t}. As with parameteric polymorphism, this extra
  \hs{Semigroup t} parameter is not available at the value level. Even if it
  were, we could not alter our behaviour by inspecting it, since Haskell only
  allows types to implement each type class in at most one way, so there would
  be no information to branch on.} There are many \emph{instances} of
\hs{Semigroup} (types which may be substituted for \hs{t}), e.g. \hs{Integer}
with \hs{op} performing addition. Many more examples can be found in the
\emph{typeclassopedia} \cite{yorgey2009typeclassopedia}. This ability to
constrain types, and the existence of laws, helps us reason about code
generically, rather than repeating the same arguments for each particular pair
of \hs{t} and \hs{op}.

\item{Equational}: Haskell uses equations at the value level, for definitions;
  at the type level, for coercions; at the documentation level, for typeclass
  laws; and at the compiler level, for ad-hoc rewrite rules. This provides us
  with many \emph{sources} of equations, as well as many possible \emph{uses}
  for any equations we might discover. Along with their support in existing
  tools such as SMT solvers, this makes equational conjectures a natural target
  for theory exploration.

\item{Modularity}: Haskell has a module system, where each module may specify an
  \emph{export list} containing the names which should be made available for
  other modules to import. When such a list is given, any expressions \emph{not}
  on the list are considered \emph{private} to that module, and are hence
  inaccessible from elsewhere. This mechanism allows modules to provide more
  guarantees than are available just in their types. For example, a module may
  represent email addresses in the following way:

\begin{lstlisting}[language=Haskell, xleftmargin=.2\textwidth, xrightmargin=.2\textwidth, upquote=true]
module Email (Email(), at, render) where

data Email = E String String

render :: Email -> String
render (E u h) = u ++ "@" ++ h

at :: String -> String -> Maybe Email
at "" h  = Nothing
at u  "" = Nothing
at u  h  = Just (E u h)
\end{lstlisting}

The \hs{Email} type guarantees that its elements have both a user part and a
host part (modulo divergence), but it does not provide any guarantees about
those parts. We also define the \hs{at} function, a so-called ``smart
constructor'', which has the additional guarantee that the \hs{Email}s it
returns contain non-empty \hs{String}s. By ommitting the \hs{E} constructor from
the export list on the first line \footnote{The syntax \hs{Email()} means we're
  exporting the \hs{Email} type, but not any of its constructors.}, the only way
\emph{other} modules can create an \hs{Email} is by using \hs{at}, which forces
the non-empty guarantee to hold globally.

\end{description}

Together, these features make Haskell code highly structured, amenable to
logical analysis and subject to many algebraic laws. However, as mentioned with
regards to type classes, Haskell itself is incapable of expressing or enforcing
these laws (at least, without difficulty \cite{lindley2014hasochism}). This
reduces the incentive to manually discover, state and prove theorems about
Haskell code, e.g. in the style of interactive theorem proving, as these results
may be invalidated by seemingly innocuous code changes. This puts Haskell in a
rather special position with regards to the discovery of interesting theorems;
namely that many discoveries may be available with very little work, simply
because the code's authors are focused on \emph{software} development rather
than \emph{proof} development. The same cannot be said, for example, of ITP
systems; although our reasoning capabilities may be stronger in an ITP setting,
much of the ``low hanging fruit'' will have already been found through the
user's dedicated efforts, and hence theory exploration would be less likely to
discover unexpected properties.

Other empirical advantages to studying Haskell, compared to other programming
languages or theorem proving systems, include:

\begin{itemize}
\item The large amount of Haskell code which is freely available online, e.g. in
  repositories like \href{http://hackage.haskell.org}{Hackage}, with which we
  can experiment.

\item The existence of theory exploration systems such as \hspec{}, and related
  tools which we can to re-use, including conjecture generators like \qspec{}
  and \textsc{Speculate}; counterexample finders like \qcheck{}, \textsc{SmallCheck},
  \textsc{SmartCheck}, \textsc{LeanCheck}, \textsc{Hedgehog}, etc.; theorem
  provers like \textsc{Hip} \cite{rosen2012proving}; and other related testing
  and term-generating systems like \textsc{MuCheck} \cite{le2014mucheck},
  \textsc{MagicHaskeller}~\cite{katayama2011magichaskeller} and
  \textsc{Djinn}~\cite{augustsson2005djinn}.

\item The remarkable amount of infrastructure which exists for working with
  Haskell code, including package managers, compilers, interpreters, parsers,
  static analysers, etc.
\end{itemize}

%% FIXME: MERGE THE FOLLOWING WITH THE ABOVE

\newcommand{\CVar}{\texttt{Var}}
\newcommand{\CLit}{\texttt{Lit}}
\newcommand{\CApp}{\texttt{App}}
\newcommand{\CLam}{\texttt{Lam}}
\newcommand{\CLet}{\texttt{Let}}
\newcommand{\CCase}{\texttt{Case}}
\newcommand{\CType}{\texttt{Type}}
\newcommand{\CLocal}{\texttt{Local}}
\newcommand{\CGlobal}{\texttt{Global}}
\newcommand{\CConstructor}{\texttt{Constructor}}
\newcommand{\CLitNum}{\texttt{LitNum}}
\newcommand{\CLitStr}{\texttt{LitStr}}
\newcommand{\CAlt}{\texttt{Alt}}
\newcommand{\CDataAlt}{\texttt{DataAlt}}
\newcommand{\CLitAlt}{\texttt{LitAlt}}
\newcommand{\CDefault}{\texttt{Default}}
\newcommand{\CNonRec}{\texttt{NonRec}}
\newcommand{\CRec}{\texttt{Rec}}
\newcommand{\CBind}{\texttt{Bind}}

\begin{figure}
  \begin{equation*}
    \begin{split}
      expr\    \rightarrow\ & \CVar\ id                          \\
                         |\ & \CLit\ literal                     \\
                         |\ & \CApp\ expr\ expr                  \\
                         |\ & \CLam\ \mathcal{L}\ expr           \\
                         |\ & \CLet\ bind\ expr                  \\
                         |\ & \CCase\ expr\ \mathcal{L}\ [alt]   \\
                         |\ & \CType                             \\
      id\      \rightarrow\ & \CLocal\       \mathcal{L}         \\
                         |\ & \CGlobal\      \mathcal{G}         \\
                         |\ & \CConstructor\ \mathcal{D}         \\
      literal\ \rightarrow\ & \CLitNum\ \mathcal{N}              \\
                         |\ & \CLitStr\ \mathcal{S}              \\
      alt\     \rightarrow\ & \CAlt\ altcon\ expr\ [\mathcal{L}] \\
      altcon\  \rightarrow\ & \CDataAlt\ \mathcal{D}             \\
                         |\ & \CLitAlt\ literal                  \\
                         |\ & \CDefault                          \\
      bind\    \rightarrow\ & \CNonRec\ binder                   \\
                         |\ & \CRec\ [binder]                    \\
      binder   \rightarrow\ & \CBind\ \mathcal{L}\ expr
    \end{split}
  \end{equation*}
  Where:
  \begin{tabular}[t]{l @{ $=$ } l}
    $\mathcal{S}$ & string literals    \\
    $\mathcal{N}$ & numeric literals   \\
    $\mathcal{L}$ & local identifiers  \\
    $\mathcal{G}$ & global identifiers \\
    $\mathcal{D}$ & constructor identifiers
  \end{tabular}

  \caption{Simplified syntax of GHC Core in BNF style. $[]$ and $(,)$ denote repetition and grouping, respectively.}
  \label{fig:coresyntax}
\end{figure}

We decided to focus on theory exploration in the Haskell programming language as
it has mature, state-of-the-art implementations (\qspec{} \cite{QuickSpec} and
\hspec{} \cite{claessen2013automating}). This is evident from the fact that the
state-of-the-art equivalent for Isabelle/HOL, the \textsc{Hipster}
\cite{Hipster} system, is actually implemented by translating to Haskell and
invoking \hspec{}.

Haskell is well-suited to programming language research; indeed, this was a goal
of the language's creators \cite{marlow2010haskell}. Like most
\emph{functional} programming languages, Haskell builds upon $\lambda$-calculus,
with extra features such as a strong type system and ``syntactic sugar'' to
improve readability. For simplicity, we will focus on an intermediate
representation of the \textsc{GHC} compiler, known as \emph{GHC Core}, rather
than the relatively large and complex syntax of Haskell proper. Core is based on
\fc{}, but for our machine learning purposes we are mostly interested in its
syntax; for a more thorough treatment of \fc{} and its use in GHC, see
\cite[Appendix C]{sulzmann2007system}.

The sub-set of Core we consider is shown in Figure \ref{fig:coresyntax};
compared to the full language \footnote{As of GHC version 7.10.2, the latest at
  the time of writing.} our major restriction is to ignore type hints (such as
explicit casts, and differences between types/kinds/coercions). For brevity, we
also omit several other forms of literal (machine words of various sizes,
individual characters, etc.), as their treatment is similar to those of strings
and numerals. We will use quoted strings to denote names and literals,
e.g. \texttt{Local "foo"}, \texttt{Global "bar"}, \texttt{Constructor "Baz"},
\texttt{LitStr "quux"} and \texttt{LitNum "42"}, and require only that they can
be compared for equality.

\begin{figure}
  \begin{haskell}
    data Nat = Z
             | S Nat

    plus :: Nat -> Nat -> Nat
    plus    Z  y = y
    plus (S x) y = S (plus x y)

    mult :: Nat -> Nat -> Nat
    mult    Z  y = Z
    mult (S x) y = plus y (mult x y)

    odd :: Nat -> Bool
    odd    Z  = False
    odd (S n) = even n

    even :: Nat -> Bool
    even    Z  = True
    even (S n) = odd n
  \end{haskell}
  \caption{A Haskell datatype for Peano numerals with some simple arithmetic
    functions, including mutually-recursive definitions for \hs{odd} and
    \hs{even}. \hs{Bool} is Haskell's built in boolean type, which can be
    regarded as \hs{data Bool \equal{} True | False}.}
  \label{fig:haskellexample}
\end{figure}

\begin{figure}
  \begin{subfigure}[plus]{\textwidth}
    \begin{small}
      \underline{\texttt{plus}}
      \begin{verbatim}
Lam "a" (Lam "y" (Case (Var (Local "a"))
                       "b"
                       (Alt (DataAlt "Z") (Var (Local "y")))
                       (Alt (DataAlt "S") (App (Var (Constructor "S"))
                                               (App (App (Var (Global "plus"))
                                                         (Var (Local  "x")))
                                                    (Var (Local "y"))))
                                          "x")))
      \end{verbatim}
    \end{small}
  \end{subfigure}
  \begin{subfigure}[mult]{\textwidth}
    \begin{small}
      \underline{\texttt{mult}}
      \begin{verbatim}
Lam "a" (Lam "y" (Case (Var (Local "a"))
                       "b"
                       (Alt (DataAlt "Z") (Var (Constructor "Z")))
                       (Alt (DataAlt "S") (App (App (Var (Global "plus"))
                                                    (Var (Local  "y")))
                                               (App (App (Var (Global "mult"))
                                                         (Var (Local  "x")))
                                                    (Var (Local  "y"))))
                                          "x")))
      \end{verbatim}
    \end{small}
  \end{subfigure}
  \begin{subfigure}[odd]{\textwidth}
    \begin{small}
      \underline{\texttt{odd}}
      \begin{verbatim}
Lam "a" (Case (Var (Local "a"))
              "b"
              (Alt (DataAlt "Z") (Var (Constructor "False")))
              (Alt (DataAlt "S") (App (Var (Global "even"))
                                      (Var (Local  "n")))
                                 "n"))
      \end{verbatim}
    \end{small}
  \end{subfigure}
  \begin{subfigure}[even]{\textwidth}
    \begin{small}
      \underline{\texttt{even}}
      \begin{verbatim}
Lam "a" (Case (Var (Local "a"))
              "b"
              (Alt (DataAlt "Z") (Var (Constructor "True")))
              (Alt (DataAlt "S") (App (Var (Global "odd"))
                                      (Var (Local  "n")))
                                 "n"))
      \end{verbatim}
    \end{small}
  \end{subfigure}
  \caption{Translations of functions in Figure \ref{fig:haskellexample} into the
    Core syntax of Figure \ref{fig:coresyntax}. Notice the introduction of
    explicit $\lambda$ abstractions (\texttt{Lam}) and the use of \texttt{Case}
    to represent piecewise definitions. Fresh variables are chosen arbitrarily
    as \texttt{"a"}, \texttt{"b"}, etc.}
  \label{fig:coreexample}
\end{figure}

Figure \ref{fig:haskellexample} shows some simple Haskell function definitions,
along with a custom datatype for Peano numerals. The translation to our Core
syntax is routine, and shown in Figure \ref{fig:coreexample}. Although the Core
is more verbose, we can see that similar structure in the Haskell definitions
gives rise to similar structure in the Core; for example, the definitions of
\hs{odd} and \hs{even} are identical in both languages, except for the
particular identifiers used. It is this close correspondence which allows us to
analyse Core expressions in place of their more complicated Haskell source.

Note that we exclude representations for type-level entities, including datatype
definitions like that of \hs{Nat}. GHC can represent these, but in this work we
only consider reducible expressions (i.e. value-level bindings of the form
\mbox{\hs{f a b ... = ...}}).
