\section{Property Checking}
\label{sec:propertychecking}

% TODO: Automated testing
% TODO: Agile
% TODO: Unit testing
% TODO:

Although unit testing is the de facto industry standard for quality assurance in
non-critical systems, the level of confidence it provides is rather low, and
totally inadequate for many (e.g. life-) critical systems. To see why, consider
the following Haskell function, along with some unit
tests\footnote{Haskell functions can be prefix or infix: prefix functions have
  alphanumeric names, like \hs{div} and \hs{factorial}, and appear in function
  calls before their arguments, e.g. \hs{div 10 5}; infix functions have
  non-alphanumeric names, like \hs{+} and \hs{==}, and appear in function calls
  between their first two arguments, e.g. \hs{4 + 3}. Prefix functions can
  appear infix using backticks, like \hs{10 `div` 5}, and infix functions can
  appear prefix using parentheses, like \hs{(+) 4 3}.}:

% BEGIN unit
\begin{haskell}
factorial 0 = 1
factorial n = n * factorial (n - 1)

fact_base      = factorial 0 == factorial 1
fact_increases = factorial 3 <= factorial 4
fact_div       = factorial 4 == factorial 5 `div` 5
\end{haskell}
% END unit

The intent of the function is to map an input $n$ to an output $n!$. The tests
check a few properties of the implementation, including the base case, that the
function is monotonically increasing, and a relationship between adjacent
outputs. However, these tests will \emph{not} expose a serious problem with the
implementation: it diverges on half of its possible inputs.

Haskell's built-in numeric types allow negative numbers, which this
implementation doesn't take into account. Whilst this is a rather trivial
example, it highlights a common problem: unit tests are insufficient to expose
incorrect assumptions. In this case, our assumption that numbers are positive
has caused a bug in the implementation \emph{and} limited the tests we've
written.

One way to avoid this problem, if we do manage to spot it, would make results
symmetric around zero, which can be checked by a \emph{regression test} such as:

% BEGIN neg
\begin{haskell}
fact_neg = factorial 1 == factorial (-1)
\end{haskell}
% END neg

The implementation can be modified by composing if with the \hs{abs}olute value
function:

% BEGIN abs
\begin{haskell}
factorial = original . abs
  where original 0 = 1
        original n = n * naive (n - 1)
\end{haskell}
% END abs

However, this is \emph{still} not enough, since this function will also accept
fractional values\footnote{Since we only use generic numeric operations, the
  function will be polymorphic with a type of the form
  \hs{forall t. Num t => t -> t}, where \hs{Num t} constrains the type variable
  \hs{t} to be numeric.}, which will also cause it to diverge. This problem can
be fixed by restricting \hs{factorial}'s type with an explicit annotation, e.g.
to only whole numbers: \hs{factorial :: forall t. Integral t => t -> t}

Clearly, by choosing which scenarios to check, we are biasing the test suite
towards those cases we've already thought of, whilst neglecting the problems we
did not expect. Haskell offers a partial solution to this problem in the form of \emph{property checking}. Tools such as \quickcheck{} separate tests into three components: a \emph{property} to check, which unlike a unit test may contain
\emph{free variables}; a source of values to instantiate these free variables; and a
stopping criterion.

\subsection{\quickcheck{}}
\label{sec:quickcheck}

\quickcheck{}~\cite{claessen2011quickcheck} is the most widely used property
checking library for
Haskell~\footnote{According to downloads counted at
  \url{http://hackage.haskell.org/packages/browse}, and packages depending on
  it at \url{https://packdeps.haskellers.com/reverse/QuickCheck}, accessed on
  2019-05-21.}. Here is how we might restate our unit tests as \quickcheck{}
properties:

% BEGIN quickchecked
\begin{haskell}
fact_base        = factorial 0 == factorial 1
fact_increases n = factorial n <= factorial (n + 1)
fact_div       n = factorial n == factorial (n + 1) `div` (n + 1)
fact_neg       n = factorial n == factorial (-n)
\end{haskell}
% END quickchecked

The free variables (all called \hs{n} in this case) are abstracted as function
parameters; these parameters are implicitly \emph{universally quantified},
i.e. we've gone from a unit test asserting $factorial(3) \leq factorial(4)$ to a
property asserting $\forall n, factorial(n) \leq factorial(n+1)$. Notice that
unit tests like \hs{fact_base} are valid properties; they just happen to assert
rather weak statements, since they contain no free variables.

To check these properties, \quickcheck{} treats closed terms (like \hs{fact_base})
just like unit tests: pass if they evaluate to \hs{True}, fail otherwise. For
open terms, a random selection of values are generated and passed in via the
function parameter; the results are then treated in the same way as closed
terms. The default stopping criterion for \quickcheck{} (for each test) is when a
single generated test fails, or when 100 generated tests pass.

The ability to state \emph{universal} properties in this way avoids some of the
bias we encountered with unit tests. In the \hs{factorial} example, this
manifests in two ways:

\begin{itemize}
\item We are forced to confront the generality of \hs{factorial}'s type, since
  \quickcheck{} cannot test polymorphic functions directly; they must be
  \emph{monomorphised} first (instantiated to a particular concrete type), to
  decide which data generator to use (i.e. which instance of \quickcheck{}'s
  \hs{Arbitrary} type class). This can be done either by specifying a generator
  directly (avoiding the type class mechanism), or by adding explicit type
  annotations, or by calling \quickcheck{}'s \hs{monomorphic}
  function.\footnote{\hs{monomorphic} picks \hs{Integer} for all polymorphic
    type variables. This na\"ive choice instantiates many common type classes
    like \hs{Num}, \hs{Show} and \hs{Eq}; variables whose constraints are not
    satisfied will fail to type check (e.g. \hs{Bounded}, since \hs{Integer} is
    a bignum).} This will bring the problem of fractional values. Instead of
  monomorphising the properties, we might choose to monomorphise \hs{factorial}

\item By default, \quickcheck{} picks a generator based on the \emph{type} of
  value to be generated: since \hs{Int} includes positive and negative values,
  the \hs{Int} generator will output both. This will expose the problem with
  negative numbers, which we weren't expecting.
\end{itemize}

Whilst property checking generalises and improves upon unit testing, the problem
of tests being biased towards expected cases remains, since the properties to be
checked are still manually specified. Property checking can be complemented by
\emph{theory exploration} (TE), which avoids this bias by \emph{discovering}
such properties; through a combination of brute-force enumeration, random
testing and (in the case of \hipspec{} and \hipster{}) automated theorem
proving. Property checkers like \quickcheck{} are an important component of TE
systems, since their data generators can the search process, and they can
prevent ``obvious'' falsehoods from being output by checking them first.
