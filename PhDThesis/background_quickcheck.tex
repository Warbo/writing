\section{QuickCheck}
\label{sec:quickcheck}

Although unit testing is the de facto industry standard for quality assurance in
non-critical systems, the level of confidence it provides is rather low, and
totally inadequate for many (e.g. life-) critical systems. To see why, consider
the following Haskell function, along with some unit
tests\footnote{Backticks allow functions to be written as infix operators, so
  \hs{x `div` y} is equivalent to \hs{div x y}.}:

\begin{haskell}
factorial 0 = 1
factorial n = n * factorial (n-1)

fact_base      = factorial 0 == factorial 1
fact_increases = factorial 3 <= factorial 4
fact_div       = factorial 4 == factorial 5 `div` 5
\end{haskell}

The intent of the function is to map an input $n$ to an output $n!$. The tests
check a few properties of the implementation, including the base case, that the
function is monotonically increasing, and a relationship between adjacent
outputs. However, these tests will \emph{not} expose a serious problem with the
implementation: it diverges on half of its possible inputs!

All of Haskell's built-in numeric types allow negative numbers, which this
implementation doesn't take into account. Whilst this is a rather trivial
example, it highlights a common problem: unit tests are insufficient to expose
incorrect assumptions. In this case, our assumption that numbers are positive
has caused a bug in the implementation \emph{and} limited the tests we've
written.

If we do manage to spot this error, we might capture it in a \emph{regression
  test} and update the definition of \hs{factorial} to handle negative numbers,
e.g. by taking their absolute value:

\begin{haskell}
factorial 0 = 1
factorial n = let nPos = abs n
               in nPos * factorial (nPos - 1)

fact_neg = factorial 1 == factorial (-1)
\end{haskell}

However, this is \emph{still} not enough, since this function will also accept
fractional values\footnote{Since we only use generic numeric operations, the
  function will be polymorphic with a type of the form \hs{forall t. Num t => t
    -> t}, where \hs{Num t} constrains the type variable \hs{t} to be
  numeric. In fact, Haskell will infer extra constraints such as \hs{Eq t} since
  we have used \hs{==} in the unit tests.}, which will also cause it to
diverge. Clearly, by choosing what to test we are biasing the test suite towards
those cases we've already taken into account, whilst neglecting the problems we
did not expect.

Haskell offers a partial solution to this problem in the form of \emph{property
  checking}. Tools such as \qcheck{} separate tests into three components: a
\emph{property} to check, which unlike a unit test may contain \emph{free
  variables}; a source of values to instantiate these free variables; and a
stopping criterion. Here is how we might restate our unit tests as properties:

\begin{haskell}
fact_base        = factorial 0 == factorial 1
fact_increases n = factorial n <= factorial (n+1)
fact_div       n = factorial n == factorial (n+1) `div` (n+1)
fact_neg       n = factorial n == factorial (-n)
\end{haskell}

The free variables (all called \hs{n} in this case) are abstracted as function
parameters; these parameters are implicitly \emph{universally quantified},
i.e. we've gone from a unit test asserting $factorial(3) \leq factorial(4)$ to a
property asserting $\forall n, factorial(n) \leq factorial(n+1)$. Notice that
unit tests like \hs{fact_base} are valid properties; they just assert rather
weak statements.

To check these properties, \qcheck{} treats closed terms (like \hs{fact_base})
just like unit tests: pass if they evaluate to \hs{True}, fail otherwise. For
open terms, a random selection of values are generated and passed in via the
function parameter; the results are then treated in the same way as closed
terms. The default stopping criterion for \qcheck{} (for each test) is when a
single generated test fails, or when 100 generated tests pass.

The ability to state \emph{universal} properties in this way avoids some of the
bias we encountered with unit tests. In the \hs{factorial} example, this
manifests in two ways:

\begin{itemize}
\item \qcheck{} cannot test polymorphic functions; they must be
  \emph{monomorphised} first (instantiated to a particular concrete type). This
  is a technical limitation, since \qcheck{} must know which type of values to
  generate, but in our example it would bring the issue with fractional values
  to our attention.

\item The generators used by \qcheck{} depend only on the \emph{type} of value
  they are generating: since \hs{Int} includes positive and negative values, the
  \hs{Int} generator will output both. This will expose the problem with
  negative numbers, which we weren't expecting.
\end{itemize}

Property checking is certainly an improvement over unit testing, but the problem
of tests being biased towards expected cases remains, since we are manually
specifying the properties to be checked.

We can reduce this bias further through the use of \emph{theory exploration}
tools, such as \qspec{} and \hspec{}. These programs \emph{discover} properties
of a ``theory'' (e.g. a library), through a combination of brute-force
enumeration, random testing and (in the case of \hspec{}) automated theorem
proving.
