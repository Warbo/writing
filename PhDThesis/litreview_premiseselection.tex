\section{Relevance Filtering}
\label{sec:relevance}

\emph{Relevance filtering} takes a specified ``target'' (such as the current
goal of a theorem prover), a large list of inputs (e.g. existing lemmas from a
theory) and produces a sub-set of those inputs which are predicted to be useful
or relevant to the target. The example of choosing a sub-set of lemmas to help
prove a theorem is known as \emph{premise
  selection}~\cite{kuhlwein2012overview}.

Relevance filtering simplifies such proof search by focusing on only those
clauses (axioms, definitions, lemmas, etc.) which are deemed relevant, shrinking
the search space. The technique is used in \textsc{Sledgehammer} during its
translation of Isabelle/HOL theories to statements in first order logic: rather
than translating the entire theory, only a sub-set of relevant clauses are
included. This speeds up the proof search, but it creates the new problem of
determining when a clause is relevant: how do we know what will be required,
before we have the proof?

The initial approach used in \textsc{Sledgehammer}, known as \textsc{MePO} (from
\emph{Meng-Paulson} \cite{meng2009lightweight}), gives each clause a score based
on the proportion $m / n$ of its symbols which are ``relevant'' (where $n$ is
the number of symbols in the clause and $m$ is the number which are relevant).
Initially, the relevant symbols are those which occur in the goal, but whenever
a clause is found which scores more than a particular threshold, all of its
symbols are then also considered relevant. There are other heuristics applied
too, such as increasing the score of user-provided facts (given by keywords
like \texttt{using}), locally-scoped facts, first-order facts and
rarely-occuring facts. To choose $r$ relevant clauses for an ATP invocation, we
simply order the clauses by decreasing score and take the first $r$ of them.

Recently, a variety of alternative algorithms have also been
investigated~\cite{kuhlwein2013mash,hoder2011sine,urban2013blistr,gauthier2015premise,alama2014premise,carlson1999snow,kaliszyk2014machinelearner,kaliszyk2015femalecop}.
%                  MaSH             SInE,         BliStr          HOLyHammer          MOR,             SNoW,           MaLARea
Each combines an ``off the shelf'' machine learning algorithm (e.g. a
naive-Bayes classifier, support vector machines, etc.) with a custom feature
extraction approach; is trained using a corpus of existing theorems and their
proofs; and evaluated by trying to \emph{re-prove} a selection of unseen
theorems from the corpus (i.e. given only the theorem statement and possible
lemmas, not the existing proof).

For example the ``Machine Learning for SledgeHammer'' (\textsc{MaSH}) system
defines a notion of ``visibility''~\cite{kuhlwein2013mash} for determining how
relevant one statement is to another. Visibility is essentially a dependency
graph of which theorems were used in the proofs of which other theorems
(represented as abstract sets of features, rather than concrete sentences or
syntax trees). To select relevant clauses for a goal, the set of clauses which
are ``visible'' from the goal's components is generated; and this is filtered
further by (an efficient approximation of) a naive-Bayes algorithm.

Premise selection has also been applied to the re-proving problem in other
systems, such as the Multi-Output Ranking (\textsc{MOR}) tool, which uses a
support vector machine (SVM) to select relevant axioms from the Mizar
Mathematical Library (MML) for use by the Vampire ATP
system~\cite{alama2014premise}. SVM requires a ``kernel function'', effectively
a similarity measure between formulas. The \textsc{MOR} tool measures similarity
based on sub-terms which two formulas have in common: first a list of all $m$
terms appearing in the input space is extracted, and each formula is converted
to an $m$-dimensional binary vector, with the $n$th element of the vector
indicating whether the $n$th term of the input space occurs in this formula. The
kernel used by \textsc{MOR} is a Gaussian function of the dot-products of
these binary vectors (parameterised by a hyperparameter $\sigma$, chosen via
grid search).

The premise selection problem for theorem proving is related to the signature
selection problem for theory exploration, in that both are trying to pick a
small number of terms (depending on the speed of the subsequent algorithm and
how long the user is willing to wait) from a large set, to optimise the outcome
of that algorithm (e.g. finding a proof quickly; or finding many interesting
relationships quickly). The major difference is that premise selection has a
particular ``target'' theorem statement, against which to compare each
term. Indeed, machine learning approaches like \textsc{MOR} learn a separate
ranking function for each target formula.

This is not the case for theory exploration, which must consider the
interactions of each term with all of the others when deciding what to select.
