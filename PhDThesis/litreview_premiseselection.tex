\subsection{Relevance Filtering}
\label{sec:relevance}

One fruitful application of machine learning methods to formal mathematics is
\emph{relevance filtering}, which is used for \emph{premise selection} in
automated theorem provers~\cite{kuhlwein2012overview}.

Relevance filtering simplifies a proof search problem by removing from
consideration those clauses (axioms, definitions, lemmas, etc.) which are deemed
\emph{irrelevant}. The technique is used in Sledgehammer during its translation
of Isabelle/HOL theories to statements in first order logic: rather than
translating the entire theory, only a sub-set of relevant clauses are
included. This reduces the size of the problem and speeds up the proof search,
but it creates the new problem of determining when a clause is relevant: how do
we know what will be required, before we have the proof?

Sledgehammer's initial approach, known as \textsc{MePO} (from
\emph{Meng-Paulson} \cite{meng2009lightweight}), gives each clause a score based
on the proportion $m / n$ of its symbols which are ``relevant'' (where $n$ is
the number of symbols in the clause and $m$ is the number which are relevant).
Initially, the relevant symbols are those which occur in the goal, but whenever
a clause is found which scores more than a particular threshold, all of its
symbols are then also considered relevant. There are other heuristics applied
too, such as increasing the score of user-provided facts (e.g. given by keywords
like \texttt{using}), locally-scoped facts, first-order facts and
rarely-occuring facts. To choose $r$ relevant clauses for an ATP invocation, we
simply order the clauses by decreasing score and take the first $r$ of them.

Recently, a variety of alternative algorithms have also been investigated,
dubbed \textsc{MaSH}, \textsc{SInE}, \textsc{BliStr}, \textsc{HOLyHammer},
\textsc{MoMM}, \textsc{SNoW}, \textsc{MPTP 0.2}, \textsc{MaLARea} and
\textsc{MaLARea SG1}. Each one combines an ``off the shelf'' machine learning
algorithm with a custom feature extraction approach, is trained using a corpus
of existing theorems and their proofs, and evaluated by trying to re-prove a
selection of unseen theorems from the corpus (i.e. given only the theorem
statement and possible lemmas, not the existing proof).

For example the Machine Learning for SledgeHammer (\textsc{MaSH}) system
defines a notion of ``visibility''~\cite{kuhlwein2013mash}. This is essentially
a dependency graph of which theorems were used in the proofs of which other
theorems (theorems are represented as abstract sets of features, rather than
e.g. sentences or syntax trees). To select relevant clauses for a goal, the
set of clauses which are ``visible'' from the goal's components is generated;
this is further reduced by (an efficient approximation of) a naive Bayes
algorithm.

Premise selection has also been applied to the re-proving problem in other
systems, such as the Multi-Output Ranking (\textsc{MOR}) tool, which uses a
support vector machine (SVM) to select relevant axioms from the Mizar
Mathematical Library for use by the Vampire ATP system~\cite{alama2014premise}.
%TODO: describe the kernel, as that's the interesting bit.

The premise selection problem for theorem proving is related to the signature
selection problem for theory exploration, in that both are trying to pick a
small number of terms (depending on the speed of the subsequent algorithm and
how long the user is willing to wait) from a large set, to optimise the
outcome of that algorithm (e.g. finding a proof quickly; or finding many
interesting relationships quickly). The major difference is that premise
selection has a particular ``target'' theorem statement, against which to
compare each term (e.g. for the ``relevance'' of their sub-terms). This is not
the case for theory exploration, so we must consider the interactions of each
term with all of the others when deciding what to select.
