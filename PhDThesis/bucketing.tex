\chapter{The Signature Selection Problem}
\label{sec:signatureselection}

Despite the clever search strategies employed by theory exploration tools, they
all rely on enumerating combinations of definitions in some way, which can cause
infeasible running times on larger inputs (based on experiments with our Theory
Exploration Benchmark,%in \S~\ref{sec:benchmark}
this happens above a few dozen
definitions). To avoid this, users of these systems must carefully select only
small subsets of their definitions to explore at a time.

We dub such cherry-picking the \emph{signature selection problem} for theory
exploration, and we analyse its effect on the running time of theory exploration
tools on the Theory Exploration Benchmark, and on the quality of the statements
they are able to generate. An automated solution to this signature selection
problem would accept a large number of definitions and efficiently select
sub-sets which are both small enough to reasonably explore, whilst still
enabling interesting properties to be discovered.

In this chapter we will:

\begin{enumerate}
  \item Define the \emph{signature selection problem} for theory exploration.
  \item Analyse system performance on the Theory Exploration Benchmark, to
    demonstrate the effect of ``toxic'' definitions leading to timeouts.
  \item Analyse the effect of signature selection on the quantity of desirable
    properties attainable by theory exploration on this problem set.
  \item Demonstrate how the signature selection problem is distinct from the
    well-known \emph{clustering problem} in machine learning, and how clustering
    algorithms cannot be directly applied to signature selection.
\end{enumerate}

\iffalse
% TODO: Split out
Some bespoke components required for these analyses and comparisons may also be
useful on their own, or re-purposed for other tasks. These include:

\begin{enumerate}
\item A novel feature extraction method for transforming Haskell expressions
  into a form amenable to off-the-shelf machine-learning algorithms, based on
  the existing \emph{recurrent clustering} algorithm from other languages.
\item An executable implementation of this feature extraction
  approach\footnote{https://github.com/warbo/ml4hsfe}.
\item End-to-end automation for exploration arbitrary, user-provided Haskell
  packages\footnote{https://github.com/warbo/haskell-te}.
\item Dynamic evaluation of Haskell code, with access to dynamically installed
  Haskell packages\footnote{\hPackage{nix-eval}}.
\item Translation of the signature selection problem to the constrain-solving domain, and
  executable oracles for optimal (and pessimal) signature selection, when the desired
  properties are already known\footnote{Part of
    https://github.com/warbo/bucketing-algorithms}.
\end{enumerate}
\fi

\section{The Signature Selection Problem}
\label{sec:sigselect}

Although complete, the enumeration approaches used by tools like \quickspec{}
are wasteful: many terms are unlikely to appear in theorems, which requires
careful choice by the user of what to include in the signature. For example, we
know that addition and multiplication are closely related, and hence obey many
algebraic laws; it would be prudent to explore these functions together. On the
other hand, functions such as HTTP parsers and spell-checkers will not be
related in many interesting ways, so exploring their combinations is wasteful.

The signature selection problem is that of choosing sub-sets of a large
signature, such that:

\begin{itemize}
\item Few subsets are selected
\item Each sub-set can be explored quickly
\item Many of the interesting properties of the original signature can be
  found in the union of properties of the subsets
\end{itemize}

There is a clear trade-off to be made between speed (the first two requirements)
and ``quality'' (the final requirement): we can find all of the interesting
properties of a signature if we explore the whole thing at once; whilst it
is fastest to explore no definitions at all. In between are opportunities for a
Pareto-optimal balance to be struck. Whilst the number of subsets can be
determined simply by counting, the others require more thorough investigation.

\section{Exploration Speed}

The time taken to explore a subset of definitions from a signature can be easily
measured, but only in hindsight. Choosing each subset must be done with only a
\emph{prediction} of how long it will take to explore. Here we investigate the
time taken by \quickspec{} on signatures chosen from our Theory Exploration
Benchmark (containing definitions from Tons of Inductive Problems (TIP)
benchmark~\cite{claessen2015tip}), and how this is affected by their content.

\begin{figure}
  \centering
  \scalebox{0.45}{\input{images/steppedall.pgf}}
  \scalebox{0.45}{\input{images/steppednontoxic.pgf}}
  \caption{Kaplan-Meier survival plot for running \quickspec{} on inputs
    containing various numbers of definitions, sampled from TIP. The x axis
    denotes running time, which we cut short after 300 seconds. The height of
    each line shows the proportion of \quickspec{} runs which were still going at
    that time (lower is better). First plot is for all TIP definitions, second
    plot removes runs given ``toxic'' definitions.}
  \label{fig:survival}
\end{figure}

Figure~\ref{fig:survival} shows a Kaplan-Meier survival plot of \quickspec{}
running times, when given inputs containing different numbers of definitions. Many runs
finish quickly, with the remainder occupying a ``long tail'' which we cut off
after 300 seconds\footnote{Chosen based on preliminary experiments, which showed
  little difference in survival between 300 seconds and 1 hour}.

\begin{figure}
  \scalebox{0.45}{\input{images/timeoutsall.pgf}}
  \scalebox{0.45}{\input{images/timeoutsnontoxic.pgf}}
  \caption{Proportion of samples which timed out per size, with least-squares
    linear regression. First plot is for all TIP definitions, second removes
    runs given ``toxic'' definitions.}
  \label{fig:tailsize}
\end{figure}

Figure~\ref{fig:tailsize} shows that the number of definitions in the input is
linearly correlated with the proportion of timeouts, except for the very
smallest inputs (which are more constrained by the sampling procedure). One
hypothesis to explain this is the existence of ``toxic'' definitions, whose
presence in the input always leads to the exploration failing (with larger
inputs being more likely to have sampled a toxic definition).

\begin{figure}
  \scalebox{0.45}{\input{images/proportionsall.pgf}}
  \scalebox{0.45}{\input{images/proportionsnontoxic.pgf}}
  \caption{Definitions, ordered by the ratio of successes to failures of the
    runs they appeared in. First graph contains all TIP definitions, showing
    ``toxic'' definitions which always failed. Second graph only contains runs
    without any toxic definitions.}
  \label{fig:proportions}
\end{figure}

\begin{figure}
  \begin{minted}{scheme}
    (define-fun-rec mult2 ((x Nat) (y Nat) (z Nat)) Nat
      (match x
        (case Z      z)
        (case (S x2) (mult2 x2 y (plus y z)))))
  \end{minted}

  \begin{minted}{scheme}
    (define-fun-rec qexp ((x Nat) (y Nat) (z Nat)) Nat
      (match y
        (case Z     z)
        (case (S n) (qexp x n (mult x z)))))
  \end{minted}

  \begin{minted}{scheme}
    (define-fun-rec op ((x Nat) (y Nat) (z Nat) (x2 Nat)) Nat
      (match x
        (case Z
          (match z
            (case Z      x2)
            (case (S x3) (op Z  y x3 (S x2)))))
        (case (S x4)
          (match z
            (case Z      (op x4 y y  x2))
            (case (S c ) (op x  y c  (S x2)))))))
  \end{minted}

  \iffalse
  \begin{minted}{scheme}
    (define-fun-rec mul3acc ((x Nat) (y Nat) (z Nat)) Nat
      (match x
        (case Z Z)                          ;; Base case for 0 * y * z
        (case (S x2)
          (match y
            (case Z Z)                      ;; Base case for x * 0 * z
            (case (S x3)
              (match z
                (case Z Z)                  ;; Base case for x * y * 0
                (case (S x4)
                  (match x2
                    (case Z
                      (match x3
                        (case Z
                          (match x4
                            (case Z (S Z))  ;; Base case for 1 * 1 * 1
                            (case (S x5)
                              (S (add3acc (mul3acc Z Z x4)
                                          (add3acc (mul3acc (S Z) Z x4)
                                                   (mul3acc Z (S Z) x4)
                                                   (mul3acc Z Z (S Z)))
                                          (add3acc Z Z x4))))))
                        (case (S x6)
                          (S (add3acc (mul3acc Z x3 x4)
                                      (add3acc (mul3acc (S Z) x3 x4)
                                               (mul3acc Z (S Z) x4)
                                               (mul3acc Z x3 (S Z)))
                                      (add3acc Z x3 x4))))))
                    (case (S x7)
                      (S (add3acc (mul3acc x2 x3 x4)
                                  (add3acc (mul3acc (S Z) x3 x4)
                                           (mul3acc x2 (S Z) x4)
                                           (mul3acc x2 x3 (S Z)))
                                  (add3acc x2 x3 x4))))))))))))
  \end{minted}
  \fi

  \iffalse
  \begin{minted}{scheme}
    (define-fun-rec mul3 ((x Nat) (y Nat) (z Nat)) Nat
      (match x
        (case Z Z)                          ;; Base case for 0 * y * z
        (case (S x2)
          (match y
            (case Z Z)                      ;; Base case for x * 0 * z
            (case (S x3)
              (match z
                (case Z Z)                  ;; Base case for x * y * 0
                (case (S x4)
                  (match x2
                    (case Z
                      (match x3
                        (case Z
                          (match x4
                            (case Z (S Z))  ;; Base case for 1 * 1 * 1
                            (case (S x5)
                              (S (add3 (mul3 Z Z x4)
                                       (add3 (mul3 (S Z) Z x4)
                                             (mul3 Z (S Z) x4)
                                             (mul3 Z Z (S Z)))
                                       (add3 Z Z x4))))))
                        (case (S x6)
                          (S (add3 (mul3 Z x3 x4)
                                   (add3 (mul3 (S Z) x3 x4)
                                         (mul3 Z (S Z) x4)
                                         (mul3 Z x3 (S Z)))
                                   (add3 Z x3 x4))))))
                    (case (S x7)
                      (S (add3 (mul3 x2 x3 x4)
                               (add3 (mul3 (S Z) x3 x4)
                                     (mul3 x2 (S Z) x4)
                                     (mul3 x2 x3 (S Z)))
                               (add3 x2 x3 x4))))))))))))
  \end{minted}
  \fi
  \caption{``Toxic'' definitions, which consistently cause \quickspec{} to fail. Two
    other definitions (\texttt{mul3} and \texttt{mul3acc}) are ommitted due to
    their verbosity.}
  \label{fig:faildefs}
\end{figure}

This does appear to be the case, as Figure~\ref{fig:proportions} shows that five
definitions appeared \emph{only} in failing inputs; these are named
\texttt{mul3}, \texttt{mul3acc}, \texttt{mult2}, \texttt{op} and \texttt{qexp};
their definitions appear in Figure~\ref{fig:faildefs}. All of these are
functions of Peano-encoded natural numbers (\texttt{Nat}), and they cause
exploration to time out by either exhausting the RAM with too many expressions
to explore, % TODO{2019-04-27} CAN QSPEC2 AVOID THIS?
or by generating such deeply-nested outputs that comparing them takes too long.
% TODO{2019-04-27} CAN WE PUT TIME AND MEMORY LIMITS ON EACH ATTEMPTED EVALUATION?

The \texttt{mul3} and \texttt{mul3acc} definitions are rather pathological
implementations of multiplication with an accumulator parameter, with many
(non-tail) recursive calls. The \texttt{op} function appears in files named
\texttt{weird\_nat\_op}, which assert its commutativity and associativity.
Finally, the \texttt{mult2} and \texttt{qexp} functions are standard
tail-recursive definitions of multiplication and exponentiation, respectively.
All of these functions have an extra ``accumulator'' argument, which increases
the number of possible expressions to explore compared to those without.

Note that Haskell is lazily evaluated, which allows large structures like Peano
numerals to be compared using little memory: the \texttt{S} constructors are
successively unwrapped from each side, being calculated on-demand and
immediately garbage collected in a tight loop. However, this still takes a lot
of CPU time and hence causes timeouts. We confirmed this hypothesis by exploring
with a custom data generator which only generates the values \texttt{Z},
\texttt{S Z} and \texttt{S (S Z)} (0, 1 and 2), which caused the exploration to
finish quickly in these cases. Other interventions, like making the accumulator
arguments strict (to prevent space leaks), did not prevent timeouts.

To assess the impact of these toxic definitions, we removed any samples
containing them and repeated our analysis: the results are shown on the right
hand side of the previous figures. Whilst timeouts remain, they are
substantially reduced, and completely eliminated for the smallest sample sizes.
Some definitions never appeared in a failing signature.

Our analysis of signature contents on theory exploration performance leads us to
two recommendations for signature selection methods:

\begin{itemize}
\item Avoid ``toxic'' definitions: those producing exponentially large outputs,
  and/or taking many arguments (leading to a large number of combinations). In
  particular, exposing the accumulator argument of tail-recursive functions can
  make them harder to explore.
\item Smaller sets are faster to explore than larger sets, but the
\end{itemize}

It also seems prudent for theory exploration tools to limit time and memory
usage at the level of individual evaluations: these can abort evaluation of
toxic definitions, whilst still being high enough to allow the majority of
non-toxic definitions to be unhindered.

\section{Maintaining Quality}

Our final criterion concerns the ``quality'' of the selected subsets. This
depends crucially on how we define whether a property is ``interesting''. We can
use the same ground truth definition as our Theory Exploration Benchmark: a
property is ``interesting'' if it appears in a particular corpus (in our case
TIP~\cite{claessen2015tip}), and ``uninteresting'' if not. Whilst na\"ive, this
approach allows the criterion to be quantified: the quality of the selected
subsets is proportion of the whole signature's interesting properties that also
apply to the subsets.

Since signature selection is largely independent of which tool we choose to
explore the resulting subsets, we do not require that such exploration actually
\emph{finds} all of the interesting properties; only that we have not
\emph{prevented} them from being found. For example, we may be given a signature
containing addition and multiplication functions and may consider commutativity
and distributivity properties to be ``interesting''. Selecting a subset
containing both would be high quality, since all of these properties are
\emph{available} to find, even if our chosen tool doesn't spot them. In
contrast, splitting addition and multiplication into \emph{separate} subsets is
lower quality, since this prevents \emph{any} tool finding the distributivity
property (since it involves both definitions).

To understand the quality impact of limiting ourselves to such subsets, we
applied four signature selection methods to signatures sampled from the Theory
Exploration Benchmark. Signatures varied in size from a single definition up to
100 definitions and 100 signatures of each size were sampled. Each method was
run 19 times on each signature: splitting into between 1 and 20 subsets.

All four methods place each definition in exactly one subset, to focus on the
effect of splitting apart definitions. The tested methods are:

\begin{itemize}
\item \textbf{Optimal}: Uses knowledge of the ground truth to choose equal sized
  subsets with the highest quality.
\item \textbf{Pessimal}: Like optimal but chooses the lowest quality subsets.
\item \textbf{Pseudorandom}: Assigns definitions to subsets pseudorandomly, used
  as a control.
\item \textbf{Clustering}: Subsets are chosen by a recurrent clustering
  algorithm, based on their syntactic similarity.
\end{itemize}

The optimal and pessimal methods are unrealistic, since they require access to
the ground truth (which only exists for benchmark problems), but set upper and
lower bounds on the impact of separating a signature's definitions. These are
implemented via constraint solving, whose exponential running time became
infeasible for signatures containing more than 10 definitions. To avoid extreme
solutions, such as all-but-one subsets being singletons, subset cardinality
could only differ by at most one.

The pseudorandom and clustering implementations are more realistic, since they
do not use any information from the ground truth. Cardinality limits were also
not imposed for these methods.

% TODO: Insert graph of hash vs MLSpec on 1-100

% TODO: Insert graph of 1-10 with all

\iffalse
% TODO
\section{Clustering}
\label{sec:clustering}

Our approach to scaling up \quickspec{} takes inspiration from two sources. The
first is relevance filtering, which makes expensive algorithms used in theorem
proving more practical by only considering clauses deemed ``relevant'' to the
problem~\cite{meng2009lightweight}.

Relevance is determined by comparing clauses to the target theorem, but theory
exploration does not have such a distinguished term. Instead, we are interested
in relationships between \emph{all} terms in a signature, and hence we need a
different algorithm for considering the relevance of \emph{all terms} to
\emph{all other terms}.

A natural fit for this task is \emph{clustering}, which attempts to group
similar inputs together in an unsupervised way. Based on their success in
discovering relationships and patterns between expressions in Coq and ACL2 (in
the ML4PG and ACL2(ml) tools respectively), we hypothesise that clustering
methods can fulfil the role of relevance filters for theory exploration:
intelligently breaking up large signatures into smaller ones more amenable to
brute force enumeration, such that related expressions are explored together.

Due to its use by ML4PG and ACL2(ml), we use \emph{k-means} clustering,
implemented in the Weka tool \cite{Holmes.Donkin.Witten:1994} by Lloyd's
algorithm \cite{lloyd1982least}, with randomly-selected input elements as the
initial clusters. Rather than relying on the user to provide the number of
clusters $k$, we use the ``rule of thumb'' given in
\cite[pp. 365]{mardia1979multivariate} of clustering $n$ data points into
$k = \lceil \sqrt{\frac{n}{2}} \rceil$ clusters.
\fi

%\section{Experiments}

% TODO: This was taken from the Background section, so probably doesn't fit here
% without some fiddling.
% TODO: Describe all of our experimental setups:
%
%  - The use of the TEBenchmark methodology for running QSpec, including the
%    sampling process
%
%  - Running our recurrent signature selection algorithm on samples from TEBenchmark,
%    including how we picked the sizes
%
%  - The definition of HashSpec and its use as control; how we ran all of the
%    same samples through it
%
%  - The use of optimal and pessimal oracles, defined via constraint
%    satisfaction. How we it was only feasible to run these up to size 10 (11?)

%We follow the tradition of prior practitioners and use an existing corpus of
%theorems as a \emph{ground truth} against which to compare results. To analyse
%behaviour on large inputs (i.e. those for which exploration is currently
%infeasible), as required to study the effects of signature selection, we have
%used the Theory Exploration Benchmark of section~\ref{sec:tebenchmark}, which is
%based on the Tons of Inductive Problems (TIP) problem set for theorem
%provers~\cite{claessen2015tip}.
