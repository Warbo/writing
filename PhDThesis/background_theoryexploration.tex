\section{Theory Exploration}
\label{sec:theoryexploration}

(Automated) Theory Exploration (TE) is the task of taking a \emph{signature}
of definitions in some formal system (for example a programming language) and
automatically generating a set of formal statements (properties) involving
those definitions\footnote{Some tools, such as \isascheme{}, also generate their
  own definitions. This work does not focus on such functionality, treating it
  instead under the umbrella of \emph{Concept Formation}.}. These may be
conjectures or theorems (proven either by sending conjectures to an automated
theorem prover, or by having the generating procedure proceed in logically sound
steps from a set of axioms). These statements should also be ``interesting'' in
some way, to rule out unhelpful trivialities such as iterating a pattern over
and over (e.g. \hs{x + 0 = x}, \hs{(x + 0) + 0 = x}, \dots). We defer a more
thorough treatment of what counts as ``interesting'' to
\S~\ref{sec:interestingness}.

The choice of appropriate methods can differ according to the underlying logic,
etc. that is in use: in this work we ground ourselves by applying TE to
discovering properties of pure functional software libraries, mostly in the
Haskell programming language. This also gives our results immediate utility to
software engineering. We focus on the generation of interesting conjectures; to
\emph{prove} such conjectures, we defer to existing tools (such as \hip{} and
\hipspec{}).

The method of conjecture generation is a key characteristic of any theory
exploration system, although all existing implementations rely on brute force
enumeration to some degree. We focus on \quickspec{}~\cite{QuickSpec}, which
conjectures equations about Haskell code. We have used version 1 of \quickspec{}
in our experiments, due to its availability, stability and tooling integration;
hence references to \quickspec{} are to version 1 unless stated otherwise. As of
2019 there is a \quickspec{} version 2 available, which has a much improved
generation procedure which can be significantly faster than its predecessor.
Whilst \quickspec{} 2 has advanced the state of the art in theory exploration,
our preliminary experience has found that it still suffers the scaling issues we
identify in this work\footnote{Not only did we experience failures with our own
  experiments, we also suffered memory exhaustion when trying to run some of the
  examples included with the \quickspec{} 2 source code.} Since \quickspec{} 2
has less integration with the tooling we have used, we leave a more thorough
analysis of it (and related systems such as \speculate{}) for future work,
pending the necessary infrastructure changes that would require.

\begin{figure}
  \centering
  \begin{minted}{haskell}
    -- A datatype with two constructors (Note that -- introduces a comment)
    data Bool = True | False

    -- A recursive datatype (S requires a Nat as argument)
    data Nat = Z | S Nat

    -- A function turning a Bool into a Bool
    not :: Bool -> Bool
    not True  = False
    not False = True

    -- A pair of mutually recursive functions

    odd :: Nat -> Bool
    odd    Z  = False
    odd (S n) = even n

    even :: Nat -> Bool
    even    Z  = True
    even (S n) = odd n
  \end{minted}
  \caption{Haskell datatypes for booleans and natural numbers, followed by some
    simple function definitions (with type annotations).}
  \label{fig:haskellteexample}
\end{figure}

\quickspec{} is written in the Haskell programming language, and works with
definitions which are also written in Haskell. A thorough description of
Haskell, including its suitability for theory exploration, can be found in
\S~\ref{sec:haskell}. For illustrative purposes, some simple Haskell definitions
are shown in Figure~\ref{fig:haskellteexample}. \quickspec{} uses the following
procedure to generate conjectures, which are both plausible (due to testing on
many examples) and potentially interesting (due to being mutually irreducible):

\begin{enumerate}
\item Given a signature $\Sigma$ of typed expressions and set of variables $V$,
  \quickspec{} generates a list $terms$ containing the expressions from
  $\Sigma$ (including functions), the variables from $V$ and type-correct
  function applications \hs{f x}, where \hs{f} and \hs{x} are elements of
  $terms$. To ensure the list is finite, function applications are only nested
  up to a specified depth (by default, 3).
\item The elements of $terms$ are grouped into equivalence classes, based on
  their type.
\item The equivalence of terms in each class is tested using \quickcheck{}:
  variables are instantiated to particular values, generated randomly, and the
  resulting closed expressions are evaluated and compared for equality.
\item If a class is found to have non-equal members, it is split up to separate
  those members.
\item The previous steps of variable instantiation and comparison are repeated
  until the classes stabilise (i.e. no differences have been observed for some
  specified number of repetitions).
\item For each class, one member is selected and equations are conjectured that
  it is equal to each of the other members.
\item A congruence closure algorithm is applied to these equations, to discard
  any which are implied by the others.
\end{enumerate}

Such conjectures can be used in several ways: they can be presented directly
to the user, sent to a more rigorous system like \hipspec{} or \hipster{} for
proving, or even serve as a background theory for an automated theorem
prover~\cite{claessen2013automating}.

As an example, we can consider a simple signature containing the expressions
from Figure~\ref{fig:haskellteexample}, and some suitable variables:

\begin{align*}
  \Sigma_{\hs{Nat}} &= \{\hs{Z},\ \hs{S},\ \hs{plus},\ \hs{mult},\ \hs{odd},
                       \ \hs{even}\} \\
  V_{\hs{Nat}}     &= \{\hs{a :: Nat},\ \hs{b :: Nat},\ \hs{c :: Nat}\}
\end{align*}

\quickspec{}'s enumeration of these terms will resemble the following:

\begin{align*}
  terms_{\hs{Nat}} = [& \hs{Z},\ \hs{S},\ \hs{plus},\ \hs{mult},\ \hs{odd},
                     \ \hs{even},\ \hs{a},\ \hs{b},\ \hs{c},\ \hs{S Z},
                     \ \hs{S a},\ \hs{S b}, \\
                     & \hs{S c},\ \hs{plus Z},\ \hs{plus a},\ \dots ]
\end{align*}

Notice that functions such as \hs{plus} and \hs{mult} are valid terms, despite
not being applied to any arguments. In addition, all Haskell functions are unary
(due to currying), which makes it valid to apply them one argument at a time as
we construct $terms_{\hs{Nat}}$.

\begin{figure}
  % To reproduce, run 'quickSpec nat' in haskell_example/src/QuickSpecExample.hs
  \begin{haskell}
                      plus a b = plus b a
                      plus a Z = a
             plus a (plus b c) = plus b (plus a c)
                      mult a b = mult b a
                      mult a Z = Z
             mult a (mult b c) = mult b (mult a c)
                  plus a (S b) = S (plus a b)
                  mult a (S b) = plus a (mult a b)
             mult a (plus b b) = mult b (plus a a)
                     odd (S a) = even a
                odd (plus a a) = odd Z
               odd (times a a) = odd a
                    even (S a) = odd a
               even (plus a a) = even Z
              even (times a a) = even a
    plus (mult a b) (mult a c) = mult a (plus b c)
  \end{haskell}
  \caption{Equations conjectured by \quickspec{} for the functions in Figure
    \ref{fig:haskellteexample}; after simplification.}
  \label{fig:qspecresult}
\end{figure}

The elements of $terms_{\hs{Nat}}$ will be grouped into five equivalence classes,
one each for \hs{Nat}, \hs{Nat -> Nat}, \hs{Nat -> Nat -> Nat}, \hs{Nat -> Bool}
and \hs{Bool}. As the variables \hs{a}, \hs{b} and \hs{c} are instantiated to
various randomly-generated numbers, the differences between these terms will
be discovered. For example, the \hs{Nat} class will initially contain (among
other things) the terms \hs{Z}, \hs{x} and \hs{plus x Z}; this will be split
apart once \hs{x} is tested with a non-zero value, since the latter terms will
not equal \hs{Z}. Since those latter terms will always be equal, for any value
of \hs{x}, they will never be split apart. Eventually (after removing
redundancies) the equations in Figure~\ref{fig:qspecresult} are conjectured.
These may be output to the user for perusal, or for another system to process
like the \hipspec{} automated theorem prover.

Note that we must be able to generate values of a type in order to include
variables of that type. Also, only equivalence classes whose element type can be
compared are split and conjectured about in this way\footnote{Both random data
  generators and comparison functions can be derived mechanically from algebraic
  datatype declarations; hence why they are omitted from our example. Note that
  care must be taken when generating recursive structures, as na\"ive recursion
  may produce values whose expected size exceeds the machine's memory.}; in
particular, functions (like \hs{odd} and \hs{even}) are first-class values, so
they will be put in an equivalence class, but they will not be subject to
testing and comparison since Haskell has no generic notion of function
equality~\footnote{In fact \quickspec{} compares expressions using a total
  ordering rather than equality; this is even more restrictive than requiring a
  notion of equality, but reduces the number of required comparisons}.
\quickspec{} can hence conjecture that two functions are pointwise-equal (e.g.
\hs{f x == g x}) but it cannot conjecture that the functions themselves are
equal (e.g. \hs{f == g}).

Although complete, this enumeration approach is wasteful: many terms are
unlikely to appear in theorems, which requires careful choice by the user of
what to include in the signature. Here we know that addition and multiplication
are closely related, and hence obey many algebraic laws; arbitrary definitions
from a typical software library or proof development are unlikely to be related
so strongly.

\subsection{Related Fields}

\quickspec{} (and \hipspec{}) grew out of software engineering tools like
\quickcheck{}, and they are in fact compatible with Haskell's existing testing
infrastructure, such that an invocation of \texttt{cabal test} can run these
tools alongside more traditional QA tools like \quickcheck{}, \textsc{HUnit} and
\textsc{Criterion}.

There are similarities between the way a TE system like \quickspec{} can
generalise from checking \emph{particular} properties to \emph{generating} new
ones, and the way counterexample finders like \quickcheck{} can generalise from
testing \emph{particular} expressions to generating expressions to test.

\emph{Experimental mathematics} is related to theory exploration, in that both
seek to use empirical methods to discover high-level properties of
well-specified formal systems. The theory is itself may be a scientific model,
in which case theory exploration becomes part of the scientific method; in
particular, finding ``testable predictions'' (e.g. equations relating measurable
quantities, which can be experimentally tested) is a special case of theory
exploration's search for ``interesting conjectures''.

\emph{Theory formation} and \emph{concept formation} go beyond mere exploration;
inventing new definitions, which can give rise to subsequent properties. This
gives rise to a feedback loop: concepts are ``interesting'' if they have
``interesting'' properties, whilst properties are ``interesting'' due to the
concepts they involve. This gives rise to some deep philosophical questions
about the nature of mathematics, but is so open-ended that it can be difficult
to apply to day-to-day tasks; hence why we generally avoid invention systems
in this work.
