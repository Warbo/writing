\subsection{Interestingness}

A major challenge when generating conjectures is choosing how to narrow down the
resulting set to only those deemed ``interesting'', since this is an imprecise
term with many different interpretations. For example, all existing approaches
agree that simple tautologies are ``uninteresting'', but differ when it comes to
more complex statements.  Colton \etal{} surveyed tools for theory formation
and exploration, and their associated notions of ``interestingness'' for
concepts and conjectures~\cite{colton2000notion}. Six qualities were identified,
which are applied to conjectured properties as follows:

{\bf Empirical plausibility} checks whether a property holds across some
specific examples. This is especially useful for avoiding falsehoods, without
resorting to a deductive proof search.

{\bf Novelty} depends on whether the property, or one isomorphic or more
general, has already been seen.

{\bf Surprisingness} of a property is whether or not it is ``obvious'', for
example if it is an instance of a tautology.

{\bf Applicability} depends on the number of models in which a
property holds. High applicability makes a property more interesting, but so
does zero applicability (i.e. non-existence).

{\bf Comprehensibility} depends on the syntactic complexity of the property's
statement. Simpler statements are considered more interesting (which favours
tools adopting a small-to-large search order).

{\bf Utility} is the relevance or usefulness of a property to the user's
particular task. For example, if we want to find properties which are useful for
optimisation, like Equation~\ref{eq:mapreduce}, then utility would include
whether the property justifies some rewrite rule, the difference in resource
usage of the expressions involved, and how common those expressions are in real
usage.

% FIXME: This is original work, so should be in a different section
\begin{table}
  \centering
  \begin{tabular}{ |l|l|c|c|c|c|c|c| }
    \hline
    \multirow{2}{*}{\textbf{Program}}                      &
    \multirow{2}{*}{\textbf{Conjecture Types}}             &
    \multicolumn{6}{c|}{\textbf{Interestingness Measures}} \\ \hhline{~~------}
    \tRow{             &                     & \iE & \iN & \iS & \iA & \iC & \iU}
    \tRow{AM           & \tIFF, \tIMP, \tNE  &   X &   X &   X &   X &   X &   X}
    \tRow{GT           & \tIFF, \tIMP, \tNE  &   X &   X &   X &   X &   X &   X}
    \tRow{Graffiti     & \tINE               &   X &   X &   X &     &   X &   X}
    \tRow{\Bagai{}     & \tNE                &     &   X &     &   X &   X &    }
    \tRow{HR           & \tIFF, \tIMP, \tNE  &   X &   X &   X &   X &   X &   X}
    \tRow{\quickspec{} & \tEQ                &   X &   X &     &   X &   X &    }
    \tRow{\speculate{} & \tCON\ \tEQ / \tINE &   X &   X &     &   X &   X &    }
    \tRow{\isacosy{}   & \tEQ                &   X &   X &     &   X &   X &   X}
    \tRow{\isascheme{} & \tEQ                &   X &   X &     &   X &   X &   X}
  \end{tabular}
  \caption{Classification of MTE tools from~\cite{colton2000notion}, extended
    to include four more recent tools. The interestingness measures are
    \iE{}mpirical plausibility, \iN{}ovelty, \iS{}urprisingness,
    \iA{}pplicability, \iC{}omprehensibility (low complexity) and \iU{}tility.}
  \label{table:colton}
\end{table}

% JUSTIFICATIONS
%
% ICoSy uses counter-example checking; this ensures empirical plausibility
% ICoSy uses constraints to avoid special-cases; this ensures novelty
% ICoSy uses utility, since there are some hard-coded patterns which are
% looked for
%
% Speculat uses a form of unification to ensure novelty, making sure one
% equation is not a special-case of another
% Speculat uses LeanCheck to enumerate values, looking for counterexamples
% Speculat uses testing, to ensure Empirical Plausibility
%
% QSpec uses QCheck to ensure empirical plausibility
% QSpec uses a congruence closure algorithm to ensure novelty
%
% IScheme uses utility, since it determines the "quality" of a definition
% based on how many theorems it appears in?
% IScheme uses novelty, using an equational rewrite system (Knuth-Bendix
% completion) to remove redundancies
% IScheme uses utility, since it focuses on the terms and definitions of a
% user's theory. However, don't all of them?
% IScheme uses counterexample checking for empirical plausibility.

% ONLY APPLIES TO PRECONDITIONS
% APPLICABILITY: ensure a conjecture holds in many models (or exactly zero, in
% the case of Bagai et al). Does the testing-based approach of QuickSpec and
% Speculate ensure that there's a model? After all, these are concrete values
% rather than e.g. implications of some abstract specification.

% FIXME: Give brief description and reference for each system in the text.
% We could include the more lengthy descriptions of IsaCoSy and QuickSpec here?

We summarise the criteria in Table~\ref{table:colton} and state which were used
in key ATF/MTE tools. The entries for AM, GT, Graffiti, \Bagai{} and HR are
based on this survey (the latter is Colton's own tool). We extend this analysis
to some more recent tools in the remaining rows (based on our understanding of
their function): \quickspec{}~\cite{QuickSpec},
\speculate{}~\cite{braquehais2017speculate},
\isacosy{}~\cite{Johansson.Dixon.Bundy:conjecture-generation}
and \isascheme{}~\cite{MontanoRivas2011}.

These recent tools all use testing to check for counterexamples, which ensures
results are empirically plausible and applicable (conditions are satisfiable,
types are inhabited, etc.). They also use a small-to-large search order,
ensuring more easily-comprehensibile conjectures are explored first.

\isacosy{} ensures novelty using a constraint system to prevent special-cases
being generated, whilst the others use post-hoc filters based on unification and
term rewriting to achieve the same result. \isacosy{} is able to look for
commonly desirable patterns, such as commutativity and associativity, which
increases the utility of its output, whilst \isascheme{} is completely based
around such pattern-instantiation.

This diversity of approaches makes it difficult to compare the existing
evaluations of these tools directly. In particular, evaluation methods created
for one tool might not make sense for another, due to assumptions made about
the algorithms' operation. For example, the novelty filters used by \quickspec{}
and \speculate{} are applied to the whole output set, guaranteeing that no
special-cases will appear; yet we cannot assume this in the case of \isacosy{},
since its constraint solver allows special cases if they're found \emph{before}
the general case (it does not go back and discard previous results).

In addition, attempting to \emph{measure} these qualities directly is difficult,
and having too many measures complicates comparisons. System developers have
employed a more practical alternative in their evaluations, which is to perform
\emph{precision/recall} analysis against a \emph{ground-truth}. This requires
choosing a set of definitions and a set of properties (the ground truth) which
represent the ``ideal'' result of conjecture generation for those definitions.
To analyse the quality of a tool's conjectures, we run it on these chosen
definitions and compare its output to the ground truth:

\begin{itemize}
\item \emph{Precision} is the proportion of a tool's output which appears in
  the ground truth (the ratio of true positives to all positives). This
  penalises overly-liberal tools which output a large number of properties in
  the hope that some turn out to be ``good''.
\item \emph{Recall} is the proportion of the ground truth which appears in the
  tool's output (the ratio of true positives to actual positives). This
  penalises overly-conservative tools which generate very few properties as a
  way to avoid ``bad'' ones.
\end{itemize}

To score 100\% on precision and recall, all of the properties which appear in
the ground truth must have been conjectured, and nothing else. This gives us a
simple method to evaluate and compare tools without requiring a general solution
to the question of what is ``interesting''; although we must still decide what
to put in the ground truth set, and what to leave out, for each measured set of
definitions.

The precision/recall analysis shown in~\cite{claessen2013automating} is the only
quantitative comparison of recent MTE tools we have found in the literature.
Our Theory Exploration Benchmark, described in section~\ref{sec:benchmark}, is
essentially an extension of this approach, to a larger and more diverse set of
examples.

% TODO Include Colton/Bundy et al

% TODO Include something about intrinsic motivation, compression progress,
% variance among experts (decision tree stuff as well as more established
% splitting functions stuff)

% TODO Make sure we mention the recent reinforcement learning task that includes
% a constantly-changing "TV" in the 3D world, shown in some TwoMinutePapers
% videos
