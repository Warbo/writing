\section{ML4HS Implementation}
\label{sec:implementation}

We provide an implementation of our recurrent clustering algorithm in a tool
called \mlforhs{}, which consists of a loose collection of components shown
in Figure \ref{fig:ml4hs}. This arrangement makes it easy to swap out parts for
experimentation. In the following, we describe the custom components in the
order they appear in the diagram.

\begin{figure}
  \centering
  \tikzstyle{block} = [rectangle, draw, rounded corners]
  \tikzstyle{container} = [rectangle, draw, rounded corners]
  \tikzstyle{line} = [draw, -latex']
  \colorlet{shade}{rgb:black,1;white,4}

  \begin{tikzpicture}[node distance=2cm]
    \node [block] (hackage) {\hackage{}};

    \node [block, below of=hackage] (cabal) {
      \begin{tikzpicture}[node distance = 2cm]
        \node (caballabel) {Cabal};
        \node [block, anchor=north west] at (caballabel.south) (ghc) {
          \begin{tikzpicture}[node distance = 2cm]
            \node (ghclabel) {GHC};
            \node [block, anchor=north west, fill=shade] at (ghclabel.south) (plugin) {AST Plugin};
          \end{tikzpicture}
        };
      \end{tikzpicture}
    };

    \node [block, below of=cabal, fill=shade]  (sorting) {Sorting};

    \node [block, below of=sorting, fill=shade] (clustering) {
      \begin{tikzpicture}[node distance = 2cm, auto]
        \node (clusterlabel) {Recurrent Clustering};
        \node [block, anchor=north west, fill=shade] at (clusterlabel.south west) (fe) {Feature Extraction};
        \node [block, anchor=west, fill=white] at ([xshift=2em]fe.east) (weka) {Weka};

        \path [line] (fe)   -- (weka);
        \path [line] (weka) -- (fe);
      \end{tikzpicture}
    };

    \node [block, below of=clustering, fill=shade] (mlspec) {
      \begin{tikzpicture}[node distance = 2cm, auto]
        \node (mlspeclabel) {\mlspec{}};
        \node [block, anchor=north west, fill=white] at (mlspeclabel.south) (qs) {\quickspec{}};
      \end{tikzpicture}
    };

    \node [block, below of=mlspec] (user) {User};

    \path [line] (hackage)    -- (cabal);
    \path [line] (cabal)      -- (sorting);
    \path [line] (sorting)    -- (clustering);
    \path [line] (clustering) -- (mlspec);
    \path [line] (mlspec)     -- (user);
  \end{tikzpicture}
  \caption{Components of the ML4HS theory exploration system. Custom components are shaded, arrows indicate data flow.}
  \label{fig:ml4hs}
\end{figure}

\subsection{\textsc{AST Plugin}}
\label{sec:astplugin}

The GHC compiler provides mechanisms for parsing Haskell source code and
converting it to Core. It also includes a \emph{renaming} transformation, which
resolves global identifiers into a canonical form. This allows us to spot
repeated use of a term, across multiple modules and packages, with a simple
syntactic equality check.

Since we are interested in comparing definitions based on the terms they
reference, building our framework on top of GHC seems like a promising
approach. Indeed, \hipspec{} already invokes GHC's API to obtain the definitions
of Haskell functions, in order to transform them into a form suitable for ATP
systems. However, our initial experiments showed that this technique is too
fragile for use on many real Haskell projects.

This is due to many projects having a complex module structure, requiring
particular GHC flags to be given, or using pre-processors such as \texttt{cpp}
and Template Haskell to generate parts of their code. All of this complexity
means that invoking GHC ``manually'' via its API is unlikely to obtain the
definitions we require.

Thankfully there is one implementation detail which most Haskell packages agree
on: the Cabal build system. All of the above complexities will be specified in a
package's ``Cabal file'', such that the \texttt{cabal configure} and
\texttt{cabal build} commands are very likely to work for most packages, without
any extra effort. This shifted our focus to augmenting GHC and Cabal, such that
definitions can be collected during the normal Haskell build process.

GHC provides a plugin mechanism for manipulating Core during a build, intended
for optimisation passes, which we use to inspect definitions as they are being
compiled. We provide a plugin called \textsc{AstPlugin} which emits a serialised
version of each Core definition to the console (to satisfy the type system, it
also implements a dummy ``optimisation'' which returns the Core unchanged).

Compared to Haskell, Core is a much simpler language and its representation is
relatively stable compared to many existing representations of Haskell (which
often change to support various language extensions). Three areas which make
Core difficult to handle are:

\begin{description}
\item{Type variables}: Parametric polymorphism (described in more detail in \S
  \ref{sec:haskelldesc}) can be thought of as values being parameterised by
  type-level objects. In System F, this is represented explicitly by a special
  abstraction form $\Lambda$, distinct from the $\lambda$ used for values. Core
  only has one abstraction form, \CLam, for both types and values. This alters
  function properties like arity.

\item{Unified namespace}: Haskell has distinct namespaces for values, types,
  data constructors, etc. Since Core does not make these distinctions, names may
  become ambiguous. For example, a type parameter \hs{t} may be confused with a
  function argument \hs{t}. To prevent this, overlapping namespaces are
  distinguished by prefices which are distinct from the available names; for
  example a type class constraint \hs{Ord t} may give rise to a binder
  $\CLam\ \hs{"\$dOrd"}$ in Core, which is guaranteed not to conflict since this
  name would be invalid in Haskell. This causes difficulties when looking up
  names, as these prefixed forms do not easily map back to the Haskell source.

\item{Violating encapsulation}: Although Haskell allows names to be
  \emph{private} to a module, when compiling Core we have full access to private
  definitions, as well as references to private names from within other
  definitions. Hence the definitions we receive from \textsc{AstPlugin} will
  include private values which we cannot import into a theory exploration tool.
\end{description}

In practice, we work around these issues with a post-processing stage: for each
named definition appearing in the output of \textsc{AstPlugin}, we attempt to
reference that name within the GHCi interpreter. Names with the above problems
will cause an error, and are discarded.

The result of building a Haskell package with \textsc{AstPlugin} enabled is a
database of Haskell definitions, similar in some respects to \textsc{Hoogle}
\cite{mitchell2008hoogle}. Definitions are indexed by a combination of their
package name, module name and binding name. The definitions themselves are
s-expressions representing the Core AST, with non-local references replaced by a
combination of package name, module name and binding name, which makes it
trivial to look up references in the database. Each definition also has an
associated arity and type, obtained during the post-processing step mentioned
above.

\subsection{Toplogical Sorting}

As described in \S \ref{sec:symbolstofeatures}, we must topologically sort the
output of \textsc{AstPlugin} in order for our recurrent clustering to be
well-founded. Since our database keys (containing the package, module and
binding names, as described above) match our representation of non-local
references, it is simple to walk each syntax tree to obtain the set of
references it makes. In addition, the resulting set of (identifier,
list-of-referenced-identifiers) pairs exactly matches the (vertex,
list-of-successor-vertices) format used to represent directed graphs by the
popular \hs{containers} library which ships with GHC. This provides an
implementation of topological sort for strongly connected components, which we
use as-is. A simple shell script loops through these SCCs, invoking the
recurrent clustering component for each and appending the resulting features and
clusters to the database.

\subsection{Feature Extraction}

The implementation of our feature extraction algorithm is a rather direct
translation of the description given in \S \ref{sec:contributions} into
Haskell. We parse the s-expressions generated by \textsc{AstPlugin} into
algebraic data types which correspond directly to the definitions in Figure
\ref{fig:coresyntax}; this is routine, so we omit the details for
brevity. Similarly, we can represent rose trees with a datatype corresponding to
the definition given in \S \ref{sec:expressionstovectors}:

\begin{haskell}
data RoseTree = Node Feature [RoseTree]
\end{haskell}

For simplicity we use the representation \hs{Feature = Int}, as we do not have
fractional values. Since we represent the symbols $expr$, $id$, etc. from Figure
\ref{fig:coresyntax} using different datatypes, we cannot write one big
definition of $toTree$ or $\phi$ which works on all tokens. Each case shown in
Figure \ref{fig:totree} and equations \ref{eq:feature}, \ref{eq:localfeature}
and \ref{eq:globalfeature} appears in the implementation, although they are
spread across several functions.

To support looking up local identifiers, our implementation of $toTree$ takes a
context as argument, extending it as required. As an example of the complexity
this adds, here is the \CCase\ branch of $toTree$:

\begin{haskell}
toTree :: Context -> Expr -> RoseTree
toTree ctx x = case x of
  ...
  Case e l as -> Node fCase (toTree ctx e : map (toTreeAlt (l:ctx)) as)
  ...
\end{haskell}

Breaking this down we can see \hs{fCase} representing the value of
$\feature{\CCase}$, and a list of sub-trees defined in parentheses. The first
subtree is a straightforward recursive call in an unmodified context:
\mbox{\hs{toTree ctx e}}. The rest of the list is formed by applying the
function \mbox{\hs{toTreeAlt (l:ctx)}} to each element of the list \hs{as} of
\CAlt\ clauses.

The \hs{toTreeAlt} function contains those cases of $toTree$ which handle
symbols in $alt$. We prepend the identifier \hs{l} to the context, to get the
extended context \hs{l:ctx}. This is because \hs{l} will be bound the value of
\hs{e}, in order to avoid re-computing its value several times.

The other clauses are handled in a similar way. The trickiest is the \CLet\
clause, since the local identifiers aren't directly available; we must extract
them from their \CRec, \CNonRec\ and \CBind\ wrappers first, which we do using
helper functions.

As shown above, the values from equations \ref{eq:feature} are encoded directly
in $toTree$. For $phi(l \in \mathcal{L})$ we use standard Haskell functions to
look up the required indices in the context:

\begin{haskell}
phiL :: Context -> Local -> Feature
phiL ctx x = case elemIndex x ctx of
  Nothing -> error (concat ["Local '", show x,
                            "' not in context '",
                            show ctx, "'"])
  Just i  -> (2 * alpha) + i
\end{haskell}

As explained in \S \ref{sec:symbolstofeatures}, local identifiers should always
exist in the context. If this precondition doesn't hold, we abort the program
with an error rather than continuing.

Global identifiers are kept as-is until we have access to the clusters from the
last iteration. This takes place outside Haskell, using the \hs{jq} data
processing tool.

Our implementation of $level$ exactly matches equation \ref{eq:level}:

\begin{haskell}
level :: Int -> RoseTree -> [[Feature]]
level 1 (Node f _)  = [f]
level n (Node _ ts) = concatMap (level (n-1)) ts
\end{haskell}

To produce feature vectors, we do not directly construct the matrix; instead we
generate the rows and concatenate them together in one step, using the
\hs{concatMap} function:

\begin{haskell}
featureVec :: Expr -> [Feature]
featureVec e = concatMap (\m -> pad (level m tree)) [1..r]
  where tree   = toTree [] e
        pad xs = take c (xs ++ repeat 0)
\end{haskell}

By providing \hs{featureVec} with the latest set of clusters, read from the
\textsc{AstPlugin} database, we turn Core expressions into feature vectors,
which are appended to the database.

We use the Weka system to perform our k-means clustering, as it is widely used,
including by ML4PG. We select all feature vectors from our database, and write
them in CSV format for Weka to process. The Weka CLI command is invoked, which
appends a cluster number to each of these feature vectors; we read these off and
append them to the database. As long as more SCCs remain unprocessed, we keep
looping this process, using the database to communicate between the feature
extraction and clustering phases.

\subsection{\mlspec{}}
\label{sec:mlspec}

We cannot supply these clusters as-is to \quickspec{}, since it must be provided
with a \emph{signature}. These are constructed by our \mlspec{} tool,
using information from the \astplugin{} database. Tasks performed by
\mlspec{} include:

\begin{itemize}
\item{Monomorphising}: Given values of polymorphic type, e.g. \hs{safeHead ::
    forall t. [t] -> Maybe t} and \hs{[] :: forall t. [t]}, a testing-based
  system like \quickspec{} is unable to evaluate these expressions without
  instantiating the variable \hs{t} to a specific type. Such an instantiation is
  called \emph{monomorphising}, and in the case of \mlspec{} we build on
  previous work in \quickcheck{} by attempting to instantiate all type variables to
  \hs{Integer}. We discard those cases where this is invalid, such as variable
  \emph{type constructors} (e.g. \hs{forall c. c Bool -> c Bool}) or
  incompatible class constraints (e.g. \hs{forall t. IsString t => t}).

\item{Qualification}: All names are \emph{qualified} (prefixed by their module's
  name), to avoid most ambiguity. There is still the possibility that multiple
  packages will declare modules of the same name, although this is rare as it
  causes problems for any Haskell programmer trying to use those modules. In
  such cases the exploration process simply aborts.

\item{Variable definition}: Once a \quickspec{} theory has been defined containing
  all of the given terms, we inspect the types it references and append three
  variables for each to the theory (enough to discover laws such as
  associativity, but not too many to overflow the limit of \quickspec{}'s exhaustive
  search).

\item{Sandboxing}: One difficulty with Haskell's packaging infrastructure is
  that all required packages and modules must be provided up-front, usually by
  specification in a Cabal file. Since \mlspec{} builds signatures
  \emph{dynamically}, depending on the cluster information it is given, we do
  not know what packages it may need. To work around this problem,
  \mlspec{} invokes \quickspec{} for each cluster using a library we have
  built called \texttt{nix-eval}. This provides an \texttt{eval} function, like
  those commonly found in dynamic languages such as Python and Javascript, for
  evaluating Haskell expressions. The key feature of \texttt{nix-eval} is that
  these Haskell expressions may reference packages that are not installed on the
  system. When such expressions are evaluated, these packages will be
  automatically downloaded and installed into a sandbox using the Nix package
  manager, and GHC will be invoked in this sandbox to perform the evaluation.
\end{itemize}

\subsection{Difficulties}

The application to theory exploration to real-world Haskell packages is
complicated due to \quickspec{}'s use of \quickcheck{}'s \hs{Arbitrary} type
class to generate random values for instantiating variables. Whilst we can
automatically define \quickspec{} theories and invoke them with \hs{nix-eval},
not all types have \hs{Arbitrary} instances; those without cannot be given any
variables in our signature, which severely limits the possible combinations
which can be explored. In many cases, no variables can be included at all,
leaving just equations involving constants. This has so far prevented us from
measuring the direct impact on \quickspec{} performance, either directly by
exploring the sub-sets identified through recurrent clustering, or indirectly by
comparing the equations generated by a full brute-force search to our recurrent
clusters: those equations relating terms from different clusters would not be
discovered by our method. It is this ratio of equations found through brute
force to those found after narrowing-down by clusters which is one of our key
objectives to maximise at this stage; until we begin to pursue the
\emph{interestingness} of the properties.

The following less-serious problems were also encountered while applying
\mlforhs{} to \hackage{} packages:

\begin{itemize}
\item Some packages, such as \texttt{warp} and \texttt{conduits}, get no
  declarations to cluster. This is because they make all of their declarations
  privately, e.g. in ``internal'' modules, then use separate modules to export
  the public declarations. GHC's renaming phase makes all references to such
  exports canonical, by pointing them to the private declarations. This forces
  us to ignore such declarations, as \quickspec{} will not be able to access them.

\item Since we do not support type-level entities, we ignore type
  classes. Unfortunately, this also means ignoring any value-level bindings (AKA
  ``methods'') which occur in a type class instance. Instead of being clustered,
  these result in references getting $f_{recursion}$ features. This is
  especially noticable in libraries like \hs{scientific}, where only the
  functions for constructing and destructing numbers in scientific notation are
  clustered; all of the arithmetic is defined in type classes. One difficulty
  with supporting methods is that their namespace in Core is disjoint from that
  of regular Haskell identifiers: a transformation layer would be required,
  along with explicit type annotations to avoid ambiguity.
\end{itemize}
