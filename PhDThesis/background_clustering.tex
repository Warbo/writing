Our approach to scaling up \quickspec{} takes inspiration from two sources. The
first is relevance filtering, which makes expensive algorithms used in theorem
proving more practical by limiting the size of their inputs. We describe this
approach in more details in \S~\ref{sec:relevance}. Relevance filtering is a
practical tool which has existing applications in software, such as the
\emph{Sledgehammer} component of the Isabelle/HOL theorem prover.

Despite the idea's promise, we cannot simply invoke existing relevance filter
algorithms in our theory exploration setting. The reason is that relevance
filtering is a supervised learning method, i.e. it would require a distinguished
expression to compare everything against. Theory exploration does not have such
a distinguished expression; instead, we are interested in relationships between
\emph{any} terms generated from a signature, and hence we must consider the
relevance of \emph{all terms} to \emph{all other terms}.

A natural fit for this task is \emph{clustering}, which attempts to group
similar inputs together in an unsupervised way. Based on their success in
discovering relationships and patterns between expressions in Coq and ACL2 (in
the ML4PG and ACL2(ml) tools respectively), we hypothesise that clustering
methods can fulfil the role of relevance filters for theory exploration:
intelligently breaking up large signatures into smaller ones more amenable to
brute force enumeration, such that related expressions are explored together.
