%% FIXME: This is copypasta from benchmark paper

The automation of mathematical tasks has been pursued since at least the time of
mechanical calculators like the Pascaline~\cite{d'ocagne}. A recurring theme in
these efforts is the separation between those undertaken by mathematicians like
Pascal and Babbage~\cite{bowden}, and those of engineers such as
M\"uller~\cite[p. 65]{lindgren}. This pattern continues today, with the tasks we
are concerned with (automatically constructing and evaluating concepts,
conjectures, theorems, axioms, examples, etc.) being divided into two main
fields: Mathematical Theory Exploration (MTE)~\cite{buchberger:06} (also
sometimes prefaced with ``Computer-Aided'', ``Automated'' or
``Algorithm-Supported''), which is championed by mathematicians such as
Buchberger~\cite{buchberger}; and Automated Theory Formation
(ATF)~\cite{lenat:77,colton:book}, pursued by AI researchers including Lenat.
Other related terms include ``Automated Mathematical
Discovery''~\cite{epstein:91,colton2000notion,esarm2008},
``Concept Formation in Discovery Systems''~\cite{haase}, and
``Automated Theorem Discovery''~\cite{roy}.

Such a plethora of terminology can mask similarities and shared goals between
these fields. Even notable historical differences, such as the emphasis of MTE
on user-interaction and mathematical domains, in contrast to the full automation
and more general applications targeted by ATF, are disappearing in recent
implementations.
