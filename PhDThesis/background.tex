\chapter{Background}

% Background should cover approaches to automated reasoning about software, and
% formal systems in general. We can highlight how the language/representation
% imposes invariants and guarantees on what we can know, and how we can extend
% those guarantees with features like purity (prevents behaviour depending on
% ambient environment/state), lazy evaluation (prevents problems with
% non-termination), sound type systems (limits the space of possible values for
% each expression), etc. and therefore how Haskell is a natural vehicle for us to
% use. Giving background on Haskell naturally leads to descendants like Agda and
% Idris, which are also interactive theorem provers; which lets us discuss
% Isabelle, Coq, HOL, etc., how they're difficult for existing reasoning
% algorithms to handle (e.g. resolution, which works well in first-order logics),
% approaches to helping

This chapter provides background on the fields we investigate, and the
techniques we apply. It includes descriptions of the proof- and
programming-languages we use, and justifications for their use in our work. A
very short summary of relevant Machine Learning topics is also given, to provide
context for the techniques we use (namely unsupervised clustering) and the
potential alternatives and future directions that are most applicable.

\subsection{Haskell}
\label{sec:haskell}

We mostly focus our attention on the Haskell programming language, both as an
implementation vehicle and as our representation of functions, properties, etc.
to explore. We choose Haskell since it combines formal, logical underpinnings
which aid reasoning (compared to more popular languages like Java or C), yet it
is still popular enough to sustain a rich ecosystem of tooling and a large
corpus of existing code (compared to more formal languages like Coq or
Isabelle). Haskell is well-suited to programming language research; indeed, this
was a goal of the language's creators \cite{marlow2010haskell}. Like most
\emph{functional} programming languages, Haskell builds upon $\lambda$-calculus,
with extra features such as a strong type system and ``syntactic sugar'' to
improve readability.

The following features make Haskell especially useful for our purposes, although
many are also present in other languages such as StandardML and Coq (which we
also use, but only when needed for compatibility with prior work):

\begin{description}

\item{Functional}: All control flow in Haskell is performed by function
  abstraction and application, which we can reason about using standard rules of
  inference such as \emph{modus ponens}.

\item{Pure}: Execution of actions (e.g. reading files) is separate to evaluation
  of expressions; hence our reasoning can safely ignore complicated external and
  non-local interactions.

\item{Statically Typed}: Expressions are constrained by \emph{types}, which can
  be used to eliminate unwanted combinations of values, and hence reduce search
  spaces; \emph{static} types can be deduced syntactically, without having to
  execute the code.

\item{Non-strict}: If an evaluation strategy exists for $\beta$-normalising an
  expression (i.e. performing function calls) without diverging, then a
  non-strict evaluation strategy will not diverge when evaluating that
  expression. This is rather technical, but in simple terms it allows us to
  reason effectively about a Turing-complete language, where evaluation may not
  terminate. For example, when reasoning about \emph{pairs} of values \hs{(x,
    y)} and projection functions \hs{fst} and \hs{snd}, we might want to use an
  ``obvious'' rule such as
  $\forall \text{\hs{x y}}, \text{\hs{x}} = \text{\hs{fst (x, y)}}$. Haskell's
  non-strict semantics makes this equation valid; whilst it would \emph{not} be
  valid in the strict setting common to most other languages, where the
  expression \hs{fst (x, y)} will diverge if \hs{y} diverges (and hence alter
  the semantics, if \hs{x} doesn't diverge).

\item{Algebraic Data Types}: These provide a rich grammar for building up
  user-defined data representations, and an inverse mechanism to inspect these
  data by \emph{pattern-matching}. For our purposes, the useful consequences of
  ADTs and pattern-matching include their amenability for inductive proofs and
  the fact they are \emph{closed}; i.e. an ADT's declaration specifies all of
  the normal forms for that type. This makes exhaustive case analysis trivial,
  which would be impossible for \emph{open} types (for example, consider classes
  in an object oriented language, where new subclasses can be introduced at any
  time).

\item{Parametricity}: This allows Haskell \emph{values} to be parameterised over
  \emph{type-level} objects; provided those objects are never inspected. This
  has the \emph{practical} benefit of enabling \emph{polymorphism}: for example,
  we can write a polymorphic identity function \hs{id :: forall t. t ->
    t}. \footnote{Read ``\hs{a :: b}'' as ``\hs{a} has type \hs{b}'' and ``\hs{a
      -> b}'' as ``the type of functions from \hs{a} to \hs{b}''.} Conceptually,
  this function takes \emph{two} parameters: a type \hs{t} \emph{and} a value of
  type \hs{t}; yet only the latter is available in the function body,
  e.g. \hs{id x = x}. This inability to inspect type-level arguments gives us
  the \emph{theoretical} benefit of being able to characterise the behaviour of
  polymorphic functions from their type alone, a technique known as
  \emph{theorems for free} \cite{wadler1989theorems}.

\item{Type classes}: Along with their various extensions, type classes are
  interfaces which specify a set of operations over a type (or other type-level
  object, such as a \emph{type constructor}). Many type classes also specify a
  set of \emph{laws} which their operations should obey but, lacking a simple
  mechanism to enforce this, laws are usually considered as documentation. As a
  simple example, we can define a type class \hs{Semigroup} with the following
  operation and associativity law:

\begin{lstlisting}[language=Haskell, xleftmargin=.2\textwidth, xrightmargin=.2\textwidth]
op :: forall t. Semigroup t => t -> t -> t
\end{lstlisting}

$$\forall \text{\hs{x y z}}, \text{\hs{op x (op y z)}} = \text{\hs{op (op x y) z}}$$

The notation \hs{Semigroup t =>} is a \emph{type class constraint}, which
restricts the possible types \hs{t} to only those which implement
\hs{Semigroup}. \footnote{Alternatively, we can consider \hs{Semigroup t} as the
  type of ``implementations of \hs{Semigroup} for \hs{t}'', in which case
  \hs{=>} has a similar role to \hs{->} and we can consider \hs{op} to take
  \emph{four} parameters: a type \hs{t}, an implementation of \hs{Semigroup t}
  and two values of type \hs{t}. As with parameteric polymorphism, this extra
  \hs{Semigroup t} parameter is not available at the value level. Even if it
  were, we could not alter our behaviour by inspecting it, since Haskell only
  allows types to implement each type class in at most one way, so there would
  be no information to branch on.} There are many \emph{instances} of
\hs{Semigroup} (types which may be substituted for \hs{t}), e.g. \hs{Integer}
with \hs{op} performing addition. Many more examples can be found in the
\emph{typeclassopedia} \cite{yorgey2009typeclassopedia}. This ability to
constrain types, and the existence of laws, helps us reason about code
generically, rather than repeating the same arguments for each particular pair
of \hs{t} and \hs{op}.

\item{Equational}: Haskell uses equations at the value level, for definitions;
  at the type level, for coercions; at the documentation level, for typeclass
  laws; and at the compiler level, for ad-hoc rewrite rules. This provides us
  with many \emph{sources} of equations, as well as many possible \emph{uses}
  for any equations we might discover. Along with their support in existing
  tools such as SMT solvers, this makes equational conjectures a natural target
  for theory exploration.

\item{Modularity}: Haskell has a module system, where each module may specify an
  \emph{export list} containing the names which should be made available for
  other modules to import. When such a list is given, any expressions \emph{not}
  on the list are considered \emph{private} to that module, and are hence
  inaccessible from elsewhere. This mechanism allows modules to provide more
  guarantees than are available just in their types. For example, a module may
  represent email addresses in the following way:

  \begin{haskell}
    module Email (Email(), at, render) where

    data Email = E String String

    render :: Email -> String
    render (E u h) = u ++ "@" ++ h

    at :: String -> String -> Maybe Email
    at "" h  = Nothing
    at u  "" = Nothing
    at u  h  = Just (E u h)
  \end{haskell}

  This \hs{Email} type can be constructed in two ways: directly, via the \hs{E}
  constructor, which requires both a user part and a host part (both
  \hs{String}s, for simplicity); or indirectly, via the \hs{at} function, which
  will only call \hs{E} if both \hs{String}s are non-empty (a programming
  pattern known as a ``smart constructor''). The export list in the first line
  contains the \hs{at} function, but not the \hs{E} constructor~\footnote{The
    syntax \hs{Email()} means we're exporting the \hs{Email} type, but not any
    of its constructors.}, so any code using this module is only able to create
  \hs{Email} values with non-empty parts.

\end{description}

Together, these features make Haskell code highly structured, amenable to
logical analysis and subject to many algebraic laws. However, as mentioned with
regards to type classes, Haskell itself is incapable of expressing or enforcing
these laws (at least, without difficulty \cite{lindley2014hasochism}). This
reduces the incentive to manually discover, state and prove theorems about
Haskell code, e.g. in the style of interactive theorem proving, as these results
may be invalidated by seemingly innocuous code changes. This puts Haskell in a
rather special position with regards to the discovery of interesting theorems;
namely that many discoveries may be available with very little work, simply
because the code's authors are focused on \emph{software} development rather
than \emph{proof} development. The same cannot be said, for example, of ITP
systems; although our reasoning capabilities may be stronger in an ITP setting,
much of the ``low hanging fruit'' will have already been found through the
user's dedicated efforts, and hence theory exploration would be less likely to
discover unexpected properties.

Other empirical advantages to studying Haskell, compared to other programming
languages or theorem proving systems, include:

\begin{itemize}
\item The large amount of Haskell code which is freely available online, e.g. in
  repositories like \href{http://hackage.haskell.org}{Hackage}, with which we
  can experiment.

\item The existence of theory exploration systems such as \hspec{}, \qspec{} and
  \textsc{Speculate}.

\item Related tooling we can re-use such as counterexample finders (\qcheck{},
  \textsc{SmallCheck}, \textsc{SmartCheck}, \textsc{LeanCheck},
  \textsc{Hedgehog}, etc.), theorem provers
  (e.g. \textsc{Hip}~\cite{rosen2012proving}), and other testing and
  term-generating systems like \textsc{MuCheck}~\cite{le2014mucheck},
  \textsc{MagicHaskeller}~\cite{katayama2011magichaskeller} and
  \textsc{Djinn}~\cite{augustsson2005djinn}.

\item The remarkable amount of infrastructure which exists for working with
  Haskell code, including package managers, compilers, interpreters, parsers,
  static analysers, etc.
\end{itemize}

\newcommand{\CVar}{\texttt{Var}}
\newcommand{\CLit}{\texttt{Lit}}
\newcommand{\CApp}{\texttt{App}}
\newcommand{\CLam}{\texttt{Lam}}
\newcommand{\CLet}{\texttt{Let}}
\newcommand{\CCase}{\texttt{Case}}
\newcommand{\CType}{\texttt{Type}}
\newcommand{\CLocal}{\texttt{Local}}
\newcommand{\CGlobal}{\texttt{Global}}
\newcommand{\CConstructor}{\texttt{Constructor}}
\newcommand{\CLitNum}{\texttt{LitNum}}
\newcommand{\CLitStr}{\texttt{LitStr}}
\newcommand{\CAlt}{\texttt{Alt}}
\newcommand{\CDataAlt}{\texttt{DataAlt}}
\newcommand{\CLitAlt}{\texttt{LitAlt}}
\newcommand{\CDefault}{\texttt{Default}}
\newcommand{\CNonRec}{\texttt{NonRec}}
\newcommand{\CRec}{\texttt{Rec}}
\newcommand{\CBind}{\texttt{Bind}}

\begin{figure}
  \begin{equation*}
    \begin{split}
      expr\    \rightarrow\ & \CVar\ id                          \\
                         |\ & \CLit\ literal                     \\
                         |\ & \CApp\ expr\ expr                  \\
                         |\ & \CLam\ \mathcal{L}\ expr           \\
                         |\ & \CLet\ bind\ expr                  \\
                         |\ & \CCase\ expr\ \mathcal{L}\ [alt]   \\
                         |\ & \CType                             \\
      id\      \rightarrow\ & \CLocal\       \mathcal{L}         \\
                         |\ & \CGlobal\      \mathcal{G}         \\
                         |\ & \CConstructor\ \mathcal{D}         \\
      literal\ \rightarrow\ & \CLitNum\ \mathcal{N}              \\
                         |\ & \CLitStr\ \mathcal{S}              \\
      alt\     \rightarrow\ & \CAlt\ altcon\ expr\ [\mathcal{L}] \\
      altcon\  \rightarrow\ & \CDataAlt\ \mathcal{D}             \\
                         |\ & \CLitAlt\ literal                  \\
                         |\ & \CDefault                          \\
      bind\    \rightarrow\ & \CNonRec\ binder                   \\
                         |\ & \CRec\ [binder]                    \\
      binder   \rightarrow\ & \CBind\ \mathcal{L}\ expr
    \end{split}
  \end{equation*}
  Where:
  \begin{tabular}[t]{l @{ $=$ } l}
    $\mathcal{S}$ & string literals    \\
    $\mathcal{N}$ & numeric literals   \\
    $\mathcal{L}$ & local identifiers  \\
    $\mathcal{G}$ & global identifiers \\
    $\mathcal{D}$ & constructor identifiers
  \end{tabular}

  \caption{Simplified syntax of GHC Core in BNF style. $[]$ and $(,)$ denote repetition and grouping, respectively.}
  \label{fig:coresyntax}
\end{figure}

Further evidence of Haskell's suitability for theory exploration is given by the
fact that the state-of-the-art implementation for Isabelle/HOL, the
\textsc{Hipster}~\cite{Hipster} system, is actually implemented by translating
to Haskell and invoking \hspec{}~\cite{claessen2013automating}.

Whilst our systems and experiments use normal Haskell code, for simplicity we
perform some of our analyses on an intermediate representation of the
\textsc{GHC} compiler, known as \emph{GHC Core}, rather than the relatively
large and complex syntax of Haskell proper. Core is based on \fc{}, which is
described in detail in~\cite[Appendix C]{sulzmann2007system}.

Core contains explicit type annotations and coercions, which we omit as they
have no effect on runtime behaviour. The resulting sub-set of Core\footnote{As
  of GHC version 7.10.2.} is shown in Figure~\ref{fig:coresyntax}; for brevity,
we also omit several other forms of literal (machine words of various sizes,
individual characters, etc.), since their treatment is similar to those of
strings and numerals. We use quoted strings to denote names and literals,
e.g. \hs{Local "foo"}, \hs{Global "bar"}, \hs{Constructor "Baz"}, \hs{LitStr
  "quux"} and \hs{LitNum "42"}, and require only that they can be compared for
equality.

\begin{figure}
  \begin{haskell}
    data Nat = Z
             | S Nat

    plus :: Nat -> Nat -> Nat
    plus    Z  y = y
    plus (S x) y = S (plus x y)

    mult :: Nat -> Nat -> Nat
    mult    Z  y = Z
    mult (S x) y = plus y (mult x y)

    odd :: Nat -> Bool
    odd    Z  = False
    odd (S n) = even n

    even :: Nat -> Bool
    even    Z  = True
    even (S n) = odd n
  \end{haskell}
  \caption{A Haskell datatype for Peano numerals with some simple arithmetic
    functions, including mutually-recursive definitions for \hs{odd} and
    \hs{even}. \hs{Bool} is Haskell's built in boolean type, which can be
    regarded as \hs{data Bool = True | False}.}
  \label{fig:haskellexample}
\end{figure}

\begin{figure}
  \begin{subfigure}[plus]{\textwidth}
    \begin{small}
      \underline{\texttt{plus}}
      \begin{verbatim}
Lam "a" (Lam "y" (Case (Var (Local "a"))
                       "b"
                       (Alt (DataAlt "Z") (Var (Local "y")))
                       (Alt (DataAlt "S") (App (Var (Constructor "S"))
                                               (App (App (Var (Global "plus"))
                                                         (Var (Local  "x")))
                                                    (Var (Local "y"))))
                                          "x")))
      \end{verbatim}
    \end{small}
  \end{subfigure}
  \begin{subfigure}[mult]{\textwidth}
    \begin{small}
      \underline{\texttt{mult}}
      \begin{verbatim}
Lam "a" (Lam "y" (Case (Var (Local "a"))
                       "b"
                       (Alt (DataAlt "Z") (Var (Constructor "Z")))
                       (Alt (DataAlt "S") (App (App (Var (Global "plus"))
                                                    (Var (Local  "y")))
                                               (App (App (Var (Global "mult"))
                                                         (Var (Local  "x")))
                                                    (Var (Local  "y"))))
                                          "x")))
      \end{verbatim}
    \end{small}
  \end{subfigure}
  \begin{subfigure}[odd]{\textwidth}
    \begin{small}
      \underline{\texttt{odd}}
      \begin{verbatim}
Lam "a" (Case (Var (Local "a"))
              "b"
              (Alt (DataAlt "Z") (Var (Constructor "False")))
              (Alt (DataAlt "S") (App (Var (Global "even"))
                                      (Var (Local  "n")))
                                 "n"))
      \end{verbatim}
    \end{small}
  \end{subfigure}
  \begin{subfigure}[even]{\textwidth}
    \begin{small}
      \underline{\texttt{even}}
      \begin{verbatim}
Lam "a" (Case (Var (Local "a"))
              "b"
              (Alt (DataAlt "Z") (Var (Constructor "True")))
              (Alt (DataAlt "S") (App (Var (Global "odd"))
                                      (Var (Local  "n")))
                                 "n"))
      \end{verbatim}
    \end{small}
  \end{subfigure}
  \caption{Translations of functions in Figure \ref{fig:haskellexample} into the
    Core syntax of Figure \ref{fig:coresyntax}. Notice the introduction of
    explicit $\lambda$ abstractions (\texttt{Lam}) and the use of \texttt{Case}
    to represent piecewise definitions. Fresh variables are chosen arbitrarily
    as \texttt{"a"}, \texttt{"b"}, etc.}
  \label{fig:coreexample}
\end{figure}

Figure~\ref{fig:haskellexample} shows some simple Haskell function definitions,
along with a custom datatype for Peano numerals. The translation to Core syntax
is routine, and shown in Figure~\ref{fig:coreexample}. Although the Core is more
verbose, we can see that similar structure in the Haskell definitions gives rise
to similar structure in the Core; for example, the definitions of \hs{odd} and
\hs{even} are identical in both languages, except for the particular identifiers
used. It is this close correspondence which allows us to analyse Core
expressions in place of their more complicated Haskell source.

Note that we exclude representations for type-level entities, including datatype
definitions like that of \hs{Nat}. GHC can represent these, but in this work we
only consider reducible expressions (i.e. value-level bindings of the form
\mbox{\hs{f a b ... = ...}}).

\section{Property Checking}
\label{sec:propertychecking}

% TODO: Automated testing
% TODO: Agile
% TODO: Unit testing
% TODO:

Although unit testing is the de facto industry standard for quality assurance in
non-critical systems, the level of confidence it provides is rather low, and
totally inadequate for many (e.g. life-) critical systems. To see why, consider
the following Haskell function, along with some unit
tests\footnote{Haskell functions can be prefix or infix: prefix functions have
  alphanumeric names, like \hs{div} and \hs{factorial}, and appear in function
  calls before their arguments, e.g. \hs{div 10 5}; infix functions have
  non-alphanumeric names, like \hs{+} and \hs{==}, and appear in function calls
  between their first two arguments, e.g. \hs{4 + 3}. Prefix functions can
  appear infix using backticks, like \hs{10 `div` 5}, and infix functions can
  appear prefix using parentheses, like \hs{(+) 4 3}.}:

% BEGIN unit
\begin{haskell}
factorial 0 = 1
factorial n = n * factorial (n - 1)

fact_base      = factorial 0 == factorial 1
fact_increases = factorial 3 <= factorial 4
fact_div       = factorial 4 == factorial 5 `div` 5
\end{haskell}
% END unit

The intent of the function is to map an input $n$ to an output $n!$. The tests
check a few properties of the implementation, including the base case, that the
function is monotonically increasing, and a relationship between adjacent
outputs. However, these tests will \emph{not} expose a serious problem with the
implementation: it diverges on half of its possible inputs.

Haskell's built-in numeric types allow negative numbers, which this
implementation doesn't take into account. Whilst this is a rather trivial
example, it highlights a common problem: unit tests are insufficient to expose
incorrect assumptions. In this case, our assumption that numbers are positive
has caused a bug in the implementation \emph{and} limited the tests we've
written.

One way to avoid this problem, if we do manage to spot it, would make results
symmetric around zero, which can be checked by a \emph{regression test} such as:

% BEGIN neg
\begin{haskell}
fact_neg = factorial 1 == factorial (-1)
\end{haskell}
% END neg

The implementation can be modified by composing if with the \hs{abs}olute value
function:

% BEGIN abs
\begin{haskell}
factorial = original . abs
  where original 0 = 1
        original n = n * naive (n - 1)
\end{haskell}
% END abs

However, this is \emph{still} not enough, since this function will also accept
fractional values\footnote{Since we only use generic numeric operations, the
  function will be polymorphic with a type of the form
  \hs{forall t. Num t => t -> t}, where \hs{Num t} constrains the type variable
  \hs{t} to be numeric.}, which will also cause it to diverge. This problem can
be fixed by restricting \hs{factorial}'s type with an explicit annotation, e.g.
to only whole numbers: \hs{factorial :: forall t. Integral t => t -> t}

Clearly, by choosing which scenarios to check, we are biasing the test suite
towards those cases we've already thought of, whilst neglecting the problems we
did not expect. Haskell offers a partial solution to this problem in the form of \emph{property checking}. Tools such as \quickcheck{} separate tests into three components: a \emph{property} to check, which unlike a unit test may contain
\emph{free variables}; a source of values to instantiate these free variables; and a
stopping criterion.

\subsection{\quickcheck{}}
\label{sec:quickcheck}

\quickcheck{}~\cite{claessen2011quickcheck} is the most widely used property
checking library for
Haskell~\footnote{According to downloads counted at
  \url{http://hackage.haskell.org/packages/browse}, and packages depending on
  it at \url{https://packdeps.haskellers.com/reverse/QuickCheck}, accessed on
  2019-05-21.}. Here is how we might restate our unit tests as \quickcheck{}
properties:

% BEGIN quickchecked
\begin{haskell}
fact_base        = factorial 0 == factorial 1
fact_increases n = factorial n <= factorial (n + 1)
fact_div       n = factorial n == factorial (n + 1) `div` (n + 1)
fact_neg       n = factorial n == factorial (-n)
\end{haskell}
% END quickchecked

The free variables (all called \hs{n} in this case) are abstracted as function
parameters; these parameters are implicitly \emph{universally quantified},
i.e. we've gone from a unit test asserting $factorial(3) \leq factorial(4)$ to a
property asserting $\forall n, factorial(n) \leq factorial(n+1)$. Notice that
unit tests like \hs{fact_base} are valid properties; they just happen to assert
rather weak statements, since they contain no free variables.

To check these properties, \quickcheck{} treats closed terms (like \hs{fact_base})
just like unit tests: pass if they evaluate to \hs{True}, fail otherwise. For
open terms, a random selection of values are generated and passed in via the
function parameter; the results are then treated in the same way as closed
terms. The default stopping criterion for \quickcheck{} (for each test) is when a
single generated test fails, or when 100 generated tests pass.

The ability to state \emph{universal} properties in this way avoids some of the
bias we encountered with unit tests. In the \hs{factorial} example, this
manifests in two ways:

\begin{itemize}
\item We are forced to confront the generality of \hs{factorial}'s type, since
  \quickcheck{} cannot test polymorphic functions directly; they must be
  \emph{monomorphised} first (instantiated to a particular concrete type), to
  decide which data generator to use (i.e. which instance of \quickcheck{}'s
  \hs{Arbitrary} type class). This can be done either by specifying a generator
  directly (avoiding the type class mechanism), or by adding explicit type
  annotations, or by calling \quickcheck{}'s \hs{monomorphic}
  function.\footnote{\hs{monomorphic} picks \hs{Integer} for all polymorphic
    type variables. This na\"ive choice instantiates many common type classes
    like \hs{Num}, \hs{Show} and \hs{Eq}; variables whose constraints are not
    satisfied will fail to type check (e.g. \hs{Bounded}, since \hs{Integer} is
    a bignum).} This will bring the problem of fractional values. Instead of
  monomorphising the properties, we might choose to monomorphise \hs{factorial}

\item By default, \quickcheck{} picks a generator based on the \emph{type} of
  value to be generated: since \hs{Int} includes positive and negative values,
  the \hs{Int} generator will output both. This will expose the problem with
  negative numbers, which we weren't expecting.
\end{itemize}

Whilst property checking generalises and improves upon unit testing, the problem
of tests being biased towards expected cases remains, since the properties to be
checked are still manually specified. Property checking can be complemented by
\emph{theory exploration} (TE), which avoids this bias by \emph{discovering}
such properties; through a combination of brute-force enumeration, random
testing and (in the case of \hipspec{} and \hipster{}) automated theorem
proving. Property checkers like \quickcheck{} are an important component of TE
systems, since their data generators can the search process, and they can
prevent ``obvious'' falsehoods from being output by checking them first.

\section{Theory Exploration}
\label{sec:theoryexploration}

% TODO Work this in
\iffalse
Theory exploration is the task of taking a \emph{signature} of definitions in
some formal system (for example a programming language) and automatically
generating a set of formal statements (properties) involving those
definitions. These may be conjectures or theorems (proven either by sending
conjectures to an automated theorem prover, or by having the generating
procedure proceed in logically sound steps); in either case, these statements
must also be ``interesting'' in some way. It is beyond the scope of this paper
to define what makes a mathematical statement ``interesting'', but this problem
has been tackled extensively in the literature~\cite{colton2000notion}.
\fi
% END TOTO

In this work we consider the application of \emph{(automated) theory
  exploration} to generate conjectures about code. Existing tools are able to
\emph{prove} those conjectures, and hence output \emph{novel} theorems without
guidance from the user. The method of conjecture generation is a key
characteristic of any theory exploration system, although all existing
implementations rely on brute force enumeration to some degree.

We focus on \qspec{}~\cite{QuickSpec}, which conjectures equations about Haskell
code (these may be fed into another tool, such as \hspec{}, for proving). We
have used version 1 of \qspec{} in our experiments, due to its availability,
stability and tooling integration; hence references to \qspec{} are to version 1
unless stated otherwise. At the time of writing there is a \qspec{} version 2
available, which has a much improved generation procedure which can be
significantly faster than its predecessor. Whilst \qspec{} 2 has advanced the
state of the art in theory exploration, our preliminary experience has found
that it still suffers the scaling issues we identify in this work; indeed, we
even encountered memory exhaustion from some of the examples included with the
\qspec{} 2 source code! Since \qspec{} 2 has less integration with the tooling
we have used, we leave a more thorough analysis of it (and related systems such
as \speculate{}) for future work, pending the necessary infrastructure changes
that would require.

\begin{figure}
  \centering
  \begin{minted}{haskell}
    -- A datatype with two constructors (Note that -- introduces a comment)
    data Bool = True | False

    -- A recursive datatype (S requires a Nat as argument)
    data Nat = Z | S Nat

    -- A function turning a Bool into a Bool
    not :: Bool -> Bool
    not True  = False
    not False = True

    -- A pair of mutually recursive functions

    odd :: Nat -> Bool
    odd    Z  = False
    odd (S n) = even n

    even :: Nat -> Bool
    even    Z  = True
    even (S n) = odd n
  \end{minted}
  \caption{Haskell datatypes for booleans and natural numbers, followed by some
    simple function definitions (with type annotations).}
  \label{fig:haskellteexample}
\end{figure}

\qspec{} (version 1) is written in the Haskell programming language, and works
with definitions which are also written in Haskell. A thorough description of
Haskell, including its suitability for theory exploration, can be found
in~\ref{sec:haskell}. For illustrative purposes, some simple Haskell definitions
are shown in Figure~\ref{fig:haskellteexample}. The following procedure is used to
generate conjectures, which are both plausible (due to testing on many examples)
and potentially interesting (due to being mutually irreducible):

\begin{enumerate}
\item Given a typed signature $\Sigma$ and set of variables $V$, \qspec{}
  generates a list $terms$ containing the constants (including functions) from
  $\Sigma$, the variables from $V$ and type-correct function applications
  $f(x)$, where $f$ and $x$ are elements of $terms$ \iffalse TODO: A little
  awkward; maybe use the above notation? \fi. To ensure the list is finite,
  function applications are only nested up to a specified depth (by default, 3).
\item The elements of $terms$ are grouped into equivalence classes, based on
  their type.
\item Each variable is instantiated to a particular value, generated randomly by
  \qcheck{}.
\item For each class, the members are compared (using a pre-specified function,
  such as equality \hs{==}) to see if these instantiations have caused an
  observable difference between members. If so, the class is split up to
  separate such distinguishable members.
\item The previous steps of variable instantiation and comparison are repeated
  until the classes stabilise (i.e. no differences have been observed for some
  specified number of repetitions).
\item A set of equations are then conjectured, relating each class's members.
\end{enumerate}

\iffalse
% TODO: Compare the above to this; merge/swap anything that's better
\begin{enumerate}
\item Given a typed signature $\Sigma$ and set of variables $V$, \qspec{}
  generates a list $terms$ containing the constants (including functions) from
  $\Sigma$, the variables from $V$ and type-correct function applications
  \hs{f x}, where \hs{f} and \hs{x} are elements of $terms$. To ensure the list
  is finite, function applications are only nested up to a specified depth (by
  default, 3).
\item The elements of $terms$ are grouped into equivalence classes, based on
  their type.
\item The equivalence of terms in each class is tested using \qcheck{}:
  variables are instantiated to particular values, generated randomly, and the
  resulting closed expressions are evaluated and compared for equality.
\item If a class is found to have non-equal members, it is split up to separate
  those members.
\item The previous steps of testing and splitting are repeated until the classes
  stabilise (i.e. no differences have been observed for some specified number of
  repetitions).
\item For each class, one member is selected and equations are conjectured that
  it is equal to each of the other members.
\item A congruence closure algorithm is applied to these equations, to discard
  any which are implied by the others.
\end{enumerate}
\fi

Such conjectures can be used in several ways: they can be simplified for direct
presentation to the user (for example via the congruence closure algorithm
\iffalse\cite{TODO}\fi), sent to a more rigorous system like \hspec{} or
\hipster{} for proving, or even serve as a background theory for an
automated theorem prover \cite{claessen2013automating}.

As an example, we can consider a simple signature containing the expressions
from Figure \ref{fig:haskellteexample}:

\begin{align*}
  \Sigma_{\texttt{Nat}} = \{\texttt{Z}, \texttt{S}, \texttt{plus}, \texttt{mult}, \texttt{odd}, \texttt{even}\}
\end{align*}

Together with a set of variables, say $V_{\texttt{Nat}} = \{a, b, c\}$,
\qspec{}'s enumeration will resemble the following:

\begin{align*}
  terms_{\texttt{Nat}} = [& \texttt{Z},\ \texttt{S},\ \texttt{plus},\ \texttt{mult},\ \texttt{odd},\ \texttt{even},\ a,\ b,\ c,\ \texttt{S Z},\ \texttt{S}\ a,\ \texttt{S}\ b, \\
                     & \texttt{S}\ c,\ \texttt{plus Z},\ \texttt{plus}\ a,\ \dots ]
\end{align*}

Notice that functions such as \hs{plus} and \hs{mult} are valid terms, despite
not being applied to any arguments. In addition, Haskell curries functions, so
these binary functions will be treated as unary functions which return unary
functions. This is required as the construction of $terms$ applies functions to
one argument at a time.

\begin{figure}
  % To reproduce, run 'quickSpec nat' in haskell_example/src/QuickSpecExample.hs
  \begin{haskell}
                      plus a b = plus b a
                      plus a Z = a
             plus a (plus b c) = plus b (plus a c)
                      mult a b = mult b a
                      mult a Z = Z
             mult a (mult b c) = mult b (mult a c)
                  plus a (S b) = S (plus a b)
                  mult a (S b) = plus a (mult a b)
             mult a (plus b b) = mult b (plus a a)
                     odd (S a) = even a
                odd (plus a a) = odd Z
               odd (times a a) = odd a
                    even (S a) = odd a
               even (plus a a) = even Z
              even (times a a) = even a
    plus (mult a b) (mult a c) = mult a (plus b c)
  \end{haskell}
  \caption{Equations conjectured by \qspec{} for the functions in Figure
    \ref{fig:haskellteexample}; after simplification.}
  \label{fig:qspecresult}
\end{figure}

These terms will be grouped into five classes, one each for \hs{Nat},
\hs{Nat -> Nat}, \hs{Nat -> Nat -> Nat}, \hs{Nat -> Bool} and \hs{Bool}. As the
variables $a$, $b$ and $c$ are instantiated to various randomly-generated
numbers, these equivalence classes will be divided, until eventually the
equations in Figure \ref{fig:qspecresult} are conjectured.

Although complete, this enumeration approach is wasteful: many terms are
unlikely to appear in theorems, which requires careful choice by the user of
what to include in the signature. Here we know that addition and multiplication
are closely related, and hence obey many algebraic laws.

\qspec{} (and \hspec{}) is also compatible with Haskell's existing testing
infrastructure, such that an invocation of \texttt{cabal test} can run these
tools alongside more traditional QA tools like \qcheck{}, \textsc{HUnit} and
\textsc{Criterion}.

In fact, there are similarities between the way a TE system like \qspec{} can
generalise from checking \emph{particular} properties to \emph{inventing} new
ones, and the way counterexample finders like \qcheck{} can generalise from
testing \emph{particular} expressions to \emph{inventing} expressions to
test. One of our aims is to understand the implications of this generalisation,
the lessons that each can learn from the other's approach to term generation,
and the consequences for testing and QA in general.

% TODO
%%%%% THE BELOW SEEMS TO BE INTERESTINGNESS?

%\begin{description}
%\item{Interestingness} Various alternative interestingness criteria have been
%  proposed, which we survey in \S \ref{sec:relatedwork}. Augmenting or replacing
%  the criteria may be useful, for example to distinguish useful relationships
%  from incidental coincidences; or to prevent surprising, insightful equations
%  from being discarded because they can be simplified.
%\end{description}

% Theory exploration is similar to \emph{experimental mathematics}

% - Relation to Science
%  - Testable/falsifiable hypotheses are like evaluable terms (or, more generally, conjectures which can be decided, using a reasonable amount of resources).
% - Relation to AI tasks: exploring surroundings, etc.

% - Statistics is another area that's less straightforward than normal numerical computing, since there is subjectivity and judgement involved in the answering of questions.

% Theory formation: Alison? Others.
% Theory exploration: Buchberger, Moa in Isabelle, Koen in Haskell. Others?
% Theorem proving: Well-trodden: first-order ATP, higher-order ITP, functional programming
% Communication: Latex, Wikis, APIs, communicating with aliens

% \section{Exploration in Theorem Proving}
% \label{sec:examples}

% Before exploring abstract definitions of interestingness, we can first consider
% some scenarios which arise during formal proof where we are forced to generate
% conjectures. An analysis of these situations, and the subsequent theorems they
% produce, will contribute towards an empirical justification for what is
% interesting (at least from a utilitarian point of view) and inform our later
% exploration of the literature.

% \subsection{Generalisation}

% \providecommand{\coq}[1]{\lstinline[language=ML]|#1|}

% When we \emph{generalise} a statement $S$, we obtain a new statement $S'$ of
% which $S$ is a special case. Although it seems counterintuitive, a generalised
% statement can sometimes be \emph{easier} to prove than the original. This arises
% often in inductive proofs, since the specific obligations which arise in the
% proof may be incompatible with the available inductive hypotheses.

% However, we cannot blindly generalise \emph{all} obligations we encounter, since
% \emph{over-generalising} results in obligations which are so strong that they
% are unprovable, or even false. We must therefore rely on heuristics to guide the
% generation of generalised conjectures, and hence perform a kind of exploration.

% An informative example is given by Boyer and Moore of the associativity of
% multiplication in ACL2 \cite{boyer1983proof}:

% $$(x * y) * z = x * (y * z)$$

% During the course of the proof, the following obligation arises:

% \begin{equation}
%   \tag{conc3}
%   (y + (x * y)) * z = (y * z) + ((x * y) * z)
%   \label{eq:conc3}
% \end{equation}

% ACL2 automatically generalises \eqref{eq:conc3} by replacing the repeated
% sub-term $x * y$ with a fresh variable $w$:

% \begin{equation}
%   \tag{conc4}
%   (y + w) * z = (y * z) + (w * z)
%   \label{eq:conc4}
% \end{equation}

% This generalised form is clearly the distributivity law for multiplication and
% addition, which can be proved separately to the original goal of
% associativity. It would not be controversial to claim that this distributivity
% law is interesting in its own right (relative to associativity, at least), in
% addition to its usefulness in making this proof go through.

% % TODO: Describe the ACL2 heuristics

% Generalisation also occurs frequently when reasoning about \emph{tail-recursive}
% definitions \cite{kapur2003automatic}. \footnote{A tail-recursive function can
%   be executed in constant space using a loop, whereas recursion in non-tail
%   positions may require a growing number of stack frames or nested closures. See
%   \S \ref{sec:auxiliarylemmas} for example definitions of each type.}

% \subsection{Analogy}

% One way to characterise the interestingness of a statement is by \emph{analogy}
% to existing interesting statements. By finding lemmas analogous to those of a
% different theory, we may be able to re-use tactics and other forms of
% meta-programming across both.

% Existing theory exploration systems have been successfully applied to this
% problem, however the use of pure exploration misses opportunities to
% \emph{focus} the search, since we know which lemmas are used in those theories
% where a technique succeeded. If we can find an analogy to map from such solved
% problems to our unsolved goal, we can infer the approximate form of the lemmas
% we require, and target these specifically.

% The approach taken by \textsc{ACL2(ml)} is to find lemmas which may be relevant
% to solving a goal $G$ by making analogies via unsupervised clustering
% \cite{Heras.Komendantskaya.Johansson.ea:2013}. These clusters are used in two
% ways:

% \begin{itemize}
% \item First, we use the cluster $C_G$ containing $G$ to identify analogous
%   theorems.

% \item For each theorem $T \in C_G \setminus \{G\}$, we consider those symbols
%   $S_T$ which occur in $T$ but not in $G$. Our analogous lemmas are those used
%   to prove $T$, mutated such that symbols $s \in S_T$ are replaced by members of
%   the cluster $C_s$ containing $s$.
% \end{itemize}

% The running examples for demonstrating \textsc{ACL2(ml)} are equivalence
% theorems for tail-recursive and non-tail-recursive calculations, as well as the
% effect of repeating certain list operations:

% \begin{itemize}

%   \item $\forall n, \texttt{natp}(n) \rightarrow \texttt{fact-tail}(n) = \texttt{fact}(n)$ where \texttt{natp} is the predicate that $n$ is a natural number, whilst \texttt{fact-tail} and \texttt{fact} are tail-recursive and non-tail-recursive implementations of factorial, respectively.

%   \item $\forall n, \texttt{natp}(n) \rightarrow \texttt{power-tail}(n) = \texttt{power}(n)$,  where \texttt{power-tail} and \texttt{power} calculate powers of 2.

%   \item $\forall n, \texttt{natp}(n) \rightarrow \texttt{fib-tail}(n) = \texttt{fib}(n)$,  where \texttt{fib-tail} and \texttt{fib} calculate fibonacci numbers.

%   \item $\forall x, \texttt{nat-listp}(x) \rightarrow \texttt{sort}(\texttt{sort}(x)) = \texttt{sort}(x)$, for list-of-natural-numbers predicate \texttt{nat-listp} and list-sorting function \texttt{sort}.

%   \item $\forall x, \texttt{true-listp}(x) \rightarrow \texttt{rev}(\texttt{rev}(x)) = x$, where \texttt{true-listp} ensures that $x$ is a valid singly-linked list structure and \texttt{rev} is list reversal.

%   \item $\forall x, \texttt{true-listp}(x) \rightarrow \texttt{int}(x, x) = x$, where \texttt{int} is the intersection of lists (i.e. a list of elements common to each).

% \end{itemize}

% \subsection{Auxiliary Lemmas} \label{sec:auxiliarylemmas}

% One consideration when generating conjectures is the difference between
% theorems, lemmas, corollaries, etc. From a logical point of view, these are all
% equivalent, and hence most proof assistants do not distinguish between
% them. However, their \emph{intention} may be different: in a sense, theorems are
% the interesting results; whilst lemmas are useful results, required for proving
% the theorems.

% Some systems, like Coq, allow users to \emph{label} each statement as being a
% \coq{Theorem}, a \coq{Lemma}, etc. despite their internal representations being
% the same. This shows us immediately that lemmas outnumber theorems; in the Coq
% standard library there are over five times as many lemmas as theorems
% \footnote{The latest version as of writing is \texttt{coq-8.4pl6} which, when
%   excluding comments, includes 1492 occurences of \coq{Theorem} and 7594 of
%   \coq{Lemma} in its \texttt{theories/} directory.}.

% % TODO: Analyse them

% % TODO: Theory exploration as lemma generation; give example from a HipSpec paper

% We can find a need for auxiliary lemmas, once again, in the context of
% tail-recursive functions. Consider proving the (pointwise) equality of the
% following Coq functions, defined for the Peano naturals \coq{Z} and \coq{S}:

% \begin{coqblock}
% Inductive Nat : Set := Z : Nat
%                      | S : Nat -> Nat.

% Fixpoint plus      (n m : Nat) := match n with
%                                       | Z    => m
%                                       | S n' => S (plus n' m)
%                                   end.

% Fixpoint plus_tail (n m : Nat) := match n with
%                                       | Z    => m
%                                       | S n' => plus_tail n' (S m)
%                                   end.
% \end{coqblock}

% \begin{haskell}
% Haskell equivalent:

% plus :: Nat -> Nat -> Nat
% plus      n  Z    = n
% plus      n (S m) = S (plus n m)

% plus_tail :: Nat -> Nat -> Nat
% plus_tail n  Z    = n
% plus_tail n (S m) = plus_tail (S n) m
% \end{haskell}

% Both of these functions implement addition, but the \coq{plus_tail} variant is
% tail-recursive. However, if we want to \emph{prove} that the definitions are
% (pointwise) equal, we run into difficulties. In particular, when the inductive
% step requires us to prove \coq{plus (S n) m = plus n (S m)} (which seems
% reasonable), we cannot make this go through using another inductive argument.

% \begin{coqblock}
% (* Solve equalities by beta-normalising both sides *)
% Ltac triv := try (simpl; reflexivity).

% (* Prove equivalence of plus and plus_tail *)
% Theorem equiv : forall n m, plus n m = plus_tail n m.
%   induction n; triv. (* Base case is trivial *)

%   (* Inductive case: plus (S n) m = plus_tail (S n) m *)
%   intro m.

%   (* Beta-reduce the right-hand-side (justification is trivial) *)
%   replace (plus_tail (S n) m) with (plus_tail n (S m)); triv.

%   (* Use induction hypothesis to replace plus_tail with plus *)
%   rewrite <- (IHn (S m)).
% \end{coqblock}

% Specifically, the \emph{conclusion} of a second inductive hypothesis is exactly
% the equation we need:

% \begin{coqblock}
% IHn' : (forall x, plus n' x = plus_tail n' x) -> plus (S n') m = plus n' (S m)
% \end{coqblock}

% Yet we cannot provide it with the argument it needs, as our original induction
% hypothesis is \emph{too specific} (i.e. it has too many \coq{S} constructors):

% \begin{coqblock}
% IHn : forall x, plus (S n') x = plus_tail (S n') x
% \end{coqblock}

% We are forced to abandon the proof, despite such a reasonable-looking
% intermediate goal.

% In fact, if we attempt to prove that goal \emph{separately}, we can use a
% straightforward argument by induction; even though it is actually
% \emph{stronger} due to the absence of the \coq{IHn} assumption. Using this
% separate result as a lemma, the pointwise equality is proven easily.

% \begin{coqblock}
% Lemma gen n m : plus (S n) m = plus n (S m).
%   induction n; triv. (* Base case is trivial *)

%   (* Move all S constructors outside *)
%   simpl. rewrite <- IHn. simpl.

%   (* Trivial *)
%   reflexivity.
% Defined.
% \end{coqblock}

% \begin{coqblock}
%   rewrite (gen n m).
%   reflexivity.
% Defined.
% \end{coqblock}


% FIXME: This seems to be repetitive of Theory Exploration?
% % For example theory of lists
\newcommand{\function}{\rightarrow}

\newcommand{\Zero}{\text{Z}}
\newcommand{\Succ}{\text{S}}
\newcommand{\plus}{\text{plus}}
\newcommand{\mult}{\text{times}}

\newcommand{\List}{\text{List}}
\newcommand{\ListA}{\text{List} \  a}
\newcommand{\Nil}{\text{Nil}}
\newcommand{\Cons}{\text{Cons}}
\newcommand{\Head}{\text{head}}
\newcommand{\Tail}{\text{tail}}
\newcommand{\Append}{\text{append}}
\newcommand{\Reverse}{\text{reverse}}
\newcommand{\Length}{\text{length}}
\newcommand{\Map}{\text{map}}
\newcommand{\Foldl}{\text{foldl}}
\newcommand{\Foldr}{\text{foldr}}

\section{Conjecture Generation}

\begin{figure}
  \begin{equation*}
    \begin{split}
      \forall a. \Nil            &: \ListA                                  \\
      \forall a. \Cons           &: a \rightarrow \ListA \rightarrow \ListA \\
      \Head(\Cons(x, xs))        &= x                                       \\
      \Tail(\Cons(x, xs))        &= xs                                      \\
      \Append(\Nil,         ys)  &= ys                                      \\
      \Append(\Cons(x, xs), ys)  &= \Cons(x, \Append(xs, ys))               \\
      \Reverse(\Nil)             &= \Nil                                    \\
      \Reverse(\Cons(x, xs))     &= \Append(\Reverse(xs), \Cons(x, \Nil))   \\
      \Length(\Nil)              &= \Zero                                   \\
      \Length(\Cons(x, xs))      &= \Succ (\Length(xs))                     \\
      \Map(f, \Nil)              &= \Nil                                    \\
      \Map(f, \Cons(x, xs))      &= \Cons(f(x), \Map(f, xs))                \\
      \Foldl(f, x, \Nil)         &= x                                       \\
      \Foldl(f, x, \Cons(y, ys)) &= \Foldl(f, f(x, y), ys)                  \\
      \Foldr(f, \Nil,         y) &= y                                       \\
      \Foldr(f, \Cons(x, xs), y) &= f(x, \Foldr(f, xs, y))
    \end{split}
  \end{equation*}
  \caption{A simple theory defining a $\List$ type and some associated
    operations, taken from~\cite{Johansson.Dixon.Bundy:conjecture-generation}.
    $\Zero$ and $\Succ$ are from a Peano encoding of the natural numbers.}
  \label{fig:list_theory}
\end{figure}

We focus on the task of generating conjectures from some given mathematical
theory. This theory may be, for example, a scientific model, an executable
computer program or simply an object of mathematical curiosity. A theory
defines (perhaps implicitly) a particular system of logic in which to work (for
example higher-order logic), along with a finite set of \emph{definitions} (like
those in figure~\ref{fig:list_theory}) which assign structure or meaning to
particular terms. Theoretically, this is all of the information we need to
derive the truth or falsity of all decidable statements involving these terms,
for example by enumerating all valid proofs. However, due to the exponential
search spaces involved, such brute-force enumeration is infeasible in practice;
plus the resulting proofs would be difficult to comprehend and the processed
statements would be overwhelming in number and mostly uninteresting in content.

The impracticality of enumeration demonstrates that a statement being ``true''
or ``provable'' is not \emph{sufficient} to warrant the (limited) attention of a
human operator. We would also argue that it is not \emph{necessary}, since there
are many statements considered to be interesting whose truth is either unknown,
or which later turned out to be false. We thus focus on the automated production
of \emph{interesting conjectures}, with the primary focus being on efficient
generation and interest to the user.

As an example, we use the theory of lists shown in
figure~\ref{fig:list_theory}. These definitions are so widely used in
software that they appear in the standard libraries of many programming
languages. An example of a conjecture involving these definitions is the
following universally quantified equation:

\begin{equation} \label{eq:mapreduce}
  \forall f. \forall xs. \forall ys.
    \Map(f, \Append(xs, ys)) = \Append(\Map(f, xs), \Map(f, ys))
\end{equation}

Equation~\ref{eq:mapreduce} states that combining small lists into a larger one
using $\Append$ then transforming the elements with $\Map$ is the same as
transforming the elements of the small lists then combining. This is interesting
due to its applicability as an optimisation: if a program calls $\Map$ with an
expensive transformation $f$, we can divide up the work across multiple machines
in parallel (this is the basis of the ``map/reduce'' programming paradigm).
Whether or not this equation holds depends on the choice of logical framework,
and hence (in the case of software) on the semantics of the programming
language: in particular this equation and optimisation are invalid for systems
with strict evaluation order and side-effects (common in imperative programming)
since any error/exception in the calculation of $ys$ will cause the surrounding
calls to abort; on the left-hand-side this will abort $\Append$ then $\Map$, so
$f$ will never be executed; on the right-hand-side there is an additional
$\Map(f, xs)$ call which will \emph{not} be aborted, which may execute $f$ many
times and trigger arbitrary observable effects.


\section{Machine Learning}

Machine learning (ML) is a sub-field of artificial intelligence (AI) which
emphasises the use of statistical methods to find patterns and optimise
functions, in contrast with symbolic reasoning approaches like automated theorem
proving (commonly referred to as ``Good Old-Fashioned AI'').

The machine learning umbrella spans many techniques and applications, but one
common distinction is between \emph{supervised} and \emph{unsupervised} tasks.
In supervised learning we allow the system to perform some behaviour, then an
external process assesses whether it behaved well or poorly (akin to Justice
Potter Stewart's test for obscenity: ``I know it when I see it''). The system's
parameters are optimised to try and score highly on the test. Common supervised
learning tasks include function approximation using input/output examples (e.g.
classifying scanned images of numerals) where the test is the proportion of
example inputs which got the correct output, and reinforcement learning where
the system chooses from a set of ``actions'', receives a pair of ``observation''
and ``reward'' (test score) in response, and must accumulate the most reward.

Unsupervised learning does not have the same testable ``success'' criteria as
supervised learning. Instead, the goal is to find patterns and form models of
the target domain, which is more ambitious and open-ended than supervised
learning but also much harder to define, measure and optimise. Examples of this
are prediction and compression tasks such as auto-encoders
(learned compressor/decompressor pairs, rewarded by how well they preserve data
through a ``round trip'' compression-then-decompression) and adversarial or
co-evolution tasks (coupled systems where each is rewarded by exploiting
mis-prediction in the others).

We focus on the application of statistical machine learning algorithms to
\emph{languages}, such as logical statements or programming languages, This is
made particularly difficult due to the ubiquity of recursive structures, from
inductive datatypes, to recursive function definitions; from theorem statements,
to proof objects. Such nested structures may extend to arbitrary depth, which
makes them difficult to represent in the fixed numerical terms expected by most
machine learning algorithms.

\subsection{Feature Extraction}

\emph{Feature extraction} is a common pre-processing step for machine learning
(ML). Rather than feeding ``raw'' data straight into our ML algorithm, we only
learn a sample of \emph{relevant} details, known as \emph{features}. This has
two benefits:

\begin{itemize}
\item \emph{Feature vectors} (ordered sets of features) are chosen to be more
  compact than the data they're extracted from: feature extraction is
  \emph{lossy compression}. This reduces the size of the ML problem, improving
  efficiency (e.g. running time).
\item We avoid learning irrelevant details such as the encoding system used,
  improving \emph{data} efficiency (the number of samples required to spot a
  pattern).
\item Feature vectors are designed to have a fixed size, i.e. they will all have
  length (or \emph{dimension}) $d$. Many machine learning algorithms only work
  with inputs of a uniform size; feature extraction allows us to use these
  algorithms in domains where the size of each input is not known, may vary or
  may even be unbounded. For example, element-wise comparison of feature vectors
  is trivial (compare the $i$th elements for $1 \leq i \leq d$); for expressions
  this is not so straightforward, as their nesting may give rise to very
  different shapes.
\item Unlike our expressions, which are discrete, we can continuously transform
  one feature vector into another. This allows operations like averaging,
  interpolation and gradient \emph{gradient descent} to be used, which are not
  suitable for discrete terms.
\end{itemize}

As an example, say we want to learn relationships between the following program
fragments:

\begin{haskell}
data Maybe a = Nothing | Just a

data Either a b = Left a | Right b
\end{haskell}

We might hope our algorithm discovers relationships like:

\begin{itemize}
  \item Both are valid Haskell code.
  \item Both describe datatypes.
  \item Both datatypes have two constructors.
  \item \hs{Either} is a generalisation of \hs{Maybe} (we can define \hs{Maybe a
      = Either () a} and \hs{Nothing = Left ()}).
  \item There is a symmetry in \hs{Either}: \hs{Either a b} is equivalent to
    \hs{Either b a} if we swap occurences of \hs{Left} and \hs{Right}.
  \item It is trivial to satisfy \hs{Maybe} (using \hs{Nothing}).
  \item It is not trivial to satisfy \hs{Either}; we require an \hs{a} or a
    \hs{b}.
\end{itemize}

However, this is too optimistic. Without our domain-knowledge of Haskell, an ML
algorithm cannot impose any structure on these fragments, and will treat them as
strings of bits. Our high-level hopes are obscured by low-level details: the
desirable patterns of Haskell types are mixed with undesirable patterns of ASCII
bytes, of letter frequency in English words, and so on.

In theory we could throw more computing resources and data at a problem (this is
precisely what \emph{deep learning} techniques do), but available hardware and
corpora are always limited. Instead, feature extraction lets us narrow the ML
problem to what we, with our domain knowlege, consider important.

There is no \emph{fundamental} difference between raw representations and
features: the identity function is a valid feature extractor. Likewise, there is
no crisp distinction between feature extraction and machine learning: a
sufficiently-powerful learner doesn't require feature extraction, and a
sufficiently-powerful feature extractor doesn't require any learning!
\footnote{Consider a classification problem, to assign a label $l \in L$ to each
  input. If we only extract a single feature $f \in L$, we have solved the
  classification problem without using a separate learning step.}

Rather, the terms are distinguished for purely \emph{practical} reasons: by
separating feature extraction from learning, we can distinguish straightforward,
fast data transformation (feature extraction) from complex, slow statistical
analysis (learning). This allows for modularity, separation of concerns, and in
particular allows ``off-the-shelf'' ML to be re-used across a variety of
different domains. Even with no domain knowledge, feature extraction can still
be used to improve efficiency by compressing the input (for example using
\emph{autoencoding}).

\subsubsection{Truncation and Padding}

The simplest way to limit the size of recursive structures is to truncate
anything larger than a particular size (and pad anything smaller). This is the
approach taken by ML4PG~\cite{journals/corr/abs-1302-6421}, which limits itself
to trees with at most 10 levels and 10 elements per level; each tree is
converted to a $30 \times 10$ matrix (3 values per tree node) and learning
takes place on these normalised representations.

Truncation is unsatisfactory in the way it balances \emph{data} efficiency with
\emph{time} efficiency. Specifically, truncation works best when the input data
contains no redundancy and is arranged with the most significant data first (in
a sense, it is ``big-endian''). The less these assumptions hold, the less we can
truncate. Since many ML algorithms scale poorly with input size, it is important
to truncate aggressively, but there may be useful information deep in the trees
which would be discarded.

\subsubsection{Dimension Reduction}

A more sophisticated approach to the problem of reducing input size is to view
it as a \emph{dimension reduction} technique: our inputs can be modelled as
points in a high-dimensional space, which we want to project into a
lower-dimensional space.

Truncation is a trivial dimension reduction technique: take the first $N$
coordinates. More sophisticated projection functions, such as Principle
Component Analysis (PCA) consider the \emph{distribution} of the points, and
project with the hyperplane which preserves as much of the variance as possible
(or, equivalently, reduces the \emph{mutual information} between the points).

Dimension reduction can also be approached as an unsupervised ML task (even if
the data are labelled for a supervised
task~\cite{Oveisi.Oveisi.Erfanian.ea:2012}). One popular approach is the
\emph{autoencoder}, which transforms training data into smaller feature vectors
and back again, whilst trying to minimise reconstruction error. Such compressed
representations can be used as feature vectors for subsequent learning
tasks. More semantically-meaningful features can be found using
\emph{variational autoencoding}~\cite{kingma2013auto}, which add a stochastic
sampling step before decoding, forcing similar data to produce similar vectors.

Since these techniques are effectively ML algorithms in their own right, they
suffer some of the same difficulties we encounter with recursive expressions:

\begin{itemize}
  \item They operate \emph{offline}, requiring all input points up-front
  \item All input points must have the same dimensionality
\end{itemize}

Sophisticated dimension reduction is still useful for \emph{compressing} large,
redundant features into smaller, information-dense representations, and as such
provides a good complement to more flexible but less informative techniques. For
example, we might truncate (and pad) tree structures into large, sparse vectors
of a fixed size; then use PCA to pick only a few of the most informative
dimensions to constitute our resulting feature vectors.

The requirement for offline ``batch'' processing is more difficult to overcome,
since any learning performed for feature extraction will interfere with the
core learning algorithm that's consuming these features (this is why deep
learning is often done greedily).

\subsubsection{Distributed Representations}

A \emph{distributed representation} does not isolate each semantic feature (for
example, presence of absence of some lexeme) to separate, independent elements
of a feature vector. Instead, features are represented as \emph{patterns} across
multiple elements: each feature uses many elements and each element contributes
to many features~\cite{hinton1984distributed}. Such encodings provide simple
mechanisms to represent \emph{combinations} of features (combining their
patterns element-wise), \emph{partial} features (presence of the pattern across
a subset of its elements) and \emph{generalisation} (patterns which are present
as sub-patterns of multiple features).

The compressed representations learned by autoencoders are distributed
representations, and are usually continuous. A discrete alternative is to use
\emph{hashing}, which can also be learned (e.g. using \emph{random
forests}~\cite{vens2011random}).

Particularly relevant to our work is the ability for distributed representations
to approximate recursive data such as syntax trees. This is achieved by defining
(or learning) functions to \emph{combine} the representations of parts into
those of composite objects, to \emph{test} whether or not a given feature vector
represents a composite object, and to \emph{project} composite representations
back into those of their constituents. For example, if $c$ and $d$ are the
compression and decompression routines of an auto-encoder for non-recursive
data, we can define an auto-encoder ($c'$ and $d'$) for binary trees in terms
of a combining function $\bigoplus$, a predicate $\texttt{node?}$ and projection
functions $\pi_{left}$ and $\pi_{right}$ as follows:

\begin{align*}
  c'(x) &= \begin{cases}
             c'(a) \bigoplus c'(b) & \text{if $x = \texttt{node}(a, b)$} \\
             c(x)                  & \text{otherwise}
           \end{cases} \\
  d'(x) &= \begin{cases}
             \texttt{node}(d'(\pi_{left }(x)),
                           d'(\pi_{right}(x))) & \text{if \texttt{node?}(x)} \\
             d(x)                              & \text{otherwise}
           \end{cases}
\end{align*}

With a suitable distance function between trees (e.g. edit distance), we can
learn $c$, $d$, \texttt{node?} and $\bigoplus$ together in an unsupervised
manner, by minimising the round trip error on a training corpus. The resulting
$c'$ function can then directly encode recursive data to fixed-size feature
vector. Analogous constructions can be applied to different recursive
structures.

\subsection{Sequences}

To handle input of variable size, such as natural or formal language, research
attention has been given to the handling of \emph{sequences}. This is a lossless
approach, which splits the input into fixed-size \emph{chunks} (e.g. splitting
text into a sequence of characters), which are fed into an appropriate ML
algorithm one at a time. The sequence is terminated by a sentinel; an
``end-of-sequence'' marker which, by construction, is distinguishable from the
data chunks. Tradeoffs can be made between chunk size and sequence length; for
example, treating code as a sequence of
characters~\cite{cummins2017synthesizing} or lexer tokens~\cite{cummins2017end}.

Not all ML algorithms can be adapted to accept sequences. One notable approach
is to use \emph{recurrent (artificial) neural networks} (RNNs), which allow
arbitrary connections between nodes, including cycles. Compared to
\emph{feed-forward} (FF) neural networks, which are acyclic, the \emph{future
  output} of a RNN may depend arbitrarily on its \emph{past inputs} (in fact,
RNNs are universal computers).

The main problem with RNNs, compared to the more widely-used FF approach, is the
difficulty of training them. If we extend the standard backpropagation algorithm
to handle cycles, we get the \emph{backpropagation through time}
algorithm~\cite{werbos1990backpropagation}. However, this suffers a problem
known as the \emph{vanishing gradient}: error values decay exponentially as they
propagate back through the cycles, which prevents effective learning of delayed
dependencies, undermining the main advantage of RNNs. The vanishing gradient
problem is the subject of current research, with countermeasures including
\emph{neuroevolution} (using evolutionary computation techniques to complement
or replace backpropagation) and \emph{gated neurons} (which are selectively
updated, and hence resilient to spurious overwriting; the most common example
being \emph{long short-term memory} (LSTM)~\cite{hochreiter1997long}).

An alternative to introducing recurrent connections is the use of
\emph{attention}~\cite{vaswani2017attention}. In a (typically FF) neural network
with attention, the network is given a chunk of the input and, in addition to
its usual output, also produces some ``attention'' information specifying which
chunks of the input should be provided next. The network is run again, but with
the input adjusted using this attention information; more output is produced,
along with new attention information, and so on. Such networks do not need to
``remember'' long-distance relationships, whether by recurrent connections or
gated nodes; they instead use the same input over and over again, but use their
``attention'' to emphasise those parts which are relevant to upcoming
output and hence should be taken into account when computing it.

\subsubsection{Recursive Structure}

Using sequences to represent recursive data is problematic: if we want our
learning algorithm to exploit structure (such as the depth of a token), it will
need to discover how to parse the sequences for itself, which is wasteful since
we already have a parsed representation to begin with. The
\emph{back-propagation through structure} approach~\cite{goller1996learning} is
a more direct solution to this problem, using FF neural networks to learn
recursive distributed representations~\cite{pollack1990recursive} which
correspond to the recursive structure of the inputs. Such distributed
representations can also be used for sequences, which we can use to encode
sub-trees when the branching factor of nodes is not
uniform~\cite{kwasny1995tail}. Recursive structure can also be gated inside
LSTM cells~\cite{zhu2015long}.

A simpler alternative for generating recursive distributed representations is to
use circular convolution \cite{conf/ijcai/Plate91}. Although promising results
are shown for its use in \emph{distributed tree kernels}
\cite{zanzotto2012distributed}, our preliminary experiments in applying
circular convolution to functional programming expressions found most of the
information to be lost in the process; presumably as the expressions are too
small.

\emph{Kernel methods} have also been applied to structured information, for
example in \cite{Gartner2003} the input data (including sequences, trees and
graphs) are represented using \emph{generative models}, such as hidden Markov
models, of a fixed size suitable for learning. Many more applications of kernel
methods to structured domains are given in \cite{bakir2007predicting}, which
could be used to learn more subtle relations between expressions than recurrent
clustering alone.


 One common approach to this problem is
to represent such structures as \emph{sequences}. \emph{Recurrent neural
  networks} (RNNs) are a popular choice for processing sequences, especially
when combined with mechanisms such as \emph{long short-term memory} (LSTM) for
preserving information across long sequences \cite{hochreiter1997long}. Such
systems have been used, for example, to parse and execute computer programs
\cite{zaremba2014learning}. However, learning to parse sequences seems
inefficient considering that we already have correctly-parsed ASTs.

Whilst neural networks have been applied directly to recursive structures
\cite{goller1996learning}, including using LSTM \cite{zhu2015long}, a more
popular approach is to use \emph{kernel methods}
\cite{bakir2007predicting}. These are promising as a more principled alternative
to our current hand-crafted translation of ASTs to vectors.

% FIXME: Include clustering

