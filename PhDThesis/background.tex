\section{Background}

\subsection{Conjecture Generation}

\begin{figure}
  \begin{equation*}
    \begin{split}
      \forall a. \Nil            &: \ListA                                  \\
      \forall a. \Cons           &: a \rightarrow \ListA \rightarrow \ListA \\
      \Head(\Cons(x, xs))        &= x                                       \\
      \Tail(\Cons(x, xs))        &= xs                                      \\
      \Append(\Nil,         ys)  &= ys                                      \\
      \Append(\Cons(x, xs), ys)  &= \Cons(x, \Append(xs, ys))               \\
      \Reverse(\Nil)             &= \Nil                                    \\
      \Reverse(\Cons(x, xs))     &= \Append(\Reverse(xs), \Cons(x, \Nil))   \\
      \Length(\Nil)              &= \Zero                                   \\
      \Length(\Cons(x, xs))      &= \Succ (\Length(xs))                     \\
      \Map(f, \Nil)              &= \Nil                                    \\
      \Map(f, \Cons(x, xs))      &= \Cons(f(x), \Map(f, xs))                \\
      \Foldl(f, x, \Nil)         &= x                                       \\
      \Foldl(f, x, \Cons(y, ys)) &= \Foldl(f, f(x, y), ys)                  \\
      \Foldr(f, \Nil,         y) &= y                                       \\
      \Foldr(f, \Cons(x, xs), y) &= f(x, \Foldr(f, xs, y))
    \end{split}
  \end{equation*}
  \caption{A simple theory defining a $\List$ type and some associated
    operations, taken from~\cite{Johansson.Dixon.Bundy:conjecture-generation}.
    $\Zero$ and $\Succ$ are from a Peano encoding of the natural numbers.}
  \label{figure:list_theory}
\end{figure}

We focus on the task of generating conjectures from some given mathematical
theory. This theory may be, for example, a scientific model, an executable
computer program or simply an object of mathematical curiosity. A theory
defines (perhaps implicitly) a particular system of logic in which to work (for
example higher-order logic), along with a finite set of \emph{definitions} (like
those in figure~\ref{figure:list_theory}) which assign structure or meaning to
particular terms. Theoretically, this is all of the information we need to
derive the truth or falsity of all decidable statements involving these terms,
for example by enumerating all valid proofs. However, due to the exponential
search spaces involved, such brute-force enumeration is infeasible in practice;
plus the resulting proofs would be difficult to comprehend and the processed
statements would be overwhelming in number and mostly uninteresting in content.

The impracticality of enumeration demonstrates that a statement being ``true''
or ``provable'' is not \emph{sufficient} to warrant the (limited) attention of a
human operator. We would also argue that it is not \emph{necessary}, since there
are many statements considered to be interesting whose truth is either unknown,
or which later turned out to be false. We thus focus on the automated production
of \emph{interesting conjectures}, with the primary focus being on efficient
generation and interest to the user.

As an example, we use the theory of lists shown in
figure~\ref{figure:list_theory}. These definitions are so widely used in
software that they appear in the standard libraries of many programming
languages. An example of a conjecture involving these definitions is the
following universally quantified equation:

\begin{equation} \label{eq:mapreduce}
  \forall f. \forall xs. \forall ys.
    \Map(f, \Append(xs, ys)) = \Append(\Map(f, xs), \Map(f, ys))
\end{equation}

Equation~\ref{eq:mapreduce} states that combining small lists into a larger one
using $\Append$ then transforming the elements with $\Map$ is the same as
transforming the elements of the small lists then combining. This is interesting
due to its applicability as an optimisation: if a program calls $\Map$ with an
expensive transformation $f$, we can divide up the work across multiple machines
in parallel (this is the basis of the ``map/reduce'' programming paradigm).
Whether or not this equation holds depends on the choice of logical framework,
and hence (in the case of software) on the semantics of the programming
language: in particular this equation and optimisation are invalid for systems
with strict evaluation order and side-effects (common in imperative programming)
since any error/exception in the calculation of $ys$ will cause the surrounding
calls to abort; on the left-hand-side this will abort $\Append$ then $\Map$, so
$f$ will never be executed; on the right-hand-side there is an additional
$\Map(f, xs)$ call which will \emph{not} be aborted, which may execute $f$ many
times and trigger arbitrary observable effects.

\section{Machine Learning}

Machine learning (ML) is a sub-field of artificial intelligence (AI) which
emphasises the use of statistical methods to find patterns and optimise
functions, in contrast with symbolic reasoning approaches like automated theorem
proving (commonly referred to as ``Good Old-Fashioned AI'').

The machine learning umbrella spans many techniques and applications, but one
common distinction is between \emph{supervised} and \emph{unsupervised} tasks.
In supervised learning we allow the system to perform some behaviour, then an
external process assesses whether it behaved well or poorly (akin to Justice
Potter Stewart's test for obscenity: ``I know it when I see it''). The system's
parameters are optimised to try and score highly on the test. Common supervised
learning tasks include function approximation using input/output examples (e.g.
classifying scanned images of numerals) where the test is the proportion of
example inputs which got the correct output, and reinforcement learning where
the system chooses from a set of ``actions'', receives a pair of ``observation''
and ``reward'' (test score) in response, and must accumulate the most reward.

Unsupervised learning does not have the same testable ``success'' criteria as
supervised learning. Instead, the goal is to find patterns and form models of
the target domain, which is more ambitious and open-ended than supervised
learning but also much harder to define, measure and optimise. Examples of this
are prediction and compression tasks such as auto-encoders
(learned compressor/decompressor pairs, rewarded by how well they preserve data
through a round-trip) and adversarial or co-evolution tasks (coupled systems
where each is rewarded by exploiting mis-prediction in the others).
