\chapter{Background}

% Background should cover approaches to automated reasoning about software, and
% formal systems in general. We can highlight how the language/representation
% imposes invariants and guarantees on what we can know, and how we can extend
% those guarantees with features like purity (prevents behaviour depending on
% ambient environment/state), lazy evaluation (prevents problems with
% non-termination), sound type systems (limits the space of possible values for
% each expression), etc. and therefore how Haskell is a natural vehicle for us to
% use. Giving background on Haskell naturally leads to descendants like Agda and
% Idris, which are also interactive theorem provers; which lets us discuss
% Isabelle, Coq, HOL, etc., how they're difficult for existing reasoning
% algorithms to handle (e.g. resolution, which works well in first-order logics),
% approaches to helping

\section{Haskell}
\label{sec:haskell}

\begin{figure}
  \centering
  \begin{haskell}
-- A datatype with two constructors
data Bool = True | False

-- A recursive datatype (S requires a Nat as argument)
data Nat = Z | S Nat

-- A polymorphic (AKA "generic") datatype
data List t = Nil | Cons t (List t)

-- Arithmetic functions

plus :: Nat -> Nat -> Nat
plus    Z  y = y
plus (S x) y = S (plus x y)

mult :: Nat -> Nat -> Nat
mult    Z  y = Z
mult (S x) y = plus y (mult x y)

-- Mutually-recursive functions

odd :: Nat -> Bool
odd    Z  = False
odd (S n) = even n

even :: Nat -> Bool
even    Z  = True
even (S n) = odd n
  \end{haskell}
  \caption{Haskell code, defining datatypes and functions involving them (note
    that \hs{--} introduces a comment). \hs{Bool} is the Booleans, \hs{Nat} is a
    Peano encoding of the natural numbers and \hs{List t} are polymorphic lists
    with elements of type \hs{t}. Juxtaposition denotes a function call, e.g.
    \hs{f x}, and functions may be defined by pattern-matching (case analysis)
    on their arguments. \hs{A :: B} is an optional type annotation, stating that
    value \hs{A} has type \hs{B}. \hs{A -> B} is the type of functions from
    \hs{A} to \hs{B}.}
  \label{fig:haskellexample}
\end{figure}

We mostly focus our attention on the Haskell programming language, both as an
implementation vehicle and as our representation of functions, properties, etc.
to explore. We choose Haskell since it combines formal, logical underpinnings
which aid reasoning (compared to more popular languages like Java or C), yet it
is still popular enough to sustain a rich ecosystem of tooling and a large
corpus of existing code (compared to more formal languages like Coq or
Isabelle). Haskell is well-suited to programming language research; indeed, this
was a goal of the language's creators \cite{marlow2010haskell}. An example of
Haskell code is given in Figure~\ref{fig:haskellexample}.

Like most \emph{functional} programming languages, Haskell builds upon
$\lambda$-calculus, with extra features such as a strong type system and
``syntactic sugar'' to improve readability. The following features make Haskell
especially useful for our purposes, although many are also present in other
languages such as StandardML and Coq (which we also use, but only when needed
for compatibility with prior work):

\begin{description}

\item{Functional}: Most control flow in Haskell (all except
  \emph{pattern-matching}, described below) is performed by function abstraction
  and application , which we can reason about using standard rules of inference
  such as \emph{modus ponens}. For example, reading from external sources (like
  environment variables) is \emph{impure}, so the only way to parameterise a
  value of type \hs{B} with a value of type \hs{A} is using a function of type
  \hs{A -> B}. Conversely, applying a function of type \hs{A -> B} to a value of
  type \hs{A} can only ever produce a value of type \hs{B} (it may instead crash
  the program or loop forever, but neither of those are \emph{observable}, i.e.
  we can't branch on them).

\item{Pure}: Execution of actions (e.g. reading or writing files) is separate to
  evaluation of expressions; hence our reasoning can safely ignore complicated
  external and non-local interactions. Purity ensures \emph{referential
    transparency}: references can be substituted for their referent with
  no change in semantics (as in standard mathematical practice). This implies
  that calling a function twice with the same argument must produce the same
  result (modulo crashes or infinite loops). For example, consider applying the
  function \hs{pair x = (x, x)} to some arbitrary function call \hs{foo y}. The
  references \hs{x} in the resulting pair \hs{(x, x)} both refer to \hs{foo y}
  so, by referential transparency, can be substituted to get
  \hs{(foo y, foo y)}. Both members of \hs{(x, x)} are identical, by definition;
  hence, to preserve the semantics, both (and indeed \emph{all}) calls to
  \hs{foo y} must also be identical. This holds regardless of what we choose for
  \hs{foo} and \hs{y}, and implies that function behaviour cannot depend on
  external interactions or non-deterministic processes (which may change between
  calls).

\item{Statically Typed}: Expressions are constrained by \emph{types}, which can
  be used to eliminate unwanted combinations of values (for example
  \hs{plus True} is not a valid expression), and hence reduce search spaces;
  \emph{static} types can be deduced syntactically, without having to execute
  the code. The type of Haskell expressions can also be \emph{inferred}
  automatically.\footnote{Except for certain cases, such as those caused by the
    \emph{monomorphism restriction}~\cite{marlow2010haskell}[$\S$4.5.5]}

\item{Curried}: All functions in Haskell take a single argument (as in
  $\lambda$-calculus), which makes them easier to manipulate programatically.
  Currying allows multi-argument functions to be simulated, by accepting one
  argument and returning a function to accept the rest. The \hs{mult} function
  in Figure~\ref{fig:haskellexample} has type \hs{Nat -> Nat -> Nat} meaning it
  takes a \hs{Nat} as argument and returns a function of type
  \hs{Nat -> Nat}. Function calls are written with whitespace, so \hs{mult x y}
  calls the \hs{mult} function with the argument \hs{x}, then calls the result
  of that with the argument \hs{y}. This allows \emph{partial application} such
  as \hs{double = mult (S (S Z))}, but for our purposes it is important for
  unifying calling conventions. For example, in Javascript \hs{mult x y} would
  be either \js{mult(x, y)} or \js{mult(x)(y)}, and depends on the definition of
  \js{mult} (the problem compounds with more arguments). In Haskell there is no
  distinction between these forms.

\item{Non-strict}: ``Strict'' evaluation strategies evaluate the arguments of a
  function call before the function body; non-strict does the opposite. The
  Haskell standards do not specify a particular evaluation strategy, but they
  do require that it be non-strict (for efficiency, most implementations use
  \emph{lazy} evaluation to avoid duplicating work). Strictness can result in
  infinite loops and other errors which may be avoided by non-strictness,
  making the latter more useful for reasoning in the face of such errors.
  For example, given a pair of values \hs{(x, y)} and a projection function
  \hs{fst}, we might make the ``obvious''  conjecture that
  \hs{fst (x, y) = x}.\footnote{Incidentally, this is also a valid definition of
    the \hs{fst} function} This statement is true for non-strict languages, but
  \emph{not} for strict languages. Crucially, a strict language will attempt to
  calculate the value of \hs{y}, which may cause an infinite loop or other
  error; a non-strict language like Haskell will ignore \hs{y} since it isn't
  used in the body of the \hs{fst} function.

\item{Algebraic Data Types}: These provide a rich grammar for building up
  user-defined data representations, and an inverse mechanism to inspect these
  data by \emph{pattern-matching} (Haskell's other form of control flow). The
  \hs{Bool}, \hs{Nat} and \hs{List t} definitions in
  Figure~\ref{fig:haskellexample} are ADTs; whilst the functions use
  pattern-matching to branch on their first argument. For our purposes, the
  useful consequences of ADTs and pattern-matching include their amenability for
  inductive proofs and the fact they are \emph{closed}; i.e. an ADT's
  declaration specifies all of the normal forms for that type. This makes
  exhaustive case analysis trivial, which would be impossible for \emph{open}
  types (for example, consider classes in an object oriented language, where new
  subclasses can be introduced at any time).

\item{Parametricity}: This allows Haskell \emph{values} to be parameterised over
  \emph{type-level} objects; provided those objects are never inspected. This
  enables \emph{polymorphism}. The \hs{List t} type in
  Figure~\ref{fig:haskellexample} is an example: there are many useful functions
  involving lists which work in the same way regardless of the element type
  (e.g. getting the length, appending, reversing, etc.). An even simpler example
  is the polymorphic identity function \hs{id x = x}. The type of \hs{id} is
  \hs{forall t. t -> t}\footnote{The \hs{forall t.} is optional; type-level
    identifiers beginning with a lowercase letter are assumed to be universally
    quantified variables.}, which we can view as taking \emph{two} parameters:
  a type \hs{t} and a value of type \hs{t}. Only the latter argument appears in
  the definition (as \hs{x}), meaning that we can't use the type \hs{t} to
  determine the function's behaviour. Indeed, in the case of \hs{id} we can't
  branch on the value of \hs{x} either, since we don't know what type it might
  have (our definition must work \emph{for all} types \hs{t}); the only
  functions we can call on \hs{x} must also be polymorphic, and hence also
  incapable of branching. The type of \hs{id} states that it returns a value of
  type \hs{t}; without knowing what that type is, the only type-correct value it
  can return is the argument \hs{x}. Hence the \emph{type} of \hs{id} tells us
  everything about its behaviour, with this style of reasoning known as
  \emph{theorems for free}~\cite{wadler1989theorems}. Haskell definitions are
  commonly made polymorphic like this, to prevent incorrect implementations
  passing the type checker, e.g. \hs{fst :: (Nat, Nat) -> Nat} might return the
  wrong element, but \hs{fst :: (t1, t2) -> t1} can't.

\item{Type classes}: Along with their various extensions, type classes are
  interfaces which specify a set of operations over a type or other type-level
  object (like a \emph{type constructor}, e.g. \hs{List}). Many type classes
  also specify a set of \emph{laws} which their operations should obey but,
  lacking a simple mechanism to enforce this, laws are usually considered as
  documentation. As a simple example, we can define a type class \hs{Semigroup}
  with the following operation and an associativity law:

  \begin{center}
    \begin{haskell}
op :: forall t. Semigroup t => t -> t -> t
    \end{haskell}

    $\forall \text{\hs{x y z}}, \text{\hs{op x (op y z)}} =
                                \text{\hs{op (op x y) z}}$
  \end{center}

  The notation \hs{Semigroup t =>} is a \emph{type class constraint}, which
  restricts the possible types \hs{t} to only those which implement
  \hs{Semigroup}. \footnote{Alternatively, we can consider \hs{Semigroup t} as
    the type of ``implementations of \hs{Semigroup} for \hs{t}'', in which case
  \hs{=>} has a similar role to \hs{->} and we can consider \hs{op} to take
  \emph{four} parameters: a type \hs{t}, an implementation of \hs{Semigroup t}
  and two values of type \hs{t}. As with parameteric polymorphism, this extra
  \hs{Semigroup t} parameter is not available at the value level. Even if it
  were, we could not alter our behaviour by inspecting it, since Haskell only
  allows types to implement each type class in at most one way, so there would
  be no information to branch on.} There are many \emph{instances} of
  \hs{Semigroup} (types which may be substituted for \hs{t}), e.g. \hs{Integer}
  with \hs{op} performing addition. Many more examples can be found in the
  \emph{typeclassopedia} \cite{yorgey2009typeclassopedia}. This ability to
  constrain types, and the existence of laws, helps us reason about code
  generically, rather than repeating the same arguments for each particular pair
  of \hs{t} and \hs{op}.

\item{Equational}: Haskell uses equations at the value level, for definitions;
  at the type level, for coercions; at the documentation level, for typeclass
  laws; and at the compiler level, for ad-hoc rewrite rules. This provides us
  with many \emph{sources} of equations, as well as many possible \emph{uses}
  for any equations we might discover. Along with their support in existing
  tools such as SMT solvers, this makes equational conjectures a natural target
  for theory exploration.

\item{Modularity}: Haskell's module system allows definitions to be kept
  private. This mechanism allows modules to provide more guarantees than are
  available just in their types, by constraining the ways that values can be
  constructed. For example, the following module represents email addresses as a
  pair of \hs{String}s, one for the user part and one for the host:

  \begin{haskell}
-- Exports appear between the parentheses
module Email (Email(), at, render) where

data Email = E String String

render :: Email -> String
render (E user host) = user ++ "@" ++ host

-- if/then/else is sugar for pattern-matching a Bool
at :: String -> String -> Maybe Email
at user host = if user == "" || host == ""
                  then Nothing
                  else Just (E user host)
  \end{haskell}

  An \hs{Email} value can be constructed by passing any two \hs{String}s to
  \hs{E}, but \hs{E} is private (not exported). The \hs{at} function is
  exported, but only passes its arguments to \hs{E} iff they are not
  empty.\footnote{\hs{Maybe t} is a safer alternative to the \textsc{Null}
    construct of other languages~\cite{hoare2009null}. It is defined as
    \hs{data Maybe t = Nothing | Just t} and can be understood as an optional
    value, or a computation which may fail, or as a list with at most one
    element, or as a degenerate search tree with no
    backtracking~\cite{wadler1985replace}} Since this module never calls \hs{E}
  with empty \hs{String}s, and other modules must use \hs{at}, we're guaranteed
  that \emph{all} \hs{Email} values will have non-empty \hs{String}s. Such
  ``smart constructors'' can guarantee \emph{any} decidable property, at the
  cost of performing run-time checks on each invocation.\footnote{We can
    guarantee non-emptiness ``by construction'', without run-time checks or
    \hs{Maybe} wrappers, by changing the type to require at least one element.
    \hs{String} is equivalent to \hs{List Char}, with \hs{""} represented as
    \hs{Nil}. Changing \hs{Nil} to require a \hs{Char} would eliminate empty
    \hs{String}s, e.g. \hs{data NonEmpty t = Nil t | Cons t (NonEmpty t)}. We
    could also use a pair like \hs{data NonEmpty t = NE t (List t)} instead.
    Such precise types are often more desirable than smart constructors, but are
    less general since ad-hoc representations need to be invented to enforce
    each guarantee.}

\end{description}

Together, these features make Haskell code highly structured, amenable to
logical analysis and subject to many algebraic laws. However, as mentioned with
regards to type classes, Haskell itself is incapable of expressing or enforcing
these laws (at least, without difficulty \cite{lindley2014hasochism}). This
reduces the incentive to manually discover, state and prove theorems about
Haskell code, e.g. in the style of interactive theorem proving, as these results
may be invalidated by seemingly innocuous code changes. This puts Haskell in a
rather special position with regards to the discovery of interesting theorems;
namely that many discoveries may be available with very little work, simply
because the code's authors are focused on \emph{software} development rather
than \emph{proof} development. The same cannot be said, for example, of ITP
systems; although our reasoning capabilities may be stronger in an ITP setting,
much of the ``low hanging fruit'' will have already been found through the
user's dedicated efforts, and hence theory exploration would be less likely to
discover unexpected properties.

Other empirical advantages to studying Haskell, compared to other programming
languages or theorem proving systems, include:

\begin{itemize}
\item The large amount of Haskell code which is freely available online, e.g. in
  repositories like \hackage{}, with which we
  can experiment.

\item The existence of theory exploration systems such as \hipspec{},
  \quickspec{} and \speculate{}.

\item Related tooling we can re-use such as counterexample finders (\quickcheck{},
  \smallcheck{}, \smartcheck{}, \leancheck{},
  \hedgehog{}, etc.), theorem provers
  (e.g. \hip{}~\cite{rosen2012proving}), and other testing and
  term-generating systems like \mucheck{}~\cite{le2014mucheck},
  \magichaskeller{}~\cite{katayama2011magichaskeller} and
  \djinn{}~\cite{augustsson2005djinn}.

\item The remarkable amount of infrastructure which exists for working with
  Haskell code, including package managers, compilers, interpreters, parsers,
  static analysers, etc.
\end{itemize}

Further evidence of Haskell's suitability for theory exploration is given by the
fact that the state-of-the-art implementation for Isabelle/HOL, the
\hipster{}~\cite{Hipster} system, is actually implemented by translating
to Haskell and invoking \hipspec{}~\cite{claessen2013automating}.

\section{Property Checking}
\label{sec:propertychecking}

% TODO: Automated testing
% TODO: Agile
% TODO: Unit testing
% TODO:

Although unit testing is the de facto industry standard for quality assurance in
non-critical systems, the level of confidence it provides is rather low, and
totally inadequate for many (e.g. life-) critical systems. To see why, consider
the following Haskell function, along with some unit
tests\footnote{Haskell functions can be prefix or infix: prefix functions have
  alphanumeric names, like \hs{div} and \hs{factorial}, and appear in function
  calls before their arguments, e.g. \hs{div 10 5}; infix functions have
  non-alphanumeric names, like \hs{+} and \hs{==}, and appear in function calls
  between their first two arguments, e.g. \hs{4 + 3}. Prefix functions can
  appear infix using backticks, like \hs{10 `div` 5}, and infix functions can
  appear prefix using parentheses, like \hs{(+) 4 3}.}:

% BEGIN unit
\begin{haskell}
factorial 0 = 1
factorial n = n * factorial (n - 1)

fact_base      = factorial 0 == factorial 1
fact_increases = factorial 3 <= factorial 4
fact_div       = factorial 4 == factorial 5 `div` 5
\end{haskell}
% END unit

The intent of the function is to map an input $n$ to an output $n!$. The tests
check a few properties of the implementation, including the base case, that the
function is monotonically increasing, and a relationship between adjacent
outputs. However, these tests will \emph{not} expose a serious problem with the
implementation: it diverges on half of its possible inputs.

Haskell's built-in numeric types allow negative numbers, which this
implementation doesn't take into account. Whilst this is a rather trivial
example, it highlights a common problem: unit tests are insufficient to expose
incorrect assumptions. In this case, our assumption that numbers are positive
has caused a bug in the implementation \emph{and} limited the tests we've
written.

One way to avoid this problem, if we do manage to spot it, would make results
symmetric around zero, which can be checked by a \emph{regression test} such as:

% BEGIN neg
\begin{haskell}
fact_neg = factorial 1 == factorial (-1)
\end{haskell}
% END neg

The implementation can be modified by composing if with the \hs{abs}olute value
function:

% BEGIN abs
\begin{haskell}
factorial = original . abs
  where original 0 = 1
        original n = n * original (n - 1)
\end{haskell}
% END abs

However, this is \emph{still} not enough, since this function will also accept
fractional values\footnote{Since we only use generic numeric operations, the
  function will be polymorphic with a type of the form
  \hs{forall t. Num t => t -> t}, where \hs{Num t} constrains the type variable
  \hs{t} to be numeric.}, which will also cause it to diverge. This problem can
be fixed by restricting \hs{factorial}'s type with an explicit annotation, e.g.
to only whole numbers: \hs{factorial :: forall t. Integral t => t -> t}

Clearly, by choosing which scenarios to check, we are biasing the test suite
towards those cases we've already thought of, whilst neglecting the problems we
did not expect. Haskell offers a partial solution to this problem in the form of \emph{property checking}. Tools such as \quickcheck{} separate tests into three components: a \emph{property} to check, which unlike a unit test may contain
\emph{free variables}; a source of values to instantiate these free variables; and a
stopping criterion.

\subsection{\quickcheck{}}
\label{sec:quickcheck}

\quickcheck{}~\cite{claessen2011quickcheck} is the most widely used property
checking library for
Haskell~\footnote{According to downloads counted at
  \url{http://hackage.haskell.org/packages/browse}, and packages depending on
  it at \url{https://packdeps.haskellers.com/reverse/QuickCheck}, accessed on
  2019-05-21.}. Here is how we might restate our unit tests as \quickcheck{}
properties:

% BEGIN quickchecked
\begin{haskell}
fact_base        = factorial 0 == factorial 1

fact_increases, fact_div, fact_neg :: Int -> Bool  
fact_increases n = factorial n <= factorial (n + 1)
fact_div       n = factorial n == factorial (n + 1) `div` (n + 1)
fact_neg       n = factorial n == factorial (-n)
\end{haskell}
% END quickchecked

The free variables (all called \hs{n} in this case) are abstracted as function
parameters; these parameters are implicitly \emph{universally quantified},
i.e. we've gone from a unit test asserting $factorial(3) \leq factorial(4)$ to a
property asserting $\forall n, factorial(n) \leq factorial(n+1)$. Notice that
unit tests like \hs{fact_base} are valid properties; they just happen to assert
rather weak statements, since they contain no free variables.

To check these properties, \quickcheck{} treats closed terms (like \hs{fact_base})
just like unit tests: pass if they evaluate to \hs{True}, fail otherwise. For
open terms, a random selection of values are generated and passed in via the
function parameter; the results are then treated in the same way as closed
terms. The default stopping criterion for \quickcheck{} (for each test) is when a
single generated test fails, or when 100 generated tests pass.

The ability to state \emph{universal} properties in this way avoids some of the
bias we encountered with unit tests. In the \hs{factorial} example, this
manifests in two ways:

\begin{itemize}
\item We are forced to confront the generality of \hs{factorial}'s type, since
  \quickcheck{} cannot test polymorphic functions directly; they must be
  \emph{monomorphised} first (instantiated to a particular concrete type), to
  decide which data generator to use (i.e. which instance of \quickcheck{}'s
  \hs{Arbitrary} type class). This can be done either by specifying a generator
  directly (avoiding the type class mechanism), or by adding explicit type
  annotations, or by calling \quickcheck{}'s \hs{monomorphic}
  function.\footnote{\hs{monomorphic} picks \hs{Integer} for all polymorphic
    type variables. This na\"ive choice instantiates many common type classes
    like \hs{Num}, \hs{Show} and \hs{Eq}; variables whose constraints are not
    satisfied will fail to type check (e.g. \hs{Bounded}, since \hs{Integer} is
    a bignum).} This will bring the problem of fractional values. Instead of
  monomorphising the properties, we might choose to monomorphise \hs{factorial}

\item By default, \quickcheck{} picks a generator based on the \emph{type} of
  value to be generated: since \hs{Int} includes positive and negative values,
  the \hs{Int} generator will output both. This will expose the problem with
  negative numbers, which we weren't expecting.
\end{itemize}

Whilst property checking generalises and improves upon unit testing, the problem
of tests being biased towards expected cases remains, since the properties to be
checked are still manually specified. Property checking can be complemented by
\emph{theory exploration} (TE), which avoids this bias by \emph{discovering}
such properties; through a combination of brute-force enumeration, random
testing and (in the case of \hipspec{} and \hipster{}) automated theorem
proving. Property checkers like \quickcheck{} are an important component of TE
systems, since their data generators can the search process, and they can
prevent ``obvious'' falsehoods from being output by checking them first.

\section{Theory Exploration}
\label{sec:theoryexploration}

(Automated) Theory Exploration (TE) is the task of taking a \emph{signature}
of definitions \iffalse TODO: Alison Some tools generate their own definitions
\fi in some formal system (for example a programming language) and
automatically generating a set of formal statements (properties) involving
those definitions. These may be conjectures or theorems (proven either by
sending conjectures to an automated theorem prover, or by having the generating
procedure proceed in logically sound steps from a set of axioms). These
statements should also be
``interesting'' in some way, to rule out unhelpful trivialities such as
iterating a pattern over and over (e.g. \hs{x + 0 = x}, \hs{(x + 0) + 0 = x},
\dots).

\iffalse
It is beyond the scope of this paper
to define what makes a mathematical statement ``interesting'', but this problem
has been tackled extensively in the literature~\cite{colton2000notion}.
\fi
% END TOTO

The choice of appropriate methods can differ according to the underlying logic,
etc. that is in use: in this work we ground ourselves by applying TE to
discovering properties of pure functional software libraries, mostly in the
Haskell programming language. This also gives our results immediate utility to
software engineering. We focus on the generation of interesting conjectures; to
\emph{prove} such conjectures, we defer to existing tools (such as \hip{} and
\hipspec{}).

The method of conjecture generation is a key characteristic of any theory
exploration system, although all existing implementations rely on brute force
enumeration to some degree. We focus on \quickspec{}~\cite{QuickSpec}, which
conjectures equations about Haskell code. We have used version 1 of \quickspec{}
in our experiments, due to its availability, stability and tooling integration;
hence references to \quickspec{} are to version 1 unless stated otherwise. At
the time of writing \iffalse TODO: Alison: Include date \fi there is a
\quickspec{} version 2 available, which has a much improved generation procedure
which can be significantly faster than its predecessor. Whilst \quickspec{} 2
has advanced the state of the art in theory exploration, our preliminary
experience has found that it still suffers the scaling issues we identify in
this work\footnote{Not only did we experience failures with our own experiments,
  we also suffered memory exhaustion when trying to run some of the examples
  included with the \quickspec{} 2 source code.} Since \quickspec{} 2 has less
integration with the tooling we have used, we leave a more thorough analysis of
it (and related systems such as \speculate{}) for future work, pending the
necessary infrastructure changes that would require.

\begin{figure}
  \centering
  \begin{minted}{haskell}
    -- A datatype with two constructors (Note that -- introduces a comment)
    data Bool = True | False

    -- A recursive datatype (S requires a Nat as argument)
    data Nat = Z | S Nat

    -- A function turning a Bool into a Bool
    not :: Bool -> Bool
    not True  = False
    not False = True

    -- A pair of mutually recursive functions

    odd :: Nat -> Bool
    odd    Z  = False
    odd (S n) = even n

    even :: Nat -> Bool
    even    Z  = True
    even (S n) = odd n
  \end{minted}
  \caption{Haskell datatypes for booleans and natural numbers, followed by some
    simple function definitions (with type annotations).}
  \label{fig:haskellteexample}
\end{figure}

\quickspec{} (version 1) is written in the Haskell programming language, and
works with definitions which are also written in Haskell. A thorough description
of Haskell, including its suitability for theory exploration, can be found
in \S~\ref{sec:haskell}. For illustrative purposes, some simple Haskell
definitions are shown in Figure~\ref{fig:haskellteexample}. \quickspec{} uses
the following procedure to generate conjectures, which are both plausible (due
to testing on many examples) and potentially interesting (due to being mutually irreducible):

\begin{enumerate}
\item Given a signature $\Sigma$ of typed expressions and set of variables $V$,
  \quickspec{} generates a list $terms$ containing the expressions from
  $\Sigma$ (including functions), the variables from $V$ and type-correct
  function applications \hs{f x}, where \hs{f} and \hs{x} are elements of
  $terms$. To ensure the list is finite, function applications are only nested
  up to a specified depth (by default, 3).
\item The elements of $terms$ are grouped into equivalence classes, based on
  their type.
\item The equivalence of terms in each class is tested using \quickcheck{}:
  variables are instantiated to particular values, generated randomly, and the
  resulting closed expressions are evaluated and compared for equality.
\item If a class is found to have non-equal members, it is split up to separate
  those members.
\item The previous steps of variable instantiation and comparison are repeated
  until the classes stabilise (i.e. no differences have been observed for some
  specified number of repetitions).
\item For each class, one member is selected and equations are conjectured that
  it is equal to each of the other members.
\item A congruence closure algorithm is applied to these equations, to discard
  any which are implied by the others.
\end{enumerate}

Such conjectures can be used in several ways: they can be presented directly
to the user, sent to a more rigorous system like \hipspec{} or \hipster{} for
proving, or even serve as a background theory for an automated theorem
prover~\cite{claessen2013automating}.

As an example, we can consider a simple signature containing the expressions
from Figure~\ref{fig:haskellteexample}, and some suitable variables:

\begin{align*}
  \Sigma_{\hs{Nat}} &= \{\hs{Z},\ \hs{S},\ \hs{plus},\ \hs{mult},\ \hs{odd},
                       \ \hs{even}\} \\
  V_{\hs{Nat}}     &= \{\hs{a :: Nat},\ \hs{b :: Nat},\ \hs{c :: Nat}\}
\end{align*}

\quickspec{}'s enumeration of these terms will resemble the following:

\begin{align*}
  terms_{\hs{Nat}} = [& \hs{Z},\ \hs{S},\ \hs{plus},\ \hs{mult},\ \hs{odd},
                     \ \hs{even},\ \hs{a},\ \hs{b},\ \hs{c},\ \hs{S Z},
                     \ \hs{S a},\ \hs{S b}, \\
                     & \hs{S c},\ \hs{plus Z},\ \hs{plus a},\ \dots ]
\end{align*}

Notice that functions such as \hs{plus} and \hs{mult} are valid terms, despite
not being applied to any arguments. In addition, all Haskell functions are unary
(due to currying), which makes it valid to apply them one argument at a time as
we construct $terms_{\hs{Nat}}$.

\begin{figure}
  % To reproduce, run 'quickSpec nat' in haskell_example/src/QuickSpecExample.hs
  \begin{haskell}
                      plus a b = plus b a
                      plus a Z = a
             plus a (plus b c) = plus b (plus a c)
                      mult a b = mult b a
                      mult a Z = Z
             mult a (mult b c) = mult b (mult a c)
                  plus a (S b) = S (plus a b)
                  mult a (S b) = plus a (mult a b)
             mult a (plus b b) = mult b (plus a a)
                     odd (S a) = even a
                odd (plus a a) = odd Z
               odd (times a a) = odd a
                    even (S a) = odd a
               even (plus a a) = even Z
              even (times a a) = even a
    plus (mult a b) (mult a c) = mult a (plus b c)
  \end{haskell}
  \caption{Equations conjectured by \quickspec{} for the functions in Figure
    \ref{fig:haskellteexample}; after simplification.}
  \label{fig:qspecresult}
\end{figure}

The elements of $terms_{\hs{Nat}}$ will be grouped into five classes, one each
for \hs{Nat}, \hs{Nat -> Nat}, \hs{Nat -> Nat -> Nat}, \hs{Nat -> Bool} and
\hs{Bool}. As the variables \hs{a}, \hs{b} and \hs{c} are instantiated to
various randomly-generated numbers, the differences between these terms will
be discovered and the equivalence classes will be divided, until eventually
the equations in Figure \ref{fig:qspecresult} are conjectured.

Although complete, this enumeration approach is wasteful: many terms are
unlikely to appear in theorems, which requires careful choice by the user of
what to include in the signature. Here we know that addition and multiplication
are closely related, and hence obey many algebraic laws; arbitrary definitions
from a typical software library or proof development are unlikely to have be
related so strongly.

\iffalse
\quickspec{} (and \hipspec{}) is also compatible with Haskell's existing testing
infrastructure, such that an invocation of \texttt{cabal test} can run these
tools alongside more traditional QA tools like \quickcheck{}, \textsc{HUnit} and
\textsc{Criterion}.

In fact, there are similarities between the way a TE system like \quickspec{} can
generalise from checking \emph{particular} properties to \emph{inventing} new
ones, and the way counterexample finders like \quickcheck{} can generalise from
testing \emph{particular} expressions to \emph{inventing} expressions to
test. One of our aims is to understand the implications of this generalisation,
the lessons that each can learn from the other's approach to term generation,
and the consequences for testing and QA in general.
\fi

\iffalse
% TODO: Work this into the example

As an example, giving the definitions from Figure~\ref{fig:haskellteexample} to
\quickspec{}, along with suitable comparison functions, random data generators and
variables \hs{n :: Nat} and \hs{b :: Bool}, will give rise to an equivalence
class for each type \hs{Nat}, \hs{Bool}, \hs{Nat -> Nat}, \hs{Nat -> Bool} and
\hs{Bool -> Bool}. Testing will find unequal terms in some of these classes
(such as the \hs{Bool} class containing \hs{True}, \hs{False}, \hs{b},
\hs{not b}, \hs{odd n} and \hs{even n}, which are all mutually unequal), split
them up, and repeat until no more splits are found. Equations are then generated
between elements of non-singleton classes, such as $\{\text{\hs{odd n}},
\text{\hs{not (even n)}}, \text{\hs{not (not (odd n))}}, \ldots\}$, and reduced
to a minimal set (for example, discarding \hs{odd n == not (not (odd n))} as an
instance of the more general \hs{b == not (not b)}). The remaining equations
(including \hs{not (not b) == b} and \hs{not (odd n) == even n}, etc.) are
output, either for the user to peruse or for another system to process, like the
\hipspec{} automated theorem prover.

Note that we must be able to generate values of a type in order to include
variables of that type. Also, only equivalence classes whose element type can be
compared are split and conjectured about in this way; in particular, functions
(like \hs{odd} and \hs{even}) are first-class values, so they will be put in an
equivalence class, but they will not be subject to testing and comparison since
Haskell has no generic notion of function equality~\footnote{In
  fact \quickspec{} compares expressions using a total ordering rather than equality;
  this is even more restrictive than requiring a notion of equality, but reduces
  the number of required comparisons}. \quickspec{} can hence conjecture that two
functions are pointwise-equal (e.g. \hs{f x == g x}) but it cannot conjecture
that the functions themselves are equal (e.g. \hs{f == g}).
\fi

% TODO
%%%%% THE BELOW SEEMS TO BE INTERESTINGNESS?

%\begin{description}
%\item{Interestingness} Various alternative interestingness criteria have been
%  proposed, which we survey in \S \ref{sec:relatedwork}. Augmenting or replacing
%  the criteria may be useful, for example to distinguish useful relationships
%  from incidental coincidences; or to prevent surprising, insightful equations
%  from being discarded because they can be simplified.
%\end{description}

% Theory exploration is similar to \emph{experimental mathematics}

% - Relation to Science
%  - Testable/falsifiable hypotheses are like evaluable terms (or, more generally, conjectures which can be decided, using a reasonable amount of resources).
% - Relation to AI tasks: exploring surroundings, etc.

% - Statistics is another area that's less straightforward than normal numerical computing, since there is subjectivity and judgement involved in the answering of questions.

% Theory formation: Alison? Others.
% Theory exploration: Buchberger, Moa in Isabelle, Koen in Haskell. Others?
% Theorem proving: Well-trodden: first-order ATP, higher-order ITP, functional programming
% Communication: Latex, Wikis, APIs, communicating with aliens

% \section{Exploration in Theorem Proving}
% \label{sec:examples}

% Before exploring abstract definitions of interestingness, we can first consider
% some scenarios which arise during formal proof where we are forced to generate
% conjectures. An analysis of these situations, and the subsequent theorems they
% produce, will contribute towards an empirical justification for what is
% interesting (at least from a utilitarian point of view) and inform our later
% exploration of the literature.

% \subsection{Generalisation}

% \providecommand{\coq}[1]{\lstinline[language=ML]|#1|}

% When we \emph{generalise} a statement $S$, we obtain a new statement $S'$ of
% which $S$ is a special case. Although it seems counterintuitive, a generalised
% statement can sometimes be \emph{easier} to prove than the original. This arises
% often in inductive proofs, since the specific obligations which arise in the
% proof may be incompatible with the available inductive hypotheses.

% However, we cannot blindly generalise \emph{all} obligations we encounter, since
% \emph{over-generalising} results in obligations which are so strong that they
% are unprovable, or even false. We must therefore rely on heuristics to guide the
% generation of generalised conjectures, and hence perform a kind of exploration.

% An informative example is given by Boyer and Moore of the associativity of
% multiplication in ACL2 \cite{boyer1983proof}:

% $$(x * y) * z = x * (y * z)$$

% During the course of the proof, the following obligation arises:

% \begin{equation}
%   \tag{conc3}
%   (y + (x * y)) * z = (y * z) + ((x * y) * z)
%   \label{eq:conc3}
% \end{equation}

% ACL2 automatically generalises \eqref{eq:conc3} by replacing the repeated
% sub-term $x * y$ with a fresh variable $w$:

% \begin{equation}
%   \tag{conc4}
%   (y + w) * z = (y * z) + (w * z)
%   \label{eq:conc4}
% \end{equation}

% This generalised form is clearly the distributivity law for multiplication and
% addition, which can be proved separately to the original goal of
% associativity. It would not be controversial to claim that this distributivity
% law is interesting in its own right (relative to associativity, at least), in
% addition to its usefulness in making this proof go through.

% % TODO: Describe the ACL2 heuristics

% Generalisation also occurs frequently when reasoning about \emph{tail-recursive}
% definitions \cite{kapur2003automatic}. \footnote{A tail-recursive function can
%   be executed in constant space using a loop, whereas recursion in non-tail
%   positions may require a growing number of stack frames or nested closures. See
%   \S \ref{sec:auxiliarylemmas} for example definitions of each type.}

% \subsection{Analogy}

% One way to characterise the interestingness of a statement is by \emph{analogy}
% to existing interesting statements. By finding lemmas analogous to those of a
% different theory, we may be able to re-use tactics and other forms of
% meta-programming across both.

% Existing theory exploration systems have been successfully applied to this
% problem, however the use of pure exploration misses opportunities to
% \emph{focus} the search, since we know which lemmas are used in those theories
% where a technique succeeded. If we can find an analogy to map from such solved
% problems to our unsolved goal, we can infer the approximate form of the lemmas
% we require, and target these specifically.

% The approach taken by \textsc{ACL2(ml)} is to find lemmas which may be relevant
% to solving a goal $G$ by making analogies via unsupervised clustering
% \cite{Heras.Komendantskaya.Johansson.ea:2013}. These clusters are used in two
% ways:

% \begin{itemize}
% \item First, we use the cluster $C_G$ containing $G$ to identify analogous
%   theorems.

% \item For each theorem $T \in C_G \setminus \{G\}$, we consider those symbols
%   $S_T$ which occur in $T$ but not in $G$. Our analogous lemmas are those used
%   to prove $T$, mutated such that symbols $s \in S_T$ are replaced by members of
%   the cluster $C_s$ containing $s$.
% \end{itemize}

% The running examples for demonstrating \textsc{ACL2(ml)} are equivalence
% theorems for tail-recursive and non-tail-recursive calculations, as well as the
% effect of repeating certain list operations:

% \begin{itemize}

%   \item $\forall n, \texttt{natp}(n) \rightarrow \texttt{fact-tail}(n) = \texttt{fact}(n)$ where \texttt{natp} is the predicate that $n$ is a natural number, whilst \texttt{fact-tail} and \texttt{fact} are tail-recursive and non-tail-recursive implementations of factorial, respectively.

%   \item $\forall n, \texttt{natp}(n) \rightarrow \texttt{power-tail}(n) = \texttt{power}(n)$,  where \texttt{power-tail} and \texttt{power} calculate powers of 2.

%   \item $\forall n, \texttt{natp}(n) \rightarrow \texttt{fib-tail}(n) = \texttt{fib}(n)$,  where \texttt{fib-tail} and \texttt{fib} calculate fibonacci numbers.

%   \item $\forall x, \texttt{nat-listp}(x) \rightarrow \texttt{sort}(\texttt{sort}(x)) = \texttt{sort}(x)$, for list-of-natural-numbers predicate \texttt{nat-listp} and list-sorting function \texttt{sort}.

%   \item $\forall x, \texttt{true-listp}(x) \rightarrow \texttt{rev}(\texttt{rev}(x)) = x$, where \texttt{true-listp} ensures that $x$ is a valid singly-linked list structure and \texttt{rev} is list reversal.

%   \item $\forall x, \texttt{true-listp}(x) \rightarrow \texttt{int}(x, x) = x$, where \texttt{int} is the intersection of lists (i.e. a list of elements common to each).

% \end{itemize}

% \subsection{Auxiliary Lemmas} \label{sec:auxiliarylemmas}

% One consideration when generating conjectures is the difference between
% theorems, lemmas, corollaries, etc. From a logical point of view, these are all
% equivalent, and hence most proof assistants do not distinguish between
% them. However, their \emph{intention} may be different: in a sense, theorems are
% the interesting results; whilst lemmas are useful results, required for proving
% the theorems.

% Some systems, like Coq, allow users to \emph{label} each statement as being a
% \coq{Theorem}, a \coq{Lemma}, etc. despite their internal representations being
% the same. This shows us immediately that lemmas outnumber theorems; in the Coq
% standard library there are over five times as many lemmas as theorems
% \footnote{The latest version as of writing is \texttt{coq-8.4pl6} which, when
%   excluding comments, includes 1492 occurences of \coq{Theorem} and 7594 of
%   \coq{Lemma} in its \texttt{theories/} directory.}.

% % TODO: Analyse them

% % TODO: Theory exploration as lemma generation; give example from a HipSpec paper

% We can find a need for auxiliary lemmas, once again, in the context of
% tail-recursive functions. Consider proving the (pointwise) equality of the
% following Coq functions, defined for the Peano naturals \coq{Z} and \coq{S}:

% \begin{coqblock}
% Inductive Nat : Set := Z : Nat
%                      | S : Nat -> Nat.

% Fixpoint plus      (n m : Nat) := match n with
%                                       | Z    => m
%                                       | S n' => S (plus n' m)
%                                   end.

% Fixpoint plus_tail (n m : Nat) := match n with
%                                       | Z    => m
%                                       | S n' => plus_tail n' (S m)
%                                   end.
% \end{coqblock}

% \begin{haskell}
% Haskell equivalent:

% plus :: Nat -> Nat -> Nat
% plus      n  Z    = n
% plus      n (S m) = S (plus n m)

% plus_tail :: Nat -> Nat -> Nat
% plus_tail n  Z    = n
% plus_tail n (S m) = plus_tail (S n) m
% \end{haskell}

% Both of these functions implement addition, but the \coq{plus_tail} variant is
% tail-recursive. However, if we want to \emph{prove} that the definitions are
% (pointwise) equal, we run into difficulties. In particular, when the inductive
% step requires us to prove \coq{plus (S n) m = plus n (S m)} (which seems
% reasonable), we cannot make this go through using another inductive argument.

% \begin{coqblock}
% (* Solve equalities by beta-normalising both sides *)
% Ltac triv := try (simpl; reflexivity).

% (* Prove equivalence of plus and plus_tail *)
% Theorem equiv : forall n m, plus n m = plus_tail n m.
%   induction n; triv. (* Base case is trivial *)

%   (* Inductive case: plus (S n) m = plus_tail (S n) m *)
%   intro m.

%   (* Beta-reduce the right-hand-side (justification is trivial) *)
%   replace (plus_tail (S n) m) with (plus_tail n (S m)); triv.

%   (* Use induction hypothesis to replace plus_tail with plus *)
%   rewrite <- (IHn (S m)).
% \end{coqblock}

% Specifically, the \emph{conclusion} of a second inductive hypothesis is exactly
% the equation we need:

% \begin{coqblock}
% IHn' : (forall x, plus n' x = plus_tail n' x) -> plus (S n') m = plus n' (S m)
% \end{coqblock}

% Yet we cannot provide it with the argument it needs, as our original induction
% hypothesis is \emph{too specific} (i.e. it has too many \coq{S} constructors):

% \begin{coqblock}
% IHn : forall x, plus (S n') x = plus_tail (S n') x
% \end{coqblock}

% We are forced to abandon the proof, despite such a reasonable-looking
% intermediate goal.

% In fact, if we attempt to prove that goal \emph{separately}, we can use a
% straightforward argument by induction; even though it is actually
% \emph{stronger} due to the absence of the \coq{IHn} assumption. Using this
% separate result as a lemma, the pointwise equality is proven easily.

% \begin{coqblock}
% Lemma gen n m : plus (S n) m = plus n (S m).
%   induction n; triv. (* Base case is trivial *)

%   (* Move all S constructors outside *)
%   simpl. rewrite <- IHn. simpl.

%   (* Trivial *)
%   reflexivity.
% Defined.
% \end{coqblock}

% \begin{coqblock}
%   rewrite (gen n m).
%   reflexivity.
% Defined.
% \end{coqblock}

% For example theory of lists
\newcommand{\function}{\rightarrow}

\newcommand{\Zero}{\text{Z}}
\newcommand{\Succ}{\text{S}}
\newcommand{\plus}{\text{plus}}
\newcommand{\mult}{\text{times}}

\newcommand{\List}{\text{List}}
\newcommand{\ListA}{\text{List} \  a}
\newcommand{\Nil}{\text{Nil}}
\newcommand{\Cons}{\text{Cons}}
\newcommand{\Head}{\text{head}}
\newcommand{\Tail}{\text{tail}}
\newcommand{\Append}{\text{append}}
\newcommand{\Reverse}{\text{reverse}}
\newcommand{\Length}{\text{length}}
\newcommand{\Map}{\text{map}}
\newcommand{\Foldl}{\text{foldl}}
\newcommand{\Foldr}{\text{foldr}}

\section{Conjecture Generation}

\begin{figure}
  \begin{equation*}
    \begin{split}
      \forall a. \Nil            &: \ListA                                  \\
      \forall a. \Cons           &: a \rightarrow \ListA \rightarrow \ListA \\
      \Head(\Cons(x, xs))        &= x                                       \\
      \Tail(\Cons(x, xs))        &= xs                                      \\
      \Append(\Nil,         ys)  &= ys                                      \\
      \Append(\Cons(x, xs), ys)  &= \Cons(x, \Append(xs, ys))               \\
      \Reverse(\Nil)             &= \Nil                                    \\
      \Reverse(\Cons(x, xs))     &= \Append(\Reverse(xs), \Cons(x, \Nil))   \\
      \Length(\Nil)              &= \Zero                                   \\
      \Length(\Cons(x, xs))      &= \Succ (\Length(xs))                     \\
      \Map(f, \Nil)              &= \Nil                                    \\
      \Map(f, \Cons(x, xs))      &= \Cons(f(x), \Map(f, xs))                \\
      \Foldl(f, x, \Nil)         &= x                                       \\
      \Foldl(f, x, \Cons(y, ys)) &= \Foldl(f, f(x, y), ys)                  \\
      \Foldr(f, \Nil,         y) &= y                                       \\
      \Foldr(f, \Cons(x, xs), y) &= f(x, \Foldr(f, xs, y))
    \end{split}
  \end{equation*}
  \caption{A simple theory defining a $\List$ type and some associated
    operations, taken from~\cite{Johansson.Dixon.Bundy:conjecture-generation}.
    $\Zero$ and $\Succ$ are from a Peano encoding of the natural numbers.}
  \label{figure:list_theory}
\end{figure}

We focus on the task of generating conjectures from some given mathematical
theory. This theory may be, for example, a scientific model, an executable
computer program or simply an object of mathematical curiosity. A theory
defines (perhaps implicitly) a particular system of logic in which to work (for
example higher-order logic), along with a finite set of \emph{definitions} (like
those in figure~\ref{figure:list_theory}) which assign structure or meaning to
particular terms. Theoretically, this is all of the information we need to
derive the truth or falsity of all decidable statements involving these terms,
for example by enumerating all valid proofs. However, due to the exponential
search spaces involved, such brute-force enumeration is infeasible in practice;
plus the resulting proofs would be difficult to comprehend and the processed
statements would be overwhelming in number and mostly uninteresting in content.

The impracticality of enumeration demonstrates that a statement being ``true''
or ``provable'' is not \emph{sufficient} to warrant the (limited) attention of a
human operator. We would also argue that it is not \emph{necessary}, since there
are many statements considered to be interesting whose truth is either unknown,
or which later turned out to be false. We thus focus on the automated production
of \emph{interesting conjectures}, with the primary focus being on efficient
generation and interest to the user.

As an example, we use the theory of lists shown in
figure~\ref{figure:list_theory}. These definitions are so widely used in
software that they appear in the standard libraries of many programming
languages. An example of a conjecture involving these definitions is the
following universally quantified equation:

\begin{equation} \label{eq:mapreduce}
  \forall f. \forall xs. \forall ys.
    \Map(f, \Append(xs, ys)) = \Append(\Map(f, xs), \Map(f, ys))
\end{equation}

Equation~\ref{eq:mapreduce} states that combining small lists into a larger one
using $\Append$ then transforming the elements with $\Map$ is the same as
transforming the elements of the small lists then combining. This is interesting
due to its applicability as an optimisation: if a program calls $\Map$ with an
expensive transformation $f$, we can divide up the work across multiple machines
in parallel (this is the basis of the ``map/reduce'' programming paradigm).
Whether or not this equation holds depends on the choice of logical framework,
and hence (in the case of software) on the semantics of the programming
language: in particular this equation and optimisation are invalid for systems
with strict evaluation order and side-effects (common in imperative programming)
since any error/exception in the calculation of $ys$ will cause the surrounding
calls to abort; on the left-hand-side this will abort $\Append$ then $\Map$, so
$f$ will never be executed; on the right-hand-side there is an additional
$\Map(f, xs)$ call which will \emph{not} be aborted, which may execute $f$ many
times and trigger arbitrary observable effects.

\section{Machine Learning}

Machine learning (ML) is a sub-field of artificial intelligence (AI) which
emphasises the use of statistical methods to find patterns and optimise
functions, in contrast with symbolic reasoning approaches like automated theorem
proving (commonly referred to as ``Good Old-Fashioned AI'').

The machine learning umbrella spans many techniques and applications, but one
common distinction is between \emph{supervised} and \emph{unsupervised} tasks.
In supervised learning we allow the system to perform some behaviour, then an
external process assesses whether it behaved well or poorly (akin to Justice
Potter Stewart's test for obscenity: ``I know it when I see it''). The system's
parameters are optimised to try and score highly on the test. Common supervised
learning tasks include function approximation using input/output examples (e.g.
classifying scanned images of numerals) where the test is the proportion of
example inputs which got the correct output, and reinforcement learning where
the system chooses from a set of ``actions'', receives a pair of ``observation''
and ``reward'' (test score) in response, and must accumulate the most reward.

Unsupervised learning does not have the same testable ``success'' criteria as
supervised learning. Instead, the goal is to find patterns and form models of
the target domain, which is more ambitious and open-ended than supervised
learning but also much harder to define, measure and optimise. Examples of this
are prediction and compression tasks such as auto-encoders
(learned compressor/decompressor pairs, rewarded by how well they preserve data
through a round-trip) and adversarial or co-evolution tasks (coupled systems
where each is rewarded by exploiting mis-prediction in the others).

\subsection{Feature Extraction}

One major difficulty when applying statistical machine learning algorithms to
\emph{languages}, such as Haskell, is the appearance of recursive
structures. This can lead to nested expressions of arbitrary depth, which are
difficult to compare in numerical ways. One common approach to this problem is
to represent such structures as \emph{sequences}. \emph{Recurrent neural
  networks} (RNNs) are a popular choice for processing sequences, especially
when combined with mechanisms such as \emph{long short-term memory} (LSTM) for
preserving information across long sequences \cite{hochreiter1997long}. Such
systems have been used, for example, to parse and execute computer programs
\cite{zaremba2014learning}. However, learning to parse sequences seems
inefficient considering that we already have correctly-parsed ASTs.

Whilst neural networks have been applied directly to recursive structures
\cite{goller1996learning}, including using LSTM \cite{zhu2015long}, a more
popular approach is to use \emph{kernel methods}
\cite{bakir2007predicting}. These are promising as a more principled alternative
to our current hand-crafted translation of ASTs to vectors.

\subsubsection{Distributed Representations}

A \emph{distributed representation} does not isolate each semantic feature (for
example, presence of absence of some lexeme) to separate, independent
``computing elements'' (e.g. memory locations, such as the elements of a feature
vector). Instead, features are represented as \emph{patterns} across multiple
elements: each feature uses many elements and each element contributes to many
features~\cite{hinton1984distributed}. Such encodings provide simple mechanisms
to represent \emph{combinations} of features (combining their patterns
element-wise), \emph{partial} features (presence of the pattern across a subset
of its elements) and \emph{generalisation} (patterns which are present as
sub-patterns of multiple features).

% Distributed representations may be learned from data, for example by
% \emph{auto-encoding}; or they may be calculated ``from whole cloth'' using
% techniques like hashing. We are particularly interested in distributed
% representations for storing structured data (like syntax trees) in fixed-length
% feature vectors.

Our approach to scaling up \quickspec{} takes inspiration from two sources. The
first is relevance filtering, which makes expensive algorithms used in theorem
proving more practical by limiting the size of their inputs. We describe this
approach in more details in \S~\ref{sec:relevance}. Relevance filtering is a
practical tool which has existing applications in software, such as the
\emph{Sledgehammer} component of the Isabelle/HOL theorem prover.

Despite the idea's promise, we cannot simply invoke existing relevance filter
algorithms in our theory exploration setting. The reason is that relevance
filtering is a supervised learning method, i.e. it would require a distinguished
expression to compare everything against. Theory exploration does not have such
a distinguished expression; instead, we are interested in relationships between
\emph{any} terms generated from a signature, and hence we must consider the
relevance of \emph{all terms} to \emph{all other terms}.

A natural fit for this task is \emph{clustering}, which attempts to group
similar inputs together in an unsupervised way. Based on their success in
discovering relationships and patterns between expressions in Coq and ACL2 (in
the ML4PG and ACL2(ml) tools respectively), we hypothesise that clustering
methods can fulfil the role of relevance filters for theory exploration:
intelligently breaking up large signatures into smaller ones more amenable to
brute force enumeration, such that related expressions are explored together.

\subsection{K-Means}

We use the Weka tool to perform k-means clustering
\cite{Holmes.Donkin.Witten:1994}, since we are more concerned with the
application of feature extraction to Haskell and its use in theory exploration,
rather than precise tuning of learning algorithms. Since k-means is a standard
method, there are many other implementations available. More interestingly,
there are many other clustering algorithms we could use, such as
\emph{expectation maximisation} \footnote{In fact, k-means is very similar to
  expectation-maximisation, as it alternates between an \emph{expectation step}
  (finding the mean value of each cluster) and a \emph{maximisation step}
  (assigning points to the cluster they're most similar to; or alternatively,
  \emph{minimising} the distance of each point to the centre of its cluster, as
  per equation \ref{eq:kmeansobjective}).}, but experiments with ML4PG have
shown little difference in their results; in effect, the quality of our features
is the bottleneck to learning, so there is no reason to avoid a fast algorithm
like k-means.

In any case there are many conservative improvements to the standard k-means
algorithm, which could be applied to our setup. For example, a more efficient
approach like \emph{yinyang k-means} \cite{conf/icml/DingZSMM15} could make
larger input sizes more practical to cluster, especially since recurrent
clustering invokes k-means many times. The \emph{k-means++} approach
\cite{arthur2007k, bahmani2012scalable} can be used to more carefully select
the ``seed'' values for the first timestep, and the \emph{x-means} algorithm
\cite{pelleg2000x} is able to estimate how many clusters to use (our
\emph{final} clusters should be tuned to maximise the performance of the
subsequent theory exploration step, but x-means could still be useful in the
recurrent clustering steps).

\subsubsection{K-Means}
\label{sec:kmeans}

Clustering is an unsupervised machine learning task for grouping $n$ data points
using a similarity metric. There are many variations on this theme, but in our
case we make the following choices:

\begin{itemize}
\item For simplicity, we use the ``rule of thumb'' given in
  \cite[pp. 365]{mardia1979multivariate} to fix the number of clusters at
  $k = \lceil \sqrt{\frac{n}{2}} \rceil$.
\item Data points will be $d$-dimensional feature vectors, as defined above.
\item We will use euclidean distance (denoted $e$) as our similarity metric.
\item We will use \emph{k-means} clustering, implemented by Lloyd's algorithm
  \cite{lloyd1982least}.
\end{itemize}

This is a standard setup, supported by off-the-shelf tools; in particular we use
the implementation provided by Weka \cite{Holmes.Donkin.Witten:1994}, due to
its use by ML4PG, which makes our results more easily comparable.

Since k-means is iterative, we will use function notation to denote time steps,
so $x(t)$ denotes the value of $x$ at time $t$. We denote the clusters as $C^1$
to $C^k$. As the name suggests, k-means uses the mean value of each cluster,
which we denote as $\vect{m}^1$ to $\vect{m}^k$, hence:

\begin{equation*}
  m^i_j(t) = \mean{C^i_j}(t) = \frac{\sum_{\vect{x} \in C^i(t)} x_j}{|C^i(t)|} \quad \text{for $t > 0$}
\end{equation*}

Before k-means starts, we must choose \emph{seed} values for
$\vect{m}^i(0)$. Many methods have been proposed for choosing these values
\cite{arthur2007k}. For simplicity, we will choose values randomly from our
data set $S$; this is known as the Forgy method.

The elements of each cluster $C^i(t)$ are those data points closest to the mean
value at the previous time step:

\begin{equation*}
  C^i(t) = \{ \vect{x} \in S \mid i = \argmin\limits_j e(\vect{x}, \vect{m}^j(t-1)) \} \quad \text{for $t > 0$}
\end{equation*}

As $t$ increases, the clusters $C^i$ move from their initial location around the
``seeds'', to converge on a local minimum of the ``within-cluster sum of squared
error'' objective:

\begin{equation} \label{eq:kmeansobjective}
  \argmin\limits_C \sum_{i=1}^k \sum_{\vect{x} \in C^i} e(\vect{x}, \vect{m}^i)^2
\end{equation}


Our recurrent clustering approach takes inspiration from the ML4PG
\cite{journals/corr/abs-1212-3618} and ACL2(ml) \cite{heras2013proof} tools,
used for analysing proofs in Coq and ACL2, respectively. Whilst both transform
syntax trees into matrices, the algorithm of ML4PG most closely resembles ours
as it assigns tokens directly to matrix elements. In contrast, the matrices
produced by ACL2(ml) \emph{summarise} information about the tree; for example,
one column counts the number of variables appearing at each tree level, others
count the number of function symbols which are nullary, unary, binary,
etc. Whilst it may be interesting to contrast our current algorithm with an
alternative based on that of ACL2(ml), it is unclear how such summaries could be
extended to include types, which seems the next logical step for our
approach. The ML4PG algorithm extends trivially, by using (term, type) pairs
instead of just terms.

The way we \emph{use} our clusters to inform theory exploration is actually more
similar to that of ACL2(ml) than ML4PG. ML4PG can either present clusters to the
user for inspection, or produce automata for recreating proofs. In ACL2(ml), the
clusters are used to restrict the search space of a proof search, much like we
restrict the scope of theory exploration.

ACL2(ml) reasons by analogy: finding theorem statements which are similar to the
current goal, and attempting to prove the goal in a similar way. In particular,
the lemmas used to prove a theorem are mutated by substituting symbols for those
which appear in the same cluster. For example, if \texttt{plus} and
\texttt{multiply} are clustered together, and we are trying to prove a goal
involving \texttt{multiply}, then ACL2(ml) might consider an existing theorem
involving \texttt{plus}. The lemmas used to prove that theorem will be mutated,
for example replacing occurrences of \texttt{plus} with \texttt{mult}, in an
attempt to prove the goal.

Whilst we do not currently reason by analogy, this is an interesting area for
future work in theory exploration: given a set of theorems relating particular
terms, we might form conjectures regarding similar terms found through
clustering.

\iffalse
We could expand this a bit, e.g. talking about how we both use Weka, etc.
\fi

