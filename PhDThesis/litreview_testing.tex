\subsection{Property Checking}

Automated testing is a common method of software quality assurance. In Haskell,
some of the most popular testing frameworks go beyond the industry standard of
\emph{unit testing} (asserting that components of a system behave as expected on
a fixed set of inputs) to instead perform \emph{property checking}. A
``property'', in this sense, is a statement about a program's components, which
may include free variables. Such properties are typically encoded in Haskell as
functions, accepting values for the free variables as arguments, and returning a
boolean value (perhaps adorned with useful metadata). A property checker
proceeds by
\begin{enumerate*}
\item generating suitable values for the function's arguments,
\item calling the function with these values,
\item inspecting the return value and
\item repeating until some stop condition is satisfied
\end{enumerate*}.

Most frameworks will test each property a fixed number of times (e.g. 100), and
stop early if the function returns a false value. The major difference between
implementations is the way they instantiate free variables (or, in encoded form,
how they generate arguments for the property functions).

\textsc{QuickCheck}~\cite{claessen2011quickcheck} popularised the use of
property checking in Haskell, and has subsequently been ported to many other
programming languages. \textsc{QuickCheck} generates input data \emph{randomly},
either using explicit generators (via a function called \texttt{forAll}), or by
choosing a default generator for types implementing its \texttt{Arbitrary}
typeclass (the recent \texttt{hedgehog} package works in a similar way).

Random testing is effective at thoroughly exercising the branches of a program,
since inputs may be quite large and cover many cases, especially for functions
which recurse on each \emph{sub-}expressions of their input as well. The output
of a random generator will also vary across runs, \emph{increasing} our
confidence each time we check a property, rather than simply reiterating past
results. However, there are two common problems with random property checking.
Firstly we must be able to generate random values, which can be tricky for
recursive datatypes. Consider a type of binary trees, with constructors
\texttt{Leaf~::~Tree} and \texttt{Node~::~Tree~->~Tree~->~Tree}. We might
generate a \texttt{Tree} by picking uniformly between these two constructors
and, if a \texttt{Node} was picked, recursing to generate the two
sub\texttt{Tree}s. Yet this simple scheme will generate \texttt{Tree}s whose
expected size is \emph{unbounded}! This is due to the 50\% chance of generating
a \texttt{Node}, which requires two recursive calls; hence we can expect each
call to recurse $0.5~\times~2~=~1$ more time on average, and that call will also
recurse once more on average, and so on; this can easily exhaust all available
memory. Many automated data generators, such as those from Haskell's
\texttt{derive} package, suffer from this problem. It can be avoided by biasing
the generators (in the \texttt{Tree} example we would produce a \texttt{Leaf}
more than half of the time), but this bias compounds at each level of recursion,
limiting testing to very small values. More sophisticated solutions exist, such
as the \texttt{boltzmann-sampling} package (see \cite{duchon2004boltzmann}) and
even domain-specific languages~\cite{lampropoulos2017beginner}, but all of these
add complications to the property checking process.

The other problem with random generators is that it can be difficult to
understand \emph{why} a randomly generated value violated some property.
\textsc{QuickCheck} tackles this in two ways: generators have access to a
``size'' parameter, with the intention that this be used to decide how large the
output values should be (e.g. the expected length of a list, or number of nodes
in a tree). The value of the size parameter begins low and increases as more
tests pass, such that failures are likely to be triggered by small, more easily
understood counterexamples. Additionally, a mechanism for ``shrinking'' is
provided: users can supply a \texttt{shrink} function along with their
generator, which is given any counterexample found during testing and returns a
list of smaller values. In the above example we might shrink a \texttt{Node} by
returning each sub-tree, and return an empty list when asked to shrink a
\texttt{Leaf}. After a counterexample has been found, the same property is
applied to each of the shrunk values in an attempt to narrow down the source of
the problem. When a value cannot be shrunk any more, or if none of its shrunken
versions are counterexamples, that value is presented to the user in the hope
that it is simple enough to understand.

The \textsc{SmallCheck} framework takes a different approach: \emph{enumerating}
values (either using a user-provided enumerating function or using its
\texttt{Serial} typeclass), rather than generating values
randomly~\cite{runciman2008smallcheck} (a similar approach is also followed by
the \texttt{testing-feat} package). In the \texttt{Tree} example this might
check \texttt{Leaf}, then \texttt{Node Leaf Leaf}, then
\texttt{Node (Node Leaf Leaf) Leaf}, and so on. This ensures that
counterexamples are always small, that tests are not repeated with the same
values and that memory is not exhausted (since the stopping condition will
trigger long before the enumeration reaches such large values). The disadvantage
is that large inputs (which exercise properties, especially recursive ones, the
most) may occur so late in the enumeration that testing never reaches them.

The \textsc{LazySmallCheck} framework is similar in spirit to
\textsc{SmallCheck}, but utilises Haskell's lazy evaluation to avoid generating
values up front~\cite{reich2013advances}. Instead, a partial value is generated
which will trigger an exception if evaluated. Since catching a Haskell exception
is an \texttt{IO} action, pure properties which inspect such values will always
be aborted, handing control back to \textsc{LazySmallCheck} which checks the
same property on refined versions of the previous value. For example, a lazily
generated \texttt{Tree} may initially be equivalent to $\bot$. If the property
aborts, it will be run again with each possible refinement of $\bot$: in this
case inputs equivalent to \texttt{Leaf} and \texttt{Node~$\bot_1$~$\bot_2$}. If
the latter aborts due to inspecting $\bot_1$, then $\bot_1$ will be refined to
give inputs \texttt{Node~Leaf~$\bot_2$} and
\texttt{Node~(Node~$\bot_3$~$\bot_4$)~$\bot_2$} (and likewise if $\bot_2$ had
caused the exception). This refinement continues until the property no longer
aborts or until a predefined depth limit is reached. If the property returns a
true value when given such partially-defined input then it must succeed for
\emph{all} refinements of that input (since the output was determined without
evaluating the undefined sub-terms). If the property returns a false value, the
input is a \emph{minimal} counterexample, since it only specifies those
sub-terms which are required to violate the property; any undefined sub-terms
are irrelevant to the failure.

\textsc{LazySmallCheck}'s automatic elaboration of free variables is similar in
operation to the use of search procedures in logic programming languages like
Kanren, Prolog (which has its own \textsc{QuickCheck} port) and Curry (a
functional logic language, whose \textsc{EasyCheck} framework explicitly
unifies the concepts of generators and logic
variables~\cite{christiansen2008easycheck}). The use of uninterpreted symbols
is also reminiscent of \emph{symbolic execution}~\cite{king1976symbolic}, a form
of \emph{abstract interpretation} where unknown runtime values are represented
by opaque symbols and computation proceeds to arrive at a result, which may be a
complicated expression involving many symbols.
