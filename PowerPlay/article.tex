Automation, Learning and Interestingness in Pure Mathematics

Humans have been practicing and researching mathematics, in one form or another,
for thousands of years. The invention of electronic computers has made it
routine to automate \emph{calculations}, but the autoamtion of more ``creative''
aspects of mathematics, such as theorem proving, conjecture generation and
concept formation, is still mostly limited to academic research rather than
being a routine task for computers.

We propose to address this by automating the \emph{discovery} and
\emph{learning} of mathematical concepts. We adapt the approach of Schmidhuber
et al., which has previously been used with artificial neural networks to learn
classifiers, and adapt it to a theorem proving setting.

%% System description

The PowerPlay architecture is based around inventing and solving problems. The
encoding and interpretation of these problems depends on the implementation, so
for now we leave it abstract by declaring it as a type:

%% FIXME: Problem is a type

We also keep solutions abstract, but we parameterise them by the particular
problem they're meant to solve:

%% FIXME: Solution is a function

Given the above, a problem solver is a function from problems to their
solutions, with an option type to represent failure:

%% FIXME: Solver
