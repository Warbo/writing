\section{Recurrent Proof Clustering}\label{sec:recurrent}

The method presented in the previous section can cluster similar statements of all Coq terms, including definitions and theorems.
However, this method does not capture the interactive nature of Coq proofs.
In this section, we involve proofs into the recurrent clustering of Coq libraries.

 In~\cite{KHG13},
we introduced a feature extraction method for Coq proofs capturing
the user's interaction through the applied tactics. That method
traced
low-level properties present in proof's subgoals, e.g. ``the
top symbol'' or ``the argument type''. Further, these features were taken
in relation to the statistics of user actions on every subgoal: how many and what
kind of tactics he applied, and what kind of arguments he provided to the
tactics. Finally, a few proof-steps were taken in relation to each other.
This method  had two drawbacks.\\
(1) It was  focused on the first
five proof-steps of a proof; therefore, some information was lost. We address this issue by implementing automatic split of each proof into proof-patches; thus, allowing ML4PG to analyse a proof by the properties
of the patches that constitute the proof. \\
(2) The method assigned most feature-values blindly; thus, being insensitive to many important parameters, such as e.g. the structure of lemmas and hypotheses used as
tactic arguments within a proof.
The previous section gave us the way of involving all Coq objects into recurrent feature re-evaluation.

\begin{definition}[Coq proof]
 Given a statement $S$ in Coq, a \emph{Coq proof} of $S$ is given by a sequence of triples $((\Gamma_i,G_i,T_i))_{0\leq i\leq n}$ where $\Gamma_i$ is a context,
 $G_i$ is a goal and $T_i$ is a sequence of tactics satisfying:

  - $G_0=S$, and for all $i$, $\Gamma_i$ is the context of the goal $G_i$,

  - for all $i$ with $0<i\leq n$, $\Gamma_i,G_i$ are respectively the context and goal obtained after applying $T_{i-1}$, and
    the application of $T_n$ completes the proof.
\end{definition}

In this paper, we focus on the goals and tactics of Coq proofs; thus, we do not consider the contexts and denote the
Coq proof $((\Gamma_i,G_i,T_i))_{0\leq i\leq n}$ by $((G_i,T_i))_{0\leq i\leq n}$. Involving contexts into proof-pattern search may be a subject for future work.


\begin{table}[t]
\tbl{\scriptsize{\emph{Proof for the lemma of Example~\ref{example0} in SSReflect.}}\label{tab:sumfirstn}}{
 	\centering
 	\tiny{
 		\begin{tabular}{|l|l|}
 		\hline
 	Goals and Subgoals & Applied Tactics \\
 		\hline
 		\hline
 	{\scriptsize $G_0) \forall~n,\sum\limits_{i=0}^{n} (g(i+1) - g(i)) = g(n+1) - g(0)$} & \\
 			& $T_0)$ {\scriptsize \lstinline?case : n => [|n _].?} \\
        $G_1) \sum\limits_{i=0}^{0} (g(i+1) - g(i)) = g(1) - g(0)$ & \\
        & $T_1)$  {\scriptsize\lstinline?by rewrite big_nat1.?}\\
        $G_2)  \sum\limits_{i=0}^{n+1} (g(i+1) - g(i)) = g(n+2) - g(0)$ &\\

        & $T_2)$ {\scriptsize\lstinline?rewrite sumrB big_nat_recr big_nat_recl ?}\\
          & ~~~~~~~~{\scriptsize\lstinline?     addrC addrC -subr_sub -!addrA addrA.?} \\
        $G_3) g(n+2) + \sum\limits_{i=0}^{n} g(i+1) -  \sum\limits_{i=0}^{n} g(i+1) - g(0) =$ &\\
        $  g(n+2) - g(0)$ &\\
        &$T_3)$  {\scriptsize\lstinline?move : eq_refl.?}\\
        $G_4) \sum\limits_{i=0}^{n} g(i+1) == \sum\limits_{i=0}^{n} g(i+1) \rightarrow  $ &\\
        $g(n+2) + \sum\limits_{i=0}^{n} g(i+1) -  \sum\limits_{i=0}^{n} g(i+1) - g(0) =$ &\\
        $  g(n+2) - g(0)$ &\\
        &$T_4)$  {\scriptsize\lstinline?rewrite -subr_eq0.?}\\
        $G_5) \sum\limits_{i=0}^{n} g(i+1) - \sum\limits_{i=0}^{n} g(i+1) == 0\rightarrow  $ &\\
        $g(n+2) + \sum\limits_{i=0}^{n} g(i+1) -  \sum\limits_{i=0}^{n} g(i+1) - g(0) =$ &\\
        $  g(n+2) - g(0)$ &\\
        &$T_5)$  {\scriptsize\lstinline?move/eqP => ->.?}\\
         $G_6)  g(n+2) + 0 - g(0) = g(n+2) - g(0)$ &\\
        &$T_6)$  {\scriptsize\lstinline?by rewrite sub0r.?}\\
 		$\Box$ & \\
 		& {\scriptsize\lstinline?Qed.?}\\
 		\hline
 		\end{tabular}

 		}}
 \end{table}


\begin{example}\label{example0}
Table~\ref{tab:sumfirstn} shows the Coq proof of the following statement:
 $$\forall g:\mathbb{N} \rightarrow \mathbb{Z}\implies \sum_{0\leq i \leq n} (g(i+1) - g(i)) = g(n+1) - g(0)$$
\end{example}

One small proof may potentially
resemble a fragment of a bigger proof; also, various small ``patches'' of different big proofs may resemble.

\begin{definition}[Proof-patch]
 Given a \emph{Coq proof} $C=((G_i,T_i))_{0\leq i\leq n}$, a \emph{proof-patch} of $C$ is a subsequence of at most $5$ consecutive
 pairs of $C$.
\end{definition}

From proof-patches, we can construct their feature matrices.
We will shortly define the feature extraction function $[.]_P=\langle[.]_M,[.]_{tac}\rangle : proof~patches \rightarrow M[\mathbb{Q}]_{5\times 6}$,
where $[.]_{tac}$ is an injective function that has been introduced to assign values to tactics.
We have defined two versions of  $[.]_{tac}$: one for Coq tactics and another for SSReflect tactics.
In the SSReflect case, we divide the tactics into 7 groups and assign similar
values to each tactic in the group, see Table~\ref{tab:tactics}.
Analogously for Coq tactics, cf. Appendix~\ref{sec:coqtactics}.

\begin{definition}[Proof-patch matrix]\label{def:ptm}
Given a Coq proof $C=((G_i,T_i))_{0\leq i\leq n}$, and a proof patch $p=((G_{i_0},T_{i_0}),\ldots,(G_{i_4},T_{i_4}))$ of $C$,
the feature extraction function $[.]_P: proof~patches \rightarrow M[\mathbb{Q}]_{5\times 6}$ constructs the \emph{proof-patch matrix} $[p]_P$
as follows:

\begin{itemize}
 \item the $(j,0)$-th entry of $[p]_P$ is a 4-tuple $([T_{i_j}^1]_{tac}, [T_{i_j}^2]_{tac},[T_{i_j}^3]_{tac},[T_{i_j}^r]_{tac})$
 where $T_{i_j}^1, T_{i_j}^2$ and $T_{i_j}^3$ are the three first tactics of $T_{i_j}$, and $T_{i_j}^r$ is the list of the rest of tactics of $T_{i_j}$,
 \item the $(j,1)$-th entry of $[p]_P$ is the number of tactics appearing in $T_{i_j}$,
 \item the $(j,2)$-th entry of $[p]_P$ is a 4-tuple $([t_1]_{type},[t_2]_{type},[t_3]_{type},[t_{i_j}]_{type})$ where
 $t_1, t_2$ and $t_3$ are the three first argument-types of $T_{i_j}$, and $t_{i_j}$ is the set of the rest of
 the argument-types of $T_{i_j}$ (insensitive to order or repetition),
 \item the $(j,3)$-th entry of $[p]_P$ is a 4-tuple  $([l_{i_{j_1}}]_{term},[l_{i_{j_2}}]_{term},[l_{i_{j_3}}]_{term},[l_{i_j}]_{term})$
 where $l_{i_{j_1}}$, $l_{i_{j_2}}$ and $l_{i_{j_3}}$ are the three first lemmas applied in $T_{i_j}$ and $l_{i_j}$ is the list of the rest of lemmas
used in $T_{i_j}$ (sensitive to order and repetition),
 \item the $(j,4)$-th entry of $[p]_P$ is a triple  $([s_1]_{term},[s_2]_{term},[s_3]_{term})$ where $s_1,s_2$ and $s_3$ are respectively the top, second, and third
 symbol of $G_{i_j}$,
 \item the $(j,5)$-th entry of $[p]_P$ is the number of subgoals after applying $T_{i_j}$ to $G_{i_j}$.
\end{itemize}
\end{definition}


\begin{table}[t]
\tbl{\scriptsize{\emph{Formulas computing the value of SSReflect tactics.}
they serve to assign closer values to the tactics within each of the seven groups, and more distant numbers
across the groups.  If a new tactic is defined, ML4PG automatically assigns a new number to it, using the next available natural number $n$
in the formula $n+\sum_{j=1}^i \frac{1}{10\times 2^{j-1}}$.}\label{tab:tactics}}{
\centering
{\scriptsize
\begin{tabular}{ll}
\hline
$\ast$ Bookkeeping ($b=\{$\lstinline?move:, move => ?$\}$): &$[b_i]_{tac}=1+\sum_{j=1}^i \frac{1}{10\times 2^{j-1}}$ (where $b_i$ is the $i$th element of $b$).\\
$\ast$ Case and Induction ($c=\{$\lstinline?case, elim?$\}$):& $[c_i]_{tac}=2+\sum_{j=1}^i \frac{1}{10\times 2^{j-1}}$.\\
$\ast$ Discharge ($d=\{$\lstinline?apply, exact, congr?$\}$):& $[d_i]_{tac}=3+\sum_{j=1}^i \frac{1}{10\times 2^{j-1}}$.\\
$\ast$ Simplification ($s=\{$\lstinline?//, /=, //=?$\}$):& $[s_i]_{tac}=4+\sum_{j=1}^i \frac{1}{10\times 2^{j-1}}$.\\
$\ast$ Rewrite:& $[$\lstinline?rewrite?$]_{tac} = 5$. \\
$\ast$ Forward Chaining  ($f=\{$\lstinline?have, suff, wlog?$\}$):& $[f_i]_{tac}=6+\sum_{j=1}^i \frac{1}{10\times 2^{j-1}}$.\\
$\ast$ Views and reflection  ($v=\{$\lstinline?move/, apply/, elim/, case/?$\}$):& $[v_i]_{tac}=7+\sum_{j=1}^i \frac{1}{10\times 2^{j-1}}$.\\
\hline
\end{tabular}}}
\end{table}



\begin{example}\label{example1}
Given the proof of Example~\ref{example0} and the proof-patch $((G_{i},T_{i}))_{0\leq i \leq 4}$,
the top table of Table~\ref{tab:patches} shows their proof-patch matrix.
\end{example}


The \emph{proof-patch method} considers several proof-patches to collect information from a concrete proof.
In particular, given a Coq proof $C=((G_i,T_i))_{0\leq i\leq n}$ , the proof $C$ can be split into patches $C_0,\ldots,C_m$ where
$m=\lceil\frac{n}{5}\rceil+1$. The patches are defined as follows: $C_j=((G_j,T_j),\ldots,(G_{j+4},T_{j+4}))$
for $0\leq j < m$ (some patches might contain less than $5$ proof-steps), and $C_m=((G_{n-4},T_{n-4}),\ldots,(G_{n},T_{n}))$ --- the last patch captures
the last five proof-steps.


\begin{example}
Using the proof-patch method, we  can split the proof presented in Example~\ref{example1} into three proof-patches $((G_i,T_i))_{0\leq i \leq 4}$,
$((G_5,T_5),(G_6,T_6))$ and $((G_i,T_i))_{2\leq i \leq 6}$; the corresponding proof-patch matrices are given in Table~\ref{tab:patches}.
\end{example}

The proof-patch method together with the feature function $[.]_P$ solve the two drawbacks of the old method~\cite{KHG13}:
the new method captures information about the whole proof and the feature values are dynamically computed to assign close
values to similar terms, types, tactics and lemma statements used as tactic arguments.


We finish this section with a case study that illustrates
the use of the proof-patch method, and shows the differences with the results obtained with the old method~\cite{KHG13}. This case study concerns the discovery of proof patterns in mathematical proofs
across formalisations of apparently disjoint mathematical theories: Linear Algebra, Combinatorics and Persistent Homology  (across 758 lemmas and 5~libraries).
In this scenario, we use statistically discovered proof patterns to advance the proof of a given ``problematic'' lemma.
Namely,  a few initial steps in its proof are clustered against several mathematical libraries.
We deliberately take lemmas belonging to very different SSReflect libraries. The lemma introduced in Example~\ref{example0} is a basic fact about
summations. Lemma~\ref{lem:nilpotent} states a result about \emph{nilpotent} matrices
(a square matrix $M$ is \emph{nilpotent}
if there exists an $n$ such that $M^n=0$). Finally, Lemma~\ref{lem:fundamental} is a generalisation of the \emph{fundamental lemma of
Persistent Homology}~\cite{HCMS12}.


\begin{lemma}\label{lem:nilpotent}
 Let $M$ be a square matrix and $n$ be a natural number such that $M^n=0$, then $(1-M)\times \sum\limits_{i=0}^{n-1} M^i = 1$.
\end{lemma}


\begin{lemma}\label{lem:fundamental}
Let $\beta_n^{k,l}:\mathbb{N} \times \mathbb{N} \times \mathbb{N} \rightarrow \mathbb{Z}$, then
$$\sum_{0\leq i \leq k} \sum_{l<j\leq m} (\beta_n^{i,j-1} - \beta_n^{i,j}) - (\beta_n^{i-1,j-1} - \beta_n^{i-1,j}) = \beta_n^{k,l} - \beta_n^{k,m}.$$
\end{lemma}


\begin{table}
\tbl{\scriptsize{\emph{Proof-patch matrices for the proof of Example~\ref{example1}.} \textbf{Top.} Proof-patch matrix of the patch $((G_i,T_i))_{0\leq i\leq 4}$.
 \textbf{Centre.} Proof-patch matrix of the patch $((G_i,T_i))_{5\leq i\leq 6}$ (rows that are not included in the table are filled with zeroes).
 \textbf{Bottom.} Proof-patch matrix of the patch $((G_i,T_i))_{2\leq i\leq 6}$.
Where we use notation $EL$, ML4PG gathers the lemma names: (\lstinline?addrC?, \lstinline?addrC?~, \lstinline?subr_sub?, \ldots).}\label{tab:patches}}{
\centering
\tiny{
\begin{tabular}{|l||l|l|l|l|l|l|}
\hline
 & \emph{tactics} & \emph{n} & \emph{arg type} & \emph{arg} & \emph{symbols} & \emph{goals} \\
\hline
\hline
\emph{g1}& $([case]_{tac},0,0,0)$ & $1$  & $([nat]_{type},0,0,0)$  & $([Hyp]_{term},0,0,0)$ & $([\forall]_{term},[=]_{term},[sum]_{term})$ & $2$ \\
 \hline
  \emph{g2} & $([rewrite]_{tac},0,0,0)$ & $1$  & $([Prop]_{type},0,0,0)$  & $([big\_nat1]_{term},0,0,0)$ & $([=]_{term},[\sum]_{term},[-]_{term})$& $0$ \\
 \hline
  \emph{g3} & $([rewrite]_{tac},0,0,0)$ & $1$  & $([Prop]_{type},$ & $([surB]_{term},$  & $([=]_{term},[+]_{term},[-]_{term})$& $1$ \\
 & & &$[Prop]_{type},$ & $[big\_nat\_recr]_{term}$ & & \\
 & & &$[Prop]_{type},$ & $[big\_nat\_recl]_{term}$ & & \\
 & & &$[Prop]_{type})$ & $[EL]_{term})$ & & \\
 \hline
   \emph{g4} & $([move:]_{tac},0,0,0)$ & $1$  & $([Prop]_{type},0,0,0)$ &  $([eq\_refl]_{term},0,0,0)$  & $([=]_{term},[+]_{term},[-]_{term})$ & $1$ \\
 \hline
  \emph{g5} & $([rewrite]_{tac},0,0,0)$ & $1$  & $([Prop]_{type},0,0,0)$  & $([subr\_eq0]_{term},0,0,0)$ & $([=]_{term},[+]_{term},[-]_{term})$ & $1$ \\
 \hline

 \multicolumn{7}{c}{}\\
% Second table
\hline
 & \emph{tactics} & \emph{n} & \emph{arg type} & \emph{arg} & \emph{symbols} &  \emph{goals} \\
\hline
\hline
\emph{g1} & $([move/]_{tac},[\texttt{->}]_{tac},0,0)$ & $2$  & $([Prop]_{type},0,0,0)$ &  $([eq\_refl]_{term},0,0,0)$  & $([=]_{term},[+]_{term},[-]_{term})$ & $1$ \\
 \hline
  \emph{g2} & $([rewrite]_{tac},0,0,0)$ & $1$  & $([Prop]_{type},0,0,0)$  & $([subr\_eq0]_{term},0,0,0)$ & $([=]_{term},[+]_{term},[-]_{term})$ & $1$ \\
 \hline

 \multicolumn{7}{c}{}\\
%Third table
 \hline
 & \emph{tactics} & \emph{n} & \emph{arg type} & \emph{arg} & \emph{symbols} & \emph{goals} \\
\hline
\hline
\emph{g1}&  $([rewrite]_{tac},0,0,0)$ & $1$  & $([Prop]_{type},$ & $([surB]_{term},$  & $([=]_{term},[+]_{term},[-]_{term})$& $1$ \\
 & & & $[Prop]_{type},$ & $[big\_nat\_recr]_{term}$ & & \\
 & & & $[Prop]_{type},$& $[big\_nat\_recl]_{term}$ & & \\
 & & & $[Prop]_{type})$& $[EL]_{term})$ & & \\
 \hline
  \emph{g2} & $([move:]_{tac},0,0,0)$ & $1$  & $([Prop]_{type},0,0,0)$ &  $([eq\_refl]_{term},0,0,0)$  & $([=]_{term},[+]_{term},[-]_{term})$ & $1$ \\
 \hline
  \emph{g3} & $([rewrite]_{tac},0,0,0)$ & $1$  & $([Prop]_{type},0,0,0)$  & $([subr\_eq0]_{term},0,0,0)$ & $([=]_{term},[+]_{term},[-]_{term})$ & $1$ \\
 \hline
\emph{g4}& $([move/]_{tac},[\texttt{->}]_{tac},0,0)$ & $2$  & $([Prop]_{type},0,0,0)$  & $([eqP]_{term},0,0,0)$ & $([=]_{term},[+]_{term},[-]_{term})$ & $1$ \\
 \hline
  \emph{g5} & $([rewrite]_{tac},0,0,0)$ & $1$  & $([Prop]_{type},0,0,0)$  & $([sub0r]_{term},0,0,0)$ & $([=]_{term},[+]_{term},[-]_{term})$ & $0$ \\
 \hline
  \end{tabular}
  }}
 \end{table}


When proving Lemma~\ref{lem:nilpotent}, the user may call ML4PG after completing a few standard proof steps: apply induction and solve the base case using rewriting.
At this point it is difficult, even for an expert user,
to get the intuition that he could reuse the proofs from Example~\ref{example0} and Lemma~\ref{lem:fundamental}. There are several reasons for this.
First of all, the formal proofs of these lemmas are in different libraries; then, it is difficult to establish a conceptual connection among them. Moreover,
although the three lemmas involve summations,
the types of the terms of those summations are different. Therefore, search based on types or keywords would not help. Even search
of all the lemmas involving summations does not provide a clear suggestion, since there are more than $250$ lemmas (a considerable number for handling them manually) --- a clever search-pattern will considerably reduced the number of suggested lemmas, but such a pattern is not trivial at all.




However, if only the lemmas from Example~\ref{example0} and Lemma~\ref{lem:fundamental} are suggested when proving Lemma~\ref{lem:nilpotent}, the user would be able to
spot the following common proof pattern.


\begin{PS}\label{ps:math}

\emph{
	Apply case on $n$.
  \begin{enumerate}
   \item Prove the base case (a simple task).
   \item Prove the case $0<n$:
     \begin{enumerate}
     \item expand the summation,
     \item cancel the terms pairwise,
     \item the terms remaining after the cancellation are the first and the last one.
     \end{enumerate}
  \end{enumerate}
}


\end{PS}



Using the method presented in~\cite{KHG13}, if ML4PG was invoked during the proof of Lemma~\ref{lem:nilpotent}, it would suggest the lemmas from Example~\ref{example0} and
Lemma~\ref{lem:fundamental}. However, 5 irrelevant lemmas
 about summations would also be suggested (irrelevant in the sense that they do not follow Proof Strategy~\ref{ps:math}). The cluster containing just the two desired lemmas could be obtained after
increasing the \emph{granularity} value~\cite{KHG13} --- a statistical ML4PG parameter that can be adjusted by the user to obtain more precise clusters. The new version of ML4PG suggests four proof fragments, all following Strategy~\ref{ps:math} without needing to adjust granularity.

The new method brings two improvements:
(1) the number of suggestions is increased, but, at the same time, (2)
the clusters are more accurate. The proof-patch method considers fragments of proofs that are deep in the proof and were not considered before; therefore, it can find lemmas (more precisely patches of lemmas) that were not
included previously in the clusters. In our case study, ML4PG suggests two additional interesting proof fragments. The first one is an intermediate patch of the proof of
Lemma~\ref{lem:fundamental}; then, two patches are suggested from this
lemma: the proof-patch of the inner sum, and the proof-patch of the outer sum (both of them following Proof Strategy~\ref{ps:math}). The following lemma is also
suggested.

\begin{lemma}\label{lem:nilpotent2}
Let $M$ be a nilpotent matrix, then there exists a matrix $N$ such that $N \times (1-M)=1$.
\end{lemma}

At first sight, the proof of this lemma is an unlikely candidate to follow  Proof Strategy~\ref{ps:math}, since the statement of the lemma
does not involve summations. However, inspecting its proof, we can see that it uses $\sum_{i=0}^{n-1} M^i$ as witness for $N$ and
then follows Proof Strategy~\ref{ps:math}. In this case, ML4PG suggests the patch from the last five proof-steps that correspond to
the application of Proof Strategy~\ref{ps:math}.

The new numbering of features produces more accurate clusters removing the irrelevant lemmas. In particular, using the default settings,
ML4PG only suggests the lemma from Example~\ref{example0}, the two patches from Lemma~\ref{lem:fundamental} and the last patch from
Lemma~\ref{lem:nilpotent2}. If the granularity is increased, the last patch from
Lemma~\ref{lem:nilpotent2} is the only suggestion --- note that this is the closest lemma to Lemma~\ref{lem:nilpotent}.
