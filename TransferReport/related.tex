\section{Related Work}
\label{sec:related}

\iffalse TODO: Talk about MuCheck \fi

\subsection{Haskell}
\label{sec:haskelldesc}

Haskell is a convenient choice for our purposes for several reasons. Here we discuss the relevant language features from a high-level:

\begin{description}

\item{Functional}: All control flow is performed by function abstraction and application, which we can reason about using standard rules of inference such as \emph{modus ponens}.

\item{Pure}: Execution of actions (e.g. reading files) is separate to evaluation of expressions; hence our reasoning can safely ignore complicated external and non-local interactions.

\item{Statically Typed}: Expression are constrained by \emph{types}, which can be used to eliminate unwanted combinations of values, and hence reduce search spaces; \emph{static} types can be deduced syntactically, without having to execute the code.

\item{Non-strict}: If an evaluation strategy exists for $\beta$-normalising an expression (i.e. performing function calls) without diverging, then a non-strict evaluation strategy will not diverge when evaluating that expression. This is rather technical, but in simple terms it allows us to reason effectively about a Turing-complete language, where evaluation may not terminate. For example, when reasoning about \emph{pairs} of values \hs{(x, y)} and projection functions \hs{fst} and \hs{snd}, we might want to use an ``obvious'' rule such as $\forall \text{\hs{x y}}, \text{\hs{x}} = \text{\hs{fst (x, y)}}$. Haskell's non-strict semantics makes this equation valid; whilst it would \emph{not} be valid in the strict setting common to most other languages, where the expression \hs{fst (x, y)} will diverge if \hs{y} diverges (and hence alter the semantics, if \hs{x} doesn't diverge).

\item{Algebraic Data Types}: These provide a rich grammar for building up user-defined data representations, and an inverse mechanism to inspect these data by \emph{pattern-matching}. For our purposes, the useful consequences of ADTs and pattern-matching include their amenability for inductive proofs and the fact they are \emph{closed}; i.e. an ADT's declaration specifies all of the normal forms for that type. This makes exhaustive case analysis trivial, which would be impossible for \emph{open} types (for example, consider classes in an object oriented language, where new subclasses can be introduced at any time).

\item{Parametricity}: This allows Haskell \emph{values} to be parameterised over \emph{type-level} objects; provided those objects are never inspected. This has the \emph{practical} benefit of enabling \emph{polymorphism}: for example, we can write a polymorphic identity function \hs{id :: forall t. t -> t}. \footnote{Read ``\hs{a :: b}'' as ``\hs{a} has type \hs{b}'' and ``\hs{a -> b}'' as ``the type of functions from \hs{a} to \hs{b}''.} Conceptually, this function takes \emph{two} parameters: a type \hs{t} \emph{and} a value of type \hs{t}; yet only the latter is available in the function body, e.g. \hs{id x = x}. This inability to inspect type-level arguments gives us the \emph{theoretical} benefit of being able to characterise the behaviour of polymorphic functions from their type alone, a technique known as \emph{theorems for free} \citep{wadler1989theorems}.

\item{Type classes}: Along with their various extensions, type classes are interfaces which specify a set of operations over a type (or other type-level object, such as a \emph{type constructor}). Many type classes also specify a set of \emph{laws} which their operations should obey but, lacking a simple mechanism to enforce this, laws are usually considered as documentation. As a simple example, we can define a type class \hs{Semigroup} with the following operation and associativity law:

\begin{lstlisting}[language=Haskell, xleftmargin=.2\textwidth, xrightmargin=.2\textwidth]
op :: forall t. Semigroup t => t -> t -> t
\end{lstlisting}

$$\forall \text{\hs{x y z}}, \text{\hs{op x (op y z)}} = \text{\hs{op (op x y) z}}$$

The notation \hs{Semigroup t =>} is a \emph{type class constraint}, which restricts the possible types \hs{t} to only those which implement \hs{Semigroup}. \footnote{Alternatively, we can consider \hs{Semigroup t} as the type of ``implementations of \hs{Semigroup} for \hs{t}'', in which case \hs{=>} has a similar role to \hs{->} and we can consider \hs{op} to take \emph{four} parameters: a type \hs{t}, an implementation of \hs{Semigroup t} and two values of type \hs{t}. As with parameteric polymorphism, this extra \hs{Semigroup t} parameter is not available at the value level. Even if it were, we could not alter our behaviour by inspecting it, since Haskell only allows types to implement each type class in at most one way, so there would be no information to branch on.} There are many \emph{instances} of \hs{Semigroup} (types which may be substituted for \hs{t}), e.g. \hs{Integer} with \hs{op} performing addition. Many more examples can be found in the \emph{typeclassopedia} \citep{yorgey2009typeclassopedia}. This ability to constrain types, and the existence of laws, helps us reason about code generically, rather than repeating the same arguments for each particular pair of \hs{t} and \hs{op}.

\item{Equational}: Haskell uses equations at the value level, for definitions; at the type level, for coercions; at the documentation level, for typeclass laws; and at the compiler level, for ad-hoc rewrite rules. This provides us with many \emph{sources} of equations, as well as many possible \emph{uses} for any equations we might discover. Along with their support in existing tools such as SMT solvers, this makes equational conjectures a natural target for our investigation.

\item{Modularity}: Haskell has a module system, where each module may specify an \emph{export list} containing the names which should be made available for other modules to import. When such a list is given, any expressions \emph{not} on the list are considered \emph{private} to that module, and are hence inaccessible from elsewhere. This mechanism allows modules to provide more guarantees than are available just in their types. For example, a module may represent email addresses in the following way:

\begin{lstlisting}[language=Haskell, xleftmargin=.2\textwidth, xrightmargin=.2\textwidth, upquote=true]
module Email (Email(), at, render) where

data Email = E String String

render :: Email -> String
render (E u h) = u ++ "@" ++ h

at :: String -> String -> Maybe Email
at "" h  = Nothing
at u  "" = Nothing
at u  h  = Just (E u h)
\end{lstlisting}

The \hs{Email} type guarantees that its elements have both a user part and a host part (modulo divergence), but it does not provide any guarantees about those parts. We also define the \hs{at} function, a so-called ``smart constructor'', which has the additional guarantee that the \hs{Email}s it returns contain non-empty \hs{String}s. By ommitting the \hs{E} constructor from the export list on the first line \footnote{The syntax \hs{Email()} means we're exporting the \hs{Email} type, but not any of its constructors.}, the only way \emph{other} modules can create an \hs{Email} is by using \hs{at}, which forces the non-empty guarantee to hold globally.

\end{description}

Together, these features make Haskell code highly structured, amenable to logical analysis and subject to many algebraic laws. However, as mentioned with regards to type classes, Haskell itself is incapable of expressing or enforcing these laws (at least, without difficulty \citep{lindley2014hasochism}). This reduces the incentive to manually discover, state and prove theorems about Haskell code, e.g. in the style of interactive theorem proving, as these results may be invalidated by seemingly innocuous code changes. We will revisit the second-class status of theorem proving in Haskell in our discussion of theory exploration (\S \ref{sec:theoryexploration}), but we will remark that Haskell is thus in a unique position with regards to the discovery of interesting theorems. Namely that many discoveries may be available with very little work, simply because the code's authors are focused on \emph{software} development rather than \emph{proof} development. The same cannot be said, for example, of ITP systems; although our reasoning capabilities may be stronger in an ITP setting, much of the ``low hanging fruit'' will have already been found through the user's dedicated efforts, and hence TE would be more likely to fit into an automation role for the user, rather than providing truly unexpected new discoveries.

Other empirical advantages to studying Haskell, compared to either other programming languages or theorem proving systems, include:

\begin{itemize}

\item The large amount of Haskell code which is freely available online, e.g. in repositories like \href{http://hackage.haskell.org}{Hackage}, with which we can experiment.

\item The existence of conjecture generation and theory exploration systems for Haskell, such as \qspec{} and \hspec{}, and counterexample finders like \qcheck{}, \textsc{SmallCheck} and \textsc{SmartCheck}.

\item The remarkable amount of infrastructure which exists for working with Haskell code, including package managers, compilers, interpreters, parsers, static analysers, etc.

\end{itemize}

\iffalse

\subsection{Conjecture Generation}

The task of \emph{conjecture generation} lies at the heart of theory exploration, and

\subsubsection{Lemma generation}

\fi

\subsection{Relevance Filtering}
\label{sec:relevance}

% TODO
\citep{kuhlwein2012overview}

The combinatorial nature of formal systems causes many proof search methods, such as resolution, to have exponential complexity \citep{haken1985intractability}; hence even a modest size increase can turn a trivial problem into an intractable one. Finding efficient alternatives for such algorithms, especially those which are NP-complete (e.g. determining satisfiability) or co-NP-complete (e.g. determining tautologies), seems unlikely, as it would imply progress on the famously intractable open problems of $\text{P} = \text{NP}$ and $\text{NP} = \text{co-NP}$. On the other hand, we can turn this difficulty around: a modest \emph{decrease} in size may turn an intractable problem into a solvable one. We can ensure that the solutions to these reduced problems coincide with the original if we only remove \emph{redundant} information. This leads to the idea of \emph{relevance filtering}.

Relevance filtering simplifies a proof search problem by removing from consideration those clauses (axioms, definitions, lemmas, etc.) which are deemed \emph{irrelevant}. The technique is used in Sledgehammer during its translation of Isabelle/HOL theories to statements in first order logic: rather than translating the entire theory, only a sub-set of relevant clauses are included. This reduces the size of the problem and speeds up the proof search, but it creates the new problem of determining when a clause is relevant: how do we know what will be required, before we have the proof?

The initial approach, known as \textsc{MePO} (from \emph{Meng-Paulson} \citep{meng2009lightweight}), gives each clause a score based on the proportion $m / n$ of its symbols which are ``relevant'' (where $n$ is the number of symbols in the clause and $m$ is the number which are relevant). Initially, the relevant symbols are those which occur in the goal, but whenever a clause is found which scores more than a particular threshold, all of its symbols are then also considered relevant. There are other heuristics applied too, such as increasing the score of user-provided facts (e.g. given by keywords like \texttt{using}), locally-scoped facts, first-order facts and rarely-occuring facts. To choose $r$ relevant clauses for an ATP invocation, we simply order the clauses by decreasing score and take the first $r$ of them.

Recently, a variety of alternative algorithms have also been investigated, including:

\begin{description}

  \item{\textsc{MaSH}}: Machine Learning for SledgeHammer \citep{kuhlwein2013mash}. The distinguishing feature of \textsc{MaSH} is its use of ``visibility'', which is essentially a dependency graph of which theorems were used in the proofs of which other theorems; although theorems are represented as abstract sets of features. To select relevant clauses for a goal, the set of clauses which are visible from the goal's components is generated; this is further reduced by (an efficient approximation of) a naive Bayes algorithm.

  \item{\textsc{MOR}}: \emph{Multi-output ranking} uses a support vector machine (SVM) approach for selecting relevant axioms from the Mizar Mathematical Library for use by the Vampire ATP system \citep{alama2014premise}. \iffalse TODO: describe the kernel, as that's the interesting bit \fi It compares favourably to \textsc{SNoW} and \textsc{SInE}.

  % TODO:
  \item{\textsc{SInE}}
  \item{\textsc{BliStr}}
  \item{\textsc{HOLyHammer}}
  \item{\textsc{MoMM}}
  \item{\textsc{SNoW}}
  \item{\textsc{MPTP 0.2}}
  \item{\textsc{MaLARea}}
  \item{\textsc{MaLARea SG1}}

\end{description}

\subsection{Clustering Expressions}
\label{sec:clusteringexpressions}

% TODO: ML4PG
\citep{journals/corr/abs-1212-3618}
% TODO: ACL2(ml) (also examples section)
\citep{heras2013proof}

\subsubsection{Learning From Structured Data}

One major difficulty with formal mathematics as a domain in which to apply statistical machine learning is the use of \emph{structure} to encode information in objects. In particular, \emph{trees} appear in many places: from inductive datatypes, to recursive function definitions; from theorem statements, to proof objects. Such nested structures may extend to arbitrary depth, which makes them difficult to represent with a fixed number of features, as is expected by most machine learning algorithms. Here we review a selection of solutions to this problem, and compare their distinguishing properties.

\paragraph{Truncation and Padding}

The simplest way to limit the size of our inputs is to truncate anything larger than a particular size (and pad anything smaller). This is the approach taken by ML4PG \citep{journals/corr/abs-1302-6421}, which limits itself to trees with at most 10 levels and 10 elements per level; each tree is converted to a $30 \times 10$ matrix (3 values per tree node) and learning takes place on these normalised representations.

Truncation is unsatisfactory in the way it balances \emph{data} efficiency with \emph{time} efficiency. Specifically, truncation works best when the input data contains no redundancy and is arranged with the most significant data first (in a sense, it is ``big-endian''). The less these assumptions hold, the less we can truncate. Since many ML algorithms scale poorly with input size, we would prefer to eliminate the redundancy using a more aggressive algorithm, to keep the resulting feature size as low as possible.

\paragraph{Dimension Reduction}

A more sophisticated approach to the problem of reducing input size is to view it as a \emph{dimension reduction} technique: our inputs can be modelled as points in high-dimensional spaces, which we want to project into a lower-dimensional space ($\left\{ {0, 1} \right\}^N$ in the case of $N$-bit vectors).

Truncation is a trivial dimension reduction technique: take the first $N$ coordinates (bits). More sophisticated projection functions consider the \emph{distribution} of the points, and project with the hyperplane which preserves as much of the variance as possible (or, equivalently, reduces the \emph{mutual information} between the points).

There are many techniques to find these hyperplanes, such as \emph{principle component analysis} (PCA) and \emph{autoencoding}; however, since these techniques are effectively ML algorithms in their own right, they suffer some of the same constraints we're trying to avoid:

\begin{itemize}
  \item They operate \emph{offline}, requiring all input points up-front
  \item All input points must have the same dimensionality
\end{itemize}

In particular, the second constraint is precisely what we're trying to avoid. Sophisticated dimension reduction is still useful for \emph{compressing} large, redundant features into smaller, information-dense representations, and as such provides a good complement to truncation.

The requirement for offline ``batch'' processing is more difficult to overcome, since any learning we perform for feature extraction will interfere with the core learning algorithm that's consuming these features (this is why deep learning is often done greedily).

\paragraph{Sequencing}

The task of dimension reduction changes when we consider \emph{structured} data. Recursive structures, like trees and lists, have \emph{fractal} dimension: adding layers to a recursive structure gives us more \emph{fine-grained} features, rather than \emph{orthogonal} features. For data mining context-free languages (e.g. those of programming and theorem-proving systems), we will mainly be concerned with tree structures of variable size.

Any investigation of variable-size input would be incomplete without mentioning \emph{sequencing}. This is a lossless approach, which splits the input into fixed-size \emph{chunks}, which are fed into an appropriate ML algorithm one at a time. The sequence is terminated by a sentinel; an ``end-of-sequence'' marker which, by construction, is distinguishable from the data chunks. This technique allows us to trade \emph{space} (the size of our input) for \emph{time} (the number of chunks in a sequence).

Not all ML algorithms can be adapted to accept sequences. One notable approach is to use \emph{recurrent ANNs} (RANNs), which allow arbitrary connections between nodes, including cycles. Compared to \emph{feed-forward} ANNs (FFANNs), which are acyclic, the \emph{future output} of a RANN may depend arbitrarily on its \emph{past inputs} (in fact, RANNs are universal computers).

The main problem with RANNs, compared to the more widely-used FFANNs, is the difficulty of training them. If we extend the standard backpropagation algorithm to handle cycles, we get the \emph{backpropagation through time} algorithm \citep{werbos1990backpropagation}. However, this suffers a problem known as the \emph{vanishing gradient}: error values decay exponentially as they propagate back through the cycles, which prevents effective learning of delayed dependencies, undermining the main advantage of RANNs. The vanishing gradient problem is the subject of current research, with countermeasures including \emph{neuroevolution} (using evolutionary computation techniques to train an ANN) and \emph{long short-term memory} (LSTM; introducing a few special, untrainable nodes to persist values for long time periods \citep{hochreiter1997long}).

The application of \emph{kernel methods} to structured information is discussed in \citep{Gartner2003}, where the input data (including sequences, trees and graphs) are represented using \emph{generative models}, such as hidden Markov models, of a fixed size.

% TODO
\citep{Gartner2003}
\citep{Oveisi.Oveisi.Erfanian.ea:2012}
\citep{bakir2007predicting}
\citep{conf/ijcai/Plate91}
\citep{goller1996learning}
\citep{kwasny1995tail}
\citep{pollack1990recursive}
\citep{zanzotto2012distributed}

\iffalse

Machine learning over structured data:
1D is common: parsing natural language
2D is common; images
Trees are fractal
Backpropagation through structure
LSTM with recursive structure
Most work tries to identify structure; we already have it

Recurrent neural networks
Backpropagation through structure

\fi
