\section{Contributions}
\label{sec:contributions}

\subsection{Recurrent Clustering}
\label{sec:recurrentclustering}

\iffalse TODO: Much of this discussion reads as a minor implementation solution rather than an overview of the whole method \fi

\iffalse TODO: Focus the section more on the problem of identifiers, and how recurrent clustering uses clustering as part of feature extraction to solve this \fi

\iffalse TODO: Maybe present the translation from trees to matrices, to vectors first; leaving the conversion of terms to features until afterwards; similar to how ML4PG uses [foo] to mean ``the feature corresponding to foo, whatever that might be'' \fi

\iffalse TODO: Maybe present recurrent clustering in a ``naive'' way: define the conversion function ``[]'' recursively. Leave the dependency-ordering, etc. for the Algorithm section, as a way of ``building up to'' the result, which avoids the inefficiencies of recursing. \fi

We take the \emph{recurrent clustering} approach found in ML4PG and ACL2(ml), and implement a variant in the context of Haskell theory exploration. As a clustering algorithm, the aim of recurrent clustering is to identify similarities in a set of data points (in our case, Core expressions). Its distinguishing characteristic is to \emph{combine} feature extraction and clustering into a single recursive algorithm, to avoid problems with identifiers (described below). Here we describe our approach to recurrent clustering and compare its similarity and differences to those of ML4PG and ACL2(ml).

We consider our algorithm in two stages: the first transforms the nested structure of expressions into a flat feature vector representation; the second converts the discrete symbols of Core syntax into features (real numbers), which we will denote using the notation $\feature{}$.

\subsubsection{Flattening Expressions}

Our approach will make use of the k-means algorithm described in \S \ref{sec:kmeans}, which considers the elements of a feature vector to be orthogonal. Hence we must ensure that similar expressions not only give rise to similar numerical values, but crucially that these values appear \emph{at the same position} in the feature vectors. Since expressions can be nested in different ways, simple traversals (breadth-first, depth-first, post-order, etc.) may cause features from equivalent sub-expressions to be mis-aligned. For example, if we read features in breadth-first order we might get the following feature vectors (padded to the same length with $0$):

\begin{small}
  \begin{equation*}
    \begin{array}{r@{}l@{}l@{}}
      X\ &=\ \CAlt\ (\CDataAlt\ \id{x})\ & (\vlocal{y}) \\
      Y\ &=\ \CAlt\ \CDefault\           & (\vlocal{y})
    \end{array}
  \end{equation*}

  \begin{equation*}
    \begin{array}{r@{}l@{}l@{}l@{}l@{}l@{}l@{}l}
      breadthFirst(X)\ &=\ (\feature{\CAlt},\ &\feature{\CDataAlt},\ &\feature{\CVar},\ &\feature{\id{x}},\ &\feature{\CLocal},\ &\feature{\id{y}} &) \\
      breadthFirst(Y)\ &=\ (\feature{\CAlt},\ &\feature{\CDefault},\ &\feature{\CVar},\ &\feature{\CLocal},\ &\feature{\id{y}},\ &0 &)
    \end{array}
  \end{equation*}
\end{small}

Here the common sub-expression $\vlocal{y}$ is misaligned, such that only $\frac{1}{6}$ features are guaranteed to match (others may match by coincidence, depending on $\feature{}$). These feature vectors might be deemed very dissimilar, despite the intuitive similarity of the expressions $X$ and $Y$ from which they derive.

If we were to align these features instead, by padding the third column rather than the sixth, then $\frac{4}{6}$ of the features would be guaranteed to match, making the similarity of the vectors more closely match our intuition and depend less on coincidence.

There are many approaches we could take to increase alignment. Our method, described below, is a variation of breadth-first traversal which pads each level of nesting to a fixed size. This doesn't guarantee alignment, but it does prevent misalignment from accumulating from one level to the next. In this example, our method (when padding each level to 2 features) would give the following feature vectors:

\begin{small}
  \begin{equation*}
    \begin{array}{r@{}l@{}l@{}l@{}l@{}l@{}l@{}l@{}l@{}l}
      featureVec(X)\ &=\ (\feature{\CAlt},\ &0,\ &\feature{\CDataAlt},\ &\feature{\CVar},\ &\feature{\id{x}},\  &\feature{\CLocal},\ &\feature{\id{y}},\ &0 &) \\
      featureVec(Y)\ &=\ (\feature{\CAlt},\ &0,\ &\feature{\CDefault},\ &\feature{\CVar},\ &\feature{\CLocal},\ &0,\                 &\feature{\id{y}},\ &0 &)
    \end{array}
  \end{equation*}
\end{small}

For this reason, we cannot Core expressions  we will associate first associate features (real numbers) with the syntax of figure \ref{fig:coresyntax}. To preserve the in two stages: first we construct \emph{rose trees} of features (real numbers), which maintain the nested structure of expressions shown in figure \ref{fig:coresyntax}. We then flatten these rose trees into feature vectors.

Rose trees are defined recursively: $T$ is a rose tree if $T = (f, T_1, \dots, T_{n_T})$, where $f \in \mathbb{R}$ and $T_i$ are rose trees. $T_i$ are the \emph{children of} $T$ and $f$ is the \emph{feature at} $T$. $n_T$ may be different for each child; trees where $n_T = 0$ are \emph{leaves}.

We turn Core expressions into rose trees by converting terminal symbols (such as \hs{Lam} or \hs{Rec}) into features; we represent this conversion using $\feature{}$. Non-terminal symbols (such as $expr$ and $id$) are converted recursively and become children. This is shown in figure \ref{fig:totree}.

\begin{figure}
  \begin{align*}
    toTree(e) &=
    \begin{cases}
      (\feature{\CVar}, toTree(e_1))                                 & \text{if $e = \CVar\ e_1$} \\
      (\feature{\CLit}, toTree(e_1))                                 & \text{if $e = \CLit\ e_1$} \\
      (\feature{\CApp}, toTree(e_1), toTree(e_2))                    & \text{if $e = \CApp\ e_1\ e_2$} \\
      (\feature{\CLam}, leaf(l_1), toTree(e_1))                      & \text{if $e = \CLam\ l_1\ e_1$} \\
      (\feature{\CLet}, toTree(e_1), toTree(e_2))                    & \text{if $e = \CLet\ e_1\ e_2$} \\
      (\feature{\CCase}, toTree(e_1), leaf(l_1), toTree(a_1), \dots) & \text{if $e = \CCase\ e_1\ l_1 [a_1, \dots]$} \\
      (\feature{\CType})                                            & \text{if $e = \CType$} \\
      (\feature{\CLocal}, leaf(l_1))                                & \text{if $e = \CLocal\ l_1$} \\
      (\feature{\CGlobal}, leaf(g_1))                               & \text{if $e = \CGlobal\ g_1$} \\
      (\feature{\CLitNum})                                          & \text{if $e = \CLitNum\ n_1$} \\
      (\feature{\CLitStr})                                          & \text{if $e = \CLitStr\ s_1$} \\
      (\feature{\CAlt}, toTree(e_1), toTree(e_2), leaf{l_1}, \dots)  & \text{if $e = \CAlt\ e_1 e_2 [l_1, \dots]$}  \\
      (\feature{\CDataAlt}, leaf(g_1))                              & \text{if $e = \CDataAlt\ g_1$}  \\
      (\feature{\CLitAlt}, toTree(e_1))                             & \text{if $e = \CLitAlt\ e_1$}  \\
      (\feature{\CDefault})                                         & \text{if $e = \CDefault$}  \\
      (\feature{\CNonRec}, toTree(e_1))                             & \text{if $e = \CNonRec\ e_1$}  \\
      (\feature{\CRec}, toTree(e_1), \dots)                         & \text{if $e = \CRec\ [e_1, \dots]$} \\
      (\feature{\CBind}, leaf(l_1), toTree(e_1))                    & \text{if $e = \CBind\ l_1 e_1$}
    \end{cases} \\
    leaf(x) &= (\feature{x})
  \end{align*}
  \caption{Recursively converting Core expressions of figure \ref{fig:coresyntax} to rose trees. List elements (indicated with $\dots$) are appended as children, rather than being wrapped in another tree.}
  \label{fig:totree}
\end{figure}

To flatten a rose tree we fetch the features from each ``level'', concatenating those from adjacent children:

\begin{equation*}
  level(l, (x, T_1, \dots, T_{n_T})) =
    \begin{cases}
      (x) & \text{if $l = 1$} \\
      \vect{x}_1 \vect{x}_2 \dots \vect{x}_{T_n} & \text{if $l > 1$ and $\vect{x}_i = level(l - 1, T_i)$}
    \end{cases}
\end{equation*}

Given a rose tree $T$, we pad each level $level(1, T)$, $level(2, T)$, $\dots$, $level(r, T)$ with zeros and truncate them to length $c$. The resulting vectors are concatenated to form the feature vector for $T$.

See see figure \ref{fig:rosetreeexample} for an example of this construction.

\begin{figure}
  \begin{small}
  \begin{align*}
    toTree(\hs{plus}) &= \\
      (\feature{\CLam},
       \begin{array}[t]{@{}l@{}}
         (\feature{\id{a}}), \\
         (\feature{\CLam},
          \begin{array}[t]{@{}l@{}}
            (\feature{\id{y}}), \\
            (\feature{\CCase},
             \begin{array}[t]{@{}l@{}}
               (\feature{\CVar}, (\feature{\CLocal}, (\feature{\id{a}}))), \\
               (\feature{\id{b}}), \\
               (\feature{\CAlt},
                \begin{array}[t]{@{}l@{}}
                  (\feature{\CDataAlt}, (\feature{\id{Z}})), \\
                  (\feature{\CVar}, (\feature{\CLocal}, (\feature{\id{y}})))),
                \end{array} \\
               (\feature{\CAlt},
                \begin{array}[t]{@{}l@{}}
                  (\feature{\CDataAlt}, (\feature{\id{S}})), \\
                  (\feature{\CApp},
                   \begin{array}[t]{@{}l@{}}
                     (\feature{\CVar}, (\feature{\CGlobal}, (\feature{\id{S}}))), \\
                     (\feature{\CApp},
                      \begin{array}[t]{@{}l@{}}
                        (\feature{\CApp},
                         \begin{array}[t]{@{}l@{}}
                           (\feature{\CVar}, (\feature{\CGlobal}, (\feature{\id{plus}}))), \\
                           (\feature{\CVar}, (\feature{\CLocal}, (\feature{\id{x}})))),
                         \end{array} \\
                      (\feature{\CVar}, (\feature{\CLocal}, (\feature{\id{y}}))))),
                      \end{array}
                   \end{array} \\
                  (\feature{\id{x}})))))
                \end{array}
             \end{array}
          \end{array}
       \end{array} \\
foo &= bar
  \end{align*}
  \end{small}
  \caption{}
  \label{fig:rosetreeexample}
\end{figure}

\subsubsection{Feature Extraction}

We now define the algorithm for $\feature{}$, which turns terminal symbols into features. In the case of known language features, such as $\feature{\CLam}$ and $\feature{\CCase}$, we just enumerate the possibilities (while trying to make ``similar'' symbols have similar features).

The difficulty comes from identifiers, which we cannot hard-code into our algorithm in advance. The only meaningful way to directly compare identifiers is by equality, but since the space of identifiers is large, giving distinct features to distinct identifiers would make matches very unlikely. Since identifiers make up a significant portion of expressions, we would like a more fine-grained notion of similarity between their features than exact matching.

For example, consider the following feature vectors, which are identical except for the global identifiers:

\begin{equation*}
  \begin{split}
    \vect{x} =\ & \texttt{App}\ (\texttt{Var}\ (\texttt{Local}\ \texttt{"f"}))\ (\texttt{Var}\ (\texttt{Global}\ \texttt{"odd"})) \\
    \vect{y} =\ & \texttt{App}\ (\texttt{Var}\ (\texttt{Local}\ \texttt{"f"}))\ (\texttt{Var}\ (\texttt{Global}\ \texttt{"even"})) \\
    \vect{z} =\ & \texttt{App}\ (\texttt{Var}\ (\texttt{Local}\ \texttt{"f"}))\ (\texttt{Var}\ (\texttt{Global}\ \texttt{"plus"}))
  \end{split}
\end{equation*}

From their definitions in figure \ref{fig:coreexample} we might consider \hs{odd} to be more similar to \hs{even} than to \hs{plus}, and hence for $X$ to be more similar to $Y$ than to $Z$. In this sense, exact equality will under-estimate the similarity of $X$ and $Y$, whilst ignoring identifiers will over-estimate the similarity of $X$ and $Z$.

Recurrent clustering tackles this problem by comparing identifiers \emph{indirectly}: we look up the expressions \emph{referenced by} each identifier, and use \emph{their} similarity instead. However, this is a circular definition: to compare the similarity of, say $X$ and $Y$, we would need to look up the global identifiers and compare the expressions \hs{odd} and \hs{even}; yet \hs{odd} and \hs{even} are mutually recursive! To  themselves gives rise to a recursive process, where feature extraction is interleaved with multiple rounds of clustering. For this recursive process to be well-founded, we impose a topological ordering on expressions based on their dependencies (the expressions they reference). This is slightly complicated in Haskell (compared to Coq, for example), since general recursion is permitted and several mutually-recursive expressions may appear at the same level.

\iffalse TODO: the code should go in the implementation section rather than here \fi
\iffalse TODO: give a more abstract presentation, using mathematical notation \fi
\iffalse TODO: maybe focus on the ``interesting cases'', and defer the nitty-gritty of extending the environment, etc. to the implementation section? \fi
\iffalse TODO: Define rose trees \fi
\iffalse TODO: Show an example from earlier, e.g. factorial \fi
\iffalse TODO: Split into three parts: expressions to rose trees of features; rose trees to matrices to vectors; k-means clustering \fi
\iffalse TODO: highlight the recurrent nature of the algorithm \fi
\iffalse TODO: I would probably just restructure:
 - start with the syntax, get to show how, for every line of figure 1, extraction works
 - then transformation to feature vectors
 - then k-means
obviously, they are mutually recursive...
\fi

Our algorithm proceeds in several stages. At the top level, we send named expressions to \hs{recurrentCluster}, topologically sorted: if an element contains multiple \hs{(Global, Expr)} pairs, they are mutually-recursive:

\begin{haskell}
recurrentCluster :: [[(Global, Expr)]] -> [[Global]]
recurrentCluster = go ([],[])
  where go (fs, db) []       = db
        go (fs, db) (es:ess) = let fs' = fs ++ map (extract db) es
                                   db' = kMeans fs'
                                in go (fs', db') ess
        extract db (i, e) = (i, rt db e)
\end{haskell}

The overall result of the \hs{recurrentCluster} function is a list of clusters, containing the IDs of their elements. These are obtained by interleaving clustering (the \hs{kMeans} function) and feature extraction (the \hs{rt} function).

The feature extraction itself contains two parts: first, syntax trees matching the grammar in figure \ref{fig:coresyntax} are converted into \hs{RoseTree}s with features (\hs{Float}s) on their \hs{Node}s:

\begin{haskell}
data RoseTree = Node Feature [RoseTree]

eRT :: [Local] -> [[Global]] -> Expr -> RoseTree
eRT env db e = case e of
  ...
\end{haskell}

The easiest branches to handle are literals and types, which we represent using particular feature values (\hs{sLITNUM}, \hs{sLITSTR} and \hs{sTYPE}):

\begin{haskell}
  Lit (LitNum _) -> Node sLITNUM []
  Lit (LitStr _) -> Node sLITSTR []
  Type           -> Node sTYPE   []
\end{haskell}

Function application simply recurses into both sub-expressions:

\begin{haskell}
  App e1 e2 -> Node sAPP [eRT env db e1,
                          eRT env db e2]
\end{haskell}

To look up variables locally and globally, we use the \hs{lookupL} and \hs{lookupG} functions, respectively. These return the index containing their argument, if found. In the case of \hs{lookupL}, this acts as a de Bruijn index to give alpha-equivalent terms equal feature vectors. For \hs{lookupG}, this is the ID of the cluster it appears in; this ensures that references to similar expressions result in similar features:

\begin{haskell}
  Var (Global i) -> Node (lookupG db  i) []
  Var (Local  i) -> Node (lookupL env i) []
\end{haskell}

Finally, when we traverse binders we must extend the environment \hs{env}:

\begin{haskell}
  Lam  i  e     -> Node sLAM [eRT (i:env) db e]
  Let  bs e     -> Node sLET (map (bRT (ids bs:env) db) bs ++
                                   eRT (ids bs:env) db  e
  Case e i alts -> Node sCASE (eRT    env  db  e :
                          map (aRT (i:env) db) alts)
\end{haskell}

Patterns and bindings are handled in a similar way:

\begin{haskell}
aRT :: [Local] -> [[Global]] -> Alt -> RoseTree
aRT env db alt = case alt of
  (DataAlt _, vs, e) -> eRT (vs ++ env) db e
  (LitAlt  _, _,  e) -> eRT env db e
  (Default,   _,  e) -> eRT env db e

bRT :: [Local] -> [[Global]] -> Bind -> RoseTree
bRT env db b = case b of
  NonRec i e -> eRT env db e
  Rec es     -> Node sREC (map (eRT env db . snd) es)
\end{haskell}

The result of \hs{eRT} is a \hs{RoseTree} whose branching structure mimics that of our original expression. We next need to convert this to a \emph{matrix} of features, which we do by converting each level of the \hs{RoseTree} into a row of the matrix (using \hs{pad} to ensure a consistent size). Finally we turn the matrix into a \emph{feature vector} by concatenating the rows:

\begin{haskell}
level :: Int -> RoseTree -> [Feature]
level 0 (Node x xs) = [x]
level n (Node x xs) = concatMap (level (n-1)) xs

rt :: [[Global]] -> Expr -> [Feature]
rt db e = concat (pad cols (map (`level` eRT [] db e) [0..rows]))
\end{haskell}

Notice that this algorithm contains several parameters, including \hs{rows} and \hs{cols} which determine how to truncate the matrices (defaults are 30). The \hs{kMeans} function also contains a parameter for the cluster number; we set this as $\sqrt{n}$ where $n$ is the number of feature vectors being clustered.

\subsubsection{Comparison}

Our algorithm is most similar to that of ML4PG, as our transformation maps each element in a tree to a distinct cell in its matrix. In contrast, the matrices produced by ACL2(ml) \emph{summarise} the tree elements: providing, for each level of the tree, the number of variables, nullary symbols, unary symbols, etc.

There are two major differences between our algorithm and that of ML4PG: mutual-recursion and types.

The special handling required for mutual recursion is discussed above (namely, topological sorting of expressions and the \hs{sUNKNOWN} sentinel). Such handling is not present in ML4PG, since the Coq code it analyses must, by virtue of the language, be written in dependency order to begin with. Coq \emph{does} have limited support for mutually-recursive functions, of the following form:

\begin{coqblock}
Fixpoint even n := match n with
                       | O   => true
                       | S m => odd m
                   end
    with odd  n := match n with
                       | O   => false
                       | S m => even m
                   end.
\end{coqblock}

However, this is relatively uncommon and unsupported by ML4PG.

The more interesting differences come from our handling (or lack thereof) for types. Coq and ACL2 are at opposite ends of the typing spectrum, with the former treating types as first class entities of the language whilst the latter is untyped (or \emph{unityped}). In both cases, we have a \emph{single} language to analyse, by ML4PG and ACL2(ml) respectively.\footnote{ML4PG can also analyse Coq's \textsc{Ltac} meta-language. Haskell has its own meta-language, Template Haskell, but here we only consider the regular Haskell which it generates.}

The situation is different for Haskell, where the type level is distinct from the value level, and there are strict rules for how they can influence each other. In particular, Haskell values can depend on types (via the type class mechanism) but types cannot depend on values.

In our initial approach, we restrict ourselves to the value level. This has several consequences:

\begin{itemize}
  \item Although they are values, we cannot distinguish between data constructors, other than using exact equality.
  \item Since Core uses a single \texttt{Lam} abstraction for both value- and type-level parameters, we cannot always distinguish between them. This can cause a function's Core arity to be greater than its Haskell arity.
\end{itemize}

There is certainly promise in including types in our analysis, by pairing every term with its type as in ML4PG. This will allow fine-grained distinction of expressions which are otherwise identical, especially data constructors.
