\section{Future Work}
\label{sec:future}

Our use of clustering to pre-process \qspec{} signatures has required many decisions and tradeoffs to be made. Hence our approach is just one possibility out of many alternatives which could be investigated to push this work further.

The most obvious next step for our algorithm is to incorporate types. Types contain valuable information about an expression, and allow us to distinguish between constructors. Our algorithm closely follows that of ML4PG, which \emph{does} support types, by populating matrix cells with tokens \emph{and} their types. Unfortunately this is more complicated in Haskell than it is in Coq, since types form a separate part of the language from terms, and we do not have an interactive Core environment to query for types (unlike ML4PG, which runs inside the Proof General environment).

We can also compare the performance of our hand-selected features with \emph{learned} representations, like those reviewed in \cite{bengio2013representation}. This may provide an indication of how important it is to understand the language when identifying salient aspects of expressions, and how difficult various aspects of it might be to learn.
