\iffalse
\documentclass[]{article}

\usepackage{rotating}
\usepackage{hyperref}
\usepackage{graphicx}
\usepackage{mathtools}
\usepackage{amssymb}
\usepackage[T1]{fontenc}
%\usepackage{enumitem}
\usepackage{paralist}
\usepackage{csquotes}
\usepackage[affil-it]{authblk}
\usepackage{listings}
\usepackage{fixltx2e}
\usepackage[numbers]{natbib}
\usepackage{mathpartir}
\usepackage{mmm}

% Treat paragraph as subsubsubsection
\usepackage{titlesec}
\setcounter{secnumdepth}{4}

\DeclareMathOperator{\md5}{MD5}

\newcommand{\blank}{\cdot}

\fi

\documentclass[runningheads,a4paper]{llncs}

\usepackage{amsmath}
\usepackage{amssymb}
\setcounter{tocdepth}{3}
\usepackage{graphicx}
\usepackage[caption=false]{subfig}

\usepackage{tikz-qtree}
\usepackage{tikz}
\usetikzlibrary{shapes,arrows,decorations.pathreplacing,calc}

\usepackage{algorithm}
\usepackage[noend]{algpseudocode}

\usepackage{url}
\urldef{\mailsa}\path|cmwarburton@dundee.ac.uk, katya@computing.dundee.ac.uk|
\newcommand{\keywords}[1]{\par\addvspace\baselineskip
\noindent\keywordname\enspace\ignorespaces#1}

% Abbreviations
\newcommand{\qcheck}{\textsc{QuickCheck}}
\newcommand{\qspec}{\textsc{QuickSpec}}
\newcommand{\hspec}{\textsc{HipSpec}}
\newcommand{\equal}{=}
\newcommand{\fc}{System $F_C$}

% Semantic markup
\providecommand{\coq}[1]{\texttt{#1}}
\newenvironment{haskell}{\ttfamily}{\par}
\newenvironment{coqblock}{\ttfamily}{\par}
\newcommand{\hs}[1]{\texttt{#1}}
\newcommand*\vect[1]{\mathbf{#1}}
\newcommand*\mean[1]{\overline{#1}}
\newcommand{\argmin}{\operatornamewithlimits{argmin}}
\newcommand{\feature}[1]{\phi(#1)}
\newcommand{\id}[1]{\texttt{"#1"}}
\newcommand{\vlocal}[1]{\CVar\ (\CLocal\ \id{#1})}
\newcommand{\vglobal}[1]{\CVar\ (\CGlobal\ \id{#1})}
\newcommand{\cat}{\mbox{\ensuremath{+\!\!+\,}}}

\begin{document}

\mainmatter  % start of an individual contribution

% first the title is needed
\title{Improving Haskell Theory Exploration}

% a short form should be given in case it is too long for the running head
\titlerunning{Improving Haskell Theory Exploration}

% the name(s) of the author(s) follow(s) next
%
% NB: Chinese authors should write their first names(s) in front of
% their surnames. This ensures that the names appear correctly in
% the running heads and the author index.
%
\author{Chris Warburton%
\and Ekaterina Komendantskaya}
%
\authorrunning{Improving Haskell Theory Exploration}
% (feature abused for this document to repeat the title also on left hand pages)

% the affiliations are given next; don't give your e-mail address
% unless you accept that it will be published
\institute{University of Dundee,\\
\mailsa\\
\url{http://tocai.computing.dundee.ac.uk}}

%
% NB: a more complex sample for affiliations and the mapping to the
% corresponding authors can be found in the file "llncs.dem"
% (search for the string "\mainmatter" where a contribution starts).
% "llncs.dem" accompanies the document class "llncs.cls".
%

%\toctitle{Lecture Notes in Computer Science}
\tocauthor{Improving Haskell Theory Exploration}
\maketitle

\begin{abstract}
Theory Exploration is a promising approach to improving the quality and understanding of software. It extends previously existing methods available through testing, in languages which are amenable to formal analysis such as those based on pure functional programming. Current theory exploration techniques are limited by their use of exponential time algorithms, which although thorough are ultimately limited to finding simple properties of small systems. We propose a more powerful approach, which uses machine learning algorithms to intelligently choose which parts of a system to explore based on their similarity, hence focusing its efforts on areas which are most likely to lead to discoveries.

\end{abstract}

\section{Introduction}

As computers and software become more sophisticated, and as our reliance on them increases, the importance of \emph{understanding}, \emph{predicting} and \emph{verifying} these systems grows; which is undermined by their ever-increasing complexity. The \emph{functional programming} paradigm has been proposed for addressing these issues \cite{hughes1989functional}, by constructing programs which are more amenable to mathematical analysis. For example, in pure functional programming all values are \emph{immutable}: defined once and never changed. Hence there is no way for a value to be altered between the point it is introduced and the point it is used, unlike in many \emph{imperative} languages where we may have to search the whole program to ensure the value is not altered by any intermediate code. Similarly, the results of pure functions cannot depend on any state other than their arguments, and hence will always produce repeatable results. By making state implicit in this way, powerful type systems can be used to constrain the behaviour of programs, and to give a rich, composable structure to data.

Whilst use of pure functional programming languages, like Haskell and Idris, is relatively rare, their features are well suited to common software engineering practices like \emph{unit testing}; where tasks are broken down into small, easily-specified ``units'', and tested in isolation for a variety of use-cases. Functional ideas are thus spreading to mainstream software engineering in a more dilute form; seen, for example, in the recent inclusion of first-class functions in Java \cite{gosling2015java} and C++ \cite{willcock2006lambda}.

Functional programming is also well suited to less-widespread practices, such as \emph{property checking} (as popularised by \qcheck{}) and \emph{theorem proving}, which are promising methods for increasing confidence in software, yet can be prohibitively expensive. Here we investigate how the recent \emph{theory exploration} approach can lower the effort required to pursue these goals, and in particular how machine learning techniques can mitigate the costs of the combinatorial algorithms involved.

Our contributions are:

\begin{enumerate}
  \item The application of machine learning algorithms to theory exploration, for intelligently discovering interesting sub-sets of Haskell libraries, which are more tractable to explore.
  \item A novel feature extraction method for transforming Haskell expressions into a form amenable to off-the-shelf learning algorithms.
  \item An implementation of these feature extraction and theory exploration approaches.
  \item A comparison of our methods with existing approaches, both for theory exploration in Haskell, and for machine learning in other languages.
\end{enumerate}

We begin in \S \ref{sec:background} by providing a formal context for analysing Haskell expressions (\S \ref{sec:haskell}) and describe the \qspec{} theory exploration system (\S \ref{sec:theoryexploration}). We give a brief overview of testing approaches and how they relate to Haskell (\S \ref{sec:quickcheck}), as well as the machine learning approaches we are building on (\S \ref{sec:featureextraction}). We discuss our contributions in more depth in \S \ref{sec:contributions}, and provide implementation details in \S \ref{sec:implementation}. A variety of related work is surveyed in \S \ref{sec:related}, we briefly evaluate our implementations in \S \ref{sec:evaluation} and give several potential directions for future research in \S \ref{sec:future}.

\section{Background}
\label{background}

\iffalse
TODO{2015-11-14}

Background

2.1 Introduce language signature
 b) stuff from QuickCheck/Spec

2.2? Machine learning and feature extraction
~1 page

2.3: QuickSpec, HipSpec
\fi

\iffalse

Mathematics provides us with powerful, systematic methods of reasoning, which we can bring to bear on this challenge; in particular those of \emph{formal logic} and \emph{statistics}. By (partially) mechanising these approaches, in the fields of \emph{theorem proving} and \emph{machine learning}, respectively, we can leverage these increasing machine capabilities and direct them for the purpose of analysis. However, the question still remains: on what should we focus that analysis?

In this work, we investigate the notion of \emph{interestingness} in the exploration of formal systems (an area known as \emph{theory exploration}) as a way to make productive use of resources in an often intractable domain. To keep things concrete, we focus our formal analysis on equational formulae describing programs in the Haskell language, for reasons elaborated in \S \ref{haskell}. Conceptually, we maintain a broader view, and survey many related areas which may offer insights on the problem.

Appeals to interestingness arise when more direct measures, such as utility, are not available. For example, the inclusion of particular statements in a program or proof development can be easily justified based on their contribution to the overall solution; however, in a \emph{library} there is no particular problem being solved, in which case we must judge statements on less direct criteria, such as how ``interesting'' they may be to our users. Appeals to interestingness abound in the history of computer-assisted reasoning; for example, in 1971 Plotkin \citep{plotkin1971further} considered the task of \textquote{discovering theorems $T$ from a system of axioms $Ax$}, and in particular the questions \textquote{Under what conditions is $T$ an interesting, possible theorem in the system $Ax$?} and \textquote{Is there a way to generate (most) interesting possible theorems?}. Despite such widespread use of the term, there is no standard definition of what makes a formal object, whether it is an axiom, a conjecture, a proof, etc., ``interesting''; although many ad-hoc heuristics have been proposed.

We begin our undertaking in \S \ref{background} by introducing the Haskell language, as well as the relevant fields of verification for context. We define a formal framework for our investigation, and show how it relates to the existing theorem proving landscape. A selection of theorem proving scenarios which \emph{require} exploration are discussed in \S \ref{examples}, whilst related work, including existing defintions of interestingness, is surveyed in \S \ref{related}. We also review the use of exploration in other fields of Artificial Intelligence and Machine Learning, where researchers are experimenting with replacing \emph{explicit} goals and rewards with \emph{implicit} alternatives such as interestingness. Recent efforts in this area have lead to the emergence of principled theories, mostly based around (algorithmic) information theory, which may be adapted to our theory exploration context.

We discuss our present contributions in \S \ref{current} and future research directions in \S \ref{future}, before concluding in \S \ref{conclusion}.

\fi

\subsection{Haskell}
\label{haskell}

\begin{figure}
  \begin{equation*}
    \begin{split}
      expr\    \rightarrow\ & \texttt{Var}\ id                                       \\
                         |\ & \texttt{Lit}\ literal                                  \\
                         |\ & \texttt{App}\ expr\ expr                               \\
                         |\ & \texttt{Lam}\ \mathcal{L}\ expr                        \\
                         |\ & \texttt{Let}\ bind\ expr                               \\
                         |\ & \texttt{Case}\ expr\ \mathcal{L}\ \left[ alt \right]   \\
      id\      \rightarrow\ & \texttt{Local}\ \mathcal{L}                            \\
                         |\ & \texttt{Global}\ \mathcal{G}                           \\
      literal\ \rightarrow\ & \texttt{LitNum}\ \mathcal{N}                           \\
                         |\ & \texttt{LitStr}\ \mathcal{S}                           \\
      alt\     \rightarrow\ & ( altcon,\ [\mathcal{L}],\ expr )                      \\
      altcon\  \rightarrow\ & \texttt{DataAlt}\ \mathcal{G}                          \\
                         |\ & \texttt{LitAlt}\ literal                               \\
                         |\ & \texttt{Default}                                       \\
      bind\    \rightarrow\ & \texttt{NonRec}\ \mathcal{L}\ expr                     \\
                         |\ & \texttt{Rec}\ [ ( \mathcal{L},\ expr ) ]
    \end{split}
  \end{equation*}
  where:
  \begin{tabular}[t]{l @{ $=$ } l}
    $\mathcal{S}$ & string literals    \\
    $\mathcal{N}$ & numeric literals   \\
    $\mathcal{L}$ & local identifiers  \\
    $\mathcal{G}$ & global identifiers
  \end{tabular}

  \caption{Simplified syntax of GHC Core in BNF style. $[]$ and $(,)$ denote repetition and grouping, respectively.}
  \label{coresyntax}
\end{figure}

The full Haskell language is rather complex, so we focus instead on \emph{Core}, an intermediate representation of the Glasgow Haskell Compiler (GHC) based on \fc. For a full treatment of \fc, and its use in GHC, see \citep[Appendix C]{sulzmann2007system}. The sub-set of Core we consider is shown in figure \ref{coresyntax}; compared to the full language \footnote{As of GHC version 7.10.2, the latest at the time of writing.} we erase types and use a custom representation of names. There are also several other forms of literal (machine words of various sizes, individual characters, etc.) which we omit for brevity, as their treatment is similar to those of strings and numerals.

\subsection{QuickCheck}

Although unit testing is the de facto industry standard for quality assurance in non-critical systems, the level of confidence it provides is rather low, and totally inadequate for many (e.g. life-) critical systems. To see why, consider the following Haskell function, along with some unit tests (see \ref{haskell} for more details on Haskell):

\begin{lstlisting}[language=Haskell, xleftmargin=.2\textwidth, xrightmargin=.2\textwidth]
factorial 0 = 1
factorial n = n * factorial (n-1)

fact_base      = factorial 0 == factorial 1
fact_increases = factorial 3 <= factorial 4
fact_div       = factorial 4 == factorial 5 `div` 5
\end{lstlisting}

The intent of the function is to map an input $n$ to an output $n!$. The tests check a few properties of the implementation, including the base case, that the function is monotonically increasing, and a relationship between adjacent outputs. However, these tests will \emph{not} expose a serious problem with the implementation: it diverges on half of its possible inputs!

All of Haskell's built-in numeric types allow negative numbers, which this implementation doesn't take into account. Whilst this is a rather trivial example, it highlights a common problem: unit tests are insufficient to expose incorrect assumptions. In this case, our assumption that numbers are positive has caused a bug in the implementation \emph{and} limited the tests we've written.

If we do manage to spot this error, we might capture it in a \emph{regression test} and update the definition of \hs{factorial} to handle negative numbers, e.g. by taking their absolute value:

\begin{lstlisting}[language=Haskell, xleftmargin=.2\textwidth, xrightmargin=.2\textwidth]
factorial 0 = 1
factorial n = let nPos = abs n
               in nPos * factorial (nPos - 1)

fact_neg = factorial 1 == factorial (-1)
\end{lstlisting}

However, this is \emph{still} not enough: since this function only uses generic numeric operations, it will be polymorphic; allowing all numeric types\footnote{Its type will be of the form \hs{forall t. Num t => t -> t}, where \hs{Num t} constrains the type variable \hs{t} to be numeric. There will also be extra contraints like \hs{Eq} due to the use of \hs{==} in the unit tests.}. If we provide a fractional value, the function will again diverge. Clearly, by choosing what to test we are biasing the test suite towards those cases we've already taken into account, whilst neglecting the problems we did not expect.

Haskell offers a partial solution to this problem in the form of \emph{property checking}\footnote{See section \ref{propertychecking} for more information}. Tools such as \qcheck{} separate tests into three components: a \emph{property} to check, which unlike a unit test may contain \emph{free variables}; a source of values to instantiate these free variables; and a stopping criterion. Here is how we might restate our unit tests as properties:

\begin{lstlisting}[language=Haskell, xleftmargin=.2\textwidth, xrightmargin=.2\textwidth]
fact_base        = factorial 0 == factorial 1
fact_increases n = factorial n <= factorial (n+1)
fact_div       n = factorial n == factorial (n+1) `div` (n+1)
fact_neg       n = factorial n == factorial (-n)
\end{lstlisting}

The free variables (all called \hs{n} in this case) are abstracted as function parameters; these parameters are implicitly \emph{universally quantified}, i.e. we've gone from a unit test asserting $factorial(3) \leq factorial(4)$ to a property asserting $\forall n, factorial(n) \leq factorial(n+1)$. Notice that unit tests like \hs{fact_base} are valid properties; they just assert rather weak statements.

To check these properties, \qcheck{} treats closed terms (like \hs{fact_base}) just like unit tests: pass if they evaluate to \hs{True}, fail otherwise. For open terms, a random selection of values are generated and passed in via the function parameter; the results are then treated in the same way as closed terms. The default stopping criterion for \qcheck{} (for each test) is when a single generated test fails, or when 100 generated tests pass.

The ability to state \emph{universal} properties in this way avoids some of the bias we encountered with closed values. In the \hs{factorial} example, this manifests in two ways:

\begin{itemize}
  \item \qcheck{} cannot test polymorphic functions; they type must be \emph{monomorphised} first (instantiated to a particular concrete type). This is a technical limitation, since \qcheck{} must know which type of values to generate, but in our example it would bring the issue with fractional values to our attention.

  \item The generators used by \qcheck{} depend only on the \emph{type} of value they are generating: since \hs{Int} includes positive and negative values, the \hs{Int} generator will output both. This will expose the problem with negative numbers, which we weren't expecting.
\end{itemize}

Property checking is certainly an improvement over unit testing, but the problem of tests being biased towards expected cases remains, since we are manually specifying the properties to be checked.

We can reduce this bias further through the use of \emph{theory exploration} tools, such as \qspec{} and \hspec{}. These programs \emph{discover} properties of a ``theory'' (e.g. a library), through a combination of brute-force enumeration, random testing and (in the case of \hspec{}) automated theorem proving.\footnote{See section \ref{atp} for more information on automated theorem proving.}

\iffalse

Most work in mechanised, formal mathematics focuses on a \emph{theorem proving} paradigm, usually either \emph{interactive theorem proving} (ITP) or \emph{automated theorem proving} (ATP). Whilst the boundaries between these approaches can be somewhat blurred, we give precise definitions in \S \ref{itp} and \S \ref{atp}, respectively, and characterise both as being \emph{goal-driven}: particular conjectures must be provided as input to the process, and the output is either a proof, a refutation or ``don't know'' (in the case of incomplete procedures).

In this work, we instead adopt the paradigm of \emph{(automated) theory exploration}, discussed in \S \ref{theoryexploration}, which includes the ability to \emph{generate} conjectures, and hence output \emph{novel} theorems. Since many theories will admit an infinite number of trivial theorems (e.g. $\top$, $\top \land \top$, $(\top \land \top) \land \top$, \dots) we require some mechanism for avoiding such undesirable, yet provable, statements. This is the role of interestingness, which we can use to \emph{judge} a statement, in a potentially rich way, rather than merely constructing or failing to construct a proof.

\fi

\subsection{Interactive Theorem Proving}
\label{itp}

ITP is based around a \emph{proof checker} $C$, which is a decision procedure for determining if a given value $P$ (known as a \emph{proof object} or \emph{witness}) consititutes a proof of a given statement $S$:

$$ C \colon (P \times S) \rightarrow Boolean $$

The process of ITP can hence be understood as the search for an appropriate $P$ for our \emph{goal} $S$:

$$ ITP \colon (S \times C) \rightarrow \{ P \mid C(P, S) = True \} $$

It just so happens that a useful way to implement such a system is via pure functional programming, with the result that many ITP systems (AKA \emph{proof assistants}) such as Coq and Agda appear very similar to languages like Haskell. In particular, we can represent statements $S$ as types and proofs $P$ as values, in which case the proof checker $C$ is simply a type-checker. There are some obvious quirks, such as the need to be \emph{total} (not Turing-complete) in order to ensure soundness, but overall many of the features we introduced for Haskell, such as parametricity, type classes and ADTs are directly translatable to the ITP setting.

This coincidence is due to the \emph{Curry-Howard correspondence} \citep{wadler2015propositions}, which identifies programming languages with systems of logic. Functional programming languages correspond to intuitionistic logics, and are hence a natural fit for reasoning on computers. With additional axioms, we can extend these to more familiar classical logics, although these can make it harder to compute.

Most differences between functional programming and ITP stem from different \emph{expectations} of the user. In the case of Haskell, the desire for expressivity outweighs the desire for soundness, and hence its designers opted to make it Turing-complete. For an ITP language like Agda, soundness far outweighs the inconvenience of having to prove termination, so the opposite tradeoff is made. Similarly, since many ITP ``programs'' (proofs) will never be executed, the language can avoid optimisation in favour of simplicity (and hopefully correctness). This approach is summed up by the \emph{de Bruijn criterion}, which requires that the system \textquote{generates 'proof-objects' (of some form) that can be checked by an 'easy' algorithm} \cite[\S~2]{barendregt2001proof}. One nice consequence of this approach, which is followed for example by Isabelle \citep{nipkow2002isabelle} and Coq \citep{bertot2013interactive}, is that the proof assistants themselves can become arbitrarily complex and even buggy, yet as long as the proof checker remains simple we can maintain a high degree of confidence in the results.

Their emphasis on \emph{interactivity}, mostly caused by operating in undecidable domains, means ITP can require an enormous effort for non-trivial proof or verification tasks (e.g. see \citep{hales2015formal}). One common way to mitigate this problem is by implementing powerful \emph{tactics}: meta-programs which automate as much of a proof as possible. The most striking example of such meta-programming is the \emph{Sledgehammer} component of Isabelle/HOL \citep{journals/iandc/MengQP06}, which invokes a multitude of ATP systems on (a translated form of) the current goal to see if any can provide a proof which Isabelle's core checker will accept.

\subsection{Automated Theorem Proving}
\label{atp}

ATP systems are similar in principle to proof assistants, but are based around a \emph{proof search} algorithm rather than a proof checker. By using a fixed algorithm for proving, such as \emph{resolution} \cite[\S~9.6]{Russell:2003:AIM:773294} or \emph{superposition} \citep{bachmair1994rewrite}, ATP programs are limited to particular decidable or semi-decidable fragments of logic. In particular, most ATP systems (such as E \citep{schulz2013system} and Vampire \citep{riazanov2003implementing}) operate in classical first-order logic. This is the most striking difference from ITP systems (e.g. Coq, Agda and Isabelle) which operate in higher-order logics.

In addition to its interest to logicians, ATP has been actively researched in the field of artificial intelligence, dating back to the founding of the field at the 1956 Dartmouth conference. Even by that time Newell and Simon had developed their Logic Theory Machine \citep{newell1956logic}, which was subsequently able to prove theorems like those in Principia Mathematica \citep{newell1958elements}. The approaches now known as \emph{good old-fashioned AI} (GOFAI) were due in part to the success of automated theorem proving, as attempts were made to formulate many problems in a way amenable to these powerful first-order reasoners.

Recently there has been a trend away from this direction, towards statistical formulations amenable to machine learning, which we will review in \S \ref{related}.

\subsection{Theory Exploration}
\label{theoryexploration}

\begin{figure}
  \begin{equation*}
    \begin{split}
      name\   =\ & N_1\ |\ \dots\ |\ N_n                               \\
      type\   =\ & T_1\ |\ \dots\ |\ T_m\ |\ type \rightarrow type   \\
      sig\    =\ & \{(n_i | (name, type); sig                                     \\
      expr_{\mathcal{s}}\ =\ & v    : t\ \text{if}\ v : t \in s                        \\
              |\ & f\ x : b\ \text{if}\ f : a \in expr_s \wedge x : a \in expr_s \\
      conj_s\ =\ & \{ a =_t b \mid a : t \in expr_s \wedge b : t \in expr_s \}
    \end{split}
  \end{equation*}

  \caption{General model of \qspec{} conjecture generation, given a signature $\Sigma$ containing names $N_i$ and types $T_i$.}
  \label{coresyntax}
\end{figure}

The most striking difference between theory exploration (TE) and theorem proving is the former's lack of an explicit goal (e.g. a user-provided conjecture to prove). Instead, the \emph{implicit} goal is more open-ended: the discovery of ``interesting'' conclusions from the given premises. Of course, a key question to ask is what do we mean by ``interesting''? There is no single answer, although many approximate measures have been proposed. The choice of what counts as interesting has been used to classify various TE implementations \citep{warburtonscaling}, along with their approaches to term generation, well-formedness and proof.

Following \citep{warburtonscaling}, we call a pair $(\Sigma, V)$, of a signature and a set of variables, a \emph{theory}. \emph{Theory exploration} then refers to any process $(\Sigma, V) \overset{TE}{\rightarrow} \text{Terms}(\Sigma, V)$ for producing terms of the theory which are well-formed, provable and ``interesting''.

The conditions of well-formedness and provability can be handled through the use of type systems and automated theorem provers. Hence the method of conjecture generation, and the determination of what is interesting, are the most important properties of any TE system.

Early implementations like \textsc{Theorema} \citep{buchberger2000theory} provided interactive environments, similar to computer algebra systems and interactive theorem provers, to assist the user in finding theorems. Like in the ITP approach, the relatively simple process of verification is automated, whilst the more difficult tasks (term generation and the determination of their interest) are left up to the user.

Subsequent systems have investigated \emph{automated} theory exploration, for tasks such as lemma discovery \citep{Hipster}. By removing user interaction, these properties must be implemented by algorithms. In existing systems these are tightly coupled to improve efficiency, which makes it difficult to try different approaches independently.

As an example, \qspec{} \citep{QuickSpec} discovers equations about Haskell code, which are defined as ``interesting'' if they cannot be simplified using previously discovered equations. The intuition for such criteria is to avoid special cases of known theorems, such as $0 + 0 = 0$, $0 + 1 = 1$, etc. when we already know $0 + x = x$. Whilst this interestingness judgement is elegantly implemented with a congruence closure relation (version 1) and a term rewriting system
(version 2), the conjecture generation is performed in a brute-force way.

Although \qspec{} only \emph{tests} its equations rather than proving them, it is still used as the exploration component of more rigorous systems like \hspec{} and \textsc{Hipster}.

\hspec{} is currently the most powerful TE system for Haskell, and also forms a major component of \textsc{Hipster}. It uses off-the-shelf ATP programs (including AltErgo, Vampire, Z3, E, SPASS and CVC4) to verify the conjectures of \qspec{}. \qspec{}, in turn, enumerates all type-correct combinations of the terms in the theory up to some depth, groups them into equivalence classes using the \qcheck{} counterexample finder, then conjectures equations relating the members of these classes. This approach works well as a lemma generation system, making \hspec{} a capable inductive theorem prover as well as a theory exploration system \citep{claessen2013automating}. \hspec{} is also compatible with Haskell's existing testing infrastructure, such that an invocation of \texttt{cabal test} can run \hspec{} alongside more traditional QA tools like \qcheck{}, \textsc{HUnit} and \textsc{Criterion}.

In fact, there are similarities between the way a TE system like \hspec{} can generalise from proving \emph{particular} theorems to \emph{inventing} theorems, and the way counterexample finders like \qcheck{} can generalise from testing \emph{particular} expressions to \emph{inventing} expressions to test. There are likely lessons that each can learn from the other's approach to term generation.

We identify the following areas for researching potential improvements in existing TE systems such as \hspec{}:

\begin{description}
\item{Term generation}:
  Enumerating all type-correct terms is a brute-force solution to this question. Scalable alternatives to brute-force algorithms are a well-studied area of Artificial Intelligence and Machine Learning \iffalse TODO{2015-11-15}: Related work reference \fi. In particular, heuristic search algorithms like those surveyed in \citep{blum2011hybrid} could be used.
  We could also use Machine Learning methods like those used for \emph{relevance filtering} \ref{relevance} to identify some sub-set of a given theory, to prioritise over the rest.

\item{Interestingness}
  Various alternative interestingness criteria have been proposed, which we survey in \S \ref{relatedwork}. Augmenting or replacing the criteria may be useful, for example to distinguish useful relationships from incidental coincidences; or to prevent surprising, insightful equations from being discarded because they can be simplified.
\end{description}


% Theory exploration is similar to \emph{experimental mathematics}

% - Relation to Science
%  - Testable/falsifiable hypotheses are like evaluable terms (or, more generally, conjectures which can be decided, using a reasonable amount of resources).
% - Relation to AI tasks: exploring surroundings, etc.

% - Statistics is another area that's less straightforward than normal numerical computing, since there is subjectivity and judgement involved in the answering of questions.

% Theory formation: Alison? Others.
% Theory exploration: Buchberger, Moa in Isabelle, Koen in Haskell. Others?
% Theorem proving: Well-trodden: first-order ATP, higher-order ITP, functional programming
% Communication: Latex, Wikis, APIs, communicating with aliens

\section{Contributions}
\label{sec:contributions}

\subsection{Recurrent Clustering}
\label{sec:recurrentclustering}

We adapt the methodology of \emph{recurrent clustering} proposed in \cite{DBLP:journals/corr/HerasK14} \cite{heras2013proof}, and suggest a new recurrent clustering and feature extraction algorithm for Haskell, which we then evaluate as a relevance filter technique for theory exploration. As a clustering algorithm, the aim of recurrent clustering is to identify similarities in a set of data points (in our case, Core expressions). Rather than clustering expressions directly, we first transform them into numeric vectors in a process called \emph{feature extraction}. The distinguishing characteristic of recurrent clustering is to \emph{combine} feature extraction and clustering into a single recursive algorithm (shown as Algorithm \ref{alg:recurrent}), which allows the feature vector of an expression to depend on those of the expressions it references. Here we describe our approach to recurrent clustering and compare its similarity and differences to those of ML4PG and ACL2(ml).

We consider our algorithm in two stages: the first transforms the nested structure of expressions into a flat vector representation; the second converts the discrete symbols of Core syntax into features (real numbers), which we will denote using the function $\phi$.

\subsubsection{Expressions to Vectors}
\label{sec:expressionstovectors}

Our recurrent clustering algorithm is based on k-means clustering, which considers the elements of a feature vector to be \emph{orthogonal}. Hence we must ensure that similar expressions not only give rise to similar numerical values, but crucially that these values appear \emph{at the same position} in the feature vectors. Since different patterns of nesting can alter the ``shape'' of expressions, simple traversals (breadth-first, depth-first, post-order, etc.) may cause features from equivalent sub-expressions to be mis-aligned. For example, consider the following expressions, which represent pattern-match clauses with different patterns but the same body (\hs{\vlocal{y}}):

\begin{equation}\label{eq:xy}
  \begin{array}{r@{}l@{}l@{}}
    X\ &=\ \CAlt\ (\CDataAlt\ \id{C})\ & (\vlocal{y}) \\
    Y\ &=\ \CAlt\ \CDefault\           & (\vlocal{y})
  \end{array}
\end{equation}

If we traverse these expressions in breadth-first order, converting each token to a feature using $\phi$ and padding to the same length with $0$, we would get the following feature vectors:

\begin{small}
  \begin{equation}
    \begin{array}{r@{}l@{}l@{}l@{}l@{}l@{}l@{}l}
      breadthFirst(X)\ &=\ (\feature{\CAlt},\ &\feature{\CDataAlt},\ &\feature{\CVar},\ &\feature{\id{C}},\ &\feature{\CLocal},\ &\feature{\id{y}} &) \\
      breadthFirst(Y)\ &=\ (\feature{\CAlt},\ &\feature{\CDefault},\ &\feature{\CVar},\ &\feature{\CLocal},\ &\feature{\id{y}},\ &0 &)
    \end{array}
  \end{equation}
\end{small}

Here the features corresponding to the common sub-expression $\CLocal\ \id{y}$ are misaligned, such that only $\frac{1}{3}$ of features are guaranteed to match (others may match by coincidence, depending on $\phi$). These feature vectors might be deemed very dissimilar during clustering, despite the intuitive similarity of the expressions $X$ and $Y$ from which they derive.

If we were to align these features optimally, by padding the fourth column rather than the sixth, then $\frac{2}{3}$ of features would be guaranteed to match, making the similarity of the vectors more closely match our intuition and depend less on coincidence.

The method we use to ``flatten'' expressions, described below, is a variation of breadth-first traversal which pads each level of nesting to a fixed size $c$ (for \emph{columns}). This doesn't guarantee alignment, but it does prevent mis-alignment from accumulating across different levels of nesting. Our method would align these features into the following vectors, if $c = 2$: \footnote{In fact, the $toTree$ function would ignore the constructor identifier $\id{C}$ and never produce the feature $\feature{\id{C}}$. However, this example is still accurate in terms of laying out the features as given.}

\begin{small}
  \begin{equation}\label{eq:flattened}
    \begin{array}{r@{}l@{}l@{}l@{}l@{}l@{}l@{}l@{}l@{}l}
      featureVec(X)\ &=\ (\feature{\CAlt},\ &0,\ &\feature{\CDataAlt},\ &\feature{\CVar},\ &\feature{\id{C}},\  &\feature{\CLocal},\ &\feature{\id{y}},\ &0 &) \\
      featureVec(Y)\ &=\ (\feature{\CAlt},\ &0,\ &\feature{\CDefault},\ &\feature{\CVar},\ &\feature{\CLocal},\ &0,\                 &\feature{\id{y}},\ &0 &)
    \end{array}
  \end{equation}
\end{small}

Here $\frac{1}{2}$ of the original 6 features align, which is more than $breadthFirst$ but not optimal. Both vectors have also been padded by an extra 2 zeros compared to $breadthFirst$; raising their alignment to $\frac{5}{8}$.

To perform this flattening we first transform the nested tokens of an expression into a \emph{rose tree} of features, using the function $toTree$ given in Figure \ref{fig:totree}. Intuitively, these rose trees are simply s-expression representations of the ASTs, illustrated for the \hs{odd} function in Figure \ref{fig:rosetreeexample}.

\begin{figure}
  \centering
  \begin{scriptsize}
      \Tree[ .$\feature{\CLam}$
                $\feature{\id{a}}$
                [ .$\feature{\CCase}$
                     [ .$\feature{\CVar}$
                          [ .$\feature{\CLocal}$
                               $\feature{\id{a}}$ ]]
                     $\feature{\id{b}}$
                     [ .$\feature{\CAlt}$
                          $\feature{\CDataAlt}$
                          [ .$\feature{\CVar}$
                               $\feature{\CConstructor}$ ]]
                     [ .$\feature{\CAlt}$
                          $\feature{\CDataAlt}$
                          [ .$\feature{\CApp}$
                               [ .$\feature{\CVar}$
                                    [ .$\feature{\CGlobal}$
                                         $\feature{\id{even}}$ ]]
                               [ .$\feature{\CVar}$
                                    [ .$\feature{\CLocal}$
                                         $\feature{\id{n}}$ ]]]
                          $\feature{\id{n}}$ ]]]
  \end{scriptsize}
  \caption[Rose tree for odd]{\label{fig:rosetreeexample} Rose tree for the expression \hs{odd} from Figure \ref{fig:coreexample}. Each (sub-) rose tree is rendered with its feature at the node and sub-trees beneath.}
\end{figure}

\begin{figure}
    \begin{equation*}
      \begin{bmatrix}
        \feature{\CLam}      & 0                       & 0                 & 0                   & 0               & 0                \\
        \feature{\id{a}}     & \feature{\CCase}        & 0                 & 0                   & 0               & 0                \\
        \feature{\CVar}      & \feature{\id{b}}        & \feature{\CAlt}   & \feature{\CAlt}     & 0               & 0                \\
        \feature{\CLocal}    & \feature{\CDataAlt}     & \feature{\CVar}   & \feature{\CDataAlt} & \feature{\CApp} & \feature{\id{n}} \\
        \feature{\id{a}}     & \feature{\CConstructor} & \feature{\CVar}   & \feature{\CVar}     & 0               & 0                \\
        \feature{\CGlobal}   & \feature{\CLocal}       & 0                 & 0                   & 0               & 0                \\
        \feature{\id{even}}  & \feature{\id{n}}        & 0                 & 0                   & 0               & 0
      \end{bmatrix}
    \end{equation*}
    \caption{Matrix generated from Figure \ref{fig:rosetreeexample}, padded to 6 columns. Each level of nesting in the tree corresponds to a row in the matrix.}
    \label{fig:matrixexample}
\end{figure}

These rose trees are then turned into matrices, as shown in Figure \ref{fig:matrixexample}. Each row $i$ of the matrix contains the features at depth $i$ in the rose tree, read left-to-right, followed by any required padding. These matrices are then either truncated, or padded with rows (on the bottom) or columns (on the right) of zeros, to fit a fixed size $r \times c$.

\begin{sloppypar}
Finally, matrices are turned into vectors by simply concatenating the rows from top to bottom, hence the matrix from Figure \ref{fig:matrixexample} will begin $(\feature{\CLam}, 0, 0, 0, 0, 0, \feature{\id{a}}, \feature{\CCase}, 0, 0, 0, 0, \feature{\CVar}, \feature{\id{b}}, \feature{\CAlt}, \feature{\CAlt}, \dots$.
\end{sloppypar}

\subsubsection{Symbols to Features}
\label{sec:symbolstofeatures}

We now define the function $\phi$, which turns terminal symbols of Core syntax into features (real numbers). For known language features, such as $\feature{\CLam}$ and $\feature{\CCase}$, we can enumerate the possibilities and assign a value to each, in a similar way to \cite{DBLP:journals/corr/HerasK14} in Coq. We use a constant $\alpha$ to separate these values from those of other tokens (e.g. identifiers), but the order is essentially arbitrary: \footnote{In \cite{DBLP:journals/corr/HerasK14}, ``similar'' Gallina tokens like \coq{fix} and \coq{cofix} are grouped together to reduce redundancy; we do not group tokens, but we do put ``similar'' tokens close together, such as \CLocal\ and \CGlobal.}

\begin{equation} \label{eq:feature}
  \begin{aligned}
    \feature{\CAlt}          &= \alpha      &
    \feature{\CDataAlt}      &= \alpha + 1  &
    \feature{\CLitAlt}       &= \alpha + 2  \\
    \feature{\CDefault}      &= \alpha + 3  &
    \feature{\CNonRec}       &= \alpha + 4  &
    \feature{\CRec}          &= \alpha + 5  \\
    \feature{\CBind}         &= \alpha + 6  &
    \feature{\CLet}          &= \alpha + 7  &
    \feature{\CCase}         &= \alpha + 8  \\
    \feature{\CLocal}        &= \alpha + 9  &
    \feature{\CGlobal}       &= \alpha + 10 &
    \feature{\CConstructor}  &= \alpha + 11 \\
    \feature{\CVar}          &= \alpha + 12 &
    \feature{\CLam}          &= \alpha + 13 &
    \feature{\CApp}          &= \alpha + 14 \\
    \feature{\CType}         &= \alpha + 15 &
    \feature{\CLit}          &= \alpha + 16 &
    \feature{\CLitNum}       &= \alpha + 17 \\
    \feature{\CLitStr}       &= \alpha + 18
  \end{aligned}
\end{equation}

To encode \emph{local} identifiers $\mathcal{L}$ we would like a quantity which gives equal values for $\alpha$-equivalent expressions (i.e. renaming an identifier shouldn't affect the feature). To do this, we use the \emph{de Bruijn index} of the identifier \cite{de1972lambda}, denoted $i_l$:

\begin{equation} \label{eq:localfeature}
  \feature{l} = i_l + 2 \alpha \quad \text{if $l \in \mathcal{L}$}
\end{equation}

We again use $\alpha$ to separate these features from those of other constructs.

Since numerals, strings and constructor identifiers are discarded during conversion to rose trees, the only remaining case is global identifiers $\mathcal{G}$. Since these are declared \emph{outside} the body of an expression, we cannot perform the same indexing trick as we did for local identifiers. We also cannot directly encode the form of the identifiers, e.g. using a scheme like G{\"o}del numbering, since this is essentially arbitrary and has no effect on their semantic meaning (referencing other expressions).

Instead, we use the approach taken in the latest versions of ML4PG and encode global identifiers \emph{indirectly}, by looking up the expressions which they \emph{reference}:

\begin{equation} \label{eq:globalfeature}
  \feature{g \in \mathcal{G}} =
    \begin{cases}
      i + 3 \alpha \quad & \text{if $g \in C_i$} \\
      f_{recursion}         & \text{otherwise}
    \end{cases}
\end{equation}

Where $C_i$ are our clusters (in some arbitrary order). This is where the recurrent nature of the algorithm appears: to determine the contents of $C_i$, we must perform k-means clustering \emph{during} feature extraction; yet that clustering step, in turn, requires that we perform feature extraction.

For this recursive process to be well-founded, we perform a topological sort on declarations based on their dependencies (the expressions they reference). In this way, we can avoid looking up expressions which haven't been clustered yet. To handle mutual recursion we use Tarjan's algorithm \cite{tarjan1972depth} to produce a sorted list of \emph{strongly connected components} (SCCs), where each SCC is a mutually-recursive sub-set of the declarations. If an identifier cannot be found amongst those clustered so far, it must appear in the same SCC as the expression we are processing; hence we give that identifier the constant feature value $f_{recursion}$.

By working through the sorted list of SCCs, storing the features of each top-level expression as they are calculated, our algorithm can be computed \emph{iteratively} rather than recursively, as shown in Algorithm \ref{alg:recurrent}.

\begin{algorithm}
  \begin{algorithmic}[1]
    \Require List $d$ contains SCCs of (identifier, expression) pairs, in dependency order.
    \Procedure{RecurrentCluster}{}
      \State $\vect{C}  \gets []$
      \State $DB \gets \varnothing$
      \ForAll{$scc$ \textbf{in} $d$}
        \State $DB \gets DB \cup \{(i, featureVec(e)) \mid (i, e) \in scc\}$
        \State $\vect{C}  \gets kMeans(DB)$
      \EndFor
      \Return $\vect{C}$
    \EndProcedure
  \end{algorithmic}
  \caption{Recurrent clustering of Core expressions.}
  \label{alg:recurrent}
\end{algorithm}

\iffalse
As an example of this recurrent process, we can consider the Peano arithmetic functions from Figure \ref{fig:coreexample}. A valid topological ordering is given in Figure \ref{fig:sccexample}, which can be our value for $d$ (eliding Core expressions to save space):

\begin{equation}
  d = [\{(\hs{plus}, \dots)\}, \{(\hs{odd}, \dots), (\hs{even}, \dots)\}, \{(\hs{mult}, \dots)\}]
\end{equation}

We can then trace the execution of Algorithm \ref{alg:recurrent} as follows:

\begin{itemize}
  \item The first iteration through \textsc{RecurrentCluster}'s loop will set $scc \gets \{(\hs{plus}, \dots)\}$.
  \item With $i = \hs{plus}$ and $e$ as its Core expression, calculating $featureVec(e)$ is straightforward; the recursive call $\feature{\hs{plus}}$ will become $f_{recursion}$ (since \hs{plus} doesn't appear in $\vect{C}$).
  \item The call to $kMeans$ will produce $\vect{C} \gets [\{\hs{plus}\}]$, i.e. a single cluster containing \hs{plus}.
  \item The next iteration will set $scc \gets \{(\hs{odd}, \dots), (\hs{even}, \dots)\}$.
  \item With $i = \hs{odd}$ and $e$ as its Core expression, the call to \hs{even} will result in $f_{recursion}$.
  \item Likewise for the call to $\hs{even}$ when $i = \hs{odd}$.
  \item Since the feature vectors for \hs{odd} and \hs{even} will be identical, $kMeans$ will put them in the same cluster. To avoid the degenerate case of a single cluster, for this example we will assume that $k = 2$; in which case the other cluster must contain \hs{plus}. Their order is arbitrary, so one possibility is $\vect{C} = [\{\hs{odd}, \hs{even}\}, \{\hs{plus}\}]$.
  \item Finally \hs{mult} will be clustered. The recursive call will become $f_{recursion}$ whilst the call to \hs{plus} will become $2 + 3 \alpha$, since $\hs{plus} \in C_2$.
  \item \begin{sloppypar}Again assuming that $k = 2$, the resulting clusters will be \mbox{$\vect{C} \gets [\{\hs{odd}, \hs{even}\}, \{\hs{plus}, \hs{mult}\}]$}.\end{sloppypar}
\end{itemize}

Even in this very simple example we can see a few features of our algorithm emerge. For example, \hs{odd} and \hs{even} will always appear in the same cluster, since they only differ in their choice of constructor names (which are discarded by $toTree$) and recursive calls (which are replaced by $f_{recursion}$). A more extensive investigation of these features requires a concrete implementation, in particular to pin down values for the parameters such as $r$, $c$, $f_{recursion}$, $\alpha$ and so on.
\fi

\subsubsection{Comparison}

Our algorithm is most similar to that of ML4PG, as our transformation maps the elements of a syntax tree to distinct cells in a matrix. In contrast, the matrices produced by ACL2(ml) \emph{summarise} the tree elements: providing, for each level of the tree, the number of variables, nullary symbols, unary symbols, etc.

There are two major differences between our algorithm and that of ML4PG: mutual-recursion and types.

The special handling required for mutual recursion is discussed above (namely, topological sorting of expressions and the $f_{recursion}$ feature). Such handling is not present in ML4PG, since the Coq code it analyses must, by virtue of the language, be written in dependency order to begin with. Coq \emph{does} have limited support for mutually-recursive functions, of the following form:

\begin{verbatim}
Fixpoint even n := match n with
                       | O   => true
                       | S m => odd m
                   end
    with odd  n := match n with
                       | O   => false
                       | S m => even m
                   end.
\end{verbatim}

However, this is relatively uncommon and unsupported by ML4PG.

The more interesting differences come from our handling (or lack thereof) for types. Coq and ACL2 are at opposite ends of the typing spectrum, with the former treating types as first class entities of the language whilst the latter is untyped (or \emph{unityped}). In both cases, we have a \emph{single} language to analyse, by ML4PG and ACL2(ml) respectively.\footnote{ML4PG can also analyse Coq's \textsc{Ltac} meta-language. Haskell has its own meta-language, Template Haskell, but here we only consider the regular Haskell which it generates.}

The situation is different for Haskell, where the type level is distinct from the value level, and there are strict rules for how they can influence each other. In particular, Haskell values can depend on types (via the type class mechanism) but types cannot depend on values.

In our initial approach, we restrict ourselves to the value level. This has several consequences:

\begin{itemize}
  \item Although they are values, we cannot distinguish between data constructors, other than using exact equality. Hence they are discarded by $toTree$.
  \item Since Core uses a single \texttt{Lam} abstraction for both value- and type-level parameters, we cannot always distinguish between them. This can cause a function's Core arity to be greater than its Haskell arity.
\end{itemize}

There is certainly promise in including types in our analysis, by pairing every term with its type as in ML4PG. This will allow fine-grained distinction of expressions which are otherwise identical, especially data constructors. Integrating types into our algorithm, and extracting them from Core expressions, is hence left as future work.

\section{Implementation}
\label{sec:implementation}

We provide an implementation of our recurrent clustering algorithm in a tool called \textsc{ML4HS}. We obtain Core ASTs from Haskell packages using a plugin for the GHC compiler, which emits a serialised version of each Core definition as it is being compiled. This approach is more robust than, for example, parsing source files, since it avoids the complications of preprocessors and language extensions altering the syntax.

A post-processing stage determines which Core definitions can be used by \qspec{}, based on their type, visibility (whether they are encapsulated inside their module or visible to importers), etc. For each definition, we simply use the GHCi interpreter to check if calling \qspec{} with that argument is well-typed. This information, along with type signatures, arity, etc. are stored alongside the Core definitions in JSON format.

The definitions are then sorted topologically, based on the non-local identifiers appearing in their ASTs, and feature vectors are constructed using a Haskell implementation of the approach described in \S \ref{sec:recurrentclustering}. Since the features associated with each identifier may vary between iterations (as the clusters change), we leave the raw identifiers in the vector so their features can be extracted in the correct context.

We implement Algorithm \ref{alg:recurrent} as a simple shell script, which invokes Weka for clustering and associates each Core definition with a cluster number. These numbers are used to finish the deferred feature extraction of identifiers, the resulting feature vectors are clustered, and the process is repeated until all SCCs have been processed.

Once the recurrent clustering is complete, we generate a string of Haskell code for each of the final clusters, which will invoke \qspec{} with a suitable signature. This involves:

\begin{itemize}
  \item{Monomorphising}: If a value has polymorphic type, e.g. \hs{equal :: forall t. t -> t -> Bool}, we must choose a concrete representation to use (in this case for \hs{t}), in order to know which random generator to use. We take the approach used in \qcheck{} by attempting to instantiate all type variables to \hs{Integer}. Any values where this is invalid, such as those with incompatible class constraints (e.g. \hs{forall t. IsString t => t}, where \hs{Integer} does not implement \hs{IsString}) will not be included in the signature (this is checked at the AST post-processing stage).

  \item{Qualifying}: All names are \emph{qualified} (prefixed by their module's name), to avoid most ambiguity. There is still the possibility that multiple packages will declare modules of the same name, although this is rare as it causes problems for any Haskell programmer trying to use those modules. In such cases the exploration process simply aborts.

  \item{Appending variables}: Once a \qspec{} theory has been defined containing all of the given terms, we inspect the types it references and append three variables for each to the theory (enough to discover laws such as associativity, but not too many to overflow the limit of \qspec{}'s exhaustive search).

  \item{Sandboxing}: One difficulty with Haskell's packaging infrastructure is that all required packages and modules must be provided up-front, usually by specification in a Cabal file. Since \textsc{MLSpec} builds signatures \emph{dynamically}, depending on the cluster information it is given, we do not know what packages it may need. To work around this problem, we evaluate these strings of Haskell using a custom library called \texttt{nix-eval}. This uses the Nix package manager to obtain all of the required packages and make them available to GHC.

\end{itemize}

The equations resulting from evaluating these strings are collected and outputted as JSON values, to ease further processing (e.g. displaying in some form, sending to a theorem prover, etc.).

\section{Related Work}
\label{sec:related}

\subsection{Haskell}
\label{sec:haskelldesc}

Whilst \S \ref{sec:haskell} gave some brief background on Haskell, little explanation was given for why we chose this language rather than, for example Coq or ACL2 (for which recurrent clustering algorithms already exist), or a more widely used language like Java. Here we discuss the relevant language features from a high-level, which motivated our choice:

\begin{description}

\item{Functional}: All control flow in Haskell is performed by function abstraction and application, which we can reason about using standard rules of inference such as \emph{modus ponens}.

\item{Pure}: Execution of actions (e.g. reading files) is separate to evaluation of expressions; hence our reasoning can safely ignore complicated external and non-local interactions.

\item{Statically Typed}: Expression are constrained by \emph{types}, which can be used to eliminate unwanted combinations of values, and hence reduce search spaces; \emph{static} types can be deduced syntactically, without having to execute the code.

\item{Non-strict}: If an evaluation strategy exists for $\beta$-normalising an expression (i.e. performing function calls) without diverging, then a non-strict evaluation strategy will not diverge when evaluating that expression. This is rather technical, but in simple terms it allows us to reason effectively about a Turing-complete language, where evaluation may not terminate. For example, when reasoning about \emph{pairs} of values \hs{(x, y)} and projection functions \hs{fst} and \hs{snd}, we might want to use an ``obvious'' rule such as $\forall \text{\hs{x y}}, \text{\hs{x}} = \text{\hs{fst (x, y)}}$. Haskell's non-strict semantics makes this equation valid; whilst it would \emph{not} be valid in the strict setting common to most other languages, where the expression \hs{fst (x, y)} will diverge if \hs{y} diverges (and hence alter the semantics, if \hs{x} doesn't diverge).

\item{Algebraic Data Types}: These provide a rich grammar for building up user-defined data representations, and an inverse mechanism to inspect these data by \emph{pattern-matching}. For our purposes, the useful consequences of ADTs and pattern-matching include their amenability for inductive proofs and the fact they are \emph{closed}; i.e. an ADT's declaration specifies all of the normal forms for that type. This makes exhaustive case analysis trivial, which would be impossible for \emph{open} types (for example, consider classes in an object oriented language, where new subclasses can be introduced at any time).

\item{Parametricity}: This allows Haskell \emph{values} to be parameterised over \emph{type-level} objects; provided those objects are never inspected. This has the \emph{practical} benefit of enabling \emph{polymorphism}: for example, we can write a polymorphic identity function \hs{id :: forall t. t -> t}. \footnote{Read ``\hs{a :: b}'' as ``\hs{a} has type \hs{b}'' and ``\hs{a -> b}'' as ``the type of functions from \hs{a} to \hs{b}''.} Conceptually, this function takes \emph{two} parameters: a type \hs{t} \emph{and} a value of type \hs{t}; yet only the latter is available in the function body, e.g. \hs{id x = x}. This inability to inspect type-level arguments gives us the \emph{theoretical} benefit of being able to characterise the behaviour of polymorphic functions from their type alone, a technique known as \emph{theorems for free} \citep{wadler1989theorems}.

\item{Type classes}: Along with their various extensions, type classes are interfaces which specify a set of operations over a type (or other type-level object, such as a \emph{type constructor}). Many type classes also specify a set of \emph{laws} which their operations should obey but, lacking a simple mechanism to enforce this, laws are usually considered as documentation. As a simple example, we can define a type class \hs{Semigroup} with the following operation and associativity law:

\begin{lstlisting}[language=Haskell, xleftmargin=.2\textwidth, xrightmargin=.2\textwidth]
op :: forall t. Semigroup t => t -> t -> t
\end{lstlisting}

$$\forall \text{\hs{x y z}}, \text{\hs{op x (op y z)}} = \text{\hs{op (op x y) z}}$$

The notation \hs{Semigroup t =>} is a \emph{type class constraint}, which restricts the possible types \hs{t} to only those which implement \hs{Semigroup}. \footnote{Alternatively, we can consider \hs{Semigroup t} as the type of ``implementations of \hs{Semigroup} for \hs{t}'', in which case \hs{=>} has a similar role to \hs{->} and we can consider \hs{op} to take \emph{four} parameters: a type \hs{t}, an implementation of \hs{Semigroup t} and two values of type \hs{t}. As with parameteric polymorphism, this extra \hs{Semigroup t} parameter is not available at the value level. Even if it were, we could not alter our behaviour by inspecting it, since Haskell only allows types to implement each type class in at most one way, so there would be no information to branch on.} There are many \emph{instances} of \hs{Semigroup} (types which may be substituted for \hs{t}), e.g. \hs{Integer} with \hs{op} performing addition. Many more examples can be found in the \emph{typeclassopedia} \citep{yorgey2009typeclassopedia}. This ability to constrain types, and the existence of laws, helps us reason about code generically, rather than repeating the same arguments for each particular pair of \hs{t} and \hs{op}.

\item{Equational}: Haskell uses equations at the value level, for definitions; at the type level, for coercions; at the documentation level, for typeclass laws; and at the compiler level, for ad-hoc rewrite rules. This provides us with many \emph{sources} of equations, as well as many possible \emph{uses} for any equations we might discover. Along with their support in existing tools such as SMT solvers, this makes equational conjectures a natural target for theory exploration.

\item{Modularity}: Haskell has a module system, where each module may specify an \emph{export list} containing the names which should be made available for other modules to import. When such a list is given, any expressions \emph{not} on the list are considered \emph{private} to that module, and are hence inaccessible from elsewhere. This mechanism allows modules to provide more guarantees than are available just in their types. For example, a module may represent email addresses in the following way:

\begin{lstlisting}[language=Haskell, xleftmargin=.2\textwidth, xrightmargin=.2\textwidth, upquote=true]
module Email (Email(), at, render) where

data Email = E String String

render :: Email -> String
render (E u h) = u ++ "@" ++ h

at :: String -> String -> Maybe Email
at "" h  = Nothing
at u  "" = Nothing
at u  h  = Just (E u h)
\end{lstlisting}

The \hs{Email} type guarantees that its elements have both a user part and a host part (modulo divergence), but it does not provide any guarantees about those parts. We also define the \hs{at} function, a so-called ``smart constructor'', which has the additional guarantee that the \hs{Email}s it returns contain non-empty \hs{String}s. By ommitting the \hs{E} constructor from the export list on the first line \footnote{The syntax \hs{Email()} means we're exporting the \hs{Email} type, but not any of its constructors.}, the only way \emph{other} modules can create an \hs{Email} is by using \hs{at}, which forces the non-empty guarantee to hold globally.

\end{description}

Together, these features make Haskell code highly structured, amenable to logical analysis and subject to many algebraic laws. However, as mentioned with regards to type classes, Haskell itself is incapable of expressing or enforcing these laws (at least, without difficulty \citep{lindley2014hasochism}). This reduces the incentive to manually discover, state and prove theorems about Haskell code, e.g. in the style of interactive theorem proving, as these results may be invalidated by seemingly innocuous code changes. This puts Haskell in a rather special position with regards to the discovery of interesting theorems; namely that many discoveries may be available with very little work, simply because the code's authors are focused on \emph{software} development rather than \emph{proof} development. The same cannot be said, for example, of ITP systems; although our reasoning capabilities may be stronger in an ITP setting, much of the ``low hanging fruit'' will have already been found through the user's dedicated efforts, and hence theory exploration would be unlikely to discover unexpected properties.

Other empirical advantages to studying Haskell, compared to other programming languages or theorem proving systems, include:

\begin{itemize}

\item The large amount of Haskell code which is freely available online, e.g. in repositories like \href{http://hackage.haskell.org}{Hackage}, with which we can experiment.

\item The existence of theory exploration systems such as \hspec{}, and related tools which we may be able to re-use, including conjecture generators like \qspec{}; counterexample finders like \qcheck{}, \textsc{SmallCheck} and \textsc{SmartCheck}; theorem provers like \textsc{Hip} \citep{rosen2012proving}; and other related testing and term-generating systems like \textsc{MuCheck} \citep{le2014mucheck}, \textsc{MagicHaskeller} \citep{katayama2011magichaskeller} and \textsc{Djinn} \citep{augustsson2005djinn}.

\item The remarkable amount of infrastructure which exists for working with Haskell code, including package managers, compilers, interpreters, parsers, static analysers, etc.

\end{itemize}

\subsection{Theory Exploration}

We briefly described theory exploration in \S \ref{sec:theoryexploration}, as the task of discovering \emph{new} theorems in a software or proof library, rather than proving/disproving user-provided statements. The idea was first introduced in the \textsc{Theorema} \cite{buchberger2000theory} system of Buchberger. This provided an interactive environment, similar to computer algebra systems and interactive theorem provers. In this setting, many of our concerns such as the generation of values and deciding which properties to explore are simply delegated to the user; the software would check for correctness, store results and perform searches; again, similar to interactive theorem provers.

Subsequent systems have investigated \emph{automated} theory exploration, for tasks such as lemma discovery. By removing user interaction, these concerns about directing search must be solved by algorithms. As well as \qspec{} and \hspec{} in Haskell, automated theory exploration has been applied to Isabelle \citep{Montano-Rivas.McCasland.Dixon.ea:2012, johansson2009isacosy, Hipster}.

We have focused our attention on \qspec{}, although it does not actually \emph{prove} its results, and hence may not be considered a theory exploration system on its own. However, it does form a vital component of \hspec{}, which uses off-the-shelf automated theorem provers (ATPs) to verify \qspec{}'s conjectures, forming a complete theory exploration system as well as a capable inductive theorem prover (by exploiting theory exploration for lemma generation) \cite{claessen2013automating}. Due to \hspec{}'s use in \textsc{Hipster}, improvements to \qspec{} also benefit work being pursued in Isabelle.

\subsection{Relevance Filtering}
\label{sec:relevance}

The combinatorial nature of formal systems causes many proof search methods, such as resolution, to have exponential complexity \citep{haken1985intractability}; hence even a modest size increase can turn a trivial problem into an intractable one. Finding efficient alternatives for such algorithms, especially those which are NP-complete (e.g. determining satisfiability) or co-NP-complete (e.g. determining tautologies), seems unlikely, as it would imply progress on the famously intractable open problems of $\text{P} = \text{NP}$ and $\text{NP} = \text{co-NP}$. On the other hand, we can turn this difficulty around: a modest \emph{decrease} in size may turn an intractable problem into a solvable one. We can ensure that the solutions to these reduced problems coincide with the original if we only remove \emph{redundant} information. This leads to the idea of \emph{relevance filtering} (or, \emph{premise selection}, when viewed as the \emph{addition} of relevant information to an initially-empty problem). This is the core idea behind our restriction of theory exploration to intelligently-selected clusters of symbols, rather than whole libraries at a time.

Relevance filtering has mostly been used in automated proof search, where it simplifies problems by removing from consideration those clauses (axioms, definitions, lemmas, etc.) which are deemed \emph{irrelevant}. The technique is used in Isabelle's Sledgehammer tool, during its translation of Isabelle/HOL theories to statements in first order logic: rather than translating the entire theory, only a sub-set of relevant clauses are included. This reduces the size of the problem and speeds up the proof search, but it creates the new problem of determining when a clause is relevant: how do we know what will be required, before we have the proof?

The initial approach taken by Sledgehammer, known as \textsc{MePo} (from \emph{Meng-Paulson} \citep{meng2009lightweight}), gives each clause a score based on the proportion $\frac{m}{n}$ of its symbols which are ``relevant'' (where $n$ is the number of symbols in the clause and $m$ is the number which are relevant). Initially, the relevant symbols are those which occur in the goal to be proved, but whenever a clause is found which scores more than a particular threshold, all of its symbols are then also considered relevant. There are other heuristics applied too, such as increasing the score of user-provided facts (e.g. given by keywords like \texttt{using}), locally-scoped facts, first-order facts and rarely-occuring facts. To choose $r$ relevant clauses for an ATP invocation, we simply order the clauses by decreasing score and take the first $r$ of them.

Recently, a variety of alternative algorithms have also been investigated, for example the \textsc{MaSH} algorithm (Machine Learning for SledgeHammer) \citep{kuhlwein2013mash} uses the ``visibility'' of one theorem from another to determine the relevance of clauses. Visibility is essentially a dependency graph of which theorems were used in the proofs of which other theorems (although the theorems are actually represented as abstract sets of features). To select relevant clauses for a goal, the set of clauses which are visible from the goal's components is generated; this is further reduced by (an efficient approximation of) a na\"{\i}ve Bayes algorithm.

Another example is \emph{multi-output ranking} (MOR), which uses a support vector machine (SVM) approach for selecting relevant axioms from the Mizar Mathematical Library for use by the Vampire ATP system \citep{alama2014premise}. Many more approaches are described and evaluated in \citep{kuhlwein2012overview}, some of which may be directly applicable in the context of theory exploration.

\subsection{Recurrent Clustering}
\label{sec:clusteringexpressions}

Our recurrent clustering approach takes inspiration from the ML4PG \citep{journals/corr/abs-1212-3618} and ACL2(ml) \citep{heras2013proof} tools, used for analysing proofs in Coq and ACL2, respectively. Whilst both transform syntax trees into matrices, the algorithm of ML4PG most closely resembles ours as it assigns tokens directly to matrix elements. In contrast, the matrices produced by ACL2(ml) \emph{summarise} information about the tree; for example, one column counts the number of variables appearing at each tree level, others count the number of function symbols which are nullary, unary, binary, etc. Whilst it may be interesting to contrast our current algorithm with an alternative based on that of ACL2(ml), it is unclear how such summaries could be extended to include types, which seems the next logical step for our approach. The ML4PG algorithm extends trivially, by using (term, type) pairs instead of just terms.

The way we \emph{use} our clusters to inform theory exploration is actually more similar to that of ACL2(ml) than ML4PG. ML4PG can either present clusters to the user for inspection, or produce automata for recreating proofs. In ACL2(ml), the clusters are used to restrict the search space of a proof search, much like we restrict the scope of theory exploration.

ACL2(ml) reasons by analogy: finding theorem statements which are similar to the current goal, and attempting to prove the goal in a similar way. In particular, the lemmas used to prove a theorem are mutated by substituting symbols for those which appear in the same cluster. For example, if \texttt{plus} and \texttt{multiply} are clustered together, and we are trying to prove a goal involving \texttt{multiply}, then ACL2(ml) might consider an existing theorem involving \texttt{plus}. The lemmas used to prove that theorem will be mutated, for example replacing occurrences of \texttt{plus} with \texttt{mult}, in an attempt to prove the goal.

Whilst we do not currently reason by analogy, this is an interesting area for future work in theory exploration: given a set of theorems relating particular terms, we might form conjectures regarding similar terms found through clustering.

\iffalse
We could expand this a bit, e.g. talking about how we both use Weka, etc.
\fi

\subsection{Feature Extraction}

One major difficulty when applying statistical machine learning algorithms to \emph{languages}, such as Haskell, is the appearance of recursive structures. This can lead to nested expressions of arbitrary depth, which are difficult to compare in numerical ways. One solution, as described in \S \ref{sec:featureextraction}, is to use \emph{feature extraction}; however, our method is not the only possible way to encode recursive structures as fixed-size features.

The simplest way to encode such inputs is to simply choose a desired size, then pad anything smaller and truncate anything larger. We use this to make our matrices a uniform size, borrowing the idea from ML4PG. Care must be taken to ensure that we are not discarding too much information, that we are not producing features with too many dimensions to be practical, and that there is a uniform ``meaning'' to each feature across different feature vectors. In our case, we avoid many of these problems by transforming the recursive structure of expressions into matrices first; this gives each feature a stable meaning such as ``the $i$th token from the left at the $j$th level of nesting''.

Truncation works best when the input data is arranged with the most significant data first (in a sense, it is ``big-endian''). This is the case for Haskell expressions, since the higher levels of the syntax tree are the most semantically significant; for example, the lower levels may never even be evaluated due to laziness. This allows us to truncate more aggressively than if the leaves were most significant.

By modelling our inputs as points in high-dimensional spaces, we can consider feature extraction as projection into a lower-dimensional space (known as \emph{dimension reduction}). Truncation is a trivial dimension reduction technique; more sophisticated projection functions consider the \emph{distribution} of the input points, and project with the hyperplane which preserves as much of the variance as possible (or, equivalently, reduces the \emph{mutual information} between the points).

Techniques such as \emph{principle component analysis} (PCA) can be used to find these hyperplanes, but unfortunately require their inputs to already have a fixed, integer number of dimensions. In the case of our recursive expressions (which we may consider to have \emph{fractal} dimension), we would need another pre-processing stage to satisfy this requirement.

There are machine learning algorithms which can handle variable-size input, but these are often \emph{supervised} algorithms which require an externally-provided error function to minimise. Error functions can be given for clustering, for example k-means implicitly minimises the function given in equation \ref{eq:kmeansobjective}, but unsupervised algorithms may be preferred for efficiency as they are more direct.

One example of learning from variable-size input is to use \emph{recurrent neural networks} (RNNs). These contain cyclic connections between neurons, unlike the traditional acyclic ``feed-forward'' NNs, allowing state to persist between observations. In this way, each data point can be divided into a sequence of arbitrary length, for example an s-expression, and fed into an RNN one token at a time for processing.

Unfortunately RNNs are difficult to train. The standard way to train NNs is the back-propagation algorithm; when this is extended to handle cycles we get the \emph{backpropagation through time} algorithm \citep{werbos1990backpropagation}. However, this suffers a problem known as the \emph{vanishing} (or \emph{exploding}) \emph{gradient}: error values change exponentially as they propagate back through the cycles, which prevents effective learning of correlations across a sequence, undermining the main advantage of RNNs. The vanishing gradient problem is the subject of current research, with countermeasures including \emph{neuroevolution} (using evolutionary computation techniques rather than back-propagation) and \emph{long short-term memory} (LSTM; introducing special nodes to ``store'' state, rather than having them loop around a cycle \citep{hochreiter1997long}).

Using sequences to represent recursive structures is also problematic: if we want our learning algorithm to exploit structure (such as the depth of a token), it will have to discover how to parse the sequences for itself, which seems wasteful. The \emph{back-propagation through structure} approach \citep{goller1996learning} is a more direct solution to this problem, using a feed-forward NN to learn recursive distributed representations \citep{pollack1990recursive} which correspond to the recursive structure of the inputs. Such distributed representations can also be used for sequences, which we can use to encode sub-trees when the branching factor of nodes is not  uniform \citep{kwasny1995tail}. More recent work has investigated storing recursive structures inside LSTM cells \citep{zhu2015long}.

A simpler alternative for generating recursive distributed representations is to use circular convolution \citep{conf/ijcai/Plate91}. Although promising results are shown for its use in \emph{distributed tree kernels} \citep{zanzotto2012distributed}, our preliminary experiments in applying circular convolution to functional programming expressions found most of the information to be lost in the process; presumably as the expressions are too small.

\emph{Kernel methods} have also been applied to structured information, for example in \citep{Gartner2003} the input data (including sequences, trees and graphs) are represented using \emph{generative models}, such as hidden Markov models, of a fixed size suitable for learning. Many more applications of kernel methods to structured domains are given in \citep{bakir2007predicting}, which could be used to learn more subtle relations between expressions than recurrent clustering alone.

\subsection{K-Means}

We use the Weka tool to perform k-means clustering \citep{Holmes.Donkin.Witten:1994}, since we are more concerned with the application of feature extraction to Haskell and its use in theory exploration, rather than precise tuning of learning algorithms. Since k-means is a standard method, there are many other implementations available. More interestingly, there are many other clustering algorithms we could use, such as \emph{expectation maximisation} \footnote{In fact, k-means is very similar to expectation-maximisation, as it alternates between an \emph{expectation step} (finding the mean value of each cluster) and a \emph{maximisation step} (assigning points to the cluster they're most similar to; or alternatively, \emph{minimising} the distance of each point to the centre of its cluster, as per equation \ref{eq:kmeansobjective}).}, but experiments with ML4PG have shown little difference in their results; in effect, the quality of our features is the bottleneck to learning, so there is no reason to avoid a fast algorithm like k-means.

In any case there are many conservative improvements to the standard k-means algorithm, which could be applied to our setup. For example, a more efficient approach like \emph{yinyang k-means} \citep{conf/icml/DingZSMM15} could make larger input sizes more practical to cluster, especially since recurrent clustering invokes k-means many times. The \emph{k-means++} approach \citep{arthur2007k, bahmani2012scalable} can be used to more carefully select the ``seed'' values for the first timestep, and the \emph{x-means} algorithm \citep{pelleg2000x} is able to estimate how many clusters to use (our \emph{final} clusters should be tuned to maximise the performance of the subsequent theory exploration step, but x-means could still be useful in the recurrent clustering steps).

\iffalse
TODO: Theory Exploration?
\citep{johansson2009isacosy}
\fi

\section{Evaluation}\label{sec:evaluation}

In the previous sections, we have shown how ML4PG has been designed to provide \emph{interesting} and \emph{non-trivial} hints on user's demand, 
and to be flexible enough to do so at any stage of the proof, and relative to any chosen proof library. 
However, it is difficult to measure how ML4PG improves the interactive proof building experience, since
the usability of a hint varies from user to user. In this section, we present a quantitative method to evaluate
how useful ML4PG can be during the proof development.


Machine-learning techniques have been previously used for the automatically generation of proofs in 
ITPs, see~\cite{Duncan02,DixonF03,GNR14}. ML4PG was not initially designed with this aim;
but we can use the information obtained from clustering to automatically generate proof attempts for a given theorem. 

\begin{PEM}\label{pem}
Given the statement of a theorem $T$ and a set of lemmas $L=\{L_i\}_i$, we can use ML4PG to find a proof for $T$ as follows:
\begin{enumerate}
 \item Using term-clustering and $T\cup L$ as dataset, obtain the cluster $C$ that contains the theorem $T$ (i.e. $C=T\cup \{L_j\}_j$ where $\{L_j\}_j$ is the set of lemmas similar to $T$).
 \item Obtain the sequence of tactics $\{T_1^j,\ldots,T_{n_i}^j\}_{j}$ used to prove each lemma $L_j$ in $C$.
 \item For each $j$, try to prove $T$ using $T_1^j,\ldots,T_{n_j}^j$.
 \item If no sequence of tactics prove $T$, then for each tactic use ML4PG to infer the argument for each tactic $T_k^j$:
    \begin{itemize}
     \item If the argument of $T_k^j$ is an internal hypothesis from the context of a proof, try all the internal hypothesis from the context of the current proof. 
     \item If the argument of $T_k^j$ is an external lemma $E$, use term-clustering and $L$ as dataset to compute the lemmas in the same cluster as $E$ and try all those lemmas.
     \item[***] This can be naturally extended to tactics with several arguments, just trying all the possible combinations. 
  \end{itemize}    
\end{enumerate}
\end{PEM}

\begin{example}\label{ex:maxnACA}
Let $T$ be the lemma \lstinline?maxnACA? that states the inner commutativity of the maximum of two natural numbers in the SSReflect library:

\begin{lstlisting}
Lemma maxnACA : interchange maxn maxn. 
\end{lstlisting}

\noindent and $L$ be the \lstinline?ssrnat? library of SSReflect. Using Proof Exploration Method~\ref{pem}, we can construct a 
proof of \lstinline?maxnACA? as follows. 

\begin{enumerate}
 \item $T$ belongs to the cluster containing the lemmas $\{$\lstinline?addnACA?, \lstinline?minnACA?, \lstinline?mulnACA?$\}$ --- these
 3 lemmas state the inner commutativity of addition, multiplication and the minimum of two naturals respectively. 
 \item For this example, we will only consider the proof of the lemma \lstinline?addnACA? that is proven using the sequence of tactics \lstinline?by move=> m n p q; rewrite -!addnA (addnCA n).?
 \item The proof of \lstinline?maxnACA? fails using the sequence of tactics from \lstinline?addnACA?.
 \item The proof of \lstinline?addnACA? uses the lemmas \lstinline?addnA? and \lstinline?addnCA?. If we cluster these lemmas with the
 rest of the lemmas of the ssrnat library we find the cluster $\{$\lstinline?minnA?, \lstinline?mulnA?, \lstinline?maxnA?$\}$ for 
 \lstinline?addnA?, and the cluster $\{$\lstinline?minnAC?, \lstinline?mulnAC?, \lstinline?maxnAC?$\}$ for 
 \lstinline?addnAC?. 
 \item Using the lemmas \lstinline?maxnA? and \lstinline?maxnAC?, we construct the sequence of tactics \lstinline?by move=> m n p q; rewrite -!maxnA (maxnCA n).? that proves the lemma \lstinline?maxnACA?.
\end{enumerate}
 
\end{example}


Using 5 Coq theories of varied sizes, we perform an empirical evaluation of our proof exploration method. Our test data consists of the Basic infrastructure of SSReflect library~\cite{SSReflect}, the formalisation about Java-like bytecode presented in~\cite{HK14}, the formalisation of persistent homology~\cite{HCMS12}, the paths library of the HoTT development~\cite{hottbook}, and two formalisations of Nash Equilibrium~\cite{Ves06,nash}. Using Proof exploration method~\ref{pem}, we try to prove every lemma of these libraries with a fully-honest approach (following the terminology from~\cite{KaliszykU14}): clustering is only performed against the lemmas that have previously
proven in the given library. In addition, we study the impact of changing the granularity value.

The results of our experiments can be seen in Table~\ref{tab:reproven}. The success rate of the proof exploration method 
depends on how similar are the proofs of theorems in a given library. This explains the high success rate in the \lstinline?Paths?
library, where most of the lemmas are proven using almost the same sequence of tactics, and the low rate in the Persistent Homology 
library, where just a few lemmas are similarly proven. The granularity value does not have a big impact in the performance of 
our experiments, and almost the same amount of lemmas are proven with the different values. In some cases, like in the Nash equilibrium
library, a small granularity value generates bigger clusters that increase the exploration space allowing to prove more lemmas. 
However, reducing the granularity value can also have a negative impact; for instance, in the JVM library the number of clusters
is reduced and this means a reduction of the proven theorems. 

The evaluation method presented in this section shows an estimation of how ML4PG can help during the proof development. 
This evaluation is a bit artificial since it heavily relies on having a well-developed background theory. For instance, 
in Example~\ref{ex:maxnACA}, the proof exploration method is able to automatically generate a proof for \lstinline?maxnACA? since
the lemmas \lstinline?maxnA? and \lstinline?maxnAC? were already in the library, but this is not a common scenario when the user
is creating his library. However, ML4PG can be useful for the user even if those two lemmas are not available: ML4PG can suggest
that \lstinline?maxnACA? is similar to  \lstinline?addnACA?, and from the proof of the latter theorem, the user can infer that he needs
to state the lemmas \lstinline?maxnA? and \lstinline?maxnAC? that were missing in the library. The automatic generation of lemmas based
on clustering information was studied for the ACL2 system in~\cite{lpar13} --- the extrapolation of the technique presented in~\cite{lpar13} to Coq is not trivial since ACL2 works with first-order logic and Coq works with higher-order logic. 



% 
% \begin{itemize}
%  \item Basic infrastructure of SSReflect library~\cite{SSReflect}: 1389 theorems formalised using SSReflect.
%  \item Formalisation about Java-like bytecode~\cite{HK14}: 49 theorems formalised using SSReflect.
%  \item Formalisation of Persistent Homology~\cite{HCMS12}: 306 theorems formalised using SSReflect.
%  \item Library Paths of HoTT development~\cite{hottbook}: 80 theorems formalised using plain Coq.
%  \item Formalisation of Nash Equilibrium~\cite{Ves06,nash}: 145 theorems formalised using plain Coq.
% \end{itemize}



\begin{table}
\tbl{\scriptsize{\emph{Percentage of automatically re-proven theorems.}}\label{tab:reproven}}{
\centering
\begin{tabular}{cccccc}
 \hline
 Library & Language & Granularity 1 & Granularity 3 & Granularity 5 & Theorems \\
 \hline

 SSReflect library  & SSReflect & $36\%$ & $35\%$ & $28\%$ & 1389\\
 
  JVM & SSReflect  & $56\%$ & $58\%$ & $65\%$ & 49\\
 
 Persistent Homology & SSReflect  & $0\%$ & $10\%$ & $12\%$ & 306\\
 
 Paths (HoTT) & Coq & $92\%$ & $91\%$ & $94\%$ &80\\
 
 Nash Equilibrium & Coq & $40\%$ & $37\%$ & $36\%$ & 145\\
   \hline
\end{tabular}}
 
\end{table}

\section{Future Work}
\label{sec:future}

Our use of clustering to pre-process \qspec{} signatures has required many decisions and tradeoffs to be made. Hence our approach is just one possibility out of many alternatives which could be investigated to push this work further. In addition, there are other ways in which machine learning could aid theory exploration besides our relevance filter technique. The Gantt chart in Figure \ref{fig:gantt} shows how these relate to the short- and long-term direction being taken by this research. Below, we elaborate on the details, background and motivation for these choices.

\subsection{Clustering Extensions}
\label{sec:preprocessing}

The most glaring omission in our algorithm is its disregard for types. By ignoring types, not only are we losing valuable information about expressions, but we also lose the ability to distinguish between constructors. This is because a constructor, like \hs{True} or \hs{Just}, has no internal structure; it is just a token. The distinguishing features of constructors are their types, which not only tell us which data type they construct, but also their arity, the types of their arguments, etc.

Our algorithm closely follows that of ML4PG, which \emph{does} support types. This is handled by populating matrix cells with tokens \emph{and} their types. Unfortunately this is more complicated in Haskell than it is in Coq, since types form a separate part of the language from terms, and we do not have an interactive Core environment to query for types (unlike ML4PG, which runs inside the Proof General environment).

One partial solution would be leave most Core expressions without types, but to include them for non-local identifiers (i.e. globals and constructors), which we can look up in a database. In fact, our ML4HS framework already includes such type information in its database, alongside the Core syntax trees. Integrating this information into our algorithm is the next logical step.

We can also compare the performance of our hand-selected features with \emph{learned} representations, like those reviewed in \citep{bengio2013representation}. This may provide an indication of how important it is to understand the language when identifying salient aspects of expressions, and how difficult various aspects of it might be to learn.

With more expressive features, it may also be useful to experiment with more powerful learning algorithms. An interesting possibility is to add a feedback loop between the theory exploration phase and the clustering phase, to more directly base the similarity of expressions on whether they (are predicted to) occur together in equations.

\subsection{Theory Exploration Extensions}

Our current approach is a rather conservative change to the existing theory exploration approaches, as it is essentially a wrapper around \qspec{}. There is potential for more radical changes to be made, which alter the search process itself.

\subsubsection{Variable Instantiation}

\qcheck{} is certainly the most popular property checker for Haskell, which motivates its use in \qspec{} to instantiate variables to random values. However, this task of finding type inhabitants has also been solved in many other ways, which may be worth investigating in place of \qcheck{} (or perhaps even as part of an ensemble).

The \textsc{SmallCheck} system \citep{runciman2008smallcheck} \emph{enumerates} values rather than sampling them randomly. Whilst this does not make \textsc{SmallCheck} objectively ``better'' than \qcheck{}, one major advantage is that it may use much less memory, as the generated values are built up incrementally. In contrast, \qcheck{} may generate very large values; in particular, generating tree structures na\"{\i}vely can cause them to grow exponentially. For example, here is a potential generator for \hs{RoseTree}s:

\begin{lstlisting}[language=Haskell, xleftmargin=0.1\textwidth, xrightmargin=0.1\textwidth]
genRoseTree = do f        <- arbitrary
                 subtrees <- listOf genRoseTree
                 return (Node f subtrees)
\end{lstlisting}

The \hs{listOf genRoseTree} call will return a list of arbitrary length, where each element is generated by \hs{genRoseTree}. This allows an arbitrary number of recursive calls to be made for each invocation of \hs{genRoseTree}, which will quickly exhaust the resources of any machine. Whilst such problems may be anticipated, or quickly spotted, in a property checking setting, this can be more difficult for our automated approach. For example, if a type does not have a generator available, we cannot use a library like \hs{derive} to create one automatically, as it suffers from this na\"{\i}vity problem.

A relative of \textsc{SmallCheck} is \textsc{Lazy SmallCheck} \citep{reich2013advances}, which uses laziness to only produce parts of a datastructure as they are demanded. This may narrow down our search procedures greatly, especially when predicates are involved. \qcheck{} allows predicates to restrict the values it tests with, and hence allows \emph{conditional} equations to be discovered. However, its implementation uses a simple rejection sampling technique: values are generated just as if the predicate were not there, and afterwards are filtered to reject any which do not satisfy the predicate. This makes it difficult to use very specific predicates, as it is unlikely that many of our random samples will exactly match our criteria. On the other hand, \textsc{Lazy SmallCheck} will focus its search on exactly those parts of the datastructure which are checked by the predicate, as those are the parts being forced to evaluate. This makes it much more likely that we will find values which satisfy the predicate, allowing us to effectively explore more specific conditional properties.

Other approaches to generating inhabitants include \textsc{Djinn} \citep{augustsson2005djinn}, which uses a decision procedure for a sub-set of Haskell types which in particular can generate and apply functions (unlike the above tools, which generate values ``bottom-up'' from constructors, and only use functions when they have been explicitly written in a generator). \textsc{MuCheck} \citep{le2014mucheck} is designed for \emph{mutation testing}, and contains combinators for altering functions in common ways (e.g. changing the order of pattern match clauses); whilst not as exhaustive as the other approaches, mutating existing values in this way is claimed to yield values which correspond more closely to what a programmer might write. This is an interesting possibility for focusing theory exploration on to more ``realistic'' areas of the search space, and hence avoiding some of the more useless or bizarre expressions that random search and enumeration may produce.

In fact, the database generated by our \textsc{AstPlugin} may prove helpful in generating values, since its type information can be fed to a tool like \textsc{Djinn} to discover chains of function applications for building values, which would be particularly useful in cases where constructors are private, like in our email example. This is similar to the \textsc{Hoogle} tool, but also offers the ability to use dependency information to avoid potentially infinite recursion.

The Core syntax trees in our database could also be used to generate theories for automated theorem provers. \hspec{} currently uses the GHC API to transform Core within its own process, however that approach suffers from the problems described in \S \ref{sec:astplugin}.

\subsubsection{Interestingness}
\label{sec:interestingness}

If we do succeed in producing a fast theory exploration system, which chooses productive combinations of terms and finds a large number of properties, we encounter the problem of managing the output: finding the needles we are interested in among the haystack of trivialities and coincidences.

This is governed by the ``interestingness'' criteria of the theory exploration system: what to keep and what to discard, and even what areas of the search space to prioritise. \qspec{}'s approach, briefly mentioned in \S \ref{sec:theoryexploration}, is very simple: we discard equations which are direct consequences of others, and keep all the rest. Different, and more sophisticated notions of interestingness have been widely studied in other fields, which may be applied in the context of theory exploration.

\iffalse TODO: Section that brings back ML discussion to TE is missing \fi

\bibliographystyle{plain}
\bibliography{../Bibtex}

%\iffalse  % WITHOUT APPENDICES
\iftrue    % WITH APPENDICES

\appendix

\section{Core Syntax}\label{sec:core}

The GHC Core language is based on \fc{}, and described in detail in \cite[Appendix C]{sulzmann2007system}. For our machine learning purposes we are mostly interested in the syntax of reducible expressions (representing Haskell of the form \hs{f a b \dots = \dots}), and use the simplified syntax given below in BNF style ($[]$ and $(,)$ denote repetition and grouping, respectively):

\begin{equation}\label{fig:coresyntax}
  \begin{split}
    expr\    \rightarrow\ & \CVar\ id                          \\
                       |\ & \CLit\ literal                     \\
                       |\ & \CApp\ expr\ expr                  \\
                       |\ & \CLam\ \mathcal{L}\ expr           \\
                       |\ & \CLet\ bind\ expr                  \\
                       |\ & \CCase\ expr\ \mathcal{L}\ [alt]   \\
                       |\ & \CType                             \\
    id\      \rightarrow\ & \CLocal\       \mathcal{L}         \\
                       |\ & \CGlobal\      \mathcal{G}         \\
                       |\ & \CConstructor\ \mathcal{D}         \\
    literal\ \rightarrow\ & \CLitNum\ \mathcal{N}              \\
                       |\ & \CLitStr\ \mathcal{S}              \\
    alt\     \rightarrow\ & \CAlt\ altcon\ expr\ [\mathcal{L}] \\
    altcon\  \rightarrow\ & \CDataAlt\ \mathcal{D}             \\
                       |\ & \CLitAlt\ literal                  \\
                       |\ & \CDefault                          \\
    bind\    \rightarrow\ & \CNonRec\ binder                   \\
                       |\ & \CRec\ [binder]                    \\
    binder   \rightarrow\ & \CBind\ \mathcal{L}\ expr
  \end{split}
\end{equation}
Where:
\begin{tabular}[t]{l @{ $=$ } l}
  $\mathcal{S}$ & string literals    \\
  $\mathcal{N}$ & numeric literals   \\
  $\mathcal{L}$ & local identifiers  \\
  $\mathcal{G}$ & global identifiers \\
  $\mathcal{D}$ & constructor identifiers
\end{tabular}

The full Core language emitted by GHC (as of version 7.10.2, the latest at the time of writing) is translated automatically to this simplified form prior to recurrent clustering. Our major restriction is to ignore type-level entities (such as datatype definitions, explicit casts, and differences between types/kinds/coercions). Our implementation also handles several other forms of literal (machine words of various sizes, individual characters, etc.), but we omit them here for brevity as their treatment is similar to those of strings and numerals.

\section{Rose Trees}\label{sec:rosetree}

The $toTree$ function shown below transforms Core expressions, described in Appendix \ref{sec:core} to rose trees. We follow the presentation in \cite{blundell2012bayesian} and define rose trees recursively as follows: $T$ is a rose tree if $T = (f, T_1, \dots, T_{n_T})$, where $f \in \mathbb{R}$ and $T_i$ are rose trees. $T_i$ are the \emph{sub-trees} of $T$ and $f$ is the \emph{feature at} $T$. $n_T$ may differ for each (sub-) tree; trees where $n_T = 0$ are \emph{leaves}.

The recursive definition is mostly routine; each repeated element (shown as $\dots$) has an example to indicate their handling, e.g. for $\CRec$ we apply $toTree$ to each $e_i$. We ignore values of $\mathcal{D}$, since constructors have no internal structure for us to compare; they can only be compared based on their types, which we do not currently support. We also ignore values from $\mathcal{S}$ and $\mathcal{N}$ as it simplifies our later definition of $\phi$, and we conjecture that the effect of such ``magic values'' on clustering real code is low.

%\begin{figure}
  \begin{align*}\label{fig:totree}
    toTree(e) &=
    \begin{cases}
      (\feature{\CVar},     toTree(e_1))                                 & \text{if $e = \CVar\ e_1$} \\
      (\feature{\CLit},     toTree(e_1))                                 & \text{if $e = \CLit\ e_1$} \\
      (\feature{\CApp},     toTree(e_1), toTree(e_2))                    & \text{if $e = \CApp\ e_1\ e_2$} \\
      (\feature{\CLam},     toTree(e_1))                                 & \text{if $e = \CLam\ l_1\ e_1$} \\
      (\feature{\CLet},     toTree(e_1), toTree(e_2))                    & \text{if $e = \CLet\ e_1\ e_2$} \\
      (\feature{\CCase},    toTree(e_1), toTree(a_1), \dots)             & \text{if $e = \CCase\ e_1\ l_1\ a_1\ \dots$} \\
      (\feature{\CType})                                                & \text{if $e = \CType$} \\
      (\feature{\CLocal},   (\feature{l_1}))                            & \text{if $e = \CLocal\ l_1$} \\
      (\feature{\CGlobal},  (\feature{g_1}))                            & \text{if $e = \CGlobal\ g_1$} \\
      (\feature{\CConstructor})                                         & \text{if $e = \CConstructor\ d_1$} \\
      (\feature{\CLitNum})                                              & \text{if $e = \CLitNum\ n_1$} \\
      (\feature{\CLitStr})                                              & \text{if $e = \CLitStr\ s_1$} \\
      (\feature{\CAlt},     toTree(e_1), toTree(e_2))                   & \text{if $e = \CAlt\ e_1\ e_2\ l_1\ \dots$}  \\
      (\feature{\CDataAlt})                                             & \text{if $e = \CDataAlt\ g_1$}  \\
      (\feature{\CLitAlt},  toTree(e_1))                                & \text{if $e = \CLitAlt\ e_1$}  \\
      (\feature{\CDefault})                                             & \text{if $e = \CDefault$}  \\
      (\feature{\CNonRec},  toTree(e_1))                                & \text{if $e = \CNonRec\ e_1$}  \\
      (\feature{\CRec},     toTree(e_1), \dots)                         & \text{if $e = \CRec\ e_1\ \dots$} \\
      (\feature{\CBind},    toTree(e_1))                                & \text{if $e = \CBind\ l_1\ e_1$}
    \end{cases}
  \end{align*}

%\end{figure}

\fi


\end{document}
